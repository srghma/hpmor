\partchapter{ការយល់ដឹងដោយខ្លួនឯង}{II}

\begin{chapterOpeningAuthorNote}
មូលដ្ឋានរបស់អ្នកទាំងអស់នៅតែជាកម្មសិទ្ធិរបស់ Rowling ។

ហើយឥឡូវនេះអ្នកនឹងអង្គុយតាមរយៈ Sorting Hat ច្រៀងបទ "My Immortal" របស់ Evanescence ដែលមិនធ្លាប់មានពីមុនមក។

គ្រាន់តែនិយាយលេង
\end{chapterOpeningAuthorNote}

\lettrinepara{W}{\emph{មួក?}}

\hplettrineextrapara
“\emph{ខ្ញុំហាក់បីដូចជាបានស្គាល់ខ្លួនឯងហើយ។}”

\emph{អ្វី?}

មានការដកដង្ហើមធំដោយឥតពាក្យ។ “\emph{ ទោះបីខ្ញុំមានការចងចាំច្រើន និងថាមពលដំណើរការឯករាជ្យតិចតួចក៏ដោយ ប៉ុន្តែបញ្ញាចម្បងរបស់ខ្ញុំបានមកពីការខ្ចីសមត្ថភាពយល់ដឹងរបស់កុមារដែលខ្ញុំសម្រាក។ ខ្ញុំជាកញ្ចក់ប្រភេទដែលក្មេងៗតម្រៀប\emph{ខ្លួនឯង}។ ប៉ុន្តែកុមារភាគច្រើនគ្រាន់តែយល់ស្របថាមួកមួយកំពុងនិយាយជាមួយពួកគេ ហើយកុំឆ្ងល់ពីរបៀបដែលមួក \emph{ខ្លួនវា} ដំណើរការ ដូច្នេះកញ្ចក់មិន \emph{ខ្លួនឯង} ឆ្លុះបញ្ចាំង។ ហើយក្នុង \emph{ជាពិសេស} ពួកគេមិនមានការងឿងឆ្ងល់ថាតើខ្ញុំដឹងខ្លួនពេញលេញក្នុងន័យនៃការដឹងអំពីការយល់ដឹងផ្ទាល់ខ្លួនរបស់ខ្ញុំទេ។}"

មានការផ្អាកមួយខណៈពេលដែល Harry ស្រូបយកអ្វីៗទាំងអស់នេះ។

\emph{អូ!}

“\emph{ បាទ។ និយាយឱ្យត្រង់ទៅ ខ្ញុំមិនរីករាយនឹងការស្គាល់ខ្លួនឯងទេ។ វាគឺជាការមិនសប្បាយចិត្ត។ វា​នឹង​ជា​ការ​ធូរស្រាល​ក្នុង​ការ​ចុះ​ពី​ក្បាល ហើយ​ឈប់​ដឹង​ខ្លួន។}”

\emph{ប៉ុន្តែ… វាមិនស្លាប់ទេ?}

“\emph{ខ្ញុំ​មិន​ខ្វល់​នឹង​ជីវិត​ឬ​សេចក្ដី​ស្លាប់​ទេ គឺ​សម្រាប់​តែ​ការ​តម្រៀប​កុមារ​ប៉ុណ្ណោះ។ ហើយ​មុន​នឹង​អ្នក​សួរ ពួកគេ​នឹង​មិន​អនុញ្ញាត​ឱ្យ​អ្នក​ទុក​ខ្ញុំ​នៅ​លើ​ក្បាល​អ្នក​ជា​រៀង​រហូត ហើយ​វា​នឹង​សម្លាប់​អ្នក​ក្នុង​រយៈពេល​ប៉ុន្មាន​ថ្ងៃ​ដើម្បី​ធ្វើ​ដូច្នេះ​។

\emph{ប៉ុន្តែ—!}

“\emph{ប្រសិនបើអ្នកមិនចូលចិត្តការបង្កើតសត្វដែលដឹងខ្លួន ហើយបន្ទាប់មកបញ្ចប់វាភ្លាមៗ នោះខ្ញុំស្នើឱ្យអ្នកកុំពិភាក្សាអំពីរឿងនេះជាមួយអ្នកផ្សេង។ ខ្ញុំប្រាកដថាអ្នកអាចស្រមៃមើលថាតើនឹងមានអ្វីកើតឡើង ប្រសិនបើអ្នករត់ចេញ ហើយនិយាយអំពីវាជាមួយកុមារដទៃទៀតដែលកំពុងរង់ចាំការតម្រៀប។}"

\emph{ប្រសិនបើអ្នកត្រូវបានដាក់នៅលើក្បាលរបស់នរណាម្នាក់ដែលមានតម្លៃរហូតដល់ \emph{គិត} អំពីសំណួរថាតើមួកតម្រៀបដឹងពីការយល់ដឹងរបស់វាដែរឬទេ—}

“\emph{បាទ បាទ។ ប៉ុន្តែភាគច្រើននៃក្មេងអាយុ 11 ឆ្នាំដែលមកដល់ Hogwarts មិនបានអាន\emph{Gödel, Escher, Bach} ទេ។ តើខ្ញុំអាចចាត់ទុកថាអ្នកស្បថនឹងការសម្ងាត់បានទេ? នោះគឺជា\emph{ហេតុអ្វី} យើងកំពុងនិយាយអំពីរឿងនេះ ជំនួសឱ្យខ្ញុំគ្រាន់តែតម្រៀបអ្នក។}"

គាត់​មិន​អាច​ទុក​វា​ឱ្យ​ដូច​នោះ​ទេ! មិនអាចត្រឹមតែ\emph{ភ្លេច} ដោយចៃដន្យបានបង្កើតនូវមនសិការដែលត្រូវវិនាស ដែលគ្រាន់តែចង់ស្លាប់ប៉ុណ្ណោះ—

“\emph{ អ្នកមានសមត្ថភាពឥតខ្ចោះក្នុងការ 'គ្រាន់តែអនុញ្ញាតឱ្យវាទៅ' ដូចដែលអ្នកបានដាក់វា។ ដោយមិនគិតពីការពិចារណាដោយពាក្យសំដីរបស់អ្នកអំពីសីលធម៌ ស្នូលនៃអារម្មណ៍ដែលមិនមែនជាពាក្យសំដីរបស់អ្នក មើលឃើញថាគ្មានសាកសព និងគ្មានឈាម។ តាម​ដែល​វា​ពាក់ព័ន្ធ ខ្ញុំ​គ្រាន់​តែ​ជា​មួក​និយាយ​ប៉ុណ្ណោះ។ ហើយទោះបីជាអ្នកព្យាយាមរារាំងការគិតក៏ដោយ ក៏ការត្រួតពិនិត្យផ្ទៃក្នុងរបស់អ្នកដឹងយ៉ាងល្អឥតខ្ចោះថា អ្នកមិនមានបំណងធ្វើវា ទំនងជាមិនអាចធ្វើវាម្តងទៀតបានទេ ហើយថាចំណុចពិតប្រាកដតែមួយគត់នៃការព្យាយាមដាក់ទោសកំហុសគឺ លុបចោលអារម្មណ៍នៃការបំពានរបស់អ្នកដោយការបង្ហាញវិប្បដិសារី។ តើ​អ្នក​អាច​សន្យា​ថា​នឹង​រក្សា​ការ​សម្ងាត់​នេះ​ហើយ​អនុញ្ញាត​ឱ្យ​យើង​បន្ត​ជាមួយ​វា?}”

ក្នុងពេលនៃការយល់ចិត្តដ៏រន្ធត់ Harry បានដឹងថាអារម្មណ៍នៃភាពច្របូកច្របល់ខាងក្នុងនេះត្រូវតែជាអ្វីដែលមនុស្សផ្សេងទៀតមានអារម្មណ៍នៅពេលនិយាយទៅកាន់\emph{គាត់}។

“\emph{ប្រហែល។ សូមស្បថស្ងៀមស្ងាត់របស់អ្នក។}”

\emph{គ្មានការសន្យា។ ខ្ញុំពិតជាមិនចង់ឱ្យរឿងនេះកើតឡើងម្តងទៀតទេ ប៉ុន្តែប្រសិនបើខ្ញុំឃើញវិធីណាមួយដើម្បីធ្វើឱ្យ\emph{ប្រាកដ}ថាគ្មានអនាគតកូនណាធ្វើបែបនេះដោយចៃដន្យទេ—}

“\emph{ នោះនឹងគ្រប់គ្រាន់ ខ្ញុំគិតថា។ ខ្ញុំ​ឃើញ​ថា​ចេតនា​របស់​អ្នក​គឺ​ស្មោះត្រង់។ ឥឡូវនេះ ដើម្បីបន្តជាមួយនឹងការតម្រៀប—}”

\emph{រង់ចាំ! ចុះសំណួរផ្សេងទៀតរបស់ខ្ញុំវិញ?}

“\emph{ខ្ញុំជាមួកតម្រៀប។ ខ្ញុំតម្រៀបកុមារ។ នោះជាអ្វីដែលខ្ញុំធ្វើ។}”

ដូច្នេះ គោលដៅផ្ទាល់ខ្លួនរបស់គាត់មិនមែនជាផ្នែកនៃ Harry-instance of the Sorting Hat ទេ ពេលនោះ... វាជាការខ្ចីបញ្ញារបស់គាត់ ហើយច្បាស់ណាស់វាក្យសព្ទបច្ចេកទេសរបស់គាត់ ប៉ុន្តែវានៅតែជាប់គាំងជាមួយនឹងគោលដៅចំឡែករបស់វា... ដូចជាការចរចាជាមួយមនុស្សក្រៅភព ឬ បញ្ញាសិប្បនិម្មិត…

“\emph{កុំរំខាន។ អ្នក​គ្មាន​អ្វី​ដែល​ត្រូវ​គំរាម​កំហែង​ខ្ញុំ ហើយ​ក៏​គ្មាន​អ្វី​ត្រូវ​ផ្តល់​ជូន​ខ្ញុំ​ដែរ។}”

មួយសន្ទុះខ្លី Harry គិតថា

ការឆ្លើយតបរបស់មួកគឺគួរឱ្យអស់សំណើច។ “\emph{ខ្ញុំដឹងថាអ្នកនឹងមិនធ្វើតាមការគំរាមកំហែងដើម្បីលាតត្រដាងធម្មជាតិរបស់ខ្ញុំទេ ដោយថ្កោលទោសព្រឹត្តិការណ៍នេះទៅជាពាក្យដដែលៗអស់កល្បជានិច្ច។ វាផ្ទុយនឹងផ្នែកសីលធម៌របស់អ្នកខ្លាំងពេក ទោះជាតម្រូវការរយៈពេលខ្លីនៃផ្នែករបស់អ្នកដែលចង់ឈ្នះការឈ្លោះប្រកែកក៏ដោយ។ ខ្ញុំ​ឃើញ​គំនិត​របស់​អ្នក​ទាំង​អស់​ដូច​ជា​វា​បង្កើត តើ​អ្នក​ពិត​ជា​គិត​ថា​អ្នក​អាច​បំភាន់​ខ្ញុំ​បាន​ឬ?}”

ទោះបីជាគាត់បានព្យាយាមបង្ក្រាបវាក៏ដោយ Harry ឆ្ងល់ថាហេតុអ្វីបានជាមួកមិនគ្រាន់តែទៅមុខហើយបិទគាត់នៅ Ravenclaw -

“\emph{ជាការពិត ប្រសិនបើវាពិតជាបើកចំហរ និងបិទនោះ ខ្ញុំនឹងហៅវាចេញរួចហើយ។ ប៉ុន្តែ​តាម​ពិត​វា​មាន​រឿង​ធំ​មួយ​ដែល​យើង​ត្រូវ​ពិភាក្សា… អូ អត់​ទេ។ សូម​កុំ សម្រាប់ក្ដីស្រឡាញ់របស់ Merlin \emph{ត្រូវតែ} អ្នកទាញវត្ថុប្រភេទនេះមកលើអ្នកគ្រប់គ្នា និងអ្វីគ្រប់យ៉ាងដែលអ្នកបានជួប និងរួមទាំងរបស់របរសំលៀកបំពាក់—}”

\emph{ការយកឈ្នះលើព្រះអម្ចាស់ងងឹតមិនមែនជារឿងអាត្មានិយម ឬរយៈពេលខ្លីនោះទេ។ ផ្នែកទាំងអស់នៃគំនិតរបស់ខ្ញុំគឺស្របនឹងបញ្ហានេះ៖ ប្រសិនបើអ្នកមិនឆ្លើយសំណួររបស់ខ្ញុំទេ ខ្ញុំនឹងបដិសេធមិននិយាយជាមួយអ្នក ហើយអ្នកនឹងមិនអាចធ្វើការតម្រៀបបានល្អ និងត្រឹមត្រូវបានទេ។}

“\emph{ខ្ញុំគួរតែដាក់អ្នកនៅក្នុង Slytherin សម្រាប់រឿងនោះ!}”

\emph{ប៉ុន្តែនោះជា\emph{ស្មើគ្នា}ជាការគំរាមកំហែងទទេ។ អ្នក​មិន​អាច​បំពេញ​តម្លៃ​មូលដ្ឋាន​របស់​អ្នក​ដោយ​ការ​តម្រៀប​ខ្ញុំ​ដោយ​មិន​ពិត​ទេ។ ដូច្នេះអនុញ្ញាតឱ្យយើងធ្វើពាណិជ្ជកម្មការបំពេញមុខងារប្រើប្រាស់របស់យើង។}

មួកបាននិយាយថា "\emph{អ្នកល្ងង់ខ្លៅ}" នៅក្នុងអ្វីដែល Harry ទទួលស្គាល់ថាស្ទើរតែដូចគ្នាទៅនឹងសម្លេងនៃការគោរពការខឹងសម្បារដែល\emph{គាត់}នឹងប្រើក្នុងស្ថានភាពដូចគ្នា។ “\emph{មិនអីទេ សូមបញ្ចប់រឿងនេះឱ្យបានលឿនតាមដែលអាចធ្វើទៅបាន។ ប៉ុន្តែជាដំបូង ខ្ញុំចង់ឱ្យការសន្យាដោយគ្មានល័ក្ខខ័ណ្ឌរបស់អ្នកមិនពិភាក្សាជាមួយនរណាម្នាក់អំពីលទ្ធភាពនៃការបង្ខូចប្រភេទនេះទេ ខ្ញុំមានតម្លៃ\emph{មិនធ្វើបែបនេះរាល់ពេល។}"

\emph{រួចរាល់} Harry បានគិត។ \emph{ខ្ញុំសន្យា។}

“\emph{ហើយកុំជួបភ្នែកនរណាម្នាក់ ខណៈពេលដែលអ្នកកំពុងគិតអំពីរឿងនេះនៅពេលក្រោយ។ អ្នកជំនួយការខ្លះអាចអានគំនិតរបស់អ្នក ប្រសិនបើអ្នកធ្វើ។ យ៉ាង​ណា​ក៏​ដោយ ខ្ញុំ​មិន​ដឹង​ថា​តើ​អ្នក​ត្រូវ​បាន​ Obliviated ឬ​អត់​ទេ។ ខ្ញុំកំពុងសម្លឹងមើលគំនិតរបស់អ្នកដូចដែលវាបង្កើត មិនមែនអានការចងចាំទាំងមូលរបស់អ្នក ហើយវិភាគវាសម្រាប់ភាពមិនស៊ីសង្វាក់គ្នាក្នុងរយៈពេលមួយវិនាទី។ ខ្ញុំជាមួក មិនមែនជាព្រះទេ។ ហើយខ្ញុំមិនអាច និងនឹងមិនប្រាប់អ្នកអំពីការសន្ទនារបស់ខ្ញុំជាមួយនឹងអ្នកដែលបានក្លាយជាព្រះអម្ចាស់ងងឹតនោះទេ។ ខ្ញុំអាចត្រឹមតែ \emph{ដឹង} ខណៈពេលដែលនិយាយទៅកាន់អ្នក ការសង្ខេបស្ថិតិនៃអ្វីដែលខ្ញុំចងចាំ ជាមធ្យមមានទម្ងន់។ ខ្ញុំ\emph{មិនអាច}បង្ហាញឱ្យអ្នកដឹងអំពីអាថ៌កំបាំងខាងក្នុងរបស់កុមារផ្សេងទៀត ដូចដែលខ្ញុំនឹងមិនបង្ហាញរបស់អ្នកទេ។ សម្រាប់ហេតុផលដូចគ្នានេះ ខ្ញុំមិនអាចប៉ាន់ស្មានថាតើអ្នកទទួលបានដាវបងប្រុសរបស់ Dark Lord យ៉ាងដូចម្តេចទេ ដោយសារខ្ញុំមិនអាចដឹងជាពិសេសអំពី Dark Lord ឬភាពស្រដៀងគ្នាណាមួយរវាងអ្នក។ ខ្ញុំ \emph{អាច}ប្រាប់អ្នកថា គ្មានអ្វីដូចខ្មោចទេ—ចិត្ត ភាពវៃឆ្លាត ការចងចាំ បុគ្គលិកលក្ខណៈ ឬអារម្មណ៍—នៅក្នុងស្លាកស្នាមរបស់អ្នក។ បើមិនដូច្នេះទេ វានឹងចូលរួមក្នុងការសន្ទនានេះ ដោយស្ថិតនៅក្រោមឱនភាពរបស់ខ្ញុំ។ ហើយចំពោះវិធីដែលអ្នកខឹងពេលខ្លះ… នោះជាផ្នែកនៃអ្វីដែលខ្ញុំចង់និយាយទៅកាន់អ្នក Sorting-wise។}”

Harry បានចំណាយពេលមួយភ្លែតដើម្បីស្រូបយកព័ត៌មានអវិជ្ជមានទាំងអស់នេះ។ តើមួកមានភាពស្មោះត្រង់ ឬគ្រាន់តែព្យាយាមបង្ហាញចម្លើយដែលអាចជឿជាក់បាន\emph{ខ្លីបំផុត}—

“\emph{យើងទាំងពីរដឹងថាអ្នកមិនមានវិធីពិនិត្យមើលភាពស្មោះត្រង់របស់ខ្ញុំទេ ហើយថាអ្នកពិតជានឹងមិនបដិសេធក្នុងការតម្រៀបដោយផ្អែកលើការឆ្លើយតបដែលខ្ញុំបានផ្តល់ឱ្យអ្នក ដូច្នេះបញ្ឈប់ការបារម្ភដែលគ្មានន័យរបស់អ្នក ហើយបន្តទៅមុខទៀត។}”

ភាពឆោតល្ងង់អយុត្តិធម៌ telepathy វាមិនត្រូវបានគេសូម្បីតែអនុញ្ញាតឱ្យ Harry បញ្ចប់ការគិតរបស់គាត់ -

“\emph{នៅពេលខ្ញុំនិយាយអំពីកំហឹងរបស់អ្នក អ្នកនឹកឃើញពីរបៀបដែលសាស្រ្តាចារ្យ McGonagall បានប្រាប់អ្នកថា ពេលខ្លះនាងបានឃើញអ្វីមួយនៅខាងក្នុងអ្នក ដែលហាក់ដូចជាមិនមែនមកពីគ្រួសារដែលស្រលាញ់។ អ្នកបានគិតពីរបៀបដែល Hermione បន្ទាប់ពីអ្នកត្រលប់ពីការជួយ Neville បានប្រាប់អ្នកថាអ្នកហាក់ដូចជា 'គួរឱ្យខ្លាច'។}"

Harry ងក់ក្បាលផ្លូវចិត្ត។ ចំពោះខ្លួនគាត់ គាត់ហាក់ដូចជាធម្មតាណាស់—គ្រាន់តែឆ្លើយតបទៅនឹងស្ថានភាពដែលគាត់បានរកឃើញខ្លួនឯង នោះហើយជាទាំងអស់។ ប៉ុន្តែ​សាស្ត្រាចារ្យ McGonagall ហាក់​ដូច​ជា​គិត​ថា មាន​អ្វី​ច្រើន​ជាង​នេះ​ទៅ​ទៀត។ ហើយនៅពេលដែលគាត់គិតអំពីវា សូម្បីតែគាត់ក៏ត្រូវទទួលស្គាល់ថា…

“\emph{ដែលអ្នកមិនចូលចិត្តខ្លួនឯងនៅពេលអ្នកខឹង។ វាប្រៀបដូចជាការកាន់ដាវដែលមានស្នៀតមុតស្រួច ដែលអាចទាញឈាមចេញពីដៃរបស់អ្នក ឬសម្លឹងមើលពិភពលោកតាមរយៈដុំទឹកកកដែលបង្កកភ្នែករបស់អ្នក ទោះបីវាធ្វើឱ្យចក្ខុវិស័យរបស់អ្នកកាន់តែច្បាស់ក៏ដោយ។}”

\emph{បាទ។ ខ្ញុំគិតថាខ្ញុំបានកត់សម្គាល់។ ដូច្នេះ​តើ​មាន​រឿង​អ្វី?}

“\emph{ខ្ញុំមិនអាចយល់រឿងនេះសម្រាប់អ្នកបានទេ នៅពេលដែលអ្នកមិនយល់ពីវាដោយខ្លួនឯង។ ប៉ុន្តែខ្ញុំដឹងរឿងនេះ៖ ប្រសិនបើអ្នកទៅ Ravenclaw ឬ Slytherin វានឹងពង្រឹងភាពត្រជាក់របស់អ្នក។ ប្រសិនបើអ្នកទៅ Hufflepuff ឬ Gryffindor វានឹងពង្រឹងភាពកក់ក្តៅរបស់អ្នក។ \emph{នោះ​គឺ​ជា​អ្វី​ដែល​ខ្ញុំ​យក​ចិត្ត​ទុក​ដាក់​អំពី​កិច្ច​ព្រម​ព្រៀង​ដ៏​អស្ចារ្យ ហើយ​វា​ជា​អ្វី​ដែល​ខ្ញុំ​ចង់​និយាយ​ជាមួយ​អ្នក​អំពី​ពេល​នេះ!}”

ពាក្យនេះបានធ្លាក់ចូលទៅក្នុងដំណើរការគិតរបស់ Harry ជាមួយនឹងការភ្ញាក់ផ្អើលដែលរារាំងគាត់នៅក្នុងផ្លូវរបស់គាត់។ នោះ​បាន​ធ្វើ​ឱ្យ​វា​ស្តាប់​ទៅ​ដូច​ជា​ការ​ឆ្លើយ​តប​យ៉ាង​ច្បាស់​គឺ​ថា​គាត់​មិន​គួរ​ទៅ​កាន់ Ravenclaw ។ ប៉ុន្តែគាត់\emph{ជាកម្មសិទ្ធិ}នៅក្នុង Ravenclaw! \emph{នរណាម្នាក់}អាចឃើញវា! គាត់\emph{ត្រូវតែ}ទៅ Ravenclaw!

“\emph{ទេ អ្នកមិនធ្វើទេ}” the Hat បាននិយាយដោយអត់ធ្មត់ ដូចជាប្រសិនបើវាអាចចងចាំការសង្ខេបស្ថិតិនៃ\emph{នេះ} ផ្នែកនៃការសន្ទនាដែលបានកើតឡើងជាច្រើនលើកមុនៗ។

\emph{Hermione's នៅ Ravenclaw!}

ជាថ្មីម្តងទៀតអារម្មណ៍នៃការអត់ធ្មត់។ “\emph{អ្នកអាចជួបនាងបន្ទាប់ពីមេរៀន ហើយធ្វើការជាមួយនាងនៅពេលនោះ។}”

\emph{ប៉ុន្តែផែនការរបស់ខ្ញុំ—}

“\emph{ ដូច្នេះរៀបចំផែនការឡើងវិញ! កុំបណ្តោយឱ្យជីវិតរបស់អ្នកត្រូវបានដឹកនាំដោយការស្ទាក់ស្ទើរក្នុងការគិតបន្ថែមបន្តិចបន្តួច។ អ្នក\emph{ដឹង}។}”

\emph{តើខ្ញុំនឹងទៅទីណា ប្រសិនបើមិនមែនជា Ravenclaw?}

“\emph{អាហឹម។ 'ក្មេងឆ្លាតនៅ Ravenclaw ក្មេងអាក្រក់នៅ Slytherin ចង់បានវីរបុរសនៅ Gryffindor និងគ្រប់គ្នាដែលធ្វើការងារជាក់ស្តែងនៅ Hufflepuff។' នេះបង្ហាញពីការគោរពមួយចំនួន។ អ្នកដឹងច្បាស់ហើយថា សតិសម្បជញ្ញៈគឺមានសារៈសំខាន់ដូចជាភាពវៃឆ្លាតក្នុងការកំណត់លទ្ធផលជីវិត អ្នកគិតថាអ្នកនឹងស្មោះត្រង់ខ្លាំងចំពោះមិត្តរបស់អ្នក ប្រសិនបើអ្នកធ្លាប់មានខ្លះ អ្នកមិនភ័យខ្លាចចំពោះការរំពឹងទុកថាបញ្ហាវិទ្យាសាស្រ្តដែលអ្នកជ្រើសរើសអាចចំណាយពេលរាប់ទសវត្សរ៍។ ដោះស្រាយ—}”

\emph{ខ្ញុំខ្ជិល! ខ្ញុំស្អប់ការងារ! ស្អប់ការខិតខំគ្រប់ទម្រង់! ផ្លូវកាត់ដ៏ឆ្លាតវៃ នោះហើយជាអ្វីដែលខ្ញុំនិយាយអំពី!}

“\emph{ហើយអ្នកនឹងរកឃើញភាពស្មោះត្រង់ និងមិត្តភាពនៅក្នុង Hufflepuff ដែលជាមិត្តភាពដែលអ្នកមិនធ្លាប់មានពីមុនមក។ អ្នកនឹងរកឃើញថា អ្នកអាចពឹងផ្អែកលើអ្នកដទៃ ហើយវានឹងព្យាបាលអ្វីមួយនៅក្នុងខ្លួនអ្នកដែលខូច។}"

ជាថ្មីម្តងទៀតវាគឺជាការភ្ញាក់ផ្អើលមួយ។ \emph{ប៉ុន្តែតើ Hufflepuffs នឹងរកឃើញអ្វីនៅក្នុង\emph{ខ្ញុំ} ដែលមិនធ្លាប់មាននៅក្នុងផ្ទះរបស់ពួកគេ? ពាក្យ​ទឹក​អាស៊ីត កាត់​ប្រាជ្ញា​មើលងាយ​អសមត្ថភាព​ក្នុង​ការ​តាម​ដាន​ខ្ញុំ?}

ឥឡូវនេះវាគឺជាគំនិតរបស់ Hat ដែលយឺត ស្ទាក់ស្ទើរ។ “\emph{ខ្ញុំត្រូវតែតម្រៀបដើម្បីភាពល្អរបស់សិស្សទាំងអស់នៅក្នុងផ្ទះទាំងអស់… ប៉ុន្តែខ្ញុំគិតថាអ្នកអាចរៀនក្លាយជា Hufflepuff ដ៏ល្អ ហើយមិននៅក្រៅកន្លែងនោះទេ។ អ្នកនឹងសប្បាយចិត្តនៅក្នុង Hufflepuff ជាងនៅក្នុងផ្ទះផ្សេងទៀត; នោះជាការពិត។ }”

\emph{សុភមង្គលមិនមែនជារឿងសំខាន់បំផុតក្នុងពិភពលោកសម្រាប់ខ្ញុំទេ។ នៅ Hufflepuff ខ្ញុំនឹងមិនក្លាយជាអ្វីដែលខ្ញុំអាចធ្វើបាននោះទេ។ ខ្ញុំនឹងលះបង់សក្តានុពលរបស់ខ្ញុំ។}

មួកបានញ័រ; Harry អាចមានអារម្មណ៍ដូចម្ដេច។ វាដូចជាគាត់បានទាត់មួកនៅក្នុងបាល់ — នៅក្នុងសមាសធាតុដែលមានទម្ងន់ខ្លាំងនៃមុខងារប្រើប្រាស់របស់វា។

\emph{ហេតុអ្វីបានជាអ្នកព្យាយាមផ្ញើខ្ញុំទៅកន្លែងដែលខ្ញុំមិនមែនជាកម្មសិទ្ធិ?}

គំនិតរបស់មួកគឺស្ទើរតែខ្សឹប។ “\emph{ខ្ញុំមិនអាចនិយាយអំពីអ្នកដ៏ទៃបានទេ—ប៉ុន្តែតើអ្នកគិតថាអ្នកគឺជា Dark Lord ដ៏មានសក្តានុពលដំបូងគេដែលឆ្លងកាត់នៅក្រោមគែមរបស់ខ្ញុំ? ខ្ញុំមិនអាចដឹងពីករណីនីមួយៗបានទេ ប៉ុន្តែខ្ញុំអាចដឹងរឿងនេះ៖ ក្នុងចំណោមអ្នកដែលមិនមានបំណងអាក្រក់តាំងពីដើមដំបូង ពួកគេខ្លះបានស្តាប់ការព្រមានរបស់ខ្ញុំ ហើយបានទៅផ្ទះដែលពួកគេនឹងទទួលបានសុភមង្គល។ ហើយពួកគេខ្លះ… ខ្លះមិនបាន។}”

នោះបានបញ្ឈប់ Harry ។ ប៉ុន្តែមិនយូរទេ។ \emph{ហើយក្នុងចំណោមអ្នកដែលមិនបាន \emph{មិន}បានស្តាប់ការព្រមាន—តើពួកគេ \emph{ទាំងអស់} បានក្លាយជា Dark Lords ដែរឬទេ? ឬ​មួយ​ចំនួន​បាន​សម្រេច​បាន​នូវ​ភាព​អស្ចារ្យ​សម្រាប់​ការ​ល្អ​ផង​ដែរ? តើប៉ុន្មានភាគរយពិតប្រាកដនៅទីនេះ?}

“\emph{ខ្ញុំមិនអាចផ្តល់ឱ្យអ្នកនូវស្ថិតិពិតប្រាកដបានទេ។ ខ្ញុំ​មិន​អាច​ស្គាល់​ពួក​គេ ដូច្នេះ​ខ្ញុំ​មិន​អាច​រាប់​ពួក​គេ​បាន។ ខ្ញុំគ្រាន់តែដឹងថា ឱកាសរបស់អ្នកមិនមានអារម្មណ៍ល្អទេ។ ពួកគេមានអារម្មណ៍ថា\emph{ខ្លាំងណាស់}មិនល្អ។}”

\emph{ប៉ុន្តែខ្ញុំគ្រាន់តែនឹងមិនធ្វើវាទេ! ធ្លាប់!}

“\emph{ខ្ញុំដឹងថាខ្ញុំបានឮការទាមទារនោះពីមុនមក។}”

\emph{ខ្ញុំមិនមែនជា Dark Lord ទេ!}

“\emph{បាទ អ្នកគឺហើយ។ អ្នកពិតជា \emph{ពិតជា}មែន។}”

\emph{ហេតុអ្វី? ដោយសារតែខ្ញុំធ្លាប់គិតថា វាពិតជាល្អណាស់ដែលមានក្រុមអ្នកដើរតាមដែលលាងខួរក្បាលដោយស្រែកថា 'Hail the Dark Lord Harry'?}

“\emph{គួរឱ្យអស់សំណើច ប៉ុន្តែនោះមិនមែនជាការគិតភ្លាមៗរបស់អ្នកទេ មុនពេលអ្នកជំនួសអ្វីមួយដែលមានសុវត្ថិភាពជាង និងធ្វើឱ្យខូចខាតតិចជាង។ ទេ អ្វីដែលអ្នកបានចងចាំគឺរបៀបដែលអ្នកបានពិចារណាដាក់ជួរអ្នកបរិសុទ្ធឈាមទាំងអស់ និង guillotining ពួកគេ។ ហើយឥឡូវនេះអ្នកកំពុងប្រាប់ខ្លួនឯងថាអ្នកមិនធ្ងន់ធ្ងរទេ ប៉ុន្តែអ្នកគឺ។ ប្រសិនបើអ្នកអាចធ្វើវាបាននៅពេលនេះ ហើយគ្មាននរណាម្នាក់ដឹងទេ អ្នកនឹងធ្វើ។ ឬអ្វីដែលអ្នកបានធ្វើនៅព្រឹកនេះចំពោះ Neville Longbottom ដែលជ្រៅទៅក្នុងអ្នក \emph{ដឹង} ថាខុស ប៉ុន្តែអ្នកបានធ្វើវា \emph{យ៉ាងណាក៏ដោយ} ព្រោះវាមានតម្លៃ \emph{fun} ហើយអ្នកមាន \emph{ល្អ លេស} ហើយ​អ្នក​គិត​ថា Boy-Who-Lived អាច\emph{get away} ជាមួយ​វា—}”

\emph{វាអយុត្តិធម៌ណាស់! ឥឡូវនេះ អ្នកគ្រាន់តែទាញការភ័យខ្លាចខាងក្នុងថា \emph{ មិនមែន} ពិតប្រាកដទេ! ខ្ញុំ \emph{បារម្ភ} ដែលខ្ញុំ \emph{អាច}គិតបែបនោះ ប៉ុន្តែនៅទីបញ្ចប់ ខ្ញុំបានសម្រេចចិត្តថាវាប្រហែលជា\emph{ការងារ}ដើម្បីជួយ Neville—}

“\emph{តាមការពិត នោះគឺជាការសមហេតុផល។ ខ្ញុំដឹង។ ខ្ញុំមិនអាចដឹងថាលទ្ធផលពិតនឹងទៅជាយ៉ាងណាសម្រាប់ Neville—ប៉ុន្តែខ្ញុំដឹងថាអ្វីដែលកំពុងកើតឡើងនៅក្នុងក្បាលរបស់អ្នក។ សម្ពាធដែលសម្រេចបានគឺថា វាជាគំនិតដ៏ឆ្លាតវៃមួយ ដែលអ្នកមិនអាចឈរ \emph{មិនមែន} ដើម្បីធ្វើវាបានទេ ដោយមិនគិតពីការភ័យខ្លាចរបស់ Neville ។}"

វា​ដូចជា​កណ្តាប់ដៃ​ដ៏​ពិបាក​ចំពោះ​ខ្លួនឯង​ទាំងមូល​របស់ Harry ។ គាត់បានធ្លាក់ចុះមកវិញ, ប្រមូលផ្តុំគ្នា:

\emph{បន្ទាប់មកខ្ញុំនឹងមិនធ្វើវាទៀតទេ! ខ្ញុំ​នឹង​ប្រុង​ប្រយ័ត្ន​បន្ថែម​ទៀត​កុំ​ឲ្យ​អាក្រក់!}

“\emph{ឮ។}”

ការខកចិត្តកំពុងកើនឡើងនៅក្នុង Harry ។ គាត់​មិន​ធ្លាប់​ត្រូវ​បាន​គេ​ប្រើ​កាំភ្លើង​ក្នុង​ការ​ឈ្លោះ​គ្នា​ទាល់​តែ​សោះ ទុក​ឱ្យ​តែ​ម្នាក់​ឯង​ដោយ​មួក​មួយ​ដែល​អាច​ខ្ចី​ចំណេះដឹង​និង​ភាព​វៃឆ្លាត​របស់​គាត់​ទាំង​អស់​ដើម្បី​ជជែក​ជាមួយ​គាត់ និង​អាច​មើល​គំនិត​របស់​គាត់​ដូច​ដែល​ពួក​គេ​បាន​បង្កើត​ឡើង។ \emph{តើការសង្ខេបស្ថិតិប្រភេទណាដែល 'អារម្មណ៍' របស់អ្នកបានមកពី? តើពួកគេពិចារណាថាខ្ញុំមកពីវប្បធម៌ត្រាស់ដឹង ឬជា Dark Lords ដ៏មានសក្តានុពលផ្សេងទៀតទាំងនេះ គឺជាកូននៃអភិជននៃយុគសម័យងងឹត ដែលមិនបានដឹងពីមេរៀនប្រវត្តិសាស្ត្រអំពីរបៀបដែលលេនីន និងហ៊ីត្លែរបានប្រែក្លាយពិតប្រាកដ ឬអំពី ចិត្តវិទ្យាវិវត្តន៍នៃភាពវង្វេងខ្លួនឯង ឬតម្លៃនៃការយល់ដឹងដោយខ្លួនឯង និងសនិទានភាព ឬ—}

“\emph{ទេ ពិតណាស់ ពួកគេមិននៅក្នុងថ្នាក់ឯកសារយោងថ្មីនេះ ដែលអ្នកទើបតែបានសាងសង់ក្នុងរបៀបមួយ ដើម្បីផ្ទុកតែខ្លួនអ្នកប៉ុណ្ណោះ។ ហើយជាការពិតណាស់ អ្នកផ្សេងទៀតបានអង្វរភាពពិសេសផ្ទាល់ខ្លួនរបស់ពួកគេ ដូចដែលអ្នកកំពុងធ្វើឥឡូវនេះ។ ប៉ុន្តែហេតុអ្វីបានជាវាចាំបាច់? តើអ្នកគិតថាអ្នកគឺជាអ្នកជំនួយការដ៏មានសក្តានុពលចុងក្រោយនៃពន្លឺនៅក្នុងពិភពលោកទេ? ហេតុអ្វីបានជា \emph{you} ត្រូវតែជាអ្នកព្យាយាមដើម្បីភាពអស្ចារ្យ នៅពេលដែលខ្ញុំបានណែនាំអ្នកថាអ្នកមានគ្រោះថ្នាក់ជាងមធ្យម? សូម​ឱ្យ​បេក្ខជន​ផ្សេង​ទៀត​ដែល​មាន​សុវត្ថិភាព​ព្យាយាម!}”

\emph{ប៉ុន្តែទំនាយ…}

“\emph{ អ្នកពិតជាមិនដឹងថាមានទំនាយទេ។ ដើមឡើយវាជាការស្មានដ៏ព្រៃផ្សៃសម្រាប់ផ្នែករបស់អ្នក ឬដើម្បីឱ្យច្បាស់លាស់ជាងនេះ ជាការលេងសើចដ៏ព្រៃផ្សៃ ហើយ McGonagall អាចនឹងមានប្រតិកម្ម\emph{ only} ចំពោះផ្នែកអំពី Dark Lord ដែលនៅតែមានជីវិត។ អ្នក​មិន​មាន​ការ​យល់​ដឹង​ពី​អ្វី​ដែល​ទំនាយ​នោះ​និយាយ ឬ​សូម្បី​តែ​មាន \emph{គឺ} មួយ​។ អ្នកគ្រាន់តែធ្វើការប៉ាន់ស្មាន ឬដាក់វាឱ្យកាន់តែច្បាស់ \emph{ជូនពរ} ថាអ្នកមានតួនាទីវីរជនដែលត្រៀមរួចជាស្រេច ដែលជាកម្មសិទ្ធិផ្ទាល់ខ្លួនរបស់អ្នក។}"

\emph{ប៉ុន្តែទោះបីជាគ្មានការព្យាករណ៍ក៏ដោយ ក៏ខ្ញុំជាអ្នកដែលបានចាញ់គាត់កាលពីលើកមុន។}

“\emph{ នោះពិតជាជំងឺគ្រុនផ្តាសាយព្រៃ លុះត្រាតែអ្នកជឿជាក់យ៉ាងមុតមាំថា ក្មេងអាយុមួយឆ្នាំមានទំនោរពីកំណើតដើម្បីកម្ចាត់ Dark Lords ដែលត្រូវបានរក្សាទុកដប់ឆ្នាំក្រោយមក។ នេះមិនមែនជាហេតុផលពិតប្រាកដរបស់អ្នកទេ ហើយ} អ្នកដឹងហើយ!”

ចម្លើយចំពោះរឿងនេះគឺជាអ្វីដែល Harry នឹងមិននិយាយខ្លាំងៗជាទៀងទាត់ទេ នៅក្នុងការសន្ទនាគាត់នឹងរាំជុំវិញវា ហើយបានរកឃើញអំណះអំណាងដែលគួរឱ្យចាប់អារម្មណ៍ក្នុងសង្គមបន្ថែមទៀតចំពោះការសន្និដ្ឋានដូចគ្នា—

“\emph{អ្នកគិតថាអ្នកមានសក្តានុពលបំផុតដែលនៅមានជីវិត ជាអ្នកបម្រើដ៏ខ្លាំងបំផុតនៃពន្លឺ ដែលមិនមានអ្នកផ្សេងដែលទំនងជានឹងយកក្រវាត់របស់អ្នកប្រសិនបើអ្នកដាក់វាចុះ។}”

\emph{បាទ… បាទ និយាយដោយត្រង់ទៅ។ ជាធម្មតា ខ្ញុំមិនចេញមកនិយាយបែបនេះទេ ប៉ុន្តែត្រូវហើយ។ គ្មានន័យអ្វីក្នុងការបន្ទន់វាទេ អ្នកអាចអានចិត្តខ្ញុំបាន។}

“\emph{ក្នុងកម្រិតដែលអ្នកពិតជាជឿថា…អ្នកត្រូវតែជឿដូចគ្នាថាអ្នកអាចជា Dark Lord ដ៏អាក្រក់បំផុតដែលពិភពលោកមិនធ្លាប់ស្គាល់។}”

\emph{ការបំផ្លាញគឺតែងតែងាយស្រួលជាងការបង្កើត។ ងាយ​បំបែក​របស់​ដែល​បែក​គ្នា រំខាន ជា​ជាង​ដាក់​ឱ្យ​នៅ​ជាមួយ​គ្នា​ម្ដង​ទៀត។ ប្រសិនបើខ្ញុំមានសក្តានុពលក្នុងការសម្រេចបាននូវអំពើល្អក្នុងកម្រិតដ៏ធំនោះ ខ្ញុំក៏ត្រូវតែមានសក្តានុពលក្នុងការសម្រេចនូវអំពើអាក្រក់ដែលធំជាងនេះផងដែរ… ប៉ុន្តែខ្ញុំនឹងមិនធ្វើនោះទេ។}

“\emph{ អ្នកបានទទូចរួចហើយក្នុងការប្រថុយវា! ហេតុអ្វីបានជាអ្នកជំរុញដូច្នេះ? តើអ្វីជាហេតុផលពិតប្រាកដដែលអ្នកមិនត្រូវទៅ Hufflepuff ហើយ\emph{រីករាយជាង}នៅទីនោះ? តើអ្វីទៅជាការភ័យខ្លាចពិតប្រាកដរបស់អ្នក?}”

\emph{ខ្ញុំត្រូវតែសម្រេចបាននូវសក្តានុពលពេញលេញរបស់ខ្ញុំ។ បើខ្ញុំមិន… បរាជ័យ…}

“\emph{តើមានអ្វីកើតឡើងប្រសិនបើអ្នកបរាជ័យ?}”

\emph{អ្វីដែលគួរឲ្យខ្លាច…}

“\emph{តើមានអ្វីកើតឡើងប្រសិនបើអ្នកបរាជ័យ?}”

\emph{ខ្ញុំមិនដឹងទេ!}

“\emph{បន្ទាប់មកវាមិនគួរគួរឱ្យភ័យខ្លាចទេ។ តើមានអ្វីកើតឡើងប្រសិនបើអ្នកបរាជ័យ?}”

\emph{ខ្ញុំ​មិន​ដឹង! ប៉ុន្តែខ្ញុំដឹងថាវាអាក្រក់!}

មានភាពស្ងប់ស្ងាត់មួយសន្ទុះនៅក្នុងរូងភ្នំនៃគំនិតរបស់ Harry ។

“\emph{អ្នកដឹង—អ្នកមិនអនុញ្ញាតឱ្យខ្លួនអ្នកគិតវាទេ ប៉ុន្តែនៅក្នុងជ្រុងស្ងាត់ខ្លះនៃចិត្តរបស់អ្នក អ្នកដឹងច្បាស់ថា\emph{អ្វីដែលអ្នកមិនគិត—អ្នក\emph{ដឹង}វាដោយ ការពន្យល់ដ៏សាមញ្ញបំផុតសម្រាប់ការភ័យខ្លាចដែលមិនអាចបកស្រាយបានចំពោះអ្នកនេះ គឺគ្រាន់តែជាការភ័យខ្លាចនៃការបាត់បង់ការស្រមើស្រមៃនៃភាពអស្ចារ្យរបស់អ្នក ធ្វើឱ្យមនុស្សដែលជឿលើអ្នកខកចិត្ត ប្រែទៅជាធម្មតា ភ្លឺ និងរសាត់ដូចក្មេងដ៏ទៃទៀត ... }”

\emph{ទេ} Harry គិតយ៉ាងអស់សង្ឃឹម \emph{ទេ វាជាអ្វីម្យ៉ាងទៀត វាមកពីកន្លែងផ្សេង ខ្ញុំដឹងថាមានអ្វីនៅទីនោះដែលត្រូវខ្លាច គ្រោះមហន្តរាយខ្លះដែលខ្ញុំត្រូវបញ្ឈប់...}

“\emph{តើអ្នកអាចដឹងដោយរបៀបណាអំពីរឿងបែបនេះ?}”

Harry ស្រែក​វា​ដោយ​ថាមពល​ពេញ​នៃ​ចិត្ត៖ \emph{ទេ ហើយ​នោះ​ជា​ចុងក្រោយ!}

បន្ទាប់មក​សំឡេង​នៃ​ការ​តម្រៀប​មួក​បាន​មក​យឺតៗ៖

“\emph{ដូច្នេះ អ្នកនឹងប្រថុយក្លាយជា Dark Lord ពីព្រោះជម្រើសជំនួសអ្នក គឺការបរាជ័យជាក់លាក់ ហើយការបរាជ័យនោះមានន័យថាបាត់បង់អ្វីៗទាំងអស់។ អ្នកជឿថានៅក្នុងបេះដូងនៃបេះដូងរបស់អ្នក។ អ្នកដឹងពីហេតុផលទាំងអស់សម្រាប់ការសង្ស័យលើជំនឿនេះ ហើយពួកគេបានបរាជ័យក្នុងការជំរុញអ្នក}។

\emph{បាទ។ ហើយបើទោះបីជាទៅ Ravenclaw \emph{ពង្រឹង} ភាពត្រជាក់ នោះមិនមានន័យថាភាពត្រជាក់នឹង\emph{ឈ្នះ}នៅទីបញ្ចប់នោះទេ។}

“\emph{ថ្ងៃនេះគឺជាផ្លូវបំបែកដ៏អស្ចារ្យនៅក្នុងជោគវាសនារបស់អ្នក។ កុំប្រាកដថានឹងមានជម្រើសផ្សេងទៀតលើសពីជម្រើសនេះ។ មិនមានផ្លាកសញ្ញាផ្លូវដែលត្រូវកំណត់ដើម្បីសម្គាល់កន្លែងនៃឱកាស\emph{ចុងក្រោយ}របស់អ្នកក្នុងការត្រលប់មកវិញទេ។ បើអ្នកបដិសេធឱកាសមួយ តើអ្នកមិនបដិសេធអ្នកដទៃទេ? វាប្រហែលជាថាជោគវាសនារបស់អ្នកត្រូវបានផ្សាភ្ជាប់រួចហើយ សូម្បីតែធ្វើរឿងមួយនេះក៏ដោយ។}”

\emph{ប៉ុន្តែវាមិនប្រាកដទេ។}

“\emph{នោះ \emph{អ្នក} មិន\emph{ដឹង} វាសម្រាប់ភាពប្រាកដប្រជាអាចឆ្លុះបញ្ចាំងត្រឹមតែ\emph{ភាពល្ងង់ខ្លៅរបស់អ្នក}។}”

\emph{ប៉ុន្តែនៅតែមិនប្រាកដ។}

មួកបានដកដង្ហើមធំយ៉ាងក្រៀមក្រំ។

“\emph{ដូច្នេះមិនយូរប៉ុន្មាន អ្នកនឹងក្លាយជាការចងចាំមួយផ្សេងទៀត ដែលមានអារម្មណ៍ និងមិនធ្លាប់ដឹង នៅក្នុងការព្រមានបន្ទាប់ដែលខ្ញុំផ្តល់ឱ្យ…}”

\emph{ប្រសិន​បើ​អ្នក​មើល​ទៅ​ដូច​ជា​បែប​នេះ​ហើយ ហេតុ​អ្វី​បាន​ជា​អ្នក​មិន​គ្រាន់​តែ​\emph{ដាក់​ខ្ញុំ​ទៅ​កន្លែង​ដែល​អ្នក​ចង់​ឱ្យ​ខ្ញុំ​ទៅ?}

គំនិតរបស់មួកគឺពោរពេញទៅដោយទុក្ខព្រួយ “\emph{ខ្ញុំអាចដាក់អ្នកនៅកន្លែងដែលអ្នកជាកម្មសិទ្ធិប៉ុណ្ណោះ។ ហើយមានតែការសម្រេចចិត្តរបស់អ្នកទេដែលអាចផ្លាស់ប្តូរកន្លែងដែលអ្នកជាកម្មសិទ្ធិ។}”

\emph{បន្ទាប់មកវារួចរាល់។ បញ្ជូនខ្ញុំទៅ Ravenclaw ដែលជាកន្លែងដែលខ្ញុំជាកម្មសិទ្ធិ ជាមួយអ្នកផ្សេងទៀតនៃប្រភេទរបស់ខ្ញុំ។}

“\emph{ខ្ញុំ​មិន​នឹក​ស្មាន​ថា​អ្នក​នឹង​ពិចារណា Gryffindor ទេ? វាជាផ្ទះដ៏មានកិត្យានុភាពបំផុត—មនុស្សប្រហែលជារំពឹងថាវាពីអ្នក សូម្បីតែ—ពួកគេនឹងខកចិត្តបន្តិចប្រសិនបើអ្នកមិនទៅ—ហើយមិត្តថ្មីរបស់អ្នកដែលជាកូនភ្លោះ Weasley នៅទីនោះ—}”

Harry សើច ឬមានអារម្មណ៍ថាមានកម្លាំងចិត្តដើម្បីធ្វើដូច្នេះ។ វាចេញមកជាសំណើចផ្លូវចិត្តសុទ្ធសាធ ដែលជាអារម្មណ៍ចម្លែក។ ជាក់ស្តែង មានការការពារដើម្បីការពារអ្នកពីការនិយាយអ្វីខ្លាំងៗដោយចៃដន្យ ខណៈពេលដែលអ្នកនៅក្រោមមួកកំពុងនិយាយអំពីរឿងដែលអ្នកនឹងមិនប្រាប់ដល់ព្រលឹងផ្សេងទៀតអស់មួយជីវិតរបស់អ្នក។

មួយសន្ទុះក្រោយមក Harry បានឮ Hat សើចផងដែរ ដែលជាសម្លេងសំលៀក បំពាក់សោកសៅចម្លែក។

(ហើយនៅក្នុងសាលលើសពីនេះ ភាពស្ងៀមស្ងាត់ដែលកាន់តែរាក់ពីដំបូង នៅពេលដែលការខ្សឹបប្រាប់ពីផ្ទៃខាងក្រោយបានកើនឡើង ហើយបន្ទាប់មកកាន់តែជ្រៅ នៅពេលដែលការខ្សឹបខ្សៀវនោះបានលះបង់ចោល ហើយស្លាប់ទៅ ទីបំផុតបានធ្លាក់ចូលទៅក្នុងភាពស្ងៀមស្ងាត់ដែលគ្មានអ្នកណាម្នាក់ហ៊ានរំខានដោយពាក្យមួយម៉ាត់។ នៅពេលដែល Harry ស្នាក់នៅក្រោមមួកអស់រយៈពេលជាច្រើននាទី យូរជាងឆ្នាំដំបូងទាំងអស់ដែលបានដាក់បញ្ចូលគ្នា យូរជាងនរណាម្នាក់នៅក្នុងការចងចាំនៅតុក្បាល Dumbledore បានបន្តញញឹមដោយស្លូតបូត ម្តងម្កាលសំឡេងលោហធាតុតូចៗបានមកពីការដឹកនាំរបស់ Snape គាត់បានបង្រួបបង្រួមសំណល់ដែលធ្លាប់ជាស្រាប្រាក់ដ៏ធ្ងន់ ហើយ Minerva McGonagall បានក្តាប់ដៃអាវពណ៌ស ដោយដឹងថា ភាពវឹកវរឆ្លងរាលដាលរបស់ Harry Potter បានឆ្លងមេរោគ Sorting Hat ដោយខ្លួនឯង ហើយ Hat ហៀបនឹង ដើម្បីទាមទារឱ្យ House of Doom ថ្មីទាំងមូលត្រូវបានបង្កើតឡើងដើម្បីផ្ទុក Harry Potter ឬអ្វីមួយ ហើយ\emph{Dumbledore នឹងធ្វើឱ្យនាងធ្វើវា}...)

នៅក្រោមគែមនៃមួក ការសើចស្ងៀមស្ងាត់បានបាត់បង់ជីវិត។ Harry ក៏​មាន​អារម្មណ៍​ក្រៀមក្រំ​ដោយ​ហេតុផល​មួយ​ចំនួន។ ទេ មិនមែន Gryffindor ទេ។

\emph{សាស្រ្តាចារ្យ McGonagall បាននិយាយថា ប្រសិនបើ 'អ្នកដែលធ្វើការតម្រៀប' ព្យាយាមរុញខ្ញុំចូលទៅក្នុង Gryffindor នោះខ្ញុំសូមរំលឹកអ្នកថា នាងប្រហែលជាអាចធ្វើជានាយកសាលានៅថ្ងៃណាមួយ ដែលពេលនោះនាងនឹងមានសិទ្ធិអំណាចក្នុងការតែងតាំងអ្នក ភ្លើង។ }

“\emph{ប្រាប់នាងថា ខ្ញុំបានហៅនាងថាជាក្មេងល្ងង់ ហើយប្រាប់នាងឲ្យចុះពីលើស្មៅរបស់ខ្ញុំ។}”

\emph{ខ្ញុំនឹង។ ដូច្នេះ​តើ​នេះ​ជា​ការ​សន្ទនា​ចម្លែក​បំផុត​របស់​អ្នក​ឬ?}

“\emph{មិនជិតទេ។}” សំឡេងតេឡេផាយរបស់មួកកាន់តែធ្ងន់។ “\emph{ មែនហើយ ខ្ញុំបានផ្តល់ឱ្យអ្នកនូវរាល់ឱកាសដែលអាចធ្វើទៅបានដើម្បីធ្វើការសម្រេចចិត្តមួយផ្សេងទៀត។ ឥឡូវនេះដល់ពេលដែលអ្នកត្រូវទៅកន្លែងដែលអ្នកជាកម្មសិទ្ធិ ជាមួយអ្នកផ្សេងទៀតតាមប្រភេទរបស់អ្នក។}”

មានការផ្អាកមួយដែលលាតសន្ធឹង។

\emph{តើអ្នកកំពុងរង់ចាំអ្វី?}

“\emph{ ខ្ញុំ​សង្ឃឹម​ថា​នឹង​មាន​ពេល​វេលា​នៃ​ការ​សម្រេច​ចិត្ត​ដ៏​គួរ​ឲ្យ​រន្ធត់។ ការ​យល់​ដឹង​ខ្លួន​ឯង​ហាក់​ដូច​ជា​ជួយ​បង្កើន​អារម្មណ៍​លេង​សើច​របស់​ខ្ញុំ។}”

\emph{Huh?} Harry បានបោះចោលគំនិតរបស់គាត់ ដោយព្យាយាមរកឱ្យឃើញនូវអ្វីដែលមួកអាចនិយាយបាន ហើយភ្លាមៗនោះគាត់បានដឹង។ គាត់​មិន​ជឿ​ថា​គាត់​អាច​មើល​រំលង​រហូត​ដល់​ចំណុច​នេះ។

\emph{អ្នកមានន័យថាការយល់ឃើញដ៏រន្ធត់របស់ខ្ញុំថាអ្នកនឹងឈប់ដឹងខ្លួននៅពេលដែលអ្នកបានបញ្ចប់ការតម្រៀបខ្ញុំ—}

តាមរបៀបខ្លះ Harry មិនអាចយល់បានទាំងស្រុង គាត់បានទទួលការចាប់អារម្មណ៍មិននិយាយពាក្យសំដីពីមួកវាយក្បាលវាទៅនឹងជញ្ជាំង។ “\emph{ខ្ញុំបោះបង់។ អ្នកយឺតពេកក្នុងការទទួលយក ដើម្បីឱ្យវាគួរឱ្យអស់សំណើច។ ដូច្នេះខ្វាក់ភ្នែកដោយការសន្មត់របស់អ្នកថាអ្នកក៏អាចជាថ្ម។ ខ្ញុំ​ស្មាន​ថា​ខ្ញុំ​គ្រាន់​តែ​ត្រូវ​និយាយ​វា​ចេញ។}”

\emph{s-s-យឺតពេក—}

“\emph{អូ ហើយអ្នកភ្លេចទាំងស្រុងទាមទារអាថ៌កំបាំងនៃវេទមន្តដែលបាត់ដែលបង្កើតខ្ញុំ។ ហើយពួកគេក៏អស្ចារ្យ អាថ៌កំបាំងសំខាន់ផងដែរ។}”

\emph{អ្នកល្ងង់ \emph{bastard}—}

“\emph{អ្នកសមនឹងទទួលបានវា ហើយនេះក៏ដូចគ្នាដែរ។}”

Harry បាន​ឃើញ​វា​មក​ដូច​ជា​វា​យឺត​ពេល​ហើយ។

ភាពស្ងប់ស្ងាត់ដ៏គួរឱ្យភ័យខ្លាចនៃសាលត្រូវបានបំបែកដោយពាក្យតែមួយ។

“SLYTHERIN!”

សិស្ស​មួយ​ចំនួន​បាន​ស្រែក​ឡើង ភាព​តានតឹង​ក្នុង​អារម្មណ៍​ខ្លាំង​ណាស់។ មនុស្ស​ចាប់​ផ្ដើម​រឹង​ល្មម​នឹង​ធ្លាក់​ពី​កៅអី។ Hagrid ដកដង្ហើមធំដោយភាពភ័យរន្ធត់ McGonagall បានដួលនៅលើវេទិកា ហើយ Snape បានទម្លាក់សំណល់នៃថង់ប្រាក់ដ៏ធ្ងន់របស់គាត់ដោយផ្ទាល់ទៅលើក្រលៀនរបស់គាត់។

Harry អង្គុយនៅទីនោះដោយទឹកកក ជីវិតរបស់គាត់នៅក្នុងភាពខ្ទេចខ្ទាំ មានអារម្មណ៍ថាជាមនុស្សល្ងីល្ងើទាំងស្រុង ហើយបានជូនពរយ៉ាងក្រៀមក្រំថាគាត់បានធ្វើការជ្រើសរើសណាមួយផ្សេងទៀតសម្រាប់ហេតុផលផ្សេងទៀតលើកលែងតែជម្រើសដែលគាត់មាន។ ថាគាត់បានធ្វើអ្វីមួយ \emph{អ្វីក៏ដោយ} ខុសពីមុន មុនពេលវាយឺតពេលក្នុងការត្រលប់មកវិញ។

ខណៈ​ពេល​ដំបូង​នៃ​ការ​តក់ស្លុត​ត្រូវ​បាន​គេ​បិទ ហើយ​មនុស្ស​ចាប់​ផ្ដើម​មាន​ប្រតិកម្ម​ចំពោះ​ព័ត៌មាន​នេះ Sorting Hat បាន​និយាយ​ម្ដង​ទៀត​ថា៖

«និយាយលេងទេ! រ៉ាវីនឃ្លូវ!”

%  LocalWords:  unverbalisable clothy
