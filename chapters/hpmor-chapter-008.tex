\chapter{លំអៀងវិជ្ជមាន}

\begin{chapterOpeningAuthorNote}
ពិភពលោកទាំងអស់នេះគឺ J.~K.~Rowling's លើកលែងតែ Europa ។ ព្យាយាមគ្មាន fanfics នៅទីនោះ។

អ្នកត្រួតពិនិត្យការដាស់តឿនម្នាក់បានសួរថាតើប្រសិនបើ Luna ជាអ្នកមើល នោះមានន័យថាវានឹងក្លាយជា HPDM ខាងក្រោម!Draco mpreg fic។ ខ្ញុំសោកស្ដាយដែល FFN មិនអនុញ្ញាតឱ្យខ្ញុំមានទំហំពុម្ពអក្សរធំជាង ដែលត្រូវនិយាយថា ទេ ដោយស្មោះត្រង់វាមិនបានកើតឡើងចំពោះខ្ញុំទេដែល Luna អាចជាអ្នកមើល \emph{real}—ខ្ញុំនឹងត្រូវសម្រេចចិត្តថាតើត្រូវរត់ជាមួយនោះឬអត់—ប៉ុន្តែខ្ញុំគិតថាយើងទាំងអស់គ្នាអាចសន្មតដោយសុវត្ថិភាពថាប្រសិនបើ Luna \emph{is} អ្នកមើលឆុតម្នាក់នាងបាននិយាយអ្វីមួយអំពី "ពន្លឺដាំគ្រាប់ពូជនៅក្នុងភាពងងឹត" ហើយ Xenophilius ដូចតែងតែបកស្រាយរឿងនេះតាមរបៀបខុស។

\end{chapterOpeningAuthorNote}
\begin{chapterOpeningQuote}
អនុញ្ញាតឱ្យខ្ញុំព្រមានអ្នកថា ការប្រឈមមុខនឹងភាពប៉ិនប្រសប់របស់ខ្ញុំគឺជាគម្រោងដ៏គ្រោះថ្នាក់ ហើយប្រហែលជាធ្វើឱ្យជីវិតរបស់អ្នកកាន់តែអស្ចារ្យ។
\end{chapterOpeningQuote}

\lettrine{N}{o}-ម្នាក់បានសុំជំនួយ នោះជាបញ្ហា។ ពួកគេគ្រាន់តែដើរជុំវិញការនិយាយ ញ៉ាំអាហារ ឬសម្លឹងមើលទៅលើអាកាស ខណៈពេលដែលឪពុកម្តាយរបស់ពួកគេផ្លាស់ប្តូរការនិយាយដើម។ ទោះ​ជា​មាន​ហេតុផល​ចម្លែក​អ្វី​ក៏ដោយ គ្មាន​អ្នក​ណា​អង្គុយ​អាន​សៀវភៅ ដែល​មាន​ន័យ​ថា​នាង​មិន​អាច​អង្គុយ​ក្បែរ​គេ ហើយ​យក​សៀវភៅ​ចេញ​ដោយ​ខ្លួន​ឯង​ឡើយ។ ហើយសូម្បីតែនៅពេលដែលនាងបានចាប់ផ្តើមគំនិតផ្តួចផ្តើមដោយក្លាហានដោយអង្គុយចុះ ហើយបន្តការអានលើកទីបីរបស់នាងដែលមានតម្លៃ\emph{Hogwarts: A History,} គ្មាននរណាម្នាក់ហាក់ដូចជាចង់អង្គុយក្បែរនាងនោះទេ។

ក្រៅពីជួយមនុស្សធ្វើកិច្ចការផ្ទះ ឬអ្វីផ្សេងទៀតដែលពួកគេត្រូវការ នាងពិតជាមិនដឹងពីរបៀបជួបមនុស្សទេ។ នាងមិនមាន\emph{មានអារម្មណ៍ថា}ដូចជានាងជាមនុស្សខ្មាស់អៀន។ នាង​គិត​ថា​ខ្លួន​ឯង​ជា​មនុស្ស​ស្រី​ដែល​ទទួល​បន្ទុក។ ហើយយ៉ាងណាក៏ដោយ ប្រសិនបើមិនមានសំណើមួយចំនួនតាមបន្ទាត់នៃ "ខ្ញុំមិនអាចចាំពីរបៀបធ្វើការបែងចែកវែងបានទេ" នោះវាគ្រាន់តែជា \emph{ ឆ្គង} ពេកក្នុងការឡើងទៅកាន់នរណាម្នាក់ ហើយនិយាយថា... អ្វី? នាង​មិន​ដែល​អាច​ដឹង​ថា​អ្វី​នោះ​ទេ។ ហើយ​វា​ហាក់​ដូចជា​មិន​មាន​ក្រដាស​ព័ត៌មាន​ស្ដង់ដារ​ដែល​គួរ​ឲ្យ​អស់​សំណើច។ អាជីវកម្មទាំងមូលនៃការជួបមនុស្សមិនដែលហាក់ដូចជាសមហេតុផលសម្រាប់នាងទេ។ ហេតុអ្វីបានជា\emph{នាង}ត្រូវទទួលខុសត្រូវដោយខ្លួនឯង នៅពេលដែលមានមនុស្សពីរនាក់ចូលរួម? ហេតុអ្វីបានជាមនុស្សពេញវ័យមិនជួយ? នាង​ប្រាថ្នា​ចង់​ឱ្យ​ក្មេង​ស្រី​ម្នាក់​ទៀត​ដើរ​ឡើង​ដល់ \emph{នាង} ហើយ​និយាយ​ថា "Hermione គ្រូ​ប្រាប់​ខ្ញុំ​ឱ្យ​ធ្វើ​ជា​មិត្ត​នឹង​អ្នក"។

ប៉ុន្តែសូមឱ្យវាច្បាស់ថា Hermione Granger អង្គុយតែម្នាក់ឯងនៅថ្ងៃដំបូងនៃសាលារៀននៅក្នុងបន្ទប់មួយក្នុងចំណោមបន្ទប់មួយចំនួនដែលទទេរនៅក្នុងទូរថភ្លើងចុងក្រោយបង្អស់ ជាមួយនឹងទ្វារបន្ទប់បើកទុកក្នុងករណីនរណាម្នាក់ដោយសារហេតុផលណាមួយ។ ចង់និយាយជាមួយនាង គឺ\emph{មិន} សោកសៅ ឯកោ អាប់អួរ បាក់ទឹកចិត្ត អស់សង្ឃឹម ឬឈ្លក់វង្វេងអំពីបញ្ហារបស់នាង។ ផ្ទុយទៅវិញ នាងកំពុងអានឡើងវិញ \emph{Hogwarts: A History} ជាលើកទីបី ហើយរីករាយនឹងវា ដោយគ្រាន់តែមានការរំខានតិចតួចនៅក្នុងចិត្តរបស់នាងចំពោះភាពមិនសមហេតុផលទូទៅនៃពិភពលោក។

មាន​សំឡេង​ទ្វារ​រថភ្លើង​បើក​រួច​ក៏​មាន​សំឡេង​គន្លាក់​ចុះ​មក​តាម​ផ្លូវ​រថភ្លើង។ Hermione ទុកមួយឡែក\emph{Hogwarts: A History} ហើយក្រោកឈរឡើង ហើយតោងក្បាលនាងនៅខាងក្រៅ—ក្នុងករណីមាននរណាម្នាក់ត្រូវការជំនួយ—ហើយបានឃើញក្មេងប្រុសម្នាក់នៅក្នុងអាវផាយរបស់អ្នកជំនួយការ ប្រហែលជាឆ្នាំទី 1 ឬទី 2 ដែលកម្ពស់របស់គាត់ ហើយ មើលទៅពិតជាឆ្កួតជាមួយនឹងកន្សែងរុំក្បាលរបស់គាត់។ ដើមតូចមួយឈរនៅលើឥដ្ឋក្បែរគាត់។ សូម្បីតែពេលនាងឃើញគាត់ គាត់ក៏គោះទ្វារបន្ទប់មួយទៀតបិទជិត ហើយគាត់និយាយដោយសម្លេងរអ៊ូរទាំដោយកន្សែងបង់កថា "សុំទោស តើខ្ញុំអាចសួរសំណួរបានលឿនទេ?"

នាងមិនបានឮចម្លើយពីខាងក្នុងបន្ទប់នោះទេ ប៉ុន្តែបន្ទាប់ពីក្មេងប្រុសនោះបើកទ្វារ នាងគិតថានាងបានឮគាត់និយាយ—លុះត្រាតែនាងមិនយល់ច្រឡំ—“តើមានអ្នកណាម្នាក់នៅទីនេះស្គាល់ quarks ទាំងប្រាំមួយ ឬកន្លែងដែលខ្ញុំអាចរកបានមុនគេ។ - ក្មេងស្រីឆ្នាំម្នាក់ឈ្មោះ Hermione Granger?

បន្ទាប់​ពី​ក្មេង​ប្រុស​បាន​បិទ​ទ្វារ​បន្ទប់​នោះ Hermione បាន​និយាយ​ថា “ខ្ញុំ​អាច​ជួយ​អ្នក​រឿង​មួយ​បាន​ទេ?”

មុខ​ដែល​មាន​ក្រមា​បាន​ងាក​មក​មើល​នាង ហើយ​សំឡេង​នោះ​និយាយ​ថា “មិនមែន​ទេ លុះ​ត្រា​តែ​អ្នក​អាច​ដាក់​ឈ្មោះ​ភ្នំ​ទាំង​ប្រាំមួយ ឬ​ប្រាប់​ខ្ញុំ​ពី​កន្លែងដែល​ត្រូវ​រក Hermione Granger”។

"ឡើងចុះ ចម្លែក មន្តស្នេហ៍ ការពិត សម្រស់ ហើយហេតុអ្វីបានជាអ្នកស្វែងរកនាង?"

វាពិបាកនឹងប្រាប់ពីចម្ងាយនេះ ប៉ុន្តែនាងគិតថានាងបានឃើញក្មេងប្រុសញញឹមយ៉ាងទូលំទូលាយនៅក្រោមក្រមារបស់គាត់។ “អ៎ \emph{អ្នក} ជាក្មេងស្រីឆ្នាំទី 1 ឈ្មោះ Hermione Granger” សម្លេងដ៏ស្រទន់នោះបាននិយាយ។ "នៅលើរថភ្លើងទៅ Hogwarts មិនតិចទេ" ។ ក្មេង​ប្រុស​នោះ​ចាប់​ផ្តើម​ដើរ​សំដៅ​ទៅ​រក​នាង និង​បន្ទប់​របស់​នាង ហើយ​ដើម​របស់​វា​បាន​រអិល​តាម​ក្រោយ​គាត់។ “តាមបច្ចេកទេស អ្វីដែលខ្ញុំត្រូវធ្វើគឺ\emph{រកមើល}សម្រាប់អ្នក ប៉ុន្តែវាហាក់ដូចជាខ្ញុំចង់និយាយជាមួយអ្នក ឬអញ្ជើញអ្នកឱ្យចូលរួមពិធីជប់លៀងរបស់ខ្ញុំ ឬទទួលបានវត្ថុវេទមន្តសំខាន់ៗពីអ្នក ឬស្វែងយល់ថា Hogwarts ត្រូវ​បាន​គេ​សាង​សង់​នៅ​លើ​ប្រាសាទ​បុរាណ ឬ​អ្វី​មួយ។ PC ឬ NPC នោះជាសំណួរ?

Hermione បើកមាត់ដើម្បីឆ្លើយតបនឹងរឿងនេះ ប៉ុន្តែបន្ទាប់មកនាងមិនអាចគិតពី \emph{possible} ណាមួយដែលឆ្លើយតបទៅ…\emph{ whatever} ដែលនាងទើបតែបានឮ សូម្បីតែក្មេងប្រុសដើរមករកនាង មើលខាងក្នុង បន្ទប់នោះងក់ក្បាលដោយពេញចិត្ត ហើយអង្គុយលើកៅអីដែលនៅឆ្ងាយពីខ្លួននាង។ ប្រម៉ោយ​របស់​គាត់​បាន​លូន​ចូល​តាម​ពី​ក្រោយ​គាត់ ដោយ​បាន​ធំ​ឡើង​ដល់​ទៅ​បី​ដង​នៃ​ទំហំ​ចាស់​របស់​វា ហើយ​បាន​នៅ​ជាប់​នឹង​នាង​ផ្ទាល់​ក្នុង​របៀប​ដែល​គួរ​ឲ្យ​រំខាន។

ក្មេងប្រុសបាននិយាយថា “សូមអង្គុយចុះ ហើយសូមបិទទ្វារពីក្រោយអ្នក បើអ្នកចង់។ កុំ​បារម្ភ ខ្ញុំ​មិន​ខាំ​អ្នក​ណា​ដែល​មិន​ខាំ​ខ្ញុំ​មុន​ឡើយ»។ គាត់​បាន​ដោះ​ក្រមា​ចេញ​ពី​ជុំវិញ​ក្បាល​រួច​ហើយ។

ការសន្មត់ដែលក្មេងប្រុសនេះគិតថានាងមានតម្លៃ \emph{ខ្លាចគេ} បានធ្វើឱ្យដៃរបស់នាងបានរុញទ្វារបិទ ដោយរុញវាចូលទៅក្នុងជញ្ជាំងដោយកម្លាំងមិនចាំបាច់។ នាងបានដើរជុំវិញខ្លួន ហើយឃើញមុខក្មេងជាមួយនឹងភ្នែកពណ៌បៃតងសើចចំអក និងស្នាមក្រហម-ខ្មៅងងឹតដែលខឹងសម្បារដាក់លើថ្ងាសដែលធ្វើឲ្យនាងនឹកឃើញដល់រឿងក្នុងចិត្តរបស់នាង ប៉ុន្តែពេលនេះនាងមានរឿងសំខាន់ជាងដែលត្រូវគិត។ "ខ្ញុំមិនបាននិយាយថាខ្ញុំជា Hermione Granger!"

“\emph{ខ្ញុំ} មិនបាននិយាយថាអ្នក \emph{បាននិយាយថា} អ្នកគឺជា Hermione Granger ខ្ញុំគ្រាន់តែនិយាយថាអ្នកគឺជា Hermione Granger ។ បើ​អ្នក​សួរ​ថា​ខ្ញុំ​ដឹង​ដោយ​របៀប​ណា នោះ​គឺ​មក​ពី​ខ្ញុំ​ដឹង​គ្រប់​យ៉ាង។ សួស្តីលោក លោកស្រី អ្នកនាងកញ្ញា ខ្ញុំឈ្មោះ Harry James Potter-Evans-Verres ឬ Harry Potter និយាយដោយខ្លី ខ្ញុំដឹងថាប្រហែលជាគ្មានន័យអ្វីទេចំពោះ \emph{you} សម្រាប់ការផ្លាស់ប្តូរ—”

ទីបំផុតគំនិតរបស់ Hermione បានបង្កើតទំនាក់ទំនង។ ស្លាកស្នាមនៅលើថ្ងាសរបស់គាត់ រាងដូចផ្លេកបន្ទោរ។ “Harry Potter! អ្នកស្ថិតនៅក្នុង \emph{Modern Magical History} និង \emph{The Rise and Fall of the Dark Arts} និង \emph{Great Wizarding Events of the Twentieth Century។}" តាមពិតវាជាលើកដំបូងក្នុងជីវិតរបស់នាងដែល នាងនឹង \emph { ជួប } នរណាម្នាក់ពីខាងក្នុង \emph {book} ហើយវាជាអារម្មណ៍ចម្លែក។

ក្មេងប្រុសបានព្រិចភ្នែកបីដង។ “ខ្ញុំស្ថិតក្នុង \emph{books}? ចាំ​មើល ខ្ញុំ​នៅ​ក្នុង​សៀវភៅ​… អ្វី​ដែល​ជា​គំនិត​ចម្លែក​»។

«សុខសប្បាយទេដឹង?» Hermione បាននិយាយ។ "ខ្ញុំ​នឹង​បាន​រក​ឃើញ​អ្វី​គ្រប់​យ៉ាង​ដែល​ខ្ញុំ​អាច​ធ្វើ​បាន​ប្រសិន​បើ​ជា​ខ្ញុំ​។"

ក្មេងប្រុសនិយាយយ៉ាងស្ងួត។ “Miss~Granger វាមានរយៈពេលតិចជាង 72 ~ ម៉ោងចាប់តាំងពីខ្ញុំបានទៅ Diagon Alley ហើយបានរកឃើញការអះអាងរបស់ខ្ញុំចំពោះភាពល្បីល្បាញ។ ខ្ញុំបានចំណាយពេលពីរថ្ងៃចុងក្រោយនេះទិញសៀវភៅវិទ្យាសាស្ត្រ។ \emph{ជឿខ្ញុំ} ខ្ញុំមានបំណងស្វែងរកអ្វីគ្រប់យ៉ាងដែលខ្ញុំអាចធ្វើបាន។” ក្មេងប្រុសស្ទាក់ស្ទើរ។ “តើ \emph{do} សៀវភៅនិយាយអ្វីខ្លះអំពីខ្ញុំ?”

ចិត្តរបស់ Hermione Granger បានភ្លឺឡើងវិញ នាងមិនបានដឹងថានាងនឹងត្រូវបានសាកល្បងលើសៀវភៅ \emph{ទាំងនោះ} ដូច្នេះនាងបានអានវាតែម្តងគត់ ប៉ុន្តែវាទើបតែបានមួយខែកន្លងទៅ ដូច្នេះសម្ភារៈនៅតែថ្មីនៅក្នុងចិត្តរបស់នាង។ “អ្នកគឺជាមនុស្សតែម្នាក់គត់ដែលបានរួចរស់ជីវិតពី Killing Curse ដូច្នេះអ្នកត្រូវបានគេហៅថា Boy-Who-Lived ។ អ្នកបានកើតសម្រាប់ James Potter និង Lily Potter អតីត Lily Evans នៅថ្ងៃទី 31 ខែកក្កដា ឆ្នាំ 1980។ នៅថ្ងៃទី 31 ខែតុលា ឆ្នាំ 1981 the Dark Lord He-Who-Must-Not-Be-Name ទោះបីជាខ្ញុំមិនដឹងថាហេតុអ្វីបានជាមិនវាយប្រហារផ្ទះរបស់អ្នកក៏ដោយ។ . អ្នក​ត្រូវ​បាន​គេ​រក​ឃើញ​នៅ​រស់​ដោយ​មាន​ស្លាក​ស្នាម​នៅ​លើ​ថ្ងាស​របស់​អ្នក​នៅ​ក្នុង​ផ្ទះ​ឪពុក​ម្តាយ​របស់​អ្នក​ដែល​ខូច​ខាត​នៅ​ក្បែរ​សាកសព​អ្នក​ស្គាល់​នរណា​ដែល​ត្រូវ​ឆេះ។ ប្រធាន Warlock Albus Percival Wulfric Brian Dumbledore បានបញ្ជូនអ្នកទៅកន្លែងណាមួយ គ្មាននរណាម្នាក់ដឹងថានៅឯណាទេ។ \emph{The Rise and Fall of the Dark Arts} អះអាងថាអ្នកបានរួចរស់ជីវិតដោយសារតែក្តីស្រលាញ់របស់ម្តាយអ្នក ហើយស្លាកស្នាមរបស់អ្នកមានផ្ទុកនូវថាមពលវេទមន្តរបស់ Dark Lord ទាំងអស់ ហើយថា Centaurs ខ្លាចអ្នក ប៉ុន្តែ \emph{Great Wizarding Events of សតវត្សន៍ទី 20} មិននិយាយអំពីអ្វីដូចនោះទេ ហើយ\emph{Modern Magical History} ព្រមានថាមានទ្រឹស្ដី crackpot ជាច្រើនអំពីអ្នក។"

មាត់របស់ក្មេងប្រុសបានបើក។ "តើអ្នកត្រូវបានគេប្រាប់ឱ្យរង់ចាំ Harry Potter នៅលើរថភ្លើងទៅ Hogwarts ឬអ្វីមួយដូចនោះ?"

Hermione បាននិយាយថា "ទេ" ។ “អ្នកណាប្រាប់អ្នកអំពី\emph{ខ្ញុំ}?”

“សាស្រ្តាចារ្យ McGonagall ហើយខ្ញុំជឿថាខ្ញុំឃើញមូលហេតុ។ តើ​អ្នក​មាន​ការ​ចង​ចាំ​ដ៏​អស្ចារ្យ​ទេ Hermione?

Hermione ងក់ក្បាល។ “វាមិនមែនជាការថតរូបទេ ខ្ញុំតែងតែប្រាថ្នាវា ប៉ុន្តែខ្ញុំត្រូវអានសៀវភៅសាលារបស់ខ្ញុំប្រាំដងបន្ថែមទៀត ដើម្បីទន្ទេញវាទាំងអស់”។

«ពិត​មែន» ក្មេង​ប្រុស​និយាយ​ដោយ​សំឡេង​ច្របាច់​ក​បន្តិច។ "ខ្ញុំសង្ឃឹមថាអ្នកមិនប្រកាន់ទេប្រសិនបើខ្ញុំសាកល្បងវា - វាមិនមែនថាខ្ញុំមិនជឿអ្នកទេប៉ុន្តែដូចពាក្យថា "ទុកចិត្តប៉ុន្តែផ្ទៀងផ្ទាត់" ។ គ្មាន​ចំណុច​អ្វី​ក្នុង​ការ​ងឿង​ឆ្ងល់​ថា​ពេល​ណា​ដែល​ខ្ញុំ​អាច​ធ្វើ​ការ​ពិសោធន៍​នោះ​ទេ»។

Hermione ញញឹមជាជាងញញឹម។ នាងចូលចិត្តការធ្វើតេស្តណាស់។ “ទៅមុខ។”

ក្មេង​ប្រុស​បាន​យក​ដៃ​ដាក់​ក្នុង​ថង់​ក្បែរ​ខ្លួន ហើយ​និយាយ​ថា “Magical Drafts and Potions by Arsenius Jigger”។ ពេល​គាត់​ដក​ដៃ វា​កាន់​សៀវភៅ​ដែល​គាត់​ដាក់​ឈ្មោះ។

ភ្លាមៗ Hermione ចង់បានកាបូបមួយក្នុងចំណោមថង់ទាំងនោះច្រើនជាងនាងមិនធ្លាប់ចង់បាន។

ក្មេងប្រុសបើកសៀវភៅទៅកន្លែងណាមួយនៅកណ្តាល ហើយមើលចុះក្រោម។ “ប្រសិនបើអ្នកកំពុងរកលុយ \emph{oil of sharpness}—”

“ខ្ញុំអាច\emph{មើល}ទំព័រនោះពីទីនេះ អ្នកដឹងទេ!”

ក្មេង​នោះ​ផ្អៀង​សៀវភៅ​ដើម្បី​កុំ​ឱ្យ​នាង​មើល​វា​ទៀត ហើយ​ត្រឡប់​ទំព័រ​ម្ដង​ទៀត។ "ប្រសិនបើអ្នកកំពុងញ៉ាំ \emph{ potion of spider climbing} តើគ្រឿងផ្សំបន្ទាប់ដែលអ្នកបានបន្ថែមបន្ទាប់ពីសូត្រ Acromantula នឹងក្លាយជាអ្វី?"

“បន្ទាប់ពីទម្លាក់សូត្រហើយ ចូររង់ចាំរហូតដល់ថ្នាំបានប្រែជាម្លប់នៃមេឃពេលព្រឹកព្រលឹមគ្មានពពក ៨ ដឺក្រេពីជើងមេឃ និង ៨ នាទីមុនពេលចុងព្រះអាទិត្យអាចមើលឃើញជាមុន។ កូរឱ្យប្រាំបីដង widdershins និងមួយ deasil ហើយបន្ទាប់មកបន្ថែមប្រាំបី dram នៃ unicorn bogies ។

ក្មេងប្រុសបិទសៀវភៅដោយខ្ទាស់មុតស្រួច ហើយដាក់សៀវភៅត្រឡប់ទៅក្នុងថង់របស់គាត់វិញ ដែលវាបានលេបវាដោយសំឡេងហុយៗ។ “ល្អណាស់ \emph{ល្អ} ល្អណាស់។ ខ្ញុំ​គួរ​តែ​ធ្វើ​ឱ្យ​អ្នក​ជា​សំណើ​មួយ Miss ~Granger»។

"សំណើមួយ?" Hermione និយាយដោយសង្ស័យ។ ក្មេងស្រីមិនគួរស្តាប់វាទេ។

វាក៏នៅត្រង់ចំណុចនេះដែរ ដែល Hermione បានដឹងពីរឿងផ្សេងទៀត—ជារឿងមួយ—ដែលចម្លែកអំពីក្មេងប្រុសនោះ។ ជាក់ស្តែងមនុស្សដែលមានតម្លៃ \emph{ក្នុង}សៀវភៅ តាមពិត \emph{ស្តាប់ទៅ} ដូចជាសៀវភៅនៅពេលពួកគេនិយាយ។ នេះគឺជាការរកឃើញដ៏គួរឱ្យភ្ញាក់ផ្អើល។

ក្មេង​នោះ​លូក​ដៃ​ចូល​ក្នុង​កាបូប​របស់​គាត់ ហើយ​និយាយ​ថា “កំប៉ុង​ប៉ុប” ដោយ​យក​ស៊ីឡាំង​ពណ៌​បៃតង​ភ្លឺ​មក​វិញ។ គាត់​លើក​វា​ទៅ​នាង ហើយ​និយាយ​ថា «​តើ​ខ្ញុំ​អាច​ផ្តល់​អ្វី​ឲ្យ​អ្នក​ផឹក​បានទេ​?»

Hermione បានទទួលយកភេសជ្ជៈដ៏ឈ្ងុយឆ្ងាញ់។ តាមពិតនាង\emph{គឺ}មានអារម្មណ៍ស្រេកទឹកនៅពេលនេះ។ Hermione បាននិយាយនៅពេលនាងឡើងលើថា "អរគុណច្រើន" ។ "នោះជាសំណើរបស់អ្នកមែនទេ?"

ក្មេងប្រុសបានក្អក។ គាត់បាននិយាយថា "ទេ" ។ នៅពេលដែល Hermione ចាប់ផ្តើមផឹក គាត់បាននិយាយថា "ខ្ញុំចង់ឱ្យអ្នកជួយខ្ញុំគ្រប់គ្រងសកលលោក"។

Hermione បានបញ្ចប់ការផឹករបស់នាង ហើយទម្លាក់កំប៉ុង។ “ទេ អរគុណ ខ្ញុំមិនមែនអាក្រក់ទេ”

ក្មេង​ប្រុស​មើល​មក​នាង​ដោយ​ការ​ភ្ញាក់​ផ្អើល ហាក់​ដូច​ជា​រំពឹង​នឹង​ចម្លើយ​ផ្សេង​ទៀត។ លោក​បាន​មាន​ប្រសាសន៍​ថា “មែន​ហើយ ខ្ញុំ​និយាយ​វោហាសាស្ត្រ​បន្តិច។ "នៅក្នុងន័យនៃគម្រោង Baconian អ្នកដឹងទេមិនមែនអំណាចនយោបាយទេ។ 'ឥទ្ធិពលនៃអ្វីៗទាំងអស់ដែលអាចធ្វើទៅបាន' និងដូច្នេះនៅលើ។ ខ្ញុំចង់ធ្វើការសិក្សាពិសោធអំពីអក្ខរាវិរុទ្ធ ស្វែងយល់ពីច្បាប់មូលដ្ឋាន នាំយកវេទមន្តចូលទៅក្នុងដែននៃវិទ្យាសាស្ត្រ បញ្ចូលគ្នានូវពិភពវេទមន្ត និងពិភព Muggle លើកកំពស់កម្រិតជីវភាពរបស់ភពផែនដីទាំងមូល ផ្លាស់ទីមនុស្សជាតិជាច្រើនសតវត្សទៅមុខ ស្វែងរកអាថ៌កំបាំងនៃអមតៈ ធ្វើអាណានិគម ប្រព័ន្ធព្រះអាទិត្យ រុករកកាឡាក់ស៊ី ហើយសំខាន់បំផុត ស្វែងយល់ថាតើអ្វីដែលកំពុងកើតឡើងនៅទីនេះ ពីព្រោះអ្វីៗទាំងអស់នេះគឺមិនអាចទៅរួចទេ។

នោះស្តាប់ទៅគួរឱ្យចាប់អារម្មណ៍ជាងបន្តិច។ “ហើយ?”

ក្មេងប្រុសសម្លឹងមើលនាងដោយមិនគួរឱ្យជឿ។ “\emph{ហើយ?} នោះមិន\emph{គ្រប់គ្រាន់}ទេ?”

"ហើយចង់បានអ្វីពីខ្ញុំ?" Hermione បាននិយាយ។

"ខ្ញុំចង់ឱ្យអ្នកជួយខ្ញុំធ្វើការស្រាវជ្រាវ។ ជាមួយនឹងការចងចាំសព្វវចនាធិប្បាយរបស់អ្នកបានបន្ថែមទៅលើភាពវៃឆ្លាត និងសនិទានភាពរបស់ខ្ញុំ យើងនឹងធ្វើឱ្យគម្រោង Baconian បញ្ចប់ក្នុងពេលដ៏ខ្លី ដែល "គ្មានពេលវេលា" ខ្ញុំមានន័យថាប្រហែលជាយ៉ាងហោចណាស់សាមសិបប្រាំឆ្នាំ។

Hermione ចាប់ផ្តើមរកក្មេងប្រុសនេះរំខាន។ “ខ្ញុំមិនបានឃើញអ្នកធ្វើអ្វីដែលឆ្លាតវៃទេ។ ប្រហែលជាខ្ញុំនឹងអនុញ្ញាតឱ្យ\emph{អ្នក}ជួយខ្ញុំក្នុងការស្រាវជ្រាវ\emph{របស់ខ្ញុំ}។

មាន​ភាព​ស្ងាត់​ស្ងៀម​មួយ​នៅ​ក្នុង​បន្ទប់។

ក្មេង​ប្រុស​នោះ​និយាយ​បន្ទាប់​ពី​ផ្អាក​មួយ​រយៈ​ថា “ដូច្នេះ​ឯង​សុំ​ឲ្យ​ខ្ញុំ​បង្ហាញ​ពី​ភាព​វៃ​ឆ្លាត​របស់​ខ្ញុំ”។

Hermione ងក់ក្បាល។

"ខ្ញុំសូមព្រមានអ្នកថា ការប្រឈមមុខនឹងភាពប៉ិនប្រសប់របស់ខ្ញុំ គឺជាគម្រោងដ៏គ្រោះថ្នាក់ ហើយមាននិន្នាការធ្វើឱ្យជីវិតរបស់អ្នកកាន់តែអស្ចារ្យ"។

Hermione បាននិយាយថា "ខ្ញុំមិនទាន់ចាប់អារម្មណ៍នៅឡើយទេ" ។ ដោយ​មិន​បាន​កត់​សម្គាល់ ភេសជ្ជៈ​ពណ៌​បៃតង​បាន​ហក់​មក​បបូរមាត់​នាង​ម្ដង​ទៀត។

ក្មេងប្រុសនោះបាននិយាយថា "ប្រហែលជា\emph{នេះ}នឹងធ្វើឱ្យអ្នកចាប់អារម្មណ៍"។ គាត់​ងើប​មុខ​មើល​នាង​យ៉ាង​ខ្លាំង។ "ខ្ញុំបានធ្វើការពិសោធន៍បន្តិចហើយ ហើយខ្ញុំបានរកឃើញថាខ្ញុំមិនត្រូវការ wand ទេ ខ្ញុំអាចធ្វើអ្វីៗដែលខ្ញុំចង់បានដោយគ្រាន់តែខ្ទាស់ម្រាមដៃរបស់ខ្ញុំ"។

វាបានកើតឡើងខណៈដែល Hermione ស្ថិតនៅកណ្តាលនៃការលេប ហើយនាងក៏ញាក់ និងក្អក ហើយបញ្ចេញសារធាតុរាវពណ៌បៃតងភ្លឺចេញ។

នៅលើអាវធំរបស់មេធ្មប់ថ្មី ដែលមិនធ្លាប់ពាក់ពីមុនមក នៅថ្ងៃចូលរៀនដំបូង។

Hermione ពិតជាបានស្រែក។ វា​ជា​សំឡេង​ខ្ពស់​ដែល​បន្លឺ​ឡើង​ដូច​ស៊ីរ៉ែន​វាយ​ប្រហារ​តាម​អាកាស​ក្នុង​បន្ទប់​បិទ​ជិត។ “\emph{អឺ! សម្លៀកបំពាក់របស់ខ្ញុំ!}”

"កុំភ័យខ្លាច!" បាននិយាយថាក្មេងប្រុស។ "ខ្ញុំអាចជួសជុលវាសម្រាប់អ្នក។ គ្រាន់តែមើល!” គាត់បានលើកដៃមួយហើយខ្ទាស់ម្រាមដៃរបស់គាត់។

"អ្នកនឹង-" បន្ទាប់មកនាងមើលទៅខ្លួនឯង។

វត្ថុរាវពណ៌បៃតងនៅតែនៅទីនោះ ប៉ុន្តែទោះបីជានាងមើលក៏ដោយ ក៏វាចាប់ផ្តើមបាត់ និងរលាយបាត់ទៅវិញ ហើយក្នុងរយៈពេលតែប៉ុន្មាននាទីប៉ុណ្ណោះ វាហាក់ដូចជានាងមិនដែលហៀរទឹកអ្វីមកលើខ្លួននាងឡើយ។

Hermione សម្លឹងមើលទៅក្មេងនោះ ដែលពាក់ស្នាមញញឹមយ៉ាងក្រៀមក្រំ។

វេទមន្តគ្មានពាក្យ! នៅ \emph{his} អាយុ? នៅពេលដែលគាត់ទើបតែទទួលបានសៀវភៅសិក្សា \emph{បីថ្ងៃ}មុន?

ពេល​នោះ​នាង​នឹក​ឃើញ​អ្វី​ដែល​នាង​បាន​អាន ហើយ​នាង​ក៏​ដក​ដង្ហើម​ចេញ​ពី​គាត់។ \emph{ថាមពលវេទមន្តរបស់ Dark Lord ទាំងអស់! នៅក្នុងស្នាមរបស់គាត់!}

នាងក្រោកឡើងយ៉ាងលឿនទៅជើងរបស់នាង។ “ខ្ញុំ ខ្ញុំ ខ្ញុំត្រូវទៅបង្គន់ ចាំនៅទីនេះមិនអីទេ—” នាងត្រូវស្វែងរកមនុស្សធំដែលនាងត្រូវប្រាប់ពួកគេ—

ស្នាមញញឹមរបស់ក្មេងប្រុសរសាត់ទៅ។ "វាគ្រាន់តែជាល្បិចមួយ Hermione ។ ខ្ញុំសុំទោស ខ្ញុំមិនមានបំណងចង់បំភ័យអ្នកទេ»។

ដៃរបស់នាងបានឈប់នៅលើដៃទ្វារ។ “\emph{ល្បិច}?”

"បាទ" ក្មេងប្រុសបាននិយាយ។ "អ្នកបានសុំឱ្យខ្ញុំបង្ហាញភាពឆ្លាតវៃរបស់ខ្ញុំ។ ដូច្នេះ ខ្ញុំ​បាន​ធ្វើ​អ្វី​មួយ​ដែល​ទំនង​ជា​មិន​អាច​ទៅ​រួច ដែល​តែងតែ​ជា​វិធី​ល្អ​ដើម្បី​បង្ហាញ។ ខ្ញុំមិនអាច\emph{ពិតជា}ធ្វើអ្វីបានដោយគ្រាន់តែខ្ទាស់ម្រាមដៃរបស់ខ្ញុំ។" ក្មេងប្រុសបានផ្អាក។ "យ៉ាងហោចណាស់ខ្ញុំមិនបាន\emph{គិតថា}ខ្ញុំអាចធ្វើបានទេ ខ្ញុំមិនដែលសាកល្បងវាដោយពិសោធន៍ទេ។" ក្មេង​នោះ​លើក​ដៃ​ហើយ​ខ្ទាស់​ម្រាម​ដៃ​ម្ដង​ទៀត។ “អត់ទេ គ្មានចេកទេ”

Hermione មានភាពច្របូកច្របល់ដូចដែលនាងធ្លាប់មាននៅក្នុងជីវិតរបស់នាង។

ពេលនេះក្មេងប្រុសបានញញឹមម្តងទៀតនៅមុខរបស់នាង។ “ខ្ញុំបានធ្វើ\emph{ព្រមាន}អ្នកថាការប្រជែងនឹងភាពប៉ិនប្រសប់របស់ខ្ញុំមាននិន្នាការធ្វើឱ្យជីវិតរបស់អ្នកក្លាយជាការពិត។ សូម​ចងចាំ​វា​ពេល​ក្រោយ​ដែល​ខ្ញុំ​ព្រមាន​អ្នក​អំពី​អ្វី​មួយ»។

“ប៉ុន្តែ ប៉ុន្តែ” Hermione បាននិយាយយ៉ាងតឹងរ៉ឹង។ “តើអ្នកបាន\emph{ធ្វើអ្វី} បន្ទាប់មក?”

ការក្រឡេកមើលរបស់ក្មេងប្រុសបានធ្វើការវាស់វែង ថ្លឹងទម្ងន់ ដែលនាងមិនធ្លាប់ឃើញពីមុនមក ពីអាយុរបស់នាងផ្ទាល់។ “អ្នក​គិត​ថា​អ្នក​មាន​អ្វី​ដែល​វា​ត្រូវ​ការ​ដើម្បី​ក្លាយ​ជា​អ្នក​វិទ្យាសាស្ត្រ​ក្នុង​សិទ្ធិ​របស់​អ្នក​ដោយ​មាន​ឬ​គ្មាន​ជំនួយ​ពី​ខ្ញុំ? បន្ទាប់មកសូមមើលពីរបៀបដែល\emph{អ្នក}ស៊ើបអង្កេតបាតុភូតដែលច្របូកច្របល់។”

“ខ្ញុំ…” ចិត្តរបស់ Hermione ទទេរមួយសន្ទុះ។ នាងចូលចិត្តការធ្វើតេស្ត ប៉ុន្តែនាងមិនធ្លាប់ធ្វើតេស្តដូច\emph{នេះ}ពីមុនមកទេ។ ដោយ​ភ័យ​ខ្លាច នាង​បាន​ព្យាយាម​បដិសេធ​ចំពោះ​អ្វី​ដែល​នាង​បាន​អាន​អំពី​អ្វី​ដែល​អ្នក​វិទ្យាសាស្ត្រ​ត្រូវ​ធ្វើ។ ចិត្តរបស់នាងបានរំលងឧបករណ៍ ទប់ទល់នឹងខ្លួនវា ហើយស្តោះទឹកមាត់ដាក់ការណែនាំសម្រាប់ធ្វើគម្រោងស៊ើបអង្កេតវិទ្យាសាស្ត្រ៖

\emph{ជំហានទី 1៖ បង្កើតសម្មតិកម្ម។\\
ជំហានទី 2៖ ធ្វើការពិសោធន៍ដើម្បីសាកល្បងសម្មតិកម្មរបស់អ្នក។\\
ជំហានទី 3៖ វាស់វែងលទ្ធផល។\\
ជំហានទី ៤៖ ធ្វើផ្ទាំងរូបភាពក្រដាសកាតុងធ្វើកេស។}

ជំហានទី 1 គឺដើម្បីបង្កើតសម្មតិកម្មមួយ។ មាន​ន័យ​ថា សូម​ព្យាយាម​គិត​អំពី​អ្វី​មួយ​ដែល \emph{could} បាន​កើត​ឡើង​នៅ​ពេល​នេះ។ “មិនអីទេ។ សម្មតិកម្ម​របស់​ខ្ញុំ​គឺ​ថា​អ្នក​ដាក់​មន្ត​ស្នេហ៍​លើ​អាវ​របស់​ខ្ញុំ​ដើម្បី​ធ្វើ​ឲ្យ​អ្វី​ៗ​ដែល​ហៀរ​មក​លើ​វា​បាត់»។

ក្មេងប្រុសឆ្លើយថា "មិនអីទេ នោះជាចម្លើយរបស់អ្នក?"

ភាពតក់ស្លុតបានរលត់ទៅវិញ ហើយចិត្តរបស់ Hermione កំពុងចាប់ផ្តើមដំណើរការត្រឹមត្រូវ។ “ចាំមើល វាមិនត្រឹមត្រូវទេ។ ខ្ញុំ​មិន​បាន​ឃើញ​អ្នក​ប៉ះ​វល្លិ​របស់​អ្នក ឬ​និយាយ​អក្ខរាវិរុទ្ធ​ទេ ដូច្នេះ​តើ​អ្នក​អាច​ធ្វើ​មន្ត​ស្នេហ៍​បាន​ដោយ​របៀប​ណា?»

ក្មេងប្រុសរង់ចាំ ទឹកមុខអព្យាក្រឹត។

“ប៉ុន្តែឧបមាថាអាវផាយទាំងអស់បានមកពីហាងជាមួយនឹង Charm \emph{រួចហើយ} ដើម្បីរក្សាវាឱ្យស្អាត ដែលជាប្រភេទ Charm ដ៏មានប្រយោជន៍សម្រាប់ពួកគេមាន។ អ្នក​បាន​រក​ឃើញ​វា​ដោយ​ការ​កំពប់​អ្វី​មួយ​នៅ​លើ \emph{ខ្លួនអ្នក} មុននេះ។"

ឥឡូវនេះ ចិញ្ចើមរបស់ក្មេងប្រុសបានលើក។ “តើ \emph{នោះ} ចម្លើយរបស់អ្នកមែនទេ?”

"ទេ ខ្ញុំមិនបានធ្វើជំហានទី 2 ទេ 'ធ្វើការពិសោធន៍ដើម្បីសាកល្បងសម្មតិកម្មរបស់អ្នក។'"

ក្មេងប្រុសបិទមាត់ម្តងទៀត ហើយចាប់ផ្តើមញញឹម។

Hermione បានក្រឡេកមើលកំប៉ុងភេសជ្ជៈ ដែលនាងដាក់ដោយស្វ័យប្រវត្តិទៅក្នុងអ្នកដាក់ពែងនៅបង្អួច។ នាង​បាន​យក​វា​ឡើង​ហើយ​ពិនិត្យ​មើល​ខាង​ក្នុង ហើយ​បាន​ឃើញ​ថា​វា​ពេញ​មួយ​ភាគ​បី។

Hermione បាននិយាយថា “ជាការប្រសើរណាស់ ការពិសោធន៍ដែលខ្ញុំចង់ធ្វើគឺចាក់វានៅលើអាវធំរបស់ខ្ញុំ ហើយមើលថាតើមានអ្វីកើតឡើង ហើយការព្យាករណ៍របស់ខ្ញុំគឺថាស្នាមប្រឡាក់នឹងរលាយបាត់។ លុះត្រាតែវា \emph{មិនដំណើរការ} អាវរបស់ខ្ញុំនឹងត្រូវប្រឡាក់ ហើយខ្ញុំមិនចង់បាននោះទេ។”

ក្មេងប្រុសបាននិយាយថា "ធ្វើវាឱ្យខ្ញុំ" នោះអ្នកមិនចាំបាច់បារម្ភថាអាវរបស់អ្នកមានស្នាមប្រឡាក់ទេ។

"ប៉ុន្តែ -" Hermione បាននិយាយ។ មានអ្វីមួយ\emph{ខុស}ជាមួយនឹងការគិតនោះ ប៉ុន្តែនាងមិនដឹងពីរបៀបនិយាយវាឱ្យពិតប្រាកដនោះទេ។

ក្មេង​ប្រុស​រូប​នេះ​បាន​និយាយ​ថា៖ «ខ្ញុំ​មាន​អាវ​ក្រៅ​នៅ​ក្នុង​ដើម​របស់​ខ្ញុំ»។

Hermione បានជំទាស់ថា "ប៉ុន្តែគ្មានកន្លែងណាដែលអ្នកផ្លាស់ប្តូរទេ" ។ បន្ទាប់មកនាងគិតថាប្រសើរជាង។ “ទោះបីខ្ញុំគិតថាខ្ញុំអាចចាកចេញ ហើយបិទទ្វារក៏ដោយ—”

“ខ្ញុំ​មាន​កន្លែង​មួយ​ដែល​ត្រូវ​ផ្លាស់​ប្តូរ​ក្នុង​ប្រអប់​របស់​ខ្ញុំ​ផង​ដែរ”។

Hermione បានក្រឡេកមើលដើមរបស់គាត់ ដែលនាងចាប់ផ្តើមសង្ស័យថា ពិសេសជាងរបស់ខ្លួនឯងទៅទៀត។

Hermione បាននិយាយថា “មិនអីទេ តាំងពីអ្នកនិយាយអញ្ចឹងមក” ហើយនាងក៏ចាក់ពណ៌បៃតងបន្តិចទៅលើជ្រុងមួយនៃអាវក្មេងប្រុស។ បន្ទាប់មកនាងសម្លឹងមើលវាដោយព្យាយាមចាំថាតើសារធាតុរាវដើមបានបាត់អស់រយៈពេលប៉ុន្មាន…

ហើយស្នាមប្រឡាក់ពណ៌បៃតងបានបាត់!

Hermione ដក​ដង្ហើម​ធូរ​ស្រាល មិន​តិច​ទេ ព្រោះ​នេះ​មាន​ន័យ​ថា នាង​មិន​បាន​ដោះស្រាយ​ជាមួយ​នឹង​អំណាច​វេទមន្ត​របស់ Dark Lord ទាំងអស់​ទេ។

ជាការប្រសើរណាស់, ជំហានទី 3 កំពុងវាស់លទ្ធផល ប៉ុន្តែក្នុងករណីនេះ គ្រាន់តែឃើញថាស្នាមប្រឡាក់បានបាត់ទៅហើយ។ ហើយនាងសន្មត់ថានាងប្រហែលជាអាចរំលងជំហានទី 4 អំពីផ្ទាំងរូបភាពក្រដាសកាតុងធ្វើកេស។ "ចម្លើយរបស់ខ្ញុំគឺថា អាវផាយមានមន្តស្នេហ៍ ដើម្បីរក្សាខ្លួនឱ្យស្អាត"។

ក្មេងប្រុសបាននិយាយថា "មិនអីទេ" ។

Hermione មានអារម្មណ៍ខកចិត្ត។ នាងពិតជាប្រាថ្នាថានាង\emph{នឹងមិន}មានអារម្មណ៍បែបនេះទេ ក្មេងប្រុសនេះមិនមែនជាគ្រូបង្រៀនទេ ប៉ុន្តែវានៅតែជាការសាកល្បង ហើយនាងមានសំណួរខុស ហើយដែលតែងតែមានអារម្មណ៍ថាដូចជាកណ្តាប់ដៃតូចមួយនៅក្នុងពោះ .

(វាបាននិយាយស្ទើរតែគ្រប់យ៉ាងដែលអ្នកត្រូវដឹងអំពី Hermione Granger ថានាងមិនដែលអនុញ្ញាតឱ្យវាបញ្ឈប់នាង ឬសូម្បីតែអនុញ្ញាតឱ្យវារំខានដល់សេចក្តីស្រឡាញ់របស់នាងក្នុងការសាកល្បង។ )

ក្មេង​ប្រុស​នោះ​បាន​និយាយ​ថា​៖ «​រឿង​សោក​ស្តាយ​គឺ​អ្នក​ប្រហែល​ជា​បាន​ធ្វើ​អ្វី​គ្រប់​យ៉ាង​ដែល​សៀវភៅ​ប្រាប់​អ្នក​។ អ្នក​បាន​ធ្វើ​ការ​ទស្សន៍ទាយ​ដែល​នឹង​បែងចែក​រវាង​អាវ​ទ្រនាប់​ដែល​មាន​មន្ត​ស្នេហ៍ និង​មិន​មាន​មន្ត​ស្នេហ៍ ហើយ​អ្នក​បាន​សាកល្បង​ហើយ​បដិសេធ​នូវ​សម្មតិកម្ម​ដែល​មិន​មាន​ថា​អាវ​នោះ​មិន​មាន​មន្ត​ស្នេហ៍។ ប៉ុន្តែលុះត្រាតែអ្នកអានសៀវភៅប្រភេទដ៏ល្អបំផុត នោះពួកគេនឹងមិនបង្រៀនអ្នកពីរបៀបធ្វើវិទ្យាសាស្ត្រទេ \emph{ឱ្យបានត្រឹមត្រូវ}។ ជាការប្រសើរណាស់ក្នុងការ \emph{ពិតជា} ទទួលបានចម្លើយត្រឹមត្រូវ ខ្ញុំចង់និយាយថា ហើយមិនមែនគ្រាន់តែធ្វើការបោះពុម្ពផ្សាយផ្សេងទៀតដូចដែលប៉ាតែងតែត្អូញត្អែរនោះទេ។ ដូច្នេះ​សូម​ឲ្យ​ខ្ញុំ​ព្យាយាម​ពន្យល់ — ដោយ​មិន​បញ្ចេញ​ចម្លើយ—តើ​អ្នក​បាន​ធ្វើ​អ្វី​ខុស​លើក​នេះ ហើយ​ខ្ញុំ​នឹង​ផ្តល់​ឱកាស​ឲ្យ​អ្នក​ម្តង​ទៀត។

នាងចាប់ផ្តើមអន់ចិត្តនឹងទឹកដមដ៏ខ្ពង់ខ្ពស់របស់ក្មេងប្រុសម្នាក់នេះ នៅពេលដែលគាត់ទើបតែមានអាយុ 11 ឆ្នាំដូចនាង ប៉ុន្តែវាជារឿងបន្ទាប់បន្សំក្នុងការស្វែងរកអ្វីដែលនាងបានធ្វើខុស។ “មិនអីទេ”

ការបញ្ចេញមតិរបស់ក្មេងប្រុសកាន់តែខ្លាំងឡើង។ “នេះគឺជាហ្គេមផ្អែកលើការពិសោធន៍ដ៏ល្បីមួយហៅថា កិច្ចការ 2–4–6 ហើយនេះជារបៀបដែលវាដំណើរការ។ ខ្ញុំមាន \emph{rule}—ស្គាល់ខ្ញុំ ប៉ុន្តែមិនមែនសម្រាប់អ្នកទេ—ដែលសាកសមនឹងលេខបីចំនួនបី ប៉ុន្តែមិនមែនលេខផ្សេងទៀតទេ។ 2–4–6 ជា​ឧទាហរណ៍​មួយ​នៃ triplet ដែល​សម​នឹង​ច្បាប់។ តាមពិតទៅ… អនុញ្ញាតឱ្យខ្ញុំសរសេរច្បាប់នេះ ដើម្បីឱ្យអ្នកដឹងថាវាជាច្បាប់ថេរ ហើយបត់វាឡើង ហើយប្រគល់វាទៅអ្នក។ សូម​កុំ​មើល​អី ព្រោះ​ខ្ញុំ​សន្និដ្ឋាន​ពី​ដើម​មក​ថា​អ្នក​អាច​អាន​បញ្ច្រាស់​ចុះ»។

ក្មេងប្រុសនិយាយ “ក្រដាស” និង “ខ្មៅដៃមេកានិច” ទៅកាន់កាបូបរបស់គាត់ ហើយនាងបិទភ្នែកយ៉ាងតឹង ខណៈពេលដែលគាត់សរសេរ។

ក្មេងប្រុសនោះបាននិយាយថា "នៅទីនោះ" ហើយគាត់កំពុងកាន់ក្រដាសដែលបត់យ៉ាងតឹង។ "ដាក់ក្នុងហោប៉ៅរបស់អ្នក" ហើយនាងបានធ្វើ។

ក្មេងប្រុសបាននិយាយថា "ឥឡូវនេះរបៀបដែលហ្គេមនេះដំណើរការគឺអ្នកផ្តល់ឱ្យខ្ញុំបីដងនៃលេខបី ហើយខ្ញុំនឹងប្រាប់អ្នកថា "បាទ" ប្រសិនបើលេខទាំងបីគឺជាឧទាហរណ៍នៃច្បាប់ ហើយ 'ទេ' ប្រសិនបើ ពួកគេមិនមែនទេ។ ខ្ញុំជាធម្មជាតិ ច្បាប់គឺជាច្បាប់របស់ខ្ញុំ ហើយអ្នកកំពុងស៊ើបអង្កេតខ្ញុំ។ អ្នកដឹងរួចហើយថា 2-4-6 ទទួលបាន 'បាទ' ។ នៅពេលអ្នកបានធ្វើការធ្វើតេស្តពិសោធន៍បន្ថែមទាំងអស់ដែលអ្នកចង់បាន—បានសួរខ្ញុំបីដងច្រើនដងតាមដែលអ្នកមានអារម្មណ៍ថាចាំបាច់—អ្នកឈប់ ហើយទាយក្បួន ហើយបន្ទាប់មកអ្នកអាចលាតសន្លឹកក្រដាស ហើយមើលពីរបៀបដែលអ្នកបានធ្វើ។ តើអ្នកយល់ពីហ្គេមទេ?

Hermione បាននិយាយថា "ពិតណាស់ខ្ញុំធ្វើ" ។

“ទៅ។”

Hermione បាននិយាយថា "4-6-8" ។

"បាទ" ក្មេងប្រុសបាននិយាយ។

Hermione បាននិយាយថា "10-12-14" ។

"បាទ" ក្មេងប្រុសបាននិយាយ។

Hermione ព្យាយាម​ដក​ខ្លួន​ចេញ​ឆ្ងាយ​បន្តិច ព្រោះ​ហាក់​ដូច​ជា​នាង​បាន​ធ្វើ​ការ​សាកល្បង​ទាំង​អស់​ដែល​នាង​ត្រូវ​ការ​រួច​ហើយ ប៉ុន្តែ​វា​មិន​អាច​ជា​ការ​ងាយ​ស្រួល​នោះ​ទេ តើ​វា​ឬ​ទេ?

“១–៣–៥”។

“បាទ។”

"ដក 3 ដក 1 បូក 1 ។"

“បាទ។”

Hermione មិនអាចគិតពីអ្វីផ្សេងទៀតដែលត្រូវធ្វើ។ "ច្បាប់គឺថាចំនួនត្រូវតែកើនឡើងពីរដងរៀងរាល់ពេល។"

ក្មេងប្រុសបាននិយាយថា "ឥឡូវនេះឧបមាថាខ្ញុំប្រាប់អ្នកថាការធ្វើតេស្តនេះពិបាកជាងវាមើលទៅហើយមានតែ 20% នៃមនុស្សពេញវ័យប៉ុណ្ណោះដែលធ្វើវាត្រឹមត្រូវ។"

Hermione ងក់ក្បាល។ តើនាងខកខានអ្វី? រំពេចនោះ នាងបានគិតពីការសាកល្បងដែលនាងនៅតែត្រូវធ្វើ។

“២–៥–៨!” នាងបាននិយាយដោយជោគជ័យ។

“បាទ។”

“១០–២០–៣០!”

“បាទ។”

“ចម្លើយពិតប្រាកដគឺថា លេខត្រូវឡើងដោយចំនួន\emph{same} រាល់ពេល។ វា​មិន​ចាំបាច់​មាន 2 ទេ»។

ក្មេងនោះនិយាយថា "ល្អណាស់" យកក្រដាសចេញ ហើយមើលពីរបៀបដែលអ្នកធ្វើ។

Hermione យកក្រដាសចេញពីហោប៉ៅរបស់នាង ហើយបកវាចេញ។

\emph{ចំនួនពិតបីនៅក្នុងលំដាប់កើនឡើង ទាបបំផុតដល់ខ្ពស់បំផុត។}

ថ្គាមរបស់ Hermione បានធ្លាក់ចុះ។ នាង​មាន​អារម្មណ៍​ប្លែក​ពី​រឿង​អយុត្តិធម៌​យ៉ាង​ខ្លាំង​ដែល​បាន​ធ្វើ​មក​លើ​នាង ថា​ក្មេង​ប្រុស​នោះ​ជា​មនុស្ស​កុហក​បោកប្រាស់​ដ៏​កខ្វក់ ប៉ុន្តែ​ពេល​នាង​គិត​ថយ​ក្រោយ នាង​មិន​អាច​គិត​ពី​ការ​ឆ្លើយ​តប​ខុស​ដែល​គេ​បាន​ផ្តល់​ឱ្យ​នោះ​ទេ។

ក្មេង​ប្រុស​រូប​នេះ​បាន​និយាយ​ថា៖ «អ្វី​ដែល​អ្នក​ទើប​តែ​រក​ឃើញ​គឺ​ហៅ​ថា 'លំអៀង​វិជ្ជមាន'។ “អ្នកមានច្បាប់មួយនៅក្នុងចិត្ត ហើយអ្នកបន្តគិតពីបីដង ដែលគួរតែធ្វើឱ្យច្បាប់និយាយថា 'បាទ' ។ ប៉ុន្តែ​អ្នក​មិន​បាន​ព្យាយាម​សាកល្បង​បី​ដង​ដែល​គួរ​ធ្វើ​ឱ្យ​ច្បាប់​និយាយថា 'ទេ' ។ តាមការពិត អ្នកមិនទទួលបាន \emph{single} 'ទេ' ដូច្នេះ 'លេខបី' ក៏អាចមានច្បាប់យ៉ាងងាយស្រួលដែរ។ វាដូចជារបៀបដែលមនុស្សស្រមៃការពិសោធន៍ដែលអាចបញ្ជាក់ពីសម្មតិកម្មរបស់ពួកគេជំនួសឱ្យការព្យាយាមស្រមៃការពិសោធន៍ដែលអាចក្លែងបន្លំពួកគេ - នោះមិនមែនជាកំហុសដូចគ្នាទេប៉ុន្តែវាជិត។ អ្នក​ត្រូវ​រៀន​មើល​ទៅ​លើ​ផ្នែក​អវិជ្ជមាន​នៃ​អ្វី​ៗ សម្លឹង​ទៅ​ក្នុង​ភាព​ងងឹត។ នៅពេលដែលការពិសោធន៍នេះត្រូវបានអនុវត្ត មានតែ 20% នៃមនុស្សពេញវ័យប៉ុណ្ណោះដែលទទួលបានចម្លើយត្រឹមត្រូវ។ ហើយមនុស្សជាច្រើនផ្សេងទៀតបានបង្កើតសម្មតិកម្មដ៏ស្មុគស្មាញដ៏អស្ចារ្យ ហើយដាក់ទំនុកចិត្តយ៉ាងខ្លាំងចំពោះចម្លើយខុសរបស់ពួកគេ ចាប់តាំងពីពួកគេបានធ្វើការពិសោធន៍ជាច្រើន ហើយអ្វីៗទាំងអស់ចេញមកដូចដែលពួកគេរំពឹងទុក។

ក្មេងប្រុសបាននិយាយថា "ឥឡូវនេះតើអ្នកចង់បាញ់មួយផ្សេងទៀតចំពោះបញ្ហាដើម?"

ភ្នែករបស់គាត់ពិតជាមានចេតនាឥឡូវនេះ ហាក់ដូចជាការសាកល្បង \emph{real}។

Hermione បិទភ្នែកហើយព្យាយាមផ្តោតអារម្មណ៍។ នាងកំពុងបែកញើសនៅក្រោមអាវរបស់នាង។ នាងមានអារម្មណ៍ចម្លែកថា នេះជាការពិបាកបំផុតដែលនាងមិនធ្លាប់ត្រូវបានស្នើសុំឱ្យគិតលើការធ្វើតេស្ត ឬប្រហែលជា \emph{first} ដែលនាងមិនធ្លាប់ត្រូវបានស្នើសុំឱ្យគិតលើការធ្វើតេស្ត។

តើនាងអាចធ្វើពិសោធន៍អ្វីទៀត? នាងមាន Chocolate Frog តើនាងអាចព្យាយាមជូតវាខ្លះនៅលើអាវ ហើយមើលថាតើ \emph{it} បាត់ដែរឬទេ? ប៉ុន្តែ​វា​នៅ​តែ​មិន​ហាក់​ដូច​ជា​ការ​គិត​អវិជ្ជមាន​ដែល​ក្មេង​ប្រុស​កំពុង​សុំ​នោះ​ទេ។ ដូចជានាងនៅតែស្នើសុំ 'បាទ' ប្រសិនបើស្នាមប្រឡាក់ កង្កែបសូកូឡាបាត់ ជាជាងសុំ 'ទេ' ។

ដូច្នេះ… តាមសម្មតិកម្មរបស់នាង… តើពេលណាទើបប៉ុប…\emph{មិន} បាត់ទៅ?

Hermione បាននិយាយថា "ខ្ញុំមានការពិសោធន៍មួយដើម្បីធ្វើ" ។ “ខ្ញុំចង់ចាក់ប៉ុបខ្លះនៅលើឥដ្ឋ ហើយមើលថាតើវា \emph{មិនបាត់}។ តើ​អ្នក​មាន​កន្សែង​ក្រដាស​ខ្លះ​នៅ​ក្នុង​កាបូប​របស់​អ្នក ដូច្នេះ​ខ្ញុំ​អាច​លុប​ការ​កំពប់​បាន​ប្រសិន​បើ​វា​មិន​ដំណើរការ?»

ក្មេងប្រុសបាននិយាយថា "ខ្ញុំមានកន្សែង" ។ មុខរបស់គាត់នៅតែមើលទៅអព្យាក្រឹត។

Hermione យក​កំប៉ុង​មក​ចាក់​ទឹក​បន្តិច​លើ​ឥដ្ឋ។

ពីរបីវិនាទីក្រោយមកវាបានបាត់។

ពេល​នោះ​ការ​ដឹង​ខ្លួន​បាន​វាយ​នាង ហើយ​នាង​មាន​អារម្មណ៍​ថា​ចង់​ទាត់​ខ្លួន​ឯង។ “ពិតណាស់! \emph{អ្នក}បានផ្តល់ឱ្យខ្ញុំថាអាច! វា​មិន​មែន​ជា​អាវ​ដែល​មាន​ភាព​ទាក់​ទាញ​នោះ​ទេ វា​គឺ​ជា​ការ​ពេញ​និយម​!»។

ក្មេង​ប្រុស​ក្រោក​ឈរ​ឱន​ក្បាល​នាង​យ៉ាង​ឧឡារិក។ គាត់កំពុងញញឹមយ៉ាងទូលំទូលាយឥឡូវនេះ។ “អញ្ចឹង… តើខ្ញុំអាចជួយអ្នកក្នុងការស្រាវជ្រាវរបស់អ្នកបានទេ Hermione Granger?”

“ខ្ញុំ អេ…” Hermione នៅតែមានអារម្មណ៍រំភើបញាប់ញ័រ ប៉ុន្តែនាងមិនច្បាស់អំពីរបៀបឆ្លើយ\emph{នោះ}ទេ។

ពួកគេ​ត្រូវ​បាន​រំខាន​ដោយ​ការ​គោះ​ទ្វារ​ដោយ​ការ​ដួល​សន្លប់ ទន់​ខ្សោយ ដួល​សន្លប់ ជា​ជាង \emph{ដោយ​ស្ទាក់​ស្ទើរ}។

ក្មេង​នោះ​បែរ​មើល​ទៅ​ក្រៅ​បង្អួច ហើយ​និយាយ​ថា “ខ្ញុំ​មិន​បាន​ពាក់​ក្រមា​ទេ ដូច្នេះ​តើ​អ្នក​អាច​ទទួល​បាន​ទេ?”

វាគឺនៅចំណុចនេះដែល Hermione បានដឹងពីមូលហេតុដែលក្មេងប្រុស—ទេ ក្មេងប្រុស-Who-Lived, Harry Potter—បានពាក់ក្រមាពីលើក្បាលរបស់គាត់តាំងពីដំបូង ហើយមានអារម្មណ៍ឆ្កួតបន្តិចដែលមិនបានដឹងពីមុនមក។ វាពិតជាចម្លែកណាស់ ព្រោះនាងគិតថា Harry Potter នឹងបង្ហាញខ្លួនឯងទៅកាន់ពិភពលោកដោយមោទនភាព។ ហើយគំនិតបានកើតឡើងចំពោះនាងថា គាត់ប្រហែលជាអៀនជាងមើលទៅ។

នៅពេលដែល Hermione ទាញទ្វារបើក នាងត្រូវបានស្វាគមន៍ដោយក្មេងប្រុសតូចញាប់ញ័រ ដែលមើលទៅហាក់ដូចជាគាត់គោះ។

"សុំទោស" ក្មេងប្រុសនិយាយដោយសំឡេងតូច "ខ្ញុំគឺ Neville Longbottom ។ ខ្ញុំ​កំពុង​តែ​រក​សត្វ​កង្កែប​របស់​ខ្ញុំ ខ្ញុំ​ហាក់​ដូច​ជា​រក​មិន​ឃើញ​នៅ​កន្លែង​ណា​នៅ​លើ​រទេះ​នេះ​ទេ… តើ​អ្នក​បាន​ឃើញ​សត្វ​កង្កែប​របស់​ខ្ញុំ​ទេ?

Hermione បាននិយាយថា “ទេ” ហើយបន្ទាប់មក ជំនួយរបស់នាងបានចាប់ផ្ដើមពេញមួយទំហឹង។ "តើអ្នកបានពិនិត្យបន្ទប់ផ្សេងទៀតទាំងអស់ហើយឬនៅ?"

"បាទ" ក្មេងប្រុសខ្សឹបប្រាប់។

Hermione បាន​និយាយ​យ៉ាង​រហ័ស​ថា​៖ «​បន្ទាប់​មក​យើង​នឹង​ត្រូវ​ពិនិត្យ​ទូរថភ្លើង​ផ្សេង​ទៀត​ទាំងអស់​។ “ខ្ញុំនឹងជួយអ្នក។ ខ្ញុំឈ្មោះ Hermione Granger និយាយអញ្ចឹង។

ក្មេងប្រុសមើលទៅហាក់ដូចជាដួលសន្លប់ដោយការដឹងគុណ។

“ចាំអីទៀត” សំលេងរបស់ក្មេងប្រុស \emph{other}—Harry Potter បានចេញមក។ "ខ្ញុំ​មិន​ប្រាកដ​ថា​នោះ​ជា​វិធី​ល្អ​បំផុត​ក្នុង​ការ​ធ្វើ​វា​ទេ"។

នៅឯ Neville នេះហាក់ដូចជាគាត់ប្រហែលជាយំ ហើយ Hermione បានដើរជុំវិញដោយកំហឹង។ បើ Harry Potter ជា​មនុស្ស​ដែល​បោះបង់​ក្មេង​ប្រុស​ម្នាក់​ដោយ​សារ​តែ​គាត់​មិន​ចង់​រំខាន… “តើ​យ៉ាង​ណា? ហេតុអ្វី\emph{មិនមែន}?”

Harry Potter បាននិយាយថា "វានឹងចំណាយពេលមួយរយៈដើម្បីពិនិត្យមើលរថភ្លើងទាំងមូលដោយដៃ ហើយយើងប្រហែលជានឹក toad ហើយប្រសិនបើយើងរកមិនឃើញនៅពេលយើងនៅ Hogwarts គាត់" d មានបញ្ហា។ ដូច្នេះ អ្វី​ដែល​អាច​យល់​បាន​ច្រើន​ជាង​នេះ​គឺ​ប្រសិន​បើ​គាត់​បាន​ទៅ​កន្លែង​ដឹក​ជញ្ជូន​ខាង​មុខ​ដោយ​ផ្ទាល់ ហើយ​បាន​សុំ​អភិបាល​ខេត្ត​ម្នាក់​ឱ្យ​ជួយ។ នោះគឺជារឿងដំបូងដែលខ្ញុំបានធ្វើ នៅពេលដែលខ្ញុំកំពុងស្វែងរកអ្នក Hermione ទោះបីជាពួកគេមិនដឹងពិតប្រាកដក៏ដោយ។ ប៉ុន្តែពួកគេអាចមានអក្ខរាវិរុទ្ធ ឬវត្ថុវេទមន្ត ដែលនឹងធ្វើឱ្យវាកាន់តែងាយស្រួលក្នុងការស្វែងរក toad ។ យើង​ទើប​តែ​ជា​ឆ្នាំ​ដំបូង»។

នោះ…\emph{did} មានន័យច្រើន។

"តើ​អ្នក​គិត​ថា​អ្នក​អាច​ធ្វើ​វា​ទៅ​កាន់​រទេះ​របស់​អាណា​ខេត្ត​ដោយ​ខ្លួន​ឯង​ឬ?" សួរ Harry Potter ។ "ខ្ញុំ​មាន​ហេតុផល​មិន​ចង់​បង្ហាញ​មុខ​ច្រើន​ពេក"។

ភ្លាមៗនោះ Neville ដកដង្ហើមធំ ហើយដើរថយក្រោយ។ “ខ្ញុំចាំសំឡេងនោះ! អ្នកគឺជាព្រះអម្ចាស់នៃភាពវឹកវរ! \emph{អ្នកគឺជាអ្នកដែលអោយសូកូឡាខ្ញុំ!}”

អ្វី? តើ \emph{អ្វី}?

Harry Potter បែរក្បាលចេញពីបង្អួច ហើយងើបឡើងយ៉ាងខ្លាំង។ “ខ្ញុំ\emph{មិនដែល}!” គាត់បាននិយាយថា សំឡេងពោរពេញដោយកំហឹង។ "តើខ្ញុំមើលទៅដូចមនុស្សកំណាចដែលផ្តល់បង្អែមដល់ក្មេងមែនទេ?"

Neville បើកភ្នែកធំៗ។ “\emph{អ្នក} Harry Potter? \emph{The} Harry Potter? \emph{អ្នក?}”

“ទេ ត្រឹមតែ\emph{a} Harry Potter មានខ្ញុំបីនាក់នៅលើរថភ្លើងនេះ—”

ណេវីល​បាន​បន្លឺ​សំឡេង​តូច​មួយ ហើយ​រត់​ចេញ។ មាន​សំឡេង​គន្លាក់​ជើង​ដ៏​ព្រឺព្រួច​មួយ​រំពេច​មួយ​រំពេច ហើយ​បន្ទាប់​មក​មាន​សំឡេង​បើក និង​បិទ​ទ្វារ​រទេះរុញ។

Hermione អង្គុយយ៉ាងលំបាកនៅលើកៅអីរបស់នាង។ Harry Potter បិទទ្វារ ហើយបន្ទាប់មកអង្គុយក្បែរនាង។

"សូមជួយពន្យល់ខ្ញុំផង តើមានអ្វីកើតឡើង?" Hermione និយាយដោយសំឡេងខ្សោយ។ នាង​ឆ្ងល់​ថា​តើ​ការ​នៅ​ជុំវិញ Harry Potter មាន​ន័យ​ថា​តែងតែ​យល់​ច្រឡំ​ឬ​អត់?

“អូ! អ្វីដែលបានកើតឡើងគឺថា Fred និង George និងខ្ញុំបានឃើញក្មេងប្រុសតូចដ៏កំសត់នេះនៅស្ថានីយ៍រថភ្លើង—ស្ត្រីដែលនៅក្បែរគាត់បានទៅឆ្ងាយបន្តិច ហើយគាត់មើលទៅពិតជាភ័យខ្លាចខ្លាំងណាស់ ដូចជាគាត់ប្រាកដថាគាត់កំពុង ត្រូវបានវាយប្រហារដោយ Death Eaters ឬអ្វីមួយ។ ឥឡូវនេះ មានពាក្យថាការភ័យខ្លាចច្រើនតែអាក្រក់ជាងរឿងខ្លួនឯងទៅទៀត ដូច្នេះវាបានកើតឡើងចំពោះខ្ញុំថា នេះគឺជាក្មេងម្នាក់ដែលពិតជាអាចទទួលបានអត្ថប្រយោជន៍ពីការមើលឃើញសុបិន្តអាក្រក់បំផុតរបស់គាត់ក្លាយជាការពិត ហើយវាមិនអាក្រក់ដូចអ្វីដែលគាត់ខ្លាចនោះទេ—»

Hermione អង្គុយនៅទីនោះដោយបើកមាត់របស់នាង។

“—ហើយហ្វ្រេដ និងចច បានបង្កើតនូវអក្ខរាវិរុទ្ធនេះ ដើម្បីធ្វើឱ្យកន្សែងបង់កលើមុខរបស់យើងងងឹត និងព្រិលៗ ដូចជាយើងជាស្តេចដែលមិនទាន់ស្លាប់ ហើយទាំងនោះជាក្រមាផ្នូររបស់យើង—”

នាងមិនចូលចិត្តកន្លែងនេះទៅណាទេ។

“—ហើយ​បន្ទាប់​ពី​យើង​បាន​ឲ្យ​បង្អែម​ទាំង​អស់​ដែល​ខ្ញុំ​បាន​ទិញ​ឲ្យ​គាត់​រួច​ហើយ យើង​ក៏​ដូច​ជា 'តោះ​ឲ្យ​លុយ​គាត់​ខ្លះ! ហាហាហា! មាន Knuts ខ្លះ, ក្មេងប្រុស! ចូរ​កាន់​ស៊ីក​ប្រាក់!” ហើយ​រាំ​ជុំវិញ​គាត់ ហើយ​សើច​យ៉ាង​អាក្រក់​ជាដើម។ ខ្ញុំគិតថា មានមនុស្សមួយចំនួននៅក្នុងហ្វូងមនុស្សដែលចង់ជ្រៀតជ្រែកពីដំបូង ប៉ុន្តែការព្រងើយកន្តើយពីអ្នកទស្សនាបានរារាំងពួកគេយ៉ាងហោចណាស់រហូតដល់ពួកគេឃើញអ្វីដែលយើងកំពុងធ្វើ ហើយបន្ទាប់មកខ្ញុំគិតថាពួកគេទាំងអស់គ្នាយល់ច្រឡំក្នុងការធ្វើអ្វីទាំងអស់។ ទីបំផុតគាត់បាននិយាយដោយខ្សឹបខ្សៀវតូចមួយនេះថា 'ទៅឆ្ងាយ' ដូច្នេះយើងទាំងបីនាក់បានស្រែកហើយរត់ចេញ ដោយស្រែកយំខ្លះអំពីពន្លឺដែលឆេះយើង។ សង្ឃឹម​ថា​គាត់​នឹង​មិន​ភ័យ​ខ្លាច​ដូច​ពេល​ត្រូវ​គេ​សម្លុត​ទៅ​ថ្ងៃ​អនាគត។ វាត្រូវបានគេហៅថាការព្យាបាលដោយ desensitisation ដោយវិធីនេះ។

មិនអីទេ នាង\emph{មិនបាន}ទាយត្រូវអំពីកន្លែងដែលវានឹងទៅ។

ភ្លើងឆេះនៃកំហឹងដែលជាម៉ាស៊ីនចម្បងមួយរបស់ Hermione បានផ្ទុះឡើងក្នុងជីវិត ទោះបីជាផ្នែកខ្លះនៃ\emph{បាន} របស់នាងបានឃើញនូវអ្វីដែលពួកគេកំពុងព្យាយាមធ្វើក៏ដោយ។ “វាអាក្រក់ណាស់! \emph{អ្នក} អាក្រក់ណាស់! ក្មេងកំសត់ម្នាក់នេះ! អ្វីដែលអ្នកបានធ្វើគឺ\emph{មានន័យថា}!”

“ខ្ញុំគិតថាពាក្យដែលអ្នកកំពុងស្វែងរកគឺ\emph{រីករាយ} ហើយក្នុងករណីណាក៏ដោយដែលអ្នកកំពុងសួរសំណួរខុស។ សំណួរសួរថា តើវាធ្វើល្អជាងគ្រោះថ្នាក់ ឬគ្រោះថ្នាក់ជាងល្អ? ប្រសិនបើអ្នកមានអំណះអំណាងណាមួយដើម្បីរួមចំណែកដល់ \emph{នោះ} សំណួរខ្ញុំរីករាយដែលបានឮពួកគេ ប៉ុន្តែខ្ញុំនឹងមិនរីករាយនឹងការរិះគន់ណាមួយផ្សេងទៀតរហូតដល់ដំណោះស្រាយនោះត្រូវបានដោះស្រាយ។ ខ្ញុំពិតជាយល់ស្របថាអ្វីដែលខ្ញុំបានធ្វើ\emph{មើលទៅ} ទាំងអស់គួរឱ្យខ្លាច និងសម្លុត ហើយមានន័យ ព្រោះវាពាក់ព័ន្ធនឹងក្មេងប្រុសតូចដែលគួរឱ្យខ្លាច ហើយដូច្នេះនៅលើ ប៉ុន្តែនោះមិនមែនជាបញ្ហាសំខាន់ទេឥឡូវនេះមែនទេ? នោះត្រូវបានគេហៅថា \emph{consequentialism} ដោយវិធីនេះ វាមានន័យថា ថាតើទង្វើមួយត្រូវ ឬខុសមិនត្រូវបានកំណត់ដោយថាតើវា \emph{មើលទៅ}អាក្រក់ ឬមានន័យ ឬអ្វីក៏ដោយ សំណួរតែមួយគត់គឺ តើ​វា​នឹង​ក្លាយ​ទៅ​ជា​យ៉ាង​ណា​នៅ​ទី​បញ្ចប់—តើ​មាន​ផល​វិបាក​យ៉ាង​ណា»។

Hermione បានបើកមាត់របស់នាងដើម្បីនិយាយអ្វីមួយទាំងស្រុង\emph{searing} ប៉ុន្តែជាអកុសលនាងហាក់ដូចជាមិនបានយកចិត្តទុកដាក់ផ្នែកដែលនាងគិតចង់និយាយមុនពេលបើកមាត់របស់នាង។ អ្វីដែលនាងអាចទទួលបានគឺ "ចុះបើគាត់មាន \emph{សុបិន្តអាក្រក់}?"

“និយាយតាមត្រង់ទៅ ខ្ញុំមិនគិតថាគាត់ត្រូវការជំនួយរបស់យើងដើម្បីសុបិន្តអាក្រក់ទេ ហើយប្រសិនបើគាត់សុបិន្តអាក្រក់ប្រហែល \emph{នេះ} ជំនួសវិញ នោះវានឹងជាសុបិន្តអាក្រក់ដែលទាក់ទងនឹងសត្វចម្លែកដ៏អាក្រក់ដែលផ្តល់ឱ្យអ្នកនូវសូកូឡា ហើយនោះគឺជាប្រភេទទាំងមូល។ \emph{point}។

ខួរក្បាលរបស់ Hermione ចេះតែមានការភាន់ច្រលំ រាល់ពេលដែលនាងព្យាយាមខឹង។ "តើជីវិតរបស់អ្នកតែងតែប្លែកទេ?" នាងបាននិយាយចុងក្រោយ។

មុខរបស់ Harry Potter ភ្លឺដោយមោទនភាព។ “ខ្ញុំ \emph{make} ប្លែក​ណាស់។ អ្នក​កំពុង​សម្លឹង​មើល​ផលិតផល​នៃ​ការ​ខិតខំ​ប្រឹងប្រែង​ជា​ច្រើន និង​ខាញ់​កែង​ដៃ»។

“ដូច្នេះ…” Hermione និយាយ ហើយដើរចេញដោយងឿងឆ្ងល់។

លោក Harry Potter បាននិយាយថា “តើអ្នកដឹងវិទ្យាសាស្ត្រប៉ុន្មាន? ខ្ញុំអាចធ្វើការគណនាបាន ហើយខ្ញុំដឹងពីទ្រឹស្តីប្រូបាប៊ីលីតេ និងទ្រឹស្តីការសម្រេចចិត្តរបស់ Bayesian និងវិទ្យាសាស្ត្រយល់ដឹងជាច្រើន ហើយខ្ញុំបានអាន \emph{The Feynman Lectures} (ឬភាគ 1 យ៉ាងណាក៏ដោយ) និង \emph{ការវិនិច្ឆ័យក្រោមភាពមិនប្រាកដប្រជា៖ Heuristics and Biases} និង\emph{Language in Thought and Action} និង\emph{Influence: Science and Practice} និង\emph{Rational Choice in an Uncertain World} និង\emph{Gödel, Escher, Bach} និង € 114€{ជំហានកាន់តែឆ្ងាយ} និង—”

កម្រងសំណួរបន្ត និងសំណួរតបតបានបន្តអស់រយៈពេលជាច្រើននាទី មុនពេលត្រូវបានរំខានដោយការគោះទ្វារដ៏គួរឱ្យភ័យខ្លាចមួយទៀត។ "ចូលមក" នាង និង Harry Potter បាននិយាយស្ទើរតែក្នុងពេលតែមួយ ហើយវារំកិលត្រឡប់មកវិញដើម្បីបង្ហាញ Neville Longbottom ។

Neville \emph{} ពិតជាកំពុងយំឥឡូវនេះ។ "ខ្ញុំបានទៅរទេះរុញខាងមុខ ហើយបានរកឃើញ p-prefect ប៉ុន្តែគាត់បានប្រាប់ខ្ញុំថា prefects មិនត្រូវខ្វល់ពីរឿងតូចតាចដូចជា m-mising toads"។

មុខរបស់ក្មេងប្រុសដែលរស់នៅបានផ្លាស់ប្តូរ។ បបូរមាត់របស់គាត់ដាក់ជាបន្ទាត់ស្តើង។ សំឡេង​គាត់​ពេល​គាត់​និយាយ​គឺ​ត្រជាក់ និង​ក្រៀមក្រំ។ "តើគាត់មានពណ៌អ្វី? បៃតង និងប្រាក់?

"ទេ ផ្លាកសញ្ញារបស់គាត់គឺក្រហម និងមាស។"

“\emph{Red and gold!}” ផ្ទុះឡើង Hermione។ “ប៉ុន្តែពណ៌ទាំងនោះគឺ\emph{Gryffindor's}!”

Harry Potter \emph{hissed} នៅ​ពេល​នោះ ជា​សំឡេង​ដ៏​គួរ​ឲ្យ​ភ័យ​ខ្លាច​ដែល​អាច​ចេញ​ពី​សត្វ​ពស់ ហើយ​បាន​ធ្វើ​ឲ្យ​ទាំង​នាង និង Neville ញ័រ​ខ្លួន។ “ខ្ញុំ \emph{suppose},” Harry Potter spat, “ការស្វែងរក toad ឆ្នាំដំបូងមួយចំនួនគឺមិន\emph{heroic} គ្រប់គ្រាន់ដើម្បីសក្ដិសមនៃ\emph{Gryffindor} អាណាខេត្ត។ សូមអញ្ជើញមក Neville, \emph{ខ្ញុំនឹង} មកជាមួយអ្នកនៅពេលនេះ យើងនឹងមើលថាតើ Boy-Who-Lived ទទួលបានការចាប់អារម្មណ៍ច្រើនជាងនេះដែរឬទេ។ ជាដំបូង យើងនឹងស្វែងរកចៅហ្វាយខេត្តដែលគួរតែដឹងពីអក្ខរាវិរុទ្ធ ហើយប្រសិនបើវាមិនដំណើរការ យើងនឹងស្វែងរកចៅហ្វាយខេត្តដែលមិនខ្លាចការធ្វើឱ្យដៃរបស់ពួកគេកខ្វក់ ហើយប្រសិនបើ \emph{នោះ} មិន ការងារ ខ្ញុំនឹងចាប់ផ្តើមជ្រើសរើសអ្នកគាំទ្ររបស់ខ្ញុំ ហើយប្រសិនបើយើងត្រូវតែ យើងនឹងបំបែកវីសរថភ្លើងទាំងមូលដោយវីស”។

The Boy-Who-Lived ក្រោកឈរឡើង ហើយចាប់ដៃ Neville នៅក្នុងរបស់គាត់ ហើយ Hermione ដឹងដោយមានការឈឺក្បាលក្នុងខួរក្បាលភ្លាមៗថាពួកគេមានទំហំជិតដូចគ្នា ទោះបីជាផ្នែកខ្លះរបស់នាងបានទទូចថា Harry Potter មានកម្ពស់ខ្ពស់ជាងនេះក៏ដោយ ហើយ Neville យ៉ាងហោចណាស់ប្រាំមួយអ៊ីញខ្លីជាង។

“\emph{Stay!}” Harry Potter ចាប់នាង—អត់ទេ ចាំនៅ\emph{trunk} របស់គាត់—ហើយគាត់បានបិទទ្វារពីក្រោយគាត់យ៉ាងរឹងមាំពេលគាត់ចាកចេញ។

នាងប្រហែលជាបានទៅជាមួយពួកគេ ប៉ុន្តែក្នុងរយៈពេលខ្លីមួយ Harry Potter បានប្រែទៅជាគួរឱ្យខ្លាចខ្លាំងណាស់ដែលនាងពិតជារីករាយដែលនាងមិនគិតថានឹងណែនាំវាទេ។

% The misspelling ``History: A Hogwarts'' is intentional.
ពេលនេះ ចិត្តរបស់ Hermione មានភាពច្របូកច្របល់ខ្លាំងណាស់ ដែលនាងមិននឹកស្មានថានាងអាចអានបានត្រឹមត្រូវ \emph{History: A Hogwarts}។ នាង​មាន​អារម្មណ៍​ថា​នាង​គ្រាន់​តែ​ត្រូវ​បាន​រត់​ដោយ​ម៉ាស៊ីន​ចំហុយ ហើយ​ក្លាយ​ទៅ​ជា​នំ​បញ្ចុក។ នាង​មិន​ប្រាកដ​ថា​នាង​កំពុង​គិត​អ្វី​ឬ​អ្វី​ដែល​នាង​មាន​អារម្មណ៍​ឬ​ហេតុ​អ្វី​នោះ​ទេ​។ នាងគ្រាន់តែអង្គុយក្បែរបង្អួច ហើយសម្លឹងមើលទេសភាពដែលរំកិលទៅមុខ។

យ៉ាង​ហោច​ណាស់ នាង​បាន​ដឹង​ថា​ហេតុ​អ្វី​បាន​ជា​នាង​មាន​អារម្មណ៍​សោកសៅ​បន្តិច​នៅ​ខាង​ក្នុង។

ប្រហែលជា Gryffindor មិនអស្ចារ្យដូចដែលនាងបានគិតនោះទេ។

%  LocalWords:  NPC Eek Judgment
