\chapter{វិធីសាស្ត្រវិទ្យាសាស្ត្រ}

\begin{chapterOpeningAuthorNote}
គន្លឹះនៃយុទ្ធសាស្ត្រគឺមិនមែនជ្រើសរើសផ្លូវទៅកាន់ J.~K.~Rowling នោះទេ ប៉ុន្តែត្រូវជ្រើសរើសដើម្បីឱ្យផ្លូវទាំងអស់នាំទៅរក J.~K.~Rowling។
\end{chapterOpeningAuthorNote}

\lettrine{A}{} បន្ទប់សិក្សាតូចនៅជិត ប៉ុន្តែមិនមែននៅក្នុងបន្ទប់ស្នាក់នៅរបស់ Ravenclaw ដែលជាបន្ទប់មួយក្នុងចំណោមបន្ទប់ជាច្រើនដែលមិនបានប្រើរបស់ Hogwarts ។ ថ្មពណ៌ប្រផេះលើកម្រាលឥដ្ឋ ឥដ្ឋក្រហមជញ្ជាំង ឈើប្រឡាក់ងងឹតលើពិដាន កញ្ចក់កញ្ចក់ភ្លឺចំនួនបួនដែលដាក់ចូលទៅក្នុងជញ្ជាំងទាំងបួននៃបន្ទប់។ តុរាងជារង្វង់ដែលមើលទៅដូចជាបន្ទះថ្មម៉ាបខ្មៅធំទូលាយដាក់លើជើងថ្មម៉ាបខ្មៅក្រាស់សម្រាប់សសរ ប៉ុន្តែដែលបង្ហាញថាស្រាលខ្លាំង (ទម្ងន់ និងម៉ាស់ទាំងពីរ) ហើយមិនពិបាកក្នុងការរើស ហើយផ្លាស់ទីជុំវិញប្រសិនបើចាំបាច់។ កៅអីខ្នើយពីរដែលមើលទៅហាក់ដូចជាត្រូវបានចាក់សោរជាប់នឹងកម្រាលឥដ្ឋនៅកន្លែងដែលមិនស្រួល ប៉ុន្តែទីបំផុតពួកគេទាំងពីរបានរកឃើញ ដើរជុំវិញកន្លែងដែលអ្នកឈរភ្លាមៗនៅពេលដែលអ្នកផ្អៀងលើឥរិយាបថដែលមើលទៅដូចអ្នក ហៀបនឹងអង្គុយចុះ។

វាហាក់ដូចជាមានសត្វប្រចៀវជាច្រើនហើរជុំវិញបន្ទប់។

នោះហើយជាកន្លែងដែលអ្នកប្រវត្តិសាស្រ្តនាពេលអនាគតនឹងកត់ត្រានៅថ្ងៃណាមួយ—\emph{ប្រសិនបើ} គម្រោងទាំងមូលពិតជាមានតម្លៃចំពោះអ្វីក៏ដោយ—ការសិក្សាវិទ្យាសាស្ត្រអំពីវេទមន្តបានចាប់ផ្តើមហើយ ជាមួយនឹងសិស្សសាលា Hogwarts ឆ្នាំដំបូងវ័យក្មេងពីរនាក់។

Harry James Potter-Evans-Verres អ្នកទ្រឹស្តី។

និង Hermione Jean Granger អ្នកពិសោធន៍ និងសាកល្បងប្រធានបទ។

Harry ធ្វើបានល្អជាងនៅក្នុងថ្នាក់ឥឡូវនេះ យ៉ាងហោចណាស់ថ្នាក់ដែលគាត់ចាត់ទុកថាគួរឱ្យចាប់អារម្មណ៍។ គាត់​បាន​អាន​សៀវភៅ​ច្រើន​ជាង ហើយ​ក៏​មិន​មែន​សៀវភៅ​សម្រាប់​ក្មេង​អាយុ ១១ ឆ្នាំ​ដែរ។ គាត់​បាន​អនុវត្ត​ការ​ផ្លាស់​ប្តូរ​រូប​ភាព​ម្តង​ហើយ​ម្តង​ទៀត​ក្នុង​អំឡុង​ពេល​មួយ​នៃ​ម៉ោង​បន្ថែម​របស់​គាត់​ជា​រៀង​រាល់​ថ្ងៃ ដោយ​យក​ម៉ោង​ផ្សេង​ទៀត​សម្រាប់​ការ​ចាប់​ផ្តើ​ម Occlumency ។ គាត់កំពុងរៀនថ្នាក់ដ៏មានតម្លៃ \emph{យ៉ាងពិតប្រាកដ} មិនមែនគ្រាន់តែបង្វែរកិច្ចការផ្ទះរបស់គាត់ជារៀងរាល់ថ្ងៃប៉ុណ្ណោះទេ ប៉ុន្តែគាត់ប្រើពេលទំនេររបស់គាត់ដើម្បីរៀនច្រើនជាងតម្រូវការ អានសៀវភៅផ្សេងទៀតលើសពីសៀវភៅសិក្សាដែលបានផ្តល់ឱ្យ រកមើលដើម្បីគ្រប់គ្រងមុខវិជ្ជា ហើយមិនមែន គ្រាន់​តែ​ទន្ទេញ​ចាំ​ចម្លើយ​សាកល្បង​មួយ​ចំនួន​ដើម្បី excel ។ អ្នកមិនបានឃើញច្រើននៅខាងក្រៅ Ravenclaw ទេ។ ហើយឥឡូវនេះសូម្បីតែ\emph{នៅក្នុង} Ravenclaw ដៃគូប្រកួតប្រជែងតែមួយគត់របស់គាត់គឺ Padma Patil (ដែលឪពុកម្តាយមកពីវប្បធម៌មិននិយាយភាសាអង់គ្លេសហើយដូច្នេះបានចិញ្ចឹមនាងជាមួយនឹងក្រមសីលធម៌ការងារជាក់ស្តែង) Anthony Goldstein (ចេញពីភាពតូចមួយជាក់លាក់។ ក្រុមជនជាតិភាគតិចដែលបានឈ្នះ 25% នៃរង្វាន់ណូបែល) ហើយជាការពិតណាស់ ការដើរហួសពីមនុស្សគ្រប់រូបដូចជា Titan ដើរកាត់កូនឆ្កែ Hermione Granger ។

ដើម្បីដំណើរការការពិសោធន៍ពិសេសនេះ អ្នកត្រូវការប្រធានបទសាកល្បង ដើម្បីរៀនអក្ខរាវិរុទ្ធថ្មីចំនួន 16 ដោយខ្លួនឯង ដោយគ្មានជំនួយ ឬការកែតម្រូវ។ នោះមានន័យថាមុខវិជ្ជាធ្វើតេស្តគឺ Hermione ។ រយៈពេល។

វាគួរតែត្រូវបានលើកឡើងនៅចំណុចនេះថាសត្វប្រចៀវដែលហើរជុំវិញបន្ទប់មានតម្លៃ \emph{មិនមែន} ភ្លឺ។

Harry មានបញ្ហាក្នុងការទទួលយកផលប៉ះពាល់នៃរឿងនេះ។

“\emph{Oogely boogely!}” Hermione និយាយម្តងទៀត។

ជា​ថ្មី​ម្តង​ទៀត នៅ​ចុង​ដំបង​របស់ Hermione ប្រចៀវ​មួយ​បាន​លេច​ឡើង​ភ្លាមៗ។ មួយភ្លែត ខ្យល់ទទេ។ បន្តិចទៀតនេះ bat ។ ស្លាប​របស់​វា​ហាក់​ដូច​ជា​កំពុង​ធ្វើ​ចលនា​ភ្លាម​ៗ​នៅ​ពេល​ដែល​វា​លេច​មក។

ហើយវា \emph{នៅតែមិនភ្លឺ}។

"តើខ្ញុំអាចឈប់ឥឡូវនេះបានទេ?" Hermione បាននិយាយ។

"តើអ្នកប្រាកដទេ" Harry បាននិយាយតាមរយៈអ្វីដែលហាក់ដូចជាស្ទះនៅក្នុងបំពង់ករបស់គាត់ "នោះប្រហែលជាការអនុវត្តបន្តិចទៀតអ្នកមិនអាចធ្វើឱ្យវាភ្លឺបានទេ?" គាត់កំពុងបំពាននីតិវិធីពិសោធន៍ដែលគាត់បានសរសេរទុកមុន ដែលជាអំពើបាប ហើយគាត់កំពុងបំពានវាព្រោះគាត់មិនចូលចិត្តលទ្ធផលដែលគាត់ទទួលបាន ដែលជាអំពើបាប\emph{ជីវិតរមែងស្លាប់} អ្នកអាចចូលទៅ វិទ្យាសាស្ត្រឋាននរកសម្រាប់រឿងនោះ ប៉ុន្តែវាហាក់បីដូចជាមិនមានបញ្ហាអ្វីនោះទេ។

"លើកនេះឯងផ្លាស់ប្តូរអ្វី?" Hermione បាននិយាយទាំងនឿយហត់បន្តិច។

“រយៈពេលនៃសំឡេង\emph{oo, eh} និង\emph{ee}។ វាត្រូវបានសន្មត់ថាជា 3 ទៅ 2 ទៅ 2 មិនមែន 3 ទៅ 1 ដល់ 1 ទេ”។

“\emph{Oogely boogely!}” Hermione បាននិយាយ។

សត្វប្រចៀវបង្កើតបានដោយស្លាបតែមួយ ហើយបានបង្វិលយ៉ាងទ្រុឌទ្រោមទៅកម្រាលឥដ្ឋ វិលជុំវិញជារង្វង់នៅលើថ្មពណ៌ប្រផេះ។

«ឥឡូវ​នេះ​វា​ពិត​ជា​អ្វី?» Hermione បាននិយាយ។

“៣ ទៅ ២ ដល់ ១”

“\emph{ ហួសចិត្ត!}”

លើក​នេះ ប្រចៀវ​មិន​មាន​ស្លាប​ទាល់​តែ​សោះ ហើយ​ធ្លាក់​ចុះ​ដូច​កណ្ដុរ​ងាប់។

“៣ ដល់ ១ ដល់ ២”

ហើយមើល សត្វប្រចៀវបានបង្កើតជារូបរាង ហើយវាបានហោះឡើងភ្លាមៗ ឆ្ពោះទៅរកពិដាន មានសុខភាពល្អ និងភ្លឺថ្លាពណ៌បៃតង។

Hermione ងក់ក្បាលដោយពេញចិត្ត។ “មិនអីទេ តើមានអ្វីបន្ទាប់?”

មានការផ្អាកយូរ។

“\emph{ធ្ងន់ធ្ងរ?} អ្នក \emph{ធ្ងន់ធ្ងរ} ត្រូវនិយាយថា \emph{Oogely boogely} ជាមួយនឹងរយៈពេលនៃ \emph{oo, eh}, និង\emph{ee} ស្តាប់ទៅមានសមាមាត្រ ពី 3 ទៅ 1 ទៅ 2 ឬសត្វប្រចៀវមិនភ្លឺ? \emph{ហេតុអ្វី? ហេតុអ្វី? ព្រោះ​តែ​សេចក្ដី​ស្រឡាញ់​របស់​អ្វី​ៗ​ដែល​ពិសិដ្ឋ ហេតុអ្វី?}»

“ហេតុអ្វីមិន?”

“\emph{AAAAAAAAAARRRRRRGHHHH!}”

\emph{ធូ។ ធូ។ ធូ។ }

Harry បានគិតអំពីធម្មជាតិនៃមន្តអាគមអស់មួយរយៈ ហើយបន្ទាប់មកបានរចនាការពិសោធន៍ជាបន្តបន្ទាប់ដោយផ្អែកលើការសន្និដ្ឋានដែលស្ទើរតែគ្រប់អ្នកជំនួយការដែលជឿលើវេទមន្តគឺខុស។

អ្នកមិនអាច \emph{ពិតជា} ចាំបាច់ត្រូវនិយាយថា 'Wingardium Leviosa' តាមរបៀបត្រឹមត្រូវ ដើម្បីផ្លាស់ប្តូរអ្វីមួយ ពីព្រោះ មក 'Wingardium Leviosa'? សាកលលោកនឹងពិនិត្យមើលថាអ្នកបាននិយាយ 'Wingardium Leviosa' តាមវិធីត្រឹមត្រូវ ហើយបើមិនដូច្នេះទេវានឹងមិនធ្វើឱ្យ quill អណ្តែត?

ទេ ច្បាស់ណាស់ អត់ទេ នៅពេលដែលអ្នកគិតអំពីវាយ៉ាងម៉ត់ចត់។ មាននរណាម្នាក់ ប្រហែលជាក្មេងថ្នាក់មត្តេយ្យពិតប្រាកដ ប៉ុន្តែទោះជាយ៉ាងណាក៏ដោយ អ្នកប្រើប្រាស់វេទមន្តនិយាយភាសាអង់គ្លេសខ្លះ ដែលគិតថា 'Wingardium Leviosa' ស្តាប់ទៅដូចជាឡូយ និងអណ្តែតជាដើមនោះ បាននិយាយពាក្យទាំងនោះនៅពេលសរសេរអក្ខរាវិរុទ្ធជាលើកដំបូង។ ហើយបន្ទាប់មកបានប្រាប់អ្នកផ្សេងទៀតថាវាចាំបាច់។

ប៉ុន្តែ (Harry បានវែកញែក) វាមិនមែន \emph{have} បែបនោះទេ វាមិនត្រូវបានសាងសង់នៅក្នុងសកលលោកទេ វាត្រូវបានសាងសង់ក្នុង \emph{you}។

មានរឿងចាស់មួយបានកន្លងផុតទៅក្នុងចំណោមអ្នកវិទ្យាសាស្ត្រ រឿងនិទានប្រុងប្រយ័ត្ន រឿង Blondlot និង N-Rays ។

មិនយូរប៉ុន្មានបន្ទាប់ពីការរកឃើញនៃកាំរស្មីអ៊ិច រូបវិទូជនជាតិបារាំងដ៏ល្បីឈ្មោះ Prosper-René Blondlot ដែលធ្លាប់វាស់ល្បឿនរលកវិទ្យុមុនគេ ហើយបង្ហាញថាពួកគេបានសាយភាយក្នុងល្បឿនពន្លឺបានប្រកាសពីការរកឃើញបាតុភូតថ្មីដ៏អស្ចារ្យមួយ។ , N-Rays ដែលនឹងធ្វើឱ្យអេក្រង់ភ្លឺ។ អ្នកត្រូវតែមើលទៅពិបាកមើលវា ប៉ុន្តែវានៅទីនោះ។ N-Rays មានលក្ខណៈសម្បត្តិគួរឱ្យចាប់អារម្មណ៍គ្រប់ប្រភេទ។ ពួកវាត្រូវបានពត់ដោយអាលុយមីញ៉ូម ហើយអាចត្រូវបានផ្តោតដោយ prism អាលុយមីញ៉ូម ចូលទៅក្នុងការវាយលុកខ្សែស្រឡាយដែលត្រូវបានព្យាបាលដោយ cadmium sulphide ដែលបន្ទាប់មកនឹងបញ្ចេញពន្លឺតិចៗនៅក្នុងទីងងឹត…

មិនយូរប៉ុន្មានអ្នកវិទ្យាសាស្ត្ររាប់សិបនាក់ផ្សេងទៀតបានបញ្ជាក់ពីលទ្ធផលរបស់ Blondlot ជាពិសេសនៅប្រទេសបារាំង។

ប៉ុន្តែ​នៅ​មាន​អ្នក​វិទ្យាសាស្ត្រ​ផ្សេង​ទៀត នៅ​ក្នុង​ប្រទេស​អង់គ្លេស និង​អាល្លឺម៉ង់ ដែល​បាន​និយាយ​ថា​ពួកគេ​មិន​ប្រាកដ​ថា​ពួកគេ​អាច​ឃើញ​ពន្លឺ​ស្រទន់​នោះ​ទេ។

Blondlot បាននិយាយថា ពួកគេប្រហែលជាដំឡើងម៉ាស៊ីនខុស។

ថ្ងៃមួយ Blondlot បានផ្តល់បទបង្ហាញអំពី N-Rays ។ ភ្លើងបានរលត់ហើយ ជំនួយការរបស់គាត់បានបិទពន្លឺ និងងងឹត នៅពេលដែល Blondlot ធ្វើឧបាយកលរបស់គាត់។

វា​ជា​ការ​ធ្វើ​បាតុកម្ម​ធម្មតា លទ្ធផល​ទាំង​អស់​នឹង​ទៅ​តាម​ការ​រំពឹង​ទុក។

ទោះបីជាអ្នកវិទ្យាសាស្ត្រជនជាតិអាមេរិកម្នាក់ឈ្មោះ Robert Wood បានលួចដោយស្ងៀមស្ងាត់នូវ prism អាលុយមីញ៉ូមពីកណ្តាលនៃយន្តការ Blondlot ក៏ដោយ។

ហើយនោះគឺជាចុងបញ្ចប់នៃ N-Rays ។

\emph{Reality,} Philip K. Dick ធ្លាប់បាននិយាយថា \emph{គឺជាអ្វីដែលអ្នកឈប់ជឿលើវា វាមិនទៅណាទេ}។

អំពើបាបរបស់ Blondlot បានបង្ហាញឱ្យឃើញច្បាស់នៅក្នុងការរំលឹកឡើងវិញ។ គាត់មិនគួរប្រាប់ជំនួយការរបស់គាត់ពីអ្វីដែលគាត់កំពុងធ្វើនោះទេ។ Blondlot គួរតែប្រាកដថាជំនួយការ \emph{មិនបាន} ដឹងពីអ្វីដែលកំពុងព្យាយាម ឬនៅពេលដែលវាត្រូវបានសាកល្បង មុនពេលសួរគាត់ឱ្យពណ៌នាអំពីពន្លឺរបស់អេក្រង់។ វាប្រហែលជាសាមញ្ញណាស់។

សព្វ​ថ្ងៃ​វា​ត្រូវ​បាន​គេ​ហៅ​ថា "ពិការ​ភ្នែក" ហើយ​វា​ជា​រឿង​មួយ​ដែល​អ្នក​វិទ្យាសាស្ត្រ​សម័យ​ទំនើប​បាន​យល់​ព្រម។ ប្រសិនបើអ្នកកំពុងធ្វើការពិសោធចិត្តវិទ្យា ដើម្បីមើលថាតើមនុស្សខឹងនៅពេលដែលគេវាយក្បាលដោយប្រម៉ោយក្រហមជាងឈើឆ្កាងពណ៌បៃតង នោះអ្នកមិនត្រូវមើលមុខវិជ្ជាដោយខ្លួនឯង ហើយសម្រេចចិត្តថាតើពួកគេខឹងប៉ុណ្ណានោះទេ។ អ្នក​នឹង​ថត​រូប​ពួក​គេ​បន្ទាប់​ពី​ពួក​គេ​ត្រូវ​បាន​ប៉ះ​ទង្គិច ហើយ​ផ្ញើ​រូបថត​ទៅ​ក្រុម​អ្នក​វាយ​តម្លៃ ដែល​នឹង​វាយ​តម្លៃ​លើ​មាត្រដ្ឋាន​ពី 1 ដល់ 10 ពី​កម្រិត​ដែល​មនុស្ស​ម្នាក់ៗ​មើល​ទៅ​ខឹង​សម្បា ច្បាស់​ណាស់ \emph{ដោយ​គ្មាន} ដឹងពីពណ៌អ្វីនៃ truncheon ដែលពួកគេត្រូវបានគេវាយជាមួយ។ ពិត​ហើយ​គ្មាន​ហេតុផល​ល្អ​ក្នុង​ការ​ប្រាប់​អ្នក​វាយតម្លៃ​ពី​អ្វី​ដែល​ការ​ពិសោធ​នោះ​និយាយ​ទាល់តែ​សោះ។ អ្នក \emph{ប្រាកដណាស់} នឹងមិនប្រាប់ប្រធានបទពិសោធន៍ថា \emph{អ្នកបានគិត} ពួកគេគួរតែខឹងជាង នៅពេលដែលត្រូវបុកដោយកំណាត់ក្រហម។ អ្នកគ្រាន់តែផ្តល់ឱ្យពួកគេនូវ 20 ផោន, ទាក់ទាញពួកគេចូលទៅក្នុងបន្ទប់សាកល្បង, វាយពួកគេជាមួយនឹង truncheon, ពណ៌ដែលបានកំណត់ដោយចៃដន្យ ហើយថតរូប។ តាមពិតទៅ ការវាយតប់ និងថតរូបភាពនឹងធ្វើឡើងដោយជំនួយការម្នាក់ដែលមិនត្រូវបានគេប្រាប់អំពីសម្មតិកម្ម ដូច្នេះគាត់មិនអាចមើលទៅរំពឹង វាយខ្លាំង ឬថតរូបភាពក្នុងពេលត្រឹមត្រូវនោះទេ។

Blondlot បានបំផ្លាញកេរ្តិ៍ឈ្មោះរបស់គាត់ជាមួយនឹងប្រភេទនៃកំហុសដែលនឹងទទួលបានចំណាត់ថ្នាក់បរាជ័យ និងប្រហែលជាសើចចំអកពី T.A. នៅក្នុងវគ្គសិក្សាថ្នាក់បរិញ្ញាបត្រឆ្នាំទី 1 លើការរចនាពិសោធន៍… ក្នុងឆ្នាំ 1991 ។

ប៉ុន្តែនេះមានរយៈពេលយូរជាងនេះបន្តិចក្នុងឆ្នាំ 1904 ហើយដូច្នេះវាបានចំណាយពេលច្រើនខែមុនពេលលោក Robert Wood បានបង្កើតសម្មតិកម្មជំនួសជាក់ស្តែង និងរកវិធីធ្វើតេស្ត ហើយអ្នកវិទ្យាសាស្ត្ររាប់សិបនាក់ផ្សេងទៀតត្រូវបានបូមចូល។

ជាងពីរសតវត្សបន្ទាប់ពីវិទ្យាសាស្ត្របានចាប់ផ្តើម។ ចុង​ក្រោយ​ក្នុង​ប្រវត្តិសាស្ត្រ​វិទ្យាសាស្ត្រ វា​នៅ​តែ​មិន​មាន​ភាព​ច្បាស់​លាស់។

ដែលធ្វើឱ្យវាមានតម្លៃ\emph{ទាំងស្រុង} ដែលអាចសន្និដ្ឋានបានថានៅក្នុងពិភពវេទមន្តដ៏តូច ដែលជាកន្លែងដែលវិទ្យាសាស្រ្តហាក់ដូចជាមិនសូវស្គាល់ទាល់តែសោះ គ្មាននរណាម្នាក់ធ្លាប់សាកល្បងដំបូង សាមញ្ញបំផុត និងជាក់ស្តែងបំផុតដែលអ្នកវិទ្យាសាស្ត្រសម័យទំនើបគិតនោះទេ។ ដើម្បីពិនិត្យ។

សៀវភៅទាំងនោះពោរពេញទៅដោយការណែនាំដ៏ស្មុគស្មាញសម្រាប់កិច្ចការទាំងអស់ដែលអ្នកត្រូវធ្វើ \emph{ពិតជាត្រឹមត្រូវ} ដើម្បីដេញអក្ខរាវិរុទ្ធ។ ហើយ Harry បានសន្មត់ថា ដំណើរការនៃការគោរពតាមការណែនាំទាំងនោះ ដើម្បីពិនិត្យមើលថាអ្នកបានធ្វើតាមពួកគេត្រឹមត្រូវ ប្រហែលជា\emph{បានធ្វើអ្វីមួយ។ វា\emph{បង្ខំអ្នកឱ្យផ្តោតអារម្មណ៍លើអក្ខរាវិរុទ្ធ}។ ត្រូវបានគេប្រាប់ឱ្យគ្រាន់តែគ្រវីដៃរបស់អ្នក ហើយប្រាថ្នាប្រហែលជា \emph{នឹង} មិនដំណើរការផងដែរ។ ហើយនៅពេលដែលអ្នកជឿថាអក្ខរាវិរុទ្ធគួរតែដំណើរការតាមវិធីជាក់លាក់មួយ នៅពេលដែលអ្នកបានអនុវត្តវាតាមវិធីនោះ អ្នកប្រហែលជាមិនអាចបញ្ចុះបញ្ចូលខ្លួនអ្នកថាវាអាចដំណើរការតាមវិធីណាក៏ដោយ\emph{other}…

…ប្រសិនបើអ្នកបានធ្វើរឿងសាមញ្ញ ប៉ុន្តែខុស ហើយព្យាយាមសាកល្បងទម្រង់ជំនួស \emph{ខ្លួនអ្នក}។

ប៉ុន្តែចុះយ៉ាងណាបើអ្នក \emph{មិនដឹងថា} អក្ខរាវិរុទ្ធដើមមានលក្ខណៈដូចម្តេច?

ចុះបើអ្នកផ្តល់ឱ្យ Hermione នូវបញ្ជីអក្ខរាវិរុទ្ធដែលនាងមិនទាន់បានសិក្សា យកចេញពីសៀវភៅអក្ខរាវិរុទ្ធដ៏ឆ្កួតលីលានៅក្នុងបណ្ណាល័យ Hogwarts ហើយអក្ខរាវិរុទ្ធមួយចំនួនមានការណែនាំត្រឹមត្រូវ និងដើម ខណៈពេលដែលអ្នកផ្សេងទៀតមានកាយវិការផ្លាស់ប្តូរ មួយបានផ្លាស់ប្តូរ។ ពាក្យ? ចុះ​បើ​អ្នក​រក្សា​ការ​ណែនាំ​ដដែល ប៉ុន្តែ​ប្រាប់​នាង​ថា អក្ខរាវិរុទ្ធ​ដែល​ត្រូវ​បង្កើត​ដង្កូវ​ក្រហម​ត្រូវ​បង្កើត​ដង្កូវ​ខៀវ​ជំនួស​វិញ?

ជាការប្រសើរណាស់, ក្នុងករណីនោះ, វាបានប្រែក្លាយ ...

…Harry មានបញ្ហាក្នុងការជឿជាក់លើលទ្ធផលរបស់គាត់នៅទីនេះ…

… ប្រសិនបើអ្នកបានប្រាប់ Hermione ឱ្យនិយាយថា "Oogely boogely" ជាមួយនឹងថិរវេលាស្រៈក្នុងសមាមាត្រនៃ 3 ទៅ 1 ទៅ 1 ជំនួសឱ្យសមាមាត្រត្រឹមត្រូវនៃ 3 ទៅ 1 ទៅ 2 អ្នកនៅតែទទួលបានប្រចៀវប៉ុន្តែវានឹងមិនភ្លឺទៀតទេ .

មិនមែនជំនឿនោះគឺ\emph{មិនពាក់ព័ន្ធ}នៅទីនេះទេ។ មិនថា \emph{ only} ពាក្យ និងចលនារបស់ wand សំខាន់ទេ។

ប្រសិនបើអ្នកផ្តល់ឱ្យ Hermione នូវព័ត៌មានមិនត្រឹមត្រូវទាំងស្រុងអំពីអ្វីដែលអក្ខរាវិរុទ្ធត្រូវបានគេសន្មត់ថាធ្វើ វានឹងឈប់ដំណើរការ។

ប្រសិនបើអ្នកមិនប្រាប់នាងពីអ្វីដែលអក្ខរាវិរុទ្ធត្រូវធ្វើទេ វានឹងឈប់ដំណើរការ។

ប្រសិនបើនាងដឹងក្នុងន័យមិនច្បាស់លាស់អំពីអ្វីដែលអក្ខរាវិរុទ្ធគួរធ្វើ ឬនាងគ្រាន់តែខុសមួយផ្នែក នោះអក្ខរាវិរុទ្ធនឹងដំណើរការដូចដែលបានពិពណ៌នាដំបូងនៅក្នុងសៀវភៅ មិនមែនជាវិធីដែលនាងត្រូវបានគេប្រាប់នោះទេ។

នៅពេលនេះ Harry កំពុងតែវាយក្បាលរបស់គាត់ទៅនឹងជញ្ជាំងឥដ្ឋ។ មិនពិបាកទេ។ គាត់មិនចង់ធ្វើឱ្យខូចខួរក្បាលដ៏មានតម្លៃរបស់គាត់ទេ។ ប៉ុន្តែ​ប្រសិន​បើ​គាត់​មិន​មាន​ផ្លូវ​ចេញ​សម្រាប់​ការ​ខក​ចិត្ត​ទេ គាត់​នឹង​ឆេះ​ដោយ​ឯកឯង។

\emph{ធូ។ ធូ។ ធូ។ }

វាហាក់ដូចជាសកលលោកពិតជា \emph{did} ចង់ឱ្យអ្នកនិយាយថា 'Wingardium Leviosa' ហើយវាចង់ឱ្យអ្នកនិយាយវាតាមរបៀបជាក់លាក់មួយ ហើយវាមិនខ្វល់ថា \emph{អ្នក} គិតថាការបញ្ចេញសំឡេងគួរតែជាអ្វីនោះទេ។ ច្រើនជាងវាខ្វល់ពីអារម្មណ៍របស់អ្នកចំពោះទំនាញផែនដី។

\emph{\shout{ហេតុអីក៏ថាបាន?}}

ផ្នែកដ៏អាក្រក់បំផុតរបស់វាគឺការមើលមុខរបស់ Hermione គួរឱ្យអស់សំណើច។

Hermione មាន \emph{មិនអីទេ} ជាមួយនឹងការអង្គុយជុំវិញដោយគោរពតាមការណែនាំរបស់ Harry ដោយមិនត្រូវបានគេប្រាប់ពីមូលហេតុ។

ដូច្នេះ Harry បានពន្យល់នាងអំពីអ្វីដែលពួកគេកំពុងធ្វើតេស្ត។

Harry បានពន្យល់ពីមូលហេតុដែលពួកគេកំពុងសាកល្បងវា។

Harry បានពន្យល់ពីមូលហេតុដែលប្រហែលជាគ្មានអ្នកជំនួយការបានសាកល្បងវាមុនពេលពួកគេ។

Harry បានពន្យល់ថាគាត់ពិតជាមានទំនុកចិត្តដោយយុត្តិធម៌ចំពោះការទស្សន៍ទាយរបស់គាត់។

ដោយសារតែ Harry បាននិយាយថា មាន\emph{គ្មានផ្លូវ} ដែលសកលលោកពិតជាចង់ឱ្យអ្នកនិយាយថា 'Wingardium Leviosa'។

Hermione បានចង្អុលបង្ហាញថានេះមិនមែនជាអ្វីដែលសៀវភៅរបស់នាងបាននិយាយទេ។ Hermione បានសួរថាតើ Harry ពិតជាគិតថាគាត់ឆ្លាតជាងនៅអាយុ 11 ឆ្នាំ និងទើបតែជាងមួយខែនៅក្នុងការអប់រំ Hogwarts របស់គាត់ ជាងអ្នកជំនួយការផ្សេងទៀតនៅក្នុងពិភពលោកដែលមិនយល់ស្របជាមួយគាត់។

Harry បាននិយាយពាក្យពិតដូចខាងក្រោម៖

"ពិតណាស់។"

ឥឡូវនេះ Harry កំពុងសម្លឹងមើលឥដ្ឋក្រហមដោយផ្ទាល់នៅពីមុខគាត់ ហើយសញ្ជឹងគិតអំពីថាតើគាត់នឹងត្រូវវាយក្បាលគាត់ខ្លាំងប៉ុណ្ណា ដើម្បីផ្តល់ការប៉ះទង្គិចដល់ខ្លួនគាត់ដែលនឹងរំខានដល់ការបង្កើតការចងចាំរយៈពេលវែង និងរារាំងគាត់ពីការចងចាំរឿងនេះនៅពេលក្រោយ។ Hermione មិនបានសើចទេ ប៉ុន្តែគាត់អាចមានអារម្មណ៍ថា \emph{ចេតនាចង់សើច} របស់នាងដែលបញ្ចេញពីក្រោយគាត់ដូចជាសម្ពាធដ៏គួរឱ្យភ័យខ្លាចលើស្បែករបស់គាត់ ដូចជាដឹងថាអ្នកកំពុងត្រូវបានឃាតករតាមប្រមាញ់ត្រឹមតែ\emph{ កាន់តែអាក្រក់ }

Harry បាននិយាយថា "និយាយវា" ។

សំឡេង​សប្បុរស​របស់ Hermione Granger បាន​និយាយ​ថា “ខ្ញុំ​មិន​មែន​\emph{ទៅ}។ "វាហាក់ដូចជាមិនល្អទេ។"

Harry បាននិយាយថា "គ្រាន់តែយកវាទៅជាមួយ" ។

“មិនអីទេ! ដូច្នេះអ្នកបានផ្តល់ឱ្យខ្ញុំនូវ\emph{ការបង្រៀនរយៈពេលវែង} នេះឱ្យខ្ញុំអំពីរបៀបដែលវាលំបាកក្នុងការធ្វើវិទ្យាសាស្ត្រមូលដ្ឋាន និងរបៀបដែលយើងប្រហែលជាត្រូវបន្តដោះស្រាយបញ្ហានេះសម្រាប់\emph{សាមសិបប្រាំឆ្នាំ} ហើយបន្ទាប់មកអ្នកបានទៅ ហើយរំពឹងថាយើង ដើម្បីធ្វើឱ្យការរកឃើញដ៏អស្ចារ្យបំផុតនៅក្នុងប្រវត្តិសាស្រ្តនៃមន្តអាគមនៅក្នុងម៉ោងដំបូងដែលយើងកំពុងធ្វើការជាមួយគ្នា។ អ្នកមិនគ្រាន់តែសង្ឃឹមទេ អ្នកពិតជារំពឹងវា។ អ្នកឆ្កួត។”

“អរគុណ។ ឥឡូវនេះ —”

"ខ្ញុំបានអានសៀវភៅទាំងអស់ដែលអ្នកបានឱ្យខ្ញុំហើយខ្ញុំនៅតែមិនដឹងថាត្រូវហៅវាថាអ្វី។ ទំនុកចិត្តហួសហេតុ? ការធ្វើផែនការខុស? ឥទ្ធិពល Super duper Lake Wobegon? ពួកគេនឹងត្រូវដាក់ឈ្មោះតាមអ្នក។ Harry Bias ។

“ទាំងអស់ \emph{right}!”

“ប៉ុន្តែវា \emph{គឺ} គួរឱ្យស្រលាញ់។ វា​ជា​រឿង​ក្មេង​ប្រុស​ដែល​ត្រូវ​ធ្វើ»។

“\emph{Drop dead.}”

"អូ អ្នកនិយាយរឿងមនោសញ្ចេតនាបំផុត"

\emph{ធូ។ ធូ។ ធូ។ }

"អញ្ចឹងតើមានអ្វីបន្ទាប់?" Hermione បាននិយាយ។

Harry សម្រាកក្បាលរបស់គាត់ទល់នឹងឥដ្ឋ។ ថ្ងាស​របស់​គាត់​ចាប់​ផ្ដើម​ឈឺ​ត្រង់​កន្លែង​ដែល​គាត់​ត្រូវ​គេ​វាយ។ “គ្មានអ្វីទេ។ ខ្ញុំ​ត្រូវ​តែ​ត្រឡប់​ទៅ​វិញ ហើយ​រចនា​ការ​ពិសោធ​ផ្សេង​ៗ»។

ក្នុងរយៈពេលមួយខែចុងក្រោយនេះ លោក Harry បានធ្វើការដោយប្រុងប្រយ័ត្នជាមុន វគ្គនៃការពិសោធន៍សម្រាប់ពួកគេដែលនឹងមានរយៈពេលរហូតដល់ខែធ្នូ។

វានឹងក្លាយជាសំណុំនៃការពិសោធន៍ \emph{ដ៏អស្ចារ្យ} ប្រសិនបើ \emph{ការធ្វើតេស្តដំបូងបំផុត} មិនបានក្លែងបន្លំមូលដ្ឋាន។

Harry មិន​អាច​ជឿ​ថា​គាត់​ជា​មនុស្ស​ល្ងង់​នេះ​ទេ។

Harry បាននិយាយថា "អនុញ្ញាតឱ្យខ្ញុំកែខ្លួនឯង" ។ “ខ្ញុំត្រូវរចនាការពិសោធន៍ថ្មី \emph{one}។ ខ្ញុំ​នឹង​ប្រាប់​អ្នក​នៅ​ពេល​ដែល​យើង​បាន​វា ហើយ​យើង​នឹង​ធ្វើ​វា ហើយ​បន្ទាប់​មក​ខ្ញុំ​នឹង​រចនា​កម្មវិធី​បន្ទាប់។ តើវាស្តាប់ទៅយ៉ាងម៉េច?

“វាស្តាប់ទៅដូចជា\emph{នរណាម្នាក់}បានខ្ជះខ្ជាយ\emph{ការខិតខំប្រឹងប្រែងជាច្រើន}។”

\emph{Thud.} អូ។ គាត់​បាន​ធ្វើ​វា​ពិបាក​ជាង​ការ​គ្រោង​ទុក​បន្តិច។

Hermione បាននិយាយថា "ដូច្នេះ" ។ នាង​បាន​ផ្អៀង​ទៅ​លើ​កៅអី ហើយ​មុខ​មាត់​ក៏​លេច​មក​លើ​ផ្ទៃ​មុខ​របស់​នាង​វិញ។ "តើយើងបានរកឃើញអ្វីនៅថ្ងៃនេះ?"

Harry បាននិយាយតាមរយៈធ្មេញគ្រើមថា "ខ្ញុំបានរកឃើញថា នៅពេលនិយាយអំពីការស្រាវជ្រាវជាមូលដ្ឋានពិតប្រាកដលើបញ្ហាដែលច្របូកច្របល់ពិតប្រាកដដែលអ្នកមិនមានតម្រុយពីអ្វីដែលកំពុងកើតឡើងនោះ សៀវភៅរបស់ខ្ញុំអំពីវិធីសាស្រ្តវិទ្យាសាស្ត្រមិនមានតម្លៃទេ"

“ភាសា លោក ~ Potter! យើង​ខ្លះ​ជា​ក្មេង​ស្រី​ស្លូត​ត្រង់!»។

“មិនអីទេ។ ប៉ុន្តែប្រសិនបើសៀវភៅរបស់ខ្ញុំមានតម្លៃ \emph{carp} នោះជាប្រភេទត្រីដែលមិនមានអ្វីអាក្រក់ទេ ពួកគេនឹងផ្តល់ដំបូន្មានសំខាន់ៗដូចខាងក្រោមនេះដល់ខ្ញុំ៖ នៅពេលដែលមានបញ្ហាច្របូកច្របល់ ហើយអ្នកទើបតែចាប់ផ្តើម ហើយអ្នកមាន សម្មតិកម្មមិនពិត សូមទៅសាកល្បង។ ស្វែងរកវិធីសាមញ្ញ និងងាយស្រួលមួយចំនួនដើម្បីធ្វើការត្រួតពិនិត្យជាមូលដ្ឋាន ហើយធ្វើវាភ្លាមៗ។ កុំបារម្ភអំពីការរៀបចំវគ្គពិសោធន៍ដ៏ឧឡារិក ដែលនឹងធ្វើឱ្យសំណើផ្តល់ជំនួយមើលទៅគួរអោយចាប់អារម្មណ៍ចំពោះទីភ្នាក់ងារផ្តល់មូលនិធិ។ គ្រាន់តែពិនិត្យមើលឱ្យបានលឿនតាមដែលអាចធ្វើទៅបានថាតើគំនិតរបស់អ្នកមិនពិតទេ មុនពេលអ្នកចាប់ផ្តើមវិនិយោគនូវកិច្ចខិតខំប្រឹងប្រែងជាច្រើននៅក្នុងពួកគេ។ តើ​វា​ស្តាប់​ទៅ​តាម​សីលធម៌​យ៉ាង​ណា?»

“អឹម… មិនអីទេ” Hermione បាននិយាយ។ "ប៉ុន្តែខ្ញុំក៏សង្ឃឹមសម្រាប់អ្វីមួយដូចជា 'សៀវភៅរបស់ Hermione មិនមានតម្លៃទេ។ ពួកគេត្រូវបានសរសេរដោយអ្នកជំនួយការចាស់ដែលមានប្រាជ្ញាដែលដឹងអំពីវេទមន្តច្រើនជាងខ្ញុំ។ ខ្ញុំ​គួរ​យក​ចិត្ត​ទុក​ដាក់​នឹង​អ្វី​ដែល​សៀវភៅ​របស់ Hermione និយាយ។ តើ​យើង​អាច​មាន​សីលធម៌​បែប​នេះ​ដែរ​ឬ​ទេ?

ថ្គាម​របស់ Harry ហាក់​ត្រូវ​បាន​ក្ដាប់​យ៉ាង​តឹង​ពេក​មិន​អាច​បញ្ចេញ​ពាក្យ​អ្វី​បាន​ឡើយ ដូច្នេះ​គាត់​បាន​ត្រឹម​ងក់​ក្បាល។

“អស្ចារ្យ!” Hermione បាននិយាយ។ “ខ្ញុំចូលចិត្តការពិសោធន៍នេះ។ យើង​បាន​រៀន​ច្រើន​ពី​វា ហើយ​វា​ចំណាយ​ពេល​តែ​មួយ​ម៉ោង​ប៉ុណ្ណោះ»។

“AAAAAAAAAAAHHHHHHHHHHHH!”

\later

នៅក្នុងគុកងងឹតនៃ Slytherin ។

ថ្នាក់រៀនដែលមិនបានប្រើត្រូវបានបំភ្លឺដោយភ្លើងពណ៌បៃតងដ៏គួរឱ្យខ្លាច ដែលភ្លឺជាងនេះ និងចេញមកពីពិភពគ្រីស្តាល់តូចមួយដែលមានភាពទាក់ទាញបណ្តោះអាសន្ន ប៉ុន្តែទោះជាយ៉ាងណាពន្លឺពណ៌បៃតងដ៏គួរឱ្យភ័យខ្លាច បញ្ចេញស្រមោលចម្លែកចេញពីតុដែលពោរពេញដោយធូលី។

ក្មេង​ប្រុស​ពីរ​នាក់​ដែល​មាន​មាឌ​ក្នុង​អាវ​ពណ៌​ប្រផេះ (គ្មាន​របាំង​មុខ) បាន​ចូល​ទៅ​ក្នុង​ភាព​ស្ងៀមស្ងាត់ ហើយ​អង្គុយ​លើ​កៅអី​ពីរ​ទល់​មុខ​តុ​ដូច​គ្នា។

វាជាការប្រជុំលើកទី ២ នៃការសមគំនិត Bayesian ។

Draco Malfoy មិនប្រាកដថាគាត់គួរតែទន្ទឹងរង់ចាំវាឬអត់នោះទេ។

Harry Potter វិនិច្ឆ័យដោយការបញ្ចេញទឹកមុខរបស់គាត់ ហាក់ដូចជាមិនមានការសង្ស័យលើអារម្មណ៍សមរម្យនោះទេ។

Harry Potter មើលទៅដូចជាគាត់ត្រៀមខ្លួនជាស្រេចដើម្បីសម្លាប់នរណាម្នាក់។

Harry Potter បាននិយាយថា "Hermione Granger" ដូចជា Draco កំពុងបើកមាត់របស់គាត់។ “\emph{កុំសួរ}។

\emph{គាត់មិនអាចទៅណាត់ជួបទៀតបានទេ?} គិតថា Draco ប៉ុន្តែវាមិនសមហេតុផលទេ។

Draco បាននិយាយថា “ខ្ញុំសុំទោស ប៉ុន្តែខ្ញុំត្រូវតែសួររឿងនេះ តើអ្នកបាន \emph{ពិតជា} បញ្ជាឱ្យក្មេងស្រីឈាមភក់នូវថង់ស្បែកដ៏ថ្លៃសម្រាប់ខួបកំណើតរបស់នាងមែនទេ?”

“បាទ ខ្ញុំបានធ្វើ។ អ្នក​បាន​ដឹង​ហើយ​ថា​ហេតុ​អ្វី​បាន​ជា​ប្រាកដ​ណាស់»។

Draco លូកដៃឡើងលើសក់របស់គាត់ដោយអន្ទះសារ ក្រវ៉ាត់ករបស់គាត់ដុសលើខ្នងដៃរបស់គាត់។ គាត់ \emph{មិន​បាន} ច្បាស់​ថា​ហេតុ​អ្វី​បាន​ជា, ប៉ុន្តែ​ឥឡូវ​នេះ​គាត់​មិន​អាច​និយាយ​ដូច្នេះ​។ ហើយ Slytherin \emph{ដឹង} គាត់កំពុងទាក់ទងជាមួយ Harry Potter គាត់បានធ្វើឱ្យវាច្បាស់គ្រប់គ្រាន់នៅក្នុងថ្នាក់ការពារ។ Draco បាននិយាយថា "Harry" មនុស្សដឹងថាខ្ញុំជាមិត្តនឹងអ្នក ពួកគេមិនដឹងអំពីការសមគំនិតទេ ប៉ុន្តែពួកគេដឹងថាយើងជាមិត្ត ហើយវាធ្វើឱ្យ\emph{me} មើលទៅអាក្រក់នៅពេលអ្នកធ្វើ រឿង​បែប​នោះ»។

មុខរបស់ Harry Potter កាន់តែតឹង។ "អ្នកណាម្នាក់នៅក្នុង Slytherin ដែលមិនអាចយល់ពីគំនិតនៃការប្រព្រឹត្តល្អចំពោះមនុស្សដែលអ្នកពិតជាមិនចូលចិត្ត គួរតែត្រូវបានណែនាំ និងផ្តល់អាហារដល់សត្វពស់។"

Draco បាននិយាយថា "មានមនុស្សជាច្រើននៅក្នុង Slytherin ដែល\emph{ don't}" Draco បាននិយាយថា សំឡេងរបស់គាត់ធ្ងន់ធ្ងរ។ “មនុស្សភាគច្រើនល្ងង់ ហើយអ្នកត្រូវតែមើលទៅល្អនៅចំពោះមុខពួកគេ”។ Harry Potter \emph{ត្រូវតែ} យល់ថាប្រសិនបើគាត់ចង់ទៅកន្លែងណាក្នុងជីវិត។

“តើ \emph{អ្នក} ខ្វល់ពីអ្វីដែលអ្នកដទៃគិតយ៉ាងណា? តើ​អ្នក​ពិតជា​នឹង​រស់នៅ​ក្នុង​ជីវិត​របស់​អ្នក​ដោយ​ត្រូវការ​ពន្យល់​គ្រប់​យ៉ាង​ដែល​អ្នក​ធ្វើ​ចំពោះ​មនុស្ស​ល្ងង់​បំផុត​នៅ​ Slytherin ដោយ​អនុញ្ញាត​ឱ្យ \emph{ them} វិនិច្ឆ័យ \emph{you}? ខ្ញុំសុំទោស Draco ប៉ុន្តែខ្ញុំមិនបានបន្ទាបផែនការល្បិចកលរបស់ខ្ញុំដល់កម្រិតនៃអ្វីដែល Slytherins ល្ងង់បំផុតអាចយល់បាននោះទេ ព្រោះវាអាចធ្វើឱ្យអ្នកមើលទៅអាក្រក់។ សូម្បីតែមិត្តភាពរបស់អ្នកក៏មិនមានតម្លៃដែរ។ វានឹងមានតម្លៃ \emph{យកភាពសប្បាយរីករាយទាំងអស់ចេញពីជីវិត}។ ប្រាប់ខ្ញុំ \emph{អ្នក} មិនដែលគិតដូចគ្នាទេ នៅពេលដែលនរណាម្នាក់នៅក្នុង Slytherin ល្ងង់ពេកក្នុងការដកដង្ហើម ថាវាស្ថិតនៅក្រោមសេចក្តីថ្លៃថ្នូររបស់ Malfoy ដែលត្រូវមើលងាយពួកគេ។”

Draco ពិតជាមិនមានទេ។ ធ្លាប់។ ការនិយាយទៅកាន់មនុស្សល្ងង់គឺដូចជាការដកដង្ហើម អ្នកធ្វើវាដោយមិនគិតពីវា។

"Harry" Draco និយាយចុងក្រោយ។ “គ្រាន់តែធ្វើអ្វីដែលអ្នកចង់បាន ដោយមិនខ្វល់ពីរូបរាង គឺមិនឆ្លាតនោះទេ។ \emph{Dark Lord} ព្រួយបារម្ភអំពីរបៀបដែលគាត់មើលទៅ! គាត់ត្រូវបានគេភ័យខ្លាច និងស្អប់ ហើយគាត់ដឹងថា \emph{ពិតប្រាកដ} ប្រភេទនៃភាពភ័យខ្លាច និងស្អប់ដែលគាត់ចង់បង្កើត។ \emph{អ្នកគ្រប់គ្នា}ត្រូវព្រួយបារម្ភអំពីអ្វីដែលអ្នកដទៃគិត។"

តួរអង្គញញើតគ្រវីក្បាល។ “ប្រហែល។ រំលឹកខ្ញុំពេលខ្លះដើម្បីប្រាប់អ្នកអំពីអ្វីមួយដែលហៅថាការពិសោធន៍អនុលោមភាពរបស់ Asch អ្នកប្រហែលជាគិតថាវាគួរឱ្យអស់សំណើចណាស់។ សម្រាប់ពេលនេះ ខ្ញុំនឹងកត់សម្គាល់ថាវាមានគ្រោះថ្នាក់ក្នុងការព្រួយបារម្ភអំពីអ្វីដែលមនុស្សផ្សេងទៀតគិតលើ\emph{សភាវគតិ} ពីព្រោះអ្នក \emph{ពិតជាយកចិត្តទុកដាក់} មិនមែនជាបញ្ហានៃការគណនាដោយឈាមត្រជាក់នោះទេ។ សូមចាំថា ខ្ញុំត្រូវបានវាយដំ និងបៀតបៀនដោយ Slytherins ដែលមានវ័យចំណាស់អស់រយៈពេលដប់ប្រាំនាទី ហើយក្រោយមកខ្ញុំបានក្រោកឈរឡើង ហើយបានអត់ទោសដោយសប្បុរស។ ដូចជាក្មេងប្រុសល្អ និងគុណធម៌ ដែលគួរធ្វើ។ ប៉ុន្តែការគណនាឈាមត្រជាក់របស់ខ្ញុំ Draco ប្រាប់ខ្ញុំថាខ្ញុំមាន \emph{គ្មានប្រយោជន៍} សម្រាប់មនុស្សល្ងង់បំផុតនៅ Slytherin ចាប់តាំងពី \emph{ខ្ញុំមិនមែនជាម្ចាស់សត្វពស់ទេ។} ដូច្នេះខ្ញុំគ្មានហេតុផលដែលត្រូវខ្វល់នោះទេ។ តើ​ពួកគេ​គិត​យ៉ាង​ណា​អំពី​របៀប​ដែល​ខ្ញុំ​ប្រកួត​ជាមួយ​នឹង Hermione Granger»។

Draco មិន​បាន​ក្តាប់​កណ្តាប់​ដៃ​របស់​គាត់​ដោយ​ការ​ខក​ចិត្ត។ Draco បាននិយាយថា "នាងគ្រាន់តែជាឈាមភក់ប៉ុណ្ណោះ" ដោយរក្សាសម្លេងរបស់គាត់ឱ្យស្ងប់ស្ងាត់ ជាជាងការស្រែក។ “បើមិនចូលចិត្ត រុញនាងចុះពីលើជណ្តើរ”

"Ravenclaw នឹងដឹង -"

“ឲ្យ Pansy Parkinson រុញនាងចុះជណ្តើរ! អ្នក​នឹង​មិន​ចាំបាច់​ចាត់​ចែង​នាង​ឡើយ ផ្តល់​ឱ្យ​នាង​នូវ​ជំងឺ​កាមរោគ ហើយ​នាង​នឹង​ធ្វើ​វា​បាន!”

“\emph{ខ្ញុំ} ដឹង! Hermione បានផ្តួលខ្ញុំក្នុងការប្រកួតអានសៀវភៅ នាងទទួលបានចំណាត់ថ្នាក់ល្អជាងខ្ញុំ ខ្ញុំត្រូវតែយកឈ្នះនាងដោយ \emph{ខួរក្បាល} របស់ខ្ញុំ ឬវាមិនរាប់បញ្ចូល!”

“\emph{នាងគ្រាន់តែជាឈាមភក់! ហេតុអ្វីបានជាអ្នកគោរពនាងខ្លាំង?}”

“\emph{នាងគឺជាថាមពលក្នុងចំណោម Ravenclaws! ហេតុ​អ្វី​បាន​ជា​អ្នក​ខ្វល់​ពី​អ្វី​ដែល​មនុស្ស​ល្ងង់​ដែល​គ្មាន​អំណាច​នៅ​ Slytherin គិត?}”

“\emph{វាហៅថានយោបាយ! ហើយ​ប្រសិន​បើ​អ្នក​មិន​អាច​លេង​វា អ្នក​មិន​អាច​មាន​អំណាច!}”

“\emph{ការដើរលើព្រះច័ន្ទគឺជាថាមពល! ការក្លាយជាអ្នកជំនួយដ៏អស្ចារ្យគឺជាអំណាច! មាន​អំណាច​មួយ​ប្រភេទ​ដែល​មិន​តម្រូវ​ឱ្យ​ខ្ញុំ​ចំណាយ​ពេល​ដែល​នៅ​សល់​ក្នុង​ជីវិត​របស់​ខ្ញុំ​ដើរ​រក​មនុស្ស​ល្ងង់!}”

ពួកគេទាំងពីរនាក់បានឈប់ ហើយនៅក្នុងភាពរួបរួមគ្នាស្ទើរតែឥតខ្ចោះ បានចាប់ផ្តើមដកដង្ហើមចូលជ្រៅៗ ដើម្បីធ្វើឲ្យខ្លួនឯងស្ងប់។

"សុំទោស" Harry Potter បាននិយាយបន្ទាប់ពីមួយសន្ទុះដោយជូតញើសចេញពីថ្ងាសរបស់គាត់។ “សុំទោស ដ្រាកូ។ អ្នកមានអំណាចនយោបាយច្រើន ហើយវាសមហេតុផលសម្រាប់អ្នកក្នុងការរក្សាវា។ អ្នក \emph{sould} កំពុងគណនាអ្វីដែល Slytherin គិត។ វា​ជា​ការ​ប្រកួត​ដ៏​សំខាន់ ហើយ​ខ្ញុំ​មិន​គួរ​ប្រមាថ​វា​ទេ។ ប៉ុន្តែអ្នកមិនអាចស្នើសុំ \emph{ខ្ញុំ} ឱ្យបន្ថយកម្រិតនៃហ្គេមរបស់ខ្ញុំនៅក្នុង Ravenclaw បានទេ ដើម្បីកុំឱ្យអ្នកមើលទៅអាក្រក់ដោយការសេពគប់ជាមួយខ្ញុំ។ ប្រាប់ Slytherin ថាអ្នកកំពុងខាំធ្មេញរបស់អ្នក ខណៈពេលដែលអ្នកធ្វើពុតជាមិត្តរបស់ខ្ញុំ។

នោះជាអ្វីដែល Draco \emph{had} បានប្រាប់ Slytherin ហើយគាត់នៅតែមិនប្រាកដថាវាជាការពិតឬអត់។

Draco បាននិយាយថា "យ៉ាងណាក៏ដោយ។ “ និយាយអំពីរូបភាពរបស់អ្នក។ ខ្ញុំ​ខ្លាច​ថា​ខ្ញុំ​មាន​ដំណឹង​អាក្រក់​មួយ​ចំនួន។ Rita Skeeter បានឮរឿងខ្លះអំពីអ្នក ហើយនាងកំពុងសួរសំណួរ។

Harry Potter លើកចិញ្ចើមរបស់គាត់។ "WHO?"

Draco បាននិយាយថា “នាងសរសេរសម្រាប់\emph{Daily Prophet}”។ គាត់​បាន​ព្យាយាម​រក្សា​ការ​ព្រួយ​បារម្ភ​ចេញ​ពី​សំឡេង​របស់​គាត់។ \emph{Daily Prophet} គឺជាឧបករណ៍ចម្បងមួយរបស់ព្រះបិតា គាត់បានប្រើវាដូចជាដាវរបស់អ្នកជំនួយការ។ “នោះ​ជា​អ្នក​កាសែត​ពិត​ជា​យក​ចិត្ត​ទុក​ដាក់។ Rita Skeeter សរសេរអំពីតារាល្បី ហើយនៅពេលដែលនាងដាក់វា ប្រើភួយរបស់នាងដើម្បីវាយលុកកេរ្តិ៍ឈ្មោះហួសហេតុរបស់ពួកគេ។ បើ​នាង​រក​មិន​ឃើញ​ពាក្យ​ចចាមអារ៉ាម​អំពី​អ្នក​ទេ នាង​នឹង​បង្កើត​ឡើង​ដោយ​ខ្លួន​ឯង»។

Harry Potter បាននិយាយថា “ខ្ញុំ \emph{មើល}។ មុខ​ភ្លឺ​ពណ៌​បៃតង​របស់​គាត់​មើល​ទៅ​គិត​យ៉ាង​ខ្លាំង​នៅ​ពី​ក្រោម​គោ។

Draco ស្ទាក់ស្ទើរមុនពេលនិយាយអ្វីដែលគាត់ត្រូវនិយាយបន្ទាប់។ មកដល់ពេលនេះ មាននរណាម្នាក់បានរាយការណ៍ទៅឪពុកថាគាត់កំពុងទាក់ទងជាមួយ Harry Potter ហើយឪពុកក៏នឹងដឹងដែរថា Draco មិនបានសរសេរអំពីវានៅផ្ទះទេ ហើយឪពុកនឹងយល់ថា Draco មិនបានគិតថាគាត់ពិតជាអាចរក្សាវាទុកជាការសម្ងាត់ទេ ដែលបានផ្ញើមក។ សារច្បាស់មួយថា Draco កំពុងហាត់លេងល្បែងផ្ទាល់ខ្លួនរបស់គាត់ឥឡូវនេះ ប៉ុន្តែនៅតែនៅខាងឪពុក ព្រោះប្រសិនបើ Draco ត្រូវបានល្បួងឱ្យនៅឆ្ងាយ គាត់នឹងផ្ញើរបាយការណ៍មិនពិត។

វាធ្វើតាមដែលឪពុកប្រហែលជាបានទន្ទឹងរង់ចាំនូវអ្វីដែល Draco ហៀបនឹងនិយាយបន្ទាប់។

ការលេងហ្គេមជាមួយឪពុកពិតប្រាកដគឺជាអារម្មណ៍ដែលគួរឱ្យភ្ញាក់ផ្អើល។ ទោះបីជាពួកគេនៅម្ខាងក៏ដោយ។ ម្យ៉ាងវិញទៀត វាគឺជាការរំភើបចិត្ត ប៉ុន្តែ Draco ក៏បានដឹងដែរថា នៅទីបញ្ចប់ វានឹងប្រែក្លាយថា ឪពុកបានលេងហ្គេមកាន់តែប្រសើរ។ មិនមានវិធីផ្សេងទៀតដែលវាអាចទៅបាន។

"Harry" Draco បាននិយាយចុងក្រោយ។ “នេះមិនមែនជាការណែនាំទេ។ នេះមិនមែនជាដំបូន្មានរបស់ខ្ញុំទេ។ គ្រាន់តែជាវិធីដែលវាគឺជា។ ឪពុក​របស់​ខ្ញុំ​ស្ទើរ​តែ​អាច​បំបែក​អត្ថបទ​នោះ។ ប៉ុន្តែ​វា​នឹង​ធ្វើឱ្យ​អ្នក​ខាតបង់​»​។

ឪពុកដែលរំពឹងថា Draco នឹងប្រាប់ Harry Potter យ៉ាងពិតប្រាកដ នោះមិនមែនជាអ្វីដែល Draco និយាយខ្លាំងៗនោះទេ។ Harry Potter នឹងធ្វើវាដោយខ្លួនឯងឬអត់។

ប៉ុន្តែផ្ទុយទៅវិញ Harry Potter ងក់ក្បាលដោយញញឹមនៅក្រោមក្រវិល។ "ខ្ញុំគ្មានចេតនាចង់កម្ចាត់ Rita Skeeter ទេ"

Draco មិន​បាន​សូម្បី​តែ​ព្យាយាម​រក្សា​ភាព​មិន​គួរ​ឱ្យ​ជឿ​ចេញ​ពី​សំឡេង​របស់​គាត់​។ “អ្នក\emph{មិនអាច}ប្រាប់ខ្ញុំថាអ្នកមិនខ្វល់ពីអ្វីដែល\emph{កាសែត}និយាយអំពីអ្នក!”

Harry Potter បាននិយាយថា "ខ្ញុំខ្វល់តិចជាងអ្នកគិត" ។ "ប៉ុន្តែខ្ញុំមានវិធីផ្ទាល់ខ្លួនរបស់ខ្ញុំក្នុងការដោះស្រាយជាមួយ Skeeter ។ ខ្ញុំ​មិន​ត្រូវ​ការ​ជំនួយ​ពី Lucius ទេ»។

មើលទៅមុខរបស់ Draco មុននឹងអាចបញ្ឈប់វាបាន។ អ្វីក៏ដោយដែល Harry Potter ហៀបនឹងធ្វើបន្ទាប់ វានឹងក្លាយជាអ្វីដែលឪពុកមិននឹកស្មានដល់ ហើយ Draco មានអារម្មណ៍ភ័យខ្លាចយ៉ាងខ្លាំងចំពោះកន្លែងដែលអាចនាំទៅដល់។

Draco ក៏​បាន​ដឹង​ថា​សក់​របស់​គាត់​កំពុង​មាន​ញើស​នៅ​ពី​ក្រោម​ក្រវិល។ គាត់ពិតជាមិនដែលពាក់អាវមួយក្នុងចំណោមអាវទាំងនោះពីមុនមក ហើយមិនបានដឹងថាអាវរបស់ Death Eaters ប្រហែលជាមានរបស់ដូចជា Cooling Charms នោះទេ។

Harry Potter ជូត​ញើស​ចេញ​ពី​ថ្ងាស​របស់​គាត់​ម្តង​ទៀត ដោយ​ទឹកមុខ​ញញឹមញញែម ដក​ដង្កៀប​ចេញ ចង្អុល​វា​ឡើង​លើ ដក​ដង្ហើម​វែងៗ ហើយ​និយាយ​ថា “\emph{Frigideiro!}”

មួយសន្ទុះក្រោយមក Draco មានអារម្មណ៍ត្រជាក់។

“\emph{Frigideiro! ហ្វ្រីជីឌីរ៉ូ! ហ្វ្រីជីឌីរ៉ូ! ហ្វ្រីជីឌីរ៉ូ! Frigideiro!}”

បន្ទាប់មក Harry Potter បានបន្ទាបដំបង ទោះបីជាដៃរបស់គាត់ហាក់ដូចជាញ័របន្តិចក៏ដោយ ហើយដាក់វាចូលទៅក្នុងអាវរបស់គាត់វិញ។

បន្ទប់ទាំងមូលហាក់ដូចជាត្រជាក់ជាង។ Draco ក៏អាចធ្វើបានដែរ ប៉ុន្តែនៅតែមិនអាក្រក់។

"ដូច្នេះ" Draco បាននិយាយថា។ "វិទ្យាសាស្ត្រ។ អ្នក​នឹង​ប្រាប់​ខ្ញុំ​អំពី​ឈាម»។

Harry Potter បាននិយាយថា "យើងនឹងចំណាយ \emph{ស្វែងយល់} អំពីឈាម។ "ដោយការពិសោធន៍" ។

Draco បាននិយាយថា "មិនអីទេ" ។ "តើការពិសោធន៍បែបណា?"

Harry Potter ញញឹមយ៉ាងអាក្រក់នៅក្រោមក្រម៉ារបស់គាត់ ហើយនិយាយថា "អ្នកប្រាប់ខ្ញុំមក"។

\later

Draco ធ្លាប់បានឮអំពីអ្វីដែលហៅថា Socratic Method ដែលត្រូវបានបង្រៀនដោយការសួរសំណួរ (ដាក់ឈ្មោះតាមទស្សនវិទូបុរាណម្នាក់ដែលឆ្លាតពេកក្នុងការក្លាយជា Muggle ពិតប្រាកដ ដូច្នេះហើយបានជាអ្នកជំនួយការឈាមសុទ្ធក្លែងខ្លួន)។ គ្រូម្នាក់របស់គាត់បានប្រើការបង្រៀន Socratic យ៉ាងច្រើន។ វាធ្លាប់រំខាន ប៉ុន្តែមានប្រសិទ្ធភាព។

បន្ទាប់មកមានវិធីសាស្ត្រ Potter ដែលឆ្កួត។

ដើម្បីឱ្យមានភាពយុត្តិធម៌ Draco ត្រូវទទួលស្គាល់ថា Harry Potter បានសាកល្បងវិធី Socratic ជាមុន ហើយវាមិនដំណើរការល្អពេកទេ។

Harry Potter បានសួរថាតើ Draco នឹងទៅជាយ៉ាងណាអំពី \emph{disproving} សម្មតិកម្មដែលពោរពេញដោយឈាម ដែលអ្នកជំនួយការមិនអាចធ្វើរឿងស្អាតបានឥឡូវនេះ ដែលពួកគេបានធ្វើកាលពីប្រាំបីសតវត្សមុន ដោយសារតែពួកគេបានបង្កាត់ពូជជាមួយ Muggle-កើត និង Squibs ។

Draco បាននិយាយថាគាត់មិនយល់ពីរបៀបដែល Harry Potter អាចអង្គុយនៅទីនោះដោយទឹកមុខត្រង់ ហើយអះអាងថានេះមិនមែនជាអន្ទាក់នោះទេ។

Harry Potter បានឆ្លើយតបដោយទឹកមុខត្រង់ថា ប្រសិនបើវាជាអន្ទាក់ វាច្បាស់ណាស់ថា \emph{គាត់} គួរតែយកវាមកចិញ្ចឹម និងផ្តល់ចំណីដល់សត្វពស់ ប៉ុន្តែវាគឺ \emph{មិនមែន } អន្ទាក់ វាគ្រាន់តែជាច្បាប់នៃរបៀបដែលអ្នកវិទ្យាសាស្ត្រដំណើរការ ដែលអ្នកត្រូវតែព្យាយាមបដិសេធទ្រឹស្តីរបស់អ្នក ហើយប្រសិនបើអ្នកប្រឹងប្រែងដោយស្មោះត្រង់ ហើយបរាជ័យ នោះគឺជាជ័យជំនះ។

Draco បានព្យាយាមចង្អុលបង្ហាញពីភាពឆោតល្ងង់ដ៏គួរឱ្យភ្ញាក់ផ្អើលនេះដោយផ្តល់យោបល់ថាគន្លឹះដើម្បីរស់រានមានជីវិតពីការប្រកួតគឺត្រូវបោះ Avada Kedavra នៅលើជើងរបស់អ្នកហើយនឹក។

Harry Potter មាន\emph{ងក់ក្បាល}។

Draco ងក់ក្បាល។

ពេលនោះ Harry Potter បានបង្ហាញគំនិតដែលអ្នកវិទ្យាសាស្ត្រមើលគំនិតប្រយុទ្ធ ដើម្បីមើលថាតើមួយណាឈ្នះ ហើយអ្នក \emph{មិនអាចប្រយុទ្ធដោយគ្មានគូប្រកួតបានទេ} ដូច្នេះ Draco ត្រូវការស្វែងរកគូប្រជែងសម្រាប់សម្មតិកម្មឈាមដើម្បីប្រយុទ្ធដើម្បីឱ្យឈាម។ purism អាចឈ្នះ ដែល Draco យល់បានប្រសើរជាងនេះបន្តិច បើទោះបីជា Harry Potter បាននិយាយវាដោយមើលទៅគួរឲ្យខ្ពើមរអើមក៏ដោយ។ ដូចជា វាច្បាស់ណាស់ថា ប្រសិនបើការបន្សុតឈាមគឺជាវិធីដែលពិភពលោកពិតជាមាន នោះមេឃត្រូវតែមានពណ៌ខៀវ ហើយប្រសិនបើទ្រឹស្តីខ្លះទៀតជាការពិត នោះមេឃត្រូវតែពណ៌បៃតង។ ហើយគ្មាននរណាម្នាក់បានឃើញមេឃនៅឡើយទេ។ បន្ទាប់មក អ្នកបានចេញទៅខាងក្រៅ ហើយមើល ហើយអ្នកបរិសុទ្ធឈាមបានឈ្នះ។ ហើយបន្ទាប់ពីរឿងនេះបានកើតឡើងប្រាំមួយដងជាប់ៗគ្នា មនុស្សនឹងចាប់ផ្តើមកត់សម្គាល់ពីនិន្នាការនេះ។

បន្ទាប់មក Harry Potter បានបន្តអះអាងថាគូប្រជែងទាំងអស់ដែល Draco កំពុងបង្កើតគឺខ្សោយពេក ដូច្នេះ purism ឈាមនឹងមិនទទួលបានកិត្តិយសសម្រាប់ការកម្ចាត់ពួកគេទេ ពីព្រោះការប្រយុទ្ធមិនគួរឱ្យចាប់អារម្មណ៍គ្រប់គ្រាន់ទេ។ ដ្រាកូក៏យល់ដែរ។ \emph{Wizards កាន់តែខ្សោយទៅៗ ដោយសារតែ elves ក្នុងផ្ទះកំពុងលួចវេទមន្តរបស់យើង} ក៏មិនគួរឱ្យចាប់អារម្មណ៍ចំពោះគាត់ដែរ។

(ទោះបីជា Harry Potter \emph{had} បាននិយាយថាយ៉ាងហោចណាស់មួយអាចសាកល្បងបាន ដើម្បីឱ្យពួកគេអាចព្យាយាមពិនិត្យមើលថាតើ elves ផ្ទះមានភាពរឹងមាំតាមពេលវេលាឬអត់ ហើយថែមទាំងគូររូបភាពតំណាងឱ្យការកើនឡើងនៃកម្លាំងនៃផ្ទះ elves និងរូបភាពផ្សេងទៀតតំណាងឱ្យ ការថយចុះកម្លាំងនៃអ្នកជំនួយការ ហើយប្រសិនបើរូបភាពទាំងពីរត្រូវគ្នាដែលនឹងចង្អុលទៅផ្ទះ elves ទាំងអស់បាននិយាយដោយសម្លេងធ្ងន់ធ្ងរដែល Draco មានអារម្មណ៍ថាមានកម្លាំងចិត្តដើម្បីសួរ Dobby នូវសំណួរចង្អុលមួយចំនួននៅក្រោម Veritaserum មុនពេលដកចេញពីវា។)

ហើយ Harry Potter បាននិយាយចុងក្រោយថា Draco \emph{មិនអាច} ជួសជុលសមរភូមិទេ អ្នកវិទ្យាសាស្ត្រមិនល្ងង់ទេ វានឹងមានតម្លៃ \emph{ច្បាស់} ប្រសិនបើអ្នកជួសជុលការប្រយុទ្ធ វាត្រូវតែជា \emph{ ការប្រយុទ្ធពិតប្រាកដ} រវាងទ្រឹស្តីពីរផ្សេងគ្នាដែលអាចទាំងពីរ \emph{ពិតជា} ជាការពិត ជាមួយនឹងការធ្វើតេស្តថាមានតែសម្មតិកម្ម \emph{true} ប៉ុណ្ណោះដែលនឹងឈ្នះ អ្វីដែលពិតប្រាកដ \emph{នឹង} ចេញមកតាមវិធីផ្សេងៗគ្នា អាស្រ័យ ថាតើសម្មតិកម្មមួយណាពិតជាត្រឹមត្រូវ ហើយនឹងមានអ្នកវិទ្យាសាស្ត្រដែលមានបទពិសោធន៍មើល ដើម្បីឱ្យប្រាកដថានោះជាអ្វីដែលបានកើតឡើង។ Harry Potter បានអះអាងថាខ្លួនគាត់គ្រាន់តែចង់ដឹង\emph{របៀបដែលឈាមពិតជាដំណើរការ} ហើយសម្រាប់ការដែលគាត់ត្រូវការដើម្បីមើលឈាម purism \emph{ពិតជាឈ្នះ} ហើយ Draco នឹងមិនបោកបញ្ឆោត\emph{គាត់} ជាមួយនឹងទ្រឹស្ដីដែលគ្រាន់តែនៅទីនោះ ដែលត្រូវទម្លាក់ចុះ។

ទោះបីជាបានឃើញចំណុចនេះក៏ដោយ Draco មិនអាចបង្កើត "ជម្រើសដែលអាចជឿទុកចិត្តបាន" ណាមួយដូចដែល Harry Potter បាននិយាយនោះទេ ចំពោះគំនិតដែលថាអ្នកជំនួយការមិនសូវមានថាមពលទេ ដោយសារតែពួកគេកំពុងលាយឈាមរបស់ពួកគេជាមួយនឹងភក់។ វាច្បាស់ណាស់ជាការពិត។

ពេលនោះហើយដែល Harry Potter បាននិយាយថា មានការខកចិត្តជាខ្លាំង ដែលគាត់នឹកស្មានមិនដល់ថា Draco មានតម្លៃ \emph{ពិតជា} ដ៏អាក្រក់នេះ ក្នុងការពិចារណាពីទស្សនៈផ្សេងៗគ្នា \emph{ប្រាកដណាស់} មាន Death Eaters ដែលបានដាក់ជា ខ្មាំងសត្រូវនៃ purism ឈាម ហើយបានចេញមកជាមួយនឹងអាគុយម៉ង់ដែលអាចទុកចិត្តបានច្រើនប្រឆាំងនឹងភាគីរបស់ពួកគេជាង Draco កំពុងផ្តល់ជូន។ ប្រសិនបើ Draco បានព្យាយាមធ្វើជាសមាជិកនៃបក្សពួក Dumbledore ហើយបង្កើតសម្មតិកម្មរបស់ elf ក្នុងផ្ទះ នោះគាត់នឹងមិនបោកបញ្ឆោតនរណាម្នាក់មួយវិនាទីនោះទេ។

Draco ត្រូវបានបង្ខំឱ្យទទួលស្គាល់ថានេះគឺជាចំណុចមួយ។

ដូច្នេះវិធីសាស្រ្ត Potter ។

"សូមលោកវេជ្ជបណ្ឌិត ~ Malfoy" Harry Potter ស្រែកថា "ហេតុអ្វីបានជាអ្នកមិនទទួលយកក្រដាសរបស់ខ្ញុំ?"

Harry Potter ត្រូវការនិយាយឡើងវិញនូវឃ្លា "គ្រាន់តែធ្វើពុតជាអ្នកវិទ្យាសាស្ត្រ" បីដង មុនពេលដែល Draco យល់។

នៅពេលនោះ Draco បានដឹងថាមានអ្វីមួយយ៉ាងជ្រាលជ្រៅ \emph{ខុស} នៅក្នុងខួរក្បាលរបស់ Harry Potter ហើយអ្នកណាម្នាក់ដែលសាកល្បងភាពស្របច្បាប់លើវា ប្រហែលជាមិនត្រលប់មកវិញទៀតទេ។

ក្រោយមក Harry Potter បានចូលទៅក្នុងលម្អិតបន្ថែមទៀត និងគួរឱ្យកត់សម្គាល់: Draco ធ្វើពុតជា Death Eater ដែលកំពុងដើរតួជាអ្នកកែសម្រួលទិនានុប្បវត្តិវិទ្យាសាស្ត្រ Dr ~ Malfoy ដែលចង់បដិសេធសត្រូវរបស់គាត់ Dr~Potter's paper "On the Heritability of សមត្ថភាពវេទមន្ត” ហើយប្រសិនបើ Death Eater មិនបានប្រព្រឹត្តដូចអ្នកវិទ្យាសាស្ត្រពិតប្រាកដទេ គាត់នឹងត្រូវបានបង្ហាញថាជាអ្នកបរិភោគមរណៈ ហើយត្រូវបានប្រហារជីវិត ខណៈដែល Dr~ Malfoy ក៏ត្រូវបានគូប្រជែងរបស់គាត់មើល និងត្រូវការ \emph{បង្ហាញខ្លួន។ } ដើម្បីបដិសេធក្រដាសរបស់ Dr~Potter សម្រាប់ហេតុផលវិទ្យាសាស្ត្រអព្យាក្រឹត ឬគាត់នឹងបាត់បង់តំណែងជានិពន្ធនាយកកាសែត។

វា​ជា​ការ​ងឿង​ឆ្ងល់​ដែល​មួក​តម្រៀប​មិន​បាន​និយាយ​ដោយ​ឆ្កួតៗ​នៅ St. Mungo's។

វាក៏ជារឿងដ៏ស្មុគស្មាញបំផុតដែលនរណាម្នាក់មាន \emph{មិនធ្លាប់មាន} បានសុំឱ្យ Draco ធ្វើពុត ហើយគ្មានវិធីដែលគាត់អាចបដិសេធការប្រឈមនោះទេ។

ឥឡូវនេះពួកគេដូចជា Harry Potter បានដាក់វា ទទួលបានអារម្មណ៍។

Draco បាននិយាយថា "ខ្ញុំខ្លាច Dr~Potter ដែលអ្នកសរសេរពណ៌ទឹកខ្មៅខុស"។ “បន្ទាប់!”

មុខរបស់ Dr~Potter បានធ្វើកិច្ចការដ៏ប្រសើរមួយ ក្នុងការបង្អាក់ដោយភាពអស់សង្ឃឹម ហើយ Draco មិនអាចជួយអ្វីបានក្រៅពីមានអារម្មណ៍រីករាយរបស់ Dr~Malfoy ទោះបីជាអ្នកបរិភោគមរណៈគ្រាន់តែធ្វើពុតជា Dr~ Malfoy ក៏ដោយ។

ផ្នែកនេះគឺ\emph{fun}។ គាត់អាចធ្វើវាពេញមួយថ្ងៃ។

Dr~Potter បានក្រោកពីកៅអី ដួលទាំងស្រងេះស្រងោច ហើយងាកចេញ ហើយប្រែទៅជា Harry Potter ដែលបានលើកមេដៃដល់ Draco ហើយបន្ទាប់មកប្រែទៅជា Dr~Potter ម្តងទៀត ពេលនេះខិតជិតដោយស្នាមញញឹមយ៉ាងអន្ទះសារ។

Dr~Potter អង្គុយចុះ ហើយបង្ហាញ Dr~Malfoy ជាមួយនឹងក្រដាស់មួយសន្លឹកដែលបានសរសេរថា:

\begin{center}
\emph{On the Heritability of Magical Ability}

\emph{Dr~H. J. Potter-Evans-Verres វិទ្យាស្ថានវិទ្យាសាស្ត្រកម្រិតខ្ពស់គ្រប់គ្រាន់}
\end{center}

\begin{writtenNote}
ការសង្កេតរបស់ខ្ញុំ៖

អ្នកជំនួយការថ្ងៃនេះមិនអាចធ្វើរឿងគួរឱ្យចាប់អារម្មណ៍ដូច\\
អ្វីដែលអ្នកជំនួយការធ្លាប់ធ្វើកាលពី 800 ឆ្នាំមុន។

ការសន្និដ្ឋានរបស់ខ្ញុំ៖

Wizardkind កាន់តែខ្សោយដោយការលាយបញ្ចូលគ្នា \\
ឈាមរបស់ពួកគេជាមួយ Muggle-កើត និង Squibs ។
\end{writtenNote}

Dr~Malfoy បាននិយាយថា "Dr~Potter" បាននិយាយដោយក្តីសង្ឃឹម "ខ្ញុំឆ្ងល់ថាតើ \emph{Journal of Irreproducible Results} អាចពិចារណាសម្រាប់ការបោះពុម្ពសៀវភៅរបស់ខ្ញុំដែលមានចំណងជើងថា "On the Heritability of Magical Ability"។

Draco សម្លឹងមើលមុខបន្ទប់ដោយញញឹម ខណៈពេលដែលគាត់ពិចារណាអំពីការបដិសេធដែលអាចកើតមាន។ ប្រសិនបើគាត់ជាសាស្រ្តាចារ្យ គាត់នឹងបដិសេធការសរសេរអត្ថបទខ្លីពេក ដូច្នេះ—

វេជ្ជបណ្ឌិត ~ Malfoy បាននិយាយថា "វាវែងពេកហើយ Dr ~ Potter" ។

មួយសន្ទុះ មានភាពមិនគួរឱ្យជឿពិតប្រាកដនៅលើមុខរបស់ Dr~Potter ។

“អា…” វេជ្ជបណ្ឌិត ~ Potter បាននិយាយ។ “ចុះយ៉ាងណាបើខ្ញុំកម្ចាត់បន្ទាត់ដាច់ដោយឡែកសម្រាប់ការសង្កេត និងការសន្និដ្ឋាន ហើយគ្រាន់តែដាក់ក្នុង \emph{ដូច្នេះ}—”

"បន្ទាប់មកវានឹងខ្លីពេក។ បន្ទាប់!”

Dr~Potter បានដើរចេញ។

លោក Harry Potter បាននិយាយថា “មិនអីទេ” អ្នកទទួលបាន\emph{ល្អផងដែរក្នុងរឿងនេះ។ ពីរលើកទៀតដើម្បីអនុវត្ត ហើយបន្ទាប់មកលើកទីបីគឺសម្រាប់ពិត គ្មានការរំខានទេ ខ្ញុំនឹងមករកអ្នកត្រង់ៗ ហើយពេលនោះអ្នកនឹងបដិសេធក្រដាសដោយផ្អែកលើខ្លឹមសារជាក់ស្តែង សូមចាំថា គូប្រជែងវិទ្យាសាស្ត្ររបស់អ្នកកំពុងមើល។ ”

ក្រដាសបន្ទាប់របស់ Dr~Potter គឺល្អឥតខ្ចោះក្នុងគ្រប់មធ្យោបាយ ដែលជាភាពអស្ចារ្យនៃប្រភេទរបស់វា ប៉ុន្តែជាអកុសលត្រូវតែត្រូវបានច្រានចោល ដោយសារតែទិនានុប្បវត្តិរបស់ Dr~Malfoy កំពុងមានបញ្ហាជាមួយអក្សរ E\@។ Dr~Potter បានស្នើឱ្យសរសេរវាឡើងវិញដោយគ្មានពាក្យទាំងនោះ ហើយ Dr~Malfoy បានពន្យល់ថាវាពិតជាបញ្ហាស្រៈជាង។

ក្រដាសបន្ទាប់ពីនោះត្រូវបានច្រានចោលព្រោះវាជាថ្ងៃអង្គារ។

តាមពិតវាជាថ្ងៃសៅរ៍។

Dr~Potter ព្យាយាម​ចង្អុល​ចំណុច​នេះ ហើយ​ត្រូវ​បាន​គេ​ប្រាប់​ថា "បន្ទាប់!"

(Draco ចាប់ផ្តើមយល់ពីមូលហេតុដែល Snape បានប្រើការកាន់កាប់របស់គាត់លើ Dumbledore គ្រាន់តែដើម្បីទទួលបានតំណែងដែលអនុញ្ញាតឱ្យគាត់ធ្វើអាក្រក់ចំពោះសិស្ស។ )

ហើយបន្ទាប់មក -

Dr~Potter កំពុង​ចូល​ទៅ​ជិត​ដោយ​ការ​ញញឹម​យ៉ាង​ល្អ​លើ​មុខ​របស់​គាត់។

“នេះគឺជាក្រដាសចុងក្រោយរបស់ខ្ញុំ \emph{On the Heritability of Magical Ability,}” Dr~Potter បាននិយាយដោយទំនុកចិត្ត ហើយរុញក្រដាស់ចេញ។ "ខ្ញុំបានសម្រេចចិត្តអនុញ្ញាតឱ្យទិនានុប្បវត្តិរបស់អ្នកបោះពុម្ពវា ហើយបានរៀបចំវាឱ្យល្អឥតខ្ចោះស្របតាមការណែនាំរបស់អ្នក ដូច្នេះអ្នកអាចបោះពុម្ពវាបានយ៉ាងឆាប់រហ័ស។"

The Death Eater បានសម្រេចចិត្តតាមដាន និងសម្លាប់ Dr~Potter បន្ទាប់ពីបេសកកម្មរបស់គាត់បានបញ្ចប់។ វេជ្ជបណ្ឌិត ~ Malfoy រក្សាស្នាមញញឹមដ៏គួរសមនៅលើមុខរបស់គាត់ ចាប់តាំងពីគូប្រជែងរបស់គាត់កំពុងមើល ហើយបាននិយាយថា…

(ការផ្អាកបានលាតសន្ធឹងដោយ Dr~Potter សម្លឹងមើលគាត់ដោយអត់ធ្មត់។ )

… “សូម​ឲ្យ​ខ្ញុំ​មើល​រឿង​នោះ​ចុះ។”

វេជ្ជបណ្ឌិត ~ Malfoy បានយកក្រដាសជូតមាត់ ហើយប្រើវាដោយប្រុងប្រយ័ត្ន។

The Death Eater ចាប់ផ្តើមភ័យអំពីការពិតដែលថាគាត់មិនមែនជាអ្នកវិទ្យាសាស្ត្រពិតប្រាកដ ហើយ Draco កំពុងព្យាយាមចងចាំពីរបៀបនិយាយដូច Harry Potter ។

"អ្នក, ah, ត្រូវពិចារណាការពន្យល់ដែលអាចធ្វើទៅបានផ្សេងទៀតសម្រាប់ការសង្កេតរបស់អ្នក, ក្រៅពីនេះគ្រាន់តែមួយ -"

“ពិតមែនឬ?” បានរំខាន Dr ~ Potter ។ “ដូចអ្វីដែរ? \emph{House elves កំពុងលួចវេទមន្តរបស់យើងមែនទេ?} ទិន្នន័យរបស់ខ្ញុំសារភាពថាមានតែការសន្និដ្ឋានមួយដែលអាចធ្វើទៅបាន Dr~Malfoy។ មាន\emph{គឺ}មិនមានសម្មតិកម្មដែលអាចជឿជាក់បានផ្សេងទៀតទេ។”

Draco កំពុងព្យាយាមខឹងសម្បារដើម្បីបញ្ជាខួរក្បាលរបស់គាត់ឱ្យគិត តើគាត់នឹងនិយាយអ្វីប្រសិនបើគាត់ឈរឈ្មោះជាសមាជិកនៃបក្សពួក Dumbledore អ្វី \emph{តើ} ដែលពួកគេអះអាងថាគឺជាការពន្យល់សម្រាប់ការធ្លាក់ចុះនៃ wizardkind នោះ Draco មិនដែលខ្វល់នឹងការសួរយ៉ាងពិតប្រាកដនោះទេ។ …

"ប្រសិនបើអ្នកមិនអាចគិតពីវិធីផ្សេងទៀតដើម្បីពន្យល់ទិន្នន័យរបស់ខ្ញុំទេ អ្នកនឹងត្រូវបោះពុម្ពក្រដាសរបស់ខ្ញុំ \emph{Dr~ Malfoy។}"

វាជាការសើចចំអកលើមុខរបស់ Dr~Potter ដែលបានធ្វើវា។

“អ្ហា៎?” បាន​ថត​រូប Dr ~ Malfoy។ «ធ្វើ​ម៉េច​ដឹង​ថា​វេទមន្ត​ខ្លួន​ឯង​មិន​រសាយ​ទៅ?»

ពេលវេលាបានឈប់។

Draco និង Harry Potter ផ្លាស់ប្តូររូបរាងដ៏គួរឱ្យរន្ធត់។

បន្ទាប់មក Harry Potter ស្តោះទឹកមាត់អ្វីមួយដែលប្រហែលជាពាក្យអាក្រក់ខ្លាំងណាស់ប្រសិនបើអ្នកត្រូវបានលើកឡើងដោយ Muggles ។ “\emph{ខ្ញុំ​មិន​បាន​គិត​ដល់​រឿង​នោះ​ទេ!}” Harry Potter និយាយ។ "ហើយខ្ញុំគួរតែមាន។ វេទមន្តទៅឆ្ងាយ។ \emph{ Damn, damn, damn!}”

ការជូនដំណឹងនៅក្នុងសំឡេងរបស់ Harry Potter គឺឆ្លង។ ដោយមិនគិតពីវា ដៃរបស់ Draco បានចូលទៅក្នុងអាវរបស់គាត់ ហើយតោងជាប់នឹងដំបងរបស់គាត់។ គាត់គិតថា House of Malfoy គឺ\emph{មានសុវត្ថិភាព} ដរាបណាអ្នករៀបការក្នុងគ្រួសារដែលអាចតាមដានខ្សែរឈាមរបស់ពួកគេបានបួនជំនាន់ អ្នកត្រូវបានគេសន្មត់ថា \emph{សុវត្ថិភាព} វាមិនដែលកើតឡើងចំពោះគាត់ទេ។ មុនពេលនោះ ប្រហែលជាមិនមាននរណាម្នាក់អាចធ្វើដើម្បីបញ្ឈប់ការបញ្ចប់នៃមន្តអាគមនោះទេ។ "Harry តើយើងធ្វើអ្វី?" សំឡេងរបស់ Draco កើនឡើងដោយភាពភិតភ័យ។ “\emph{តើយើងធ្វើអ្វី?}”

“\emph{អនុញ្ញាតឱ្យខ្ញុំគិត!}”

មួយសន្ទុះក្រោយមក Harry បានទាញភួយ និងក្រដាស់ក្រដាសដូចគ្នាពីតុក្បែរនោះ ដែលគាត់ធ្លាប់សរសេរក្រដាសក្លែងក្លាយ ហើយចាប់ផ្តើមសរសេរអ្វីមួយ។

Harry បាននិយាយថា "យើងនឹងដោះស្រាយវាចេញ" Harry បាននិយាយដោយសម្លេងរបស់គាត់ថា "ប្រសិនបើវេទមន្តរសាត់ចេញពីពិភពលោក យើងនឹងដឹងថាវារសាត់លឿនប៉ុណ្ណា ហើយយើងនៅសល់ពេលប៉ុន្មានដើម្បីធ្វើអ្វីមួយ ហើយបន្ទាប់មកយើងនឹង ស្វែងយល់ថាហេតុអ្វីបានជាវារសាត់ ហើយបន្ទាប់មកយើងនឹងធ្វើអ្វីមួយអំពីវា។ Draco តើ​អំណាច​វេទមន្ត​បាន​ធ្លាក់​ចុះ​ក្នុង​អត្រា​ថេរ​ឬ​ក៏​មាន​ការ​ធ្លាក់​ចុះ​ភ្លាមៗ?

“ខ្ញុំ…ខ្ញុំមិនដឹង…”

"អ្នកបានប្រាប់ខ្ញុំថាគ្មាននរណាម្នាក់ត្រូវនឹងស្ថាបនិកទាំងបួននៃ Hogwarts នោះទេ។ ដូច្នេះ​វា​បាន​បន្ត​យ៉ាង​ហោច​ណាស់​ប្រាំបី​សតវត្ស​មក​ហើយ? អ្នក​មិន​អាច​ចាំ​ថា​បាន​ឮ​អ្វី​មួយ​អំពី​បញ្ហា​ស្រាប់តែ​លេច​ឡើង​កាល​ពី​ប្រាំ​សតវត្ស​មុន​ឬ​អ្វី​ដូច​នោះ​ទេ?

Draco ព្យាយាមគិតយ៉ាងមមាញឹក។ "ខ្ញុំតែងតែលឺថាគ្មាននរណាម្នាក់ល្អដូច Merlin ហើយបន្ទាប់មកគ្មាននរណាម្នាក់ល្អដូចស្ថាបនិក Hogwarts ទេ" ។

Harry បាននិយាយថា "មិនអីទេ" ។ គាត់នៅតែកំពុងសរសេរ។ "ដោយសារតែបីសតវត្សមុនគឺជាពេលដែល Muggles ចាប់ផ្តើមមិនជឿលើវេទមន្ត ដែលខ្ញុំគិតថាអាចមានអ្វីដែលត្រូវធ្វើជាមួយវា។ ហើយប្រហែលមួយសតវត្សកន្លះមុន គឺជាពេលដែល Muggles ចាប់ផ្តើមប្រើបច្ចេកវិទ្យាមួយប្រភេទដែលឈប់ដំណើរការដោយវេទមន្ត ហើយខ្ញុំកំពុងឆ្ងល់ថាតើវាប្រហែលជាទៅវិធីផ្សេងដែរឬអត់។

Draco បានផ្ទុះចេញពីកៅអីរបស់គាត់ ខឹងខ្លាំងណាស់ គាត់ស្ទើរតែមិនអាចនិយាយបាន។ “វាគឺ\emph{Muggles}—”

“\emph{ Damn it!}” Harry គ្រហឹម។ “តើអ្នកមិនបានស្តាប់ \emph{ខ្លួនអ្នក}ទេ? វាបានបន្តអស់រយៈពេលប្រាំបីសតវត្សមកហើយយ៉ាងហោចណាស់ ហើយ Muggles មិនបានធ្វើអ្វីគួរឱ្យចាប់អារម្មណ៍នៅពេលនោះ! \emph{យើងត្រូវតែស្វែងយល់ការពិត!} The Muggles \emph{អាច} មានអ្វីដែលត្រូវធ្វើជាមួយនេះ ប៉ុន្តែប្រសិនបើពួកគេ \emph{កុំ} ហើយអ្នកទៅបន្ទោសអ្វីៗគ្រប់យ៉ាងលើពួកគេ ហើយវាបញ្ឈប់ពួកយើង ពី​ការ​រក​ឃើញ​ថា​តើ​អ្វី​ដែល​\emph{ពិត​ជា​}​នឹង​កើត​ឡើង​បន្ទាប់​មក​ថ្ងៃ​មួយ​អ្នក​នឹង​ភ្ញាក់​ពី​ដំណេក​នៅ​ពេល​ព្រឹក​និង​ដឹង​ថា wand របស់​អ្នក​គឺ​គ្រាន់​តែ​ជា​ឈើ​មួយ!”

ដង្ហើមរបស់ Draco ឈប់នៅក្នុងបំពង់ក។ ជារឿយៗឪពុករបស់គាត់បាននិយាយថា \emph{ our wands will break in our hands} នៅក្នុងសុន្ទរកថារបស់គាត់ ប៉ុន្តែ Draco មិនដែលគិតពីមុនមកអំពីអ្វីដែល\emph{មានន័យថា} វានឹងមិនកើតឡើងចំពោះ\emph{គាត់} បន្ទាប់ពីទាំងអស់។ ហើយ​ឥឡូវ​នេះ​ភ្លាម​នោះ​វា​ហាក់​ដូច​ជា​ពិត​ណាស់​។ \emph{គ្រាន់តែឈើមួយ។} Draco អាចស្រមៃមើលថាតើវានឹងទៅជាយ៉ាងណាក្នុងការដកដង្កូវរបស់គាត់ចេញ ហើយព្យាយាមអក្ខរាវិរុទ្ធ ហើយឃើញថាគ្មានអ្វីកើតឡើង...

វាអាចកើតឡើងចំពោះ\emph{អ្នករាល់គ្នា}។

វា​នឹង​មិន​មាន​អ្នក​ជំនួយ​ការ​និង​គ្មាន​វេទមន្ត​ទៀត​ទេ​។ គ្រាន់តែ Muggles ដែលមានរឿងព្រេងមួយចំនួនអំពីអ្វីដែលបុព្វបុរសរបស់ពួកគេអាចធ្វើបាន។ Muggles មួយចំនួននឹងត្រូវបានគេហៅថា Malfoy ហើយនោះនឹងជាឈ្មោះទាំងអស់ដែលនៅសល់

ជា​លើក​ដំបូង​ក្នុង​ជីវិត​របស់​គាត់ Draco បាន​ដឹង​ថា​ហេតុ​អ្វី​បាន​ជា​មាន Death Eaters។

គាត់តែងតែយល់ស្របថា ការក្លាយជា Death Eater គឺជាអ្វីដែលអ្នកបានធ្វើនៅពេលអ្នកធំឡើង។ ឥឡូវនេះ Draco \emph{យល់ហើយ} គាត់ដឹងពីមូលហេតុដែលមិត្ដភក្ដិរបស់ឪពុក និងឪពុកបានស្បថថានឹងលះបង់ជីវិតរបស់ពួកគេ ដើម្បីការពារសុបិន្តអាក្រក់មិនឱ្យកន្លងផុតទៅ មានរឿងជាច្រើនដែលអ្នកមិនគ្រាន់តែឈរមើលហើយកើតឡើង។ ប៉ុន្តែចុះយ៉ាងណាបើវានឹងកើតឡើង\emph{anyway} ចុះបើការលះបង់ទាំងអស់ មិត្តភ័ក្តិទាំងអស់ដែលពួកគេបានបាត់បង់ទៅ Dumbledore, \emph{គ្រួសារ} ដែលពួកគេនឹងត្រូវបាត់បង់ តើវាទៅជាយ៉ាងណា? \emph{គ្មានអ្វី…}

Draco បាននិយាយថា "វេទមន្ត\emph{មិនអាច} រលាយបាត់ឡើយ។ សំឡេង​របស់​គាត់​បាន​បែក។ "វា​នឹង​មិន​មែន​ជា \emph{យុត្តិធម៌}។"

Harry ឈប់សរសេរ ហើយមើលទៅ ទឹកមុខរបស់គាត់មានការបញ្ចេញកំហឹង។ "ឪពុករបស់អ្នកមិនដែលប្រាប់អ្នកថាជីវិតមិនយុត្តិធម៌ទេ?"

ឪពុកបាននិយាយថារាល់ពេលដែល Draco ប្រើពាក្យនេះ។ "ប៉ុន្តែ ប៉ុន្តែ វាពិតជាអាក្រក់ណាស់ក្នុងការជឿថា -"

“Draco អនុញ្ញាតឱ្យខ្ញុំណែនាំអ្នកនូវអ្វីមួយដែលខ្ញុំហៅថា Litany of Tarski ។ វាផ្លាស់ប្តូររាល់ពេលដែលអ្នកប្រើវា។ ក្នុងឱកាសនេះ វាដំណើរការដូចនេះ៖ \emph{ប្រសិនបើវេទមន្តរសាត់បាត់ពីពិភពលោក ខ្ញុំចង់ជឿថាវេទមន្តកំពុងរសាត់ចេញពីពិភពលោក។ ប្រសិនបើវេទមន្តមិនរសាត់ចេញពីពិភពលោក ខ្ញុំចង់មិនជឿថា វេទមន្តកំពុងតែរសាត់ចេញពីពិភពលោក។ សូមកុំឱ្យខ្ញុំជាប់នឹងជំនឿដែលខ្ញុំប្រហែលជាមិនចង់បាន។} ប្រសិនបើយើងរស់នៅក្នុងពិភពលោកដែលវេទមន្តកំពុងធ្លាក់ចុះ \emph{នោះហើយជាអ្វីដែលយើងត្រូវជឿ} យើងត្រូវដឹងពីអ្វីដែលនឹងមកដល់ ដូច្នេះយើងអាចបញ្ឈប់វាបាន ឬក្នុងករណីដ៏អាក្រក់បំផុត សូមត្រៀមខ្លួនដើម្បីធ្វើអ្វីដែលយើងអាចធ្វើបានក្នុងពេលដែលយើងនៅសល់។ ការ​មិន​ជឿ​ថា​វា​នឹង​មិន​បញ្ឈប់​វា​មិន​ឱ្យ​កើត​ឡើង​។ ដូច្នេះ \emph{only} សំណួរដែលយើងត្រូវសួរគឺថាតើវេទមន្តគឺ \emph{តាមពិត} រសាត់ហើយ ប្រសិនបើនោះជាពិភពលោកដែលយើងរស់នៅនោះ នោះជាអ្វីដែលយើងចង់ជឿ។ Litany នៃ Gendlin៖ \emph{អ្វីដែលជាការពិតគឺរួចទៅហើយ ដូច្នេះការកាន់កាប់វាមិនធ្វើឱ្យវាកាន់តែអាក្រក់នោះទេ។} យល់ទេ Draco? ខ្ញុំនឹងធ្វើឱ្យអ្នកទន្ទេញវានៅពេលក្រោយ។ វាជាអ្វីដែលអ្នកនិយាយម្តងទៀតទៅកាន់ខ្លួនអ្នកគ្រប់ពេលដែលអ្នកចាប់ផ្តើមងឿងឆ្ងល់ថាតើវាជាគំនិតល្អក្នុងការជឿអ្វីមួយដែលមិនពិត។ តាមពិតខ្ញុំចង់ឱ្យអ្នកនិយាយវាឥឡូវនេះ។ \emph{អ្វីដែលជាការពិតគឺរួចទៅហើយ ដូច្នេះការកាន់កាប់វាមិនធ្វើឱ្យវាកាន់តែអាក្រក់នោះទេ។} និយាយវា។"

Draco បាននិយាយម្តងហើយម្តងទៀតថា "អ្វីដែលជាការពិតគឺរួចទៅហើយ" សំលេងរបស់គាត់ញ័រ "ការកាន់កាប់វាមិនធ្វើឱ្យវាកាន់តែអាក្រក់ទេ" ។

"ប្រសិនបើវេទមន្តរសាត់ទៅ ខ្ញុំចង់ជឿថាវេទមន្តកំពុងរសាត់បាត់ទៅហើយ។ បើ​វេទមន្ត​មិន​ចេះ​រសាត់​ទេ ខ្ញុំ​ចង់​មិន​ជឿ​ថា​វេទមន្ត​រសាត់​បាត់​ទៅ។ និយាយ។»

Draco តបពាក្យដដែលៗ ឈឺចុកចាប់ក្នុងពោះ។

Harry បាននិយាយថា “ល្អណាស់” សូមចាំថា ប្រហែលជា \emph{មិនមែន} កំពុងកើតឡើង ហើយបន្ទាប់មក អ្នកក៏មិនចាំបាច់ជឿដែរ។ \emph{ដំបូង} យើងគ្រាន់តែចង់ដឹងថា តើមានអ្វីកើតឡើងពិតប្រាកដ តើពិភពលោកមួយណាដែលយើងរស់នៅពិតប្រាកដ។” Harry ត្រលប់ទៅការងាររបស់គាត់វិញ សរសេរអក្សរខ្លះទៀត ហើយបន្ទាប់មកបង្វែរក្រដាសជូតមាត់ដើម្បីឱ្យ Draco ឃើញវា។ Draco ផ្អៀងលើតុ ហើយ Harry នាំភ្លើងពណ៌បៃតងមកជិត។

\penalty-10
\begin{center}\itshape
{\scshape ការសង្កេត៖}\\
Wizardry មិនមានឥទ្ធិពលខ្លាំងដូចពេលដែល Hogwarts ត្រូវបានបង្កើតឡើងនោះទេ។ \penalty101

{\scshape សម្មតិកម្ម៖}\penalty102
\begin{enumerate}[1.]\firmlist
\item វេទមន្តខ្លួនវាកំពុងតែរសាត់ទៅហើយ។
\item អ្នកជំនួយការកំពុងបង្កាត់ពូជជាមួយ Muggles និង Squibs ។
\item ចំណេះដឹងដើម្បីដេញអក្ខរាវិរុទ្ធដ៏មានឥទ្ធិពលកំពុងត្រូវបានបាត់បង់។
\item អ្នកជំនួយការកំពុងញ៉ាំអាហារខុសកាលនៅវ័យកុមារ ឬអ្វីផ្សេងទៀតក្រៅពីឈាមកំពុងធ្វើឱ្យពួកគេធំឡើងខ្សោយ។
\item បច្ចេកវិទ្យា Muggle កំពុងជ្រៀតជ្រែកជាមួយវេទមន្ត។ (តាំងពី 800 ឆ្នាំមុន?)
\item អ្នកជំនួយការខ្លាំងជាងកំពុងមានកូនតិចជាងមុន។ (Draco = មានតែកូនទេ? ពិនិត្យមើលថាតើអ្នកជំនួយការដ៏មានឥទ្ធិពល 3 នាក់ Quirrell / Dumbledore / Dark Lord មានកូនឬអត់។)
\end{enumerate}
{\scshape ការធ្វើតេស្ត៖}
\end{center}

Harry បាននិយាយថា "មិនអីទេ" ។ ការដកដង្ហើមរបស់គាត់បានធូរស្រាលបន្តិច។ “ឥឡូវនេះ នៅពេលដែលអ្នកកំពុងដោះស្រាយបញ្ហាដែលច្របូកច្របល់ ហើយអ្នកមិនដឹងថាមានរឿងអ្វីកើតឡើងនោះទេ រឿងដ៏ឆ្លាតវៃដែលត្រូវធ្វើគឺស្វែងរកការសាកល្បងដ៏សាមញ្ញមួយចំនួន អ្វីដែលអ្នកអាចមើលឃើញភ្លាមៗ។ យើងត្រូវការការធ្វើតេស្តរហ័សដែលបែងចែករវាងសម្មតិកម្មទាំងនេះ។ ការ​សង្កេត​ដែល​នឹង​ចេញ​ជា​វិធី​ផ្សេង​គ្នា​សម្រាប់​យ៉ាង​ហោច​ណាស់​មួយ​ក្នុង​ចំណោម​ពួក​គេ​បើ​ធៀប​នឹង​វិធី​ផ្សេង​ទៀត​ទាំង​អស់»។

Draco សម្លឹងមើលបញ្ជីដោយការភ្ញាក់ផ្អើល។ ស្រាប់តែគាត់ដឹងថា គាត់ស្គាល់មនុស្សឈាមសុទ្ធជាច្រើន ដែលគ្រាន់តែជាកូនក្មេង។ ខ្លួនគាត់ផ្ទាល់ វ៉ាំងសង់ ហ្គ្រេហ្គោរី ជាក់ស្តែង \emph{អ្នករាល់គ្នា}។ អ្នកជំនួយការដ៏មានឥទ្ធិពលបំផុតពីរនាក់ដែលគ្រប់គ្នាបាននិយាយអំពីគឺ Dumbledore និង Dark Lord ហើយមិនមានកូនដូច Harry បានសង្ស័យទេ…

Harry បាននិយាយថា "វាពិតជាលំបាកណាស់ក្នុងការបែងចែករវាង 2 និង 6" Harry បាននិយាយថា "វាស្ថិតនៅក្នុងឈាម អ្នកត្រូវតែព្យាយាមតាមដានការធ្លាក់ចុះនៃអ្នកជំនួយការ ហើយប្រៀបធៀបថា តើមានអ្នកជំនួយការខុសៗគ្នាប៉ុន្មាននាក់ដែលមានក្មេង។ វាស់ស្ទង់សមត្ថភាពរបស់ Muggle-born បើប្រៀបធៀបទៅនឹងឈាមសុទ្ធ…” ម្រាមដៃរបស់ Harry កំពុងតែប៉ះលើតុដោយភ័យព្រួយ។ “ ចូរយើងគ្រាន់តែដុំ 6 ក្នុងមួយជាមួយ 2 ហើយហៅពួកគេថាសម្មតិកម្មឈាមសម្រាប់ពេលនេះ។ 4 គឺមិនទំនងទេ ព្រោះពេលនោះគ្រប់គ្នានឹងកត់សម្គាល់ការធ្លាក់ចុះភ្លាមៗ នៅពេលដែលអ្នកជំនួយការបានប្តូរទៅអាហារថ្មី វាពិបាកនឹងឃើញអ្វីដែលនឹងផ្លាស់ប្តូរជាលំដាប់ក្នុងរយៈពេល 800 ឆ្នាំ។ 5 គឺមិនទំនងសម្រាប់ហេតុផលដូចគ្នានេះ, មិនមានការធ្លាក់ចុះភ្លាមៗ, Muggles មិនបានធ្វើអ្វី 800 ឆ្នាំមុន។ 4 មើលទៅដូចជា 2 និង 5 មើលទៅដូចជា 1 យ៉ាងណាក៏ដោយ។ ដូច្នេះ​ជា​ចម្បង យើង​គួរ​តែ​ព្យាយាម​បែងចែក​រវាង 1, 2, និង 3»។ Harry បង្វែរក្រដាស់មកខ្លួនឯង គូសពងក្រពើជុំវិញលេខទាំងបីនោះ បែរវាមកវិញ។ “វេទមន្ត​កំពុង​តែ​រសាត់​ទៅ​ៗ ឈាម​ក៏​ចុះ​ខ្សោយ ចំណេះ​ដឹង​ក៏​បាត់។ តើ​តេស្ត​មួយ​ណា​ចេញ​មក​ខុស​គ្នា អាស្រ័យ​លើ​មួយ​ណា​ពិត? តើ​យើង​អាច​មើល​ឃើញ​អ្វី​ដែល​មាន​ន័យ​ថា​មួយ​ក្នុង​ចំណោម​ទាំង​នេះ​មិន​ពិត?»។

“\emph{ខ្ញុំ} មិនដឹងទេ!” ធ្វើឱ្យ Draco ព្រិល។ “ហេតុអីក៏សួរខ្ញុំ? អ្នកគឺជាអ្នកវិទ្យាសាស្ត្រ!”

លោក Harry បាននិយាយថា “Draco” Harry បាននិយាយថា “ខ្ញុំគ្រាន់តែដឹងពីអ្វីដែលអ្នកវិទ្យាសាស្ត្រ Muggle ដឹង! អ្នកធំឡើងក្នុងពិភពវេទមន្ត ខ្ញុំអត់ទេ! អ្នកដឹងពីវេទមន្តច្រើនជាងខ្ញុំ ហើយអ្នកដឹង \emph{អំពីវេទមន្ត} ច្រើនជាងខ្ញុំធ្វើ ហើយអ្នកបានគិតពីគំនិតនេះតាំងពីដំបូង ដូច្នេះចាប់ផ្តើមគិតដូចអ្នកវិទ្យាសាស្ត្រ ហើយដោះស្រាយរឿងនេះ!”

Draco លេបទឹកមាត់យ៉ាងខ្លាំង ហើយសម្លឹងមើលក្រដាស។

វេទមន្ត​កំពុង​តែ​រសាត់​ទៅ​ហើយ… អ្នក​ជំនួយ​ការ​កំពុង​បង្កាត់​ពូជ​ជាមួយ Muggles… ចំណេះ​ដឹង​កំពុង​តែ​បាត់…

"តើពិភពលោកមើលទៅដូចអ្វីប្រសិនបើវេទមន្តកំពុងរសាត់?" បាននិយាយថា Harry Potter ។ “អ្នកដឹងអំពីវេទមន្តកាន់តែច្រើន អ្នកគួរតែជាអ្នកដែលស្មានមិនមែនខ្ញុំ! ស្រមៃថាអ្នកកំពុងនិយាយរឿងមួយអំពីវា តើមានអ្វីកើតឡើងនៅក្នុងរឿង?

Draco ស្រមៃមើលវា។ "មន្តស្នេហ៍ដែលធ្លាប់ធ្វើការឈប់ដំណើរការ។" \emph{អ្នកជំនួយការភ្ញាក់ពីដំណេក ហើយដឹងថាដង្កៀបរបស់ពួកគេគឺជាឈើ...}

“តើ​ពិភពលោក​នឹង​ទៅជា​យ៉ាងណា ប្រសិនបើ​ឈាម​វេទមន្ត​កាន់តែ​ខ្សោយ?”

"មនុស្សមិនអាចធ្វើអ្វីដែលដូនតារបស់ពួកគេអាចធ្វើបាននោះទេ។"

"តើពិភពលោកនឹងទៅជាយ៉ាងណា ប្រសិនបើចំណេះដឹងត្រូវបានបាត់បង់?"

Draco បាននិយាយថា "មនុស្សមិនដឹងពីរបៀបដើម្បីដេញ Charms នៅកន្លែងដំបូង ... " ។ គាត់​បាន​ឈប់​ភ្ញាក់ផ្អើល​នឹង​ខ្លួន​គាត់​។ "នោះជាការសាកល្បងមែនទេ?"

Harry ងក់ក្បាលយ៉ាងម៉ឺងម៉ាត់។ "នោះជាមួយ។" គាត់​បាន​សរសេរ​វា​នៅ​លើ​ក្រដាស​ស្នាម​ក្រោម \emph{Tests:}

\emph{A។ តើ​មាន​អក្ខរាវិរុទ្ធ​ដែល​យើង​ដឹង ប៉ុន្តែ​មិន​អាច​ដេញ (1 ឬ 2) ឬ​អក្ខរាវិរុទ្ធ​ដែល​បាត់​នោះ​លែង​ស្គាល់ (3)?}

Harry បាននិយាយថា "ដូច្នេះវាបែងចែករវាង 1 និង 2 នៅលើដៃមួយនិង 3 នៅលើដៃផ្សេងទៀត" ។ "ឥឡូវនេះយើងត្រូវការវិធីមួយចំនួនដើម្បីបែងចែករវាង 1 និង 2 ។ វេទមន្តបន្ថយឈាមចុះខ្សោយ តើយើងអាចប្រាប់ពីភាពខុសគ្នាយ៉ាងដូចម្តេច?"

"តើ Charms បែបណាដែលសិស្សធ្លាប់សម្តែងក្នុងឆ្នាំដំបូងរបស់ពួកគេនៅ Hogwarts?" បាននិយាយថា Draco ។ "ប្រសិនបើពួកគេធ្លាប់អាចបញ្ចេញមន្តស្នេហ៍ដែលមានថាមពលខ្លាំងជាងនេះ ឈាមកាន់តែរឹងមាំ។

Harry Potter ងក់ក្បាល។ “ឬវេទមន្តខ្លួនវាខ្លាំងជាង។ យើង​ត្រូវ​តែ​ស្វែង​រក​វិធី​មួយ​ចំនួន​ក្នុង​ការ​ប្រាប់ \emph{difference}។" Harry ក្រោកឈរឡើងពីកៅអី ហើយចាប់ផ្តើមដើរដោយភ័យពេញថ្នាក់។ “ទេ រង់ចាំ វានៅតែដំណើរការ។ ឧបមាថាអក្ខរាវិរុទ្ធផ្សេងគ្នាប្រើបរិមាណថាមពលវេទមន្តខុសៗគ្នា។ បន្ទាប់មក ប្រសិនបើវេទមន្តជុំវិញចុះខ្សោយ នោះអក្ខរាវិរុទ្ធដ៏មានអានុភាពនឹងស្លាប់មុន ប៉ុន្តែអក្ខរាវិរុទ្ធដែលមនុស្សគ្រប់គ្នារៀនក្នុងឆ្នាំដំបូងរបស់ពួកគេនឹងនៅដដែល…” ភាពភ័យរន្ធត់របស់ Harry កើនឡើង។ “វាមិនមែនជាការសាកល្បងដ៏ល្អនោះទេ វានិយាយអំពីអ្នកជំនួយការដ៏មានអានុភាពត្រូវបានបាត់បង់ ធៀបនឹងអ្នកជំនួយការទាំងអស់ត្រូវបានបាត់បង់ ឈាមរបស់នរណាម្នាក់អាចខ្សោយពេកសម្រាប់អ្នកជំនួយការដ៏មានថាមពល ប៉ុន្តែខ្លាំងគ្រប់គ្រាន់សម្រាប់អក្ខរាវិរុទ្ធដ៏ងាយស្រួល… Draco តើអ្នកដឹងទេថាតើអ្នកជំនួយការដែលមានថាមពលខ្លាំងជាងក្នុង €199 សម័យ €{single} ដូច​ជា​អ្នក​ជំនួយ​ការ​ដ៏​មាន​ឥទ្ធិពល​ពី​សតវត្ស​នេះ តើ​មាន​អំណាច​ជាង​ដូច​កុមារ​ដែរ​ឬ​ទេ? ប្រសិនបើ Dark Lord បានដេញ Cooling Charm នៅពេលគាត់មានអាយុ 11 ឆ្នាំ តើគាត់អាចកកបន្ទប់ទាំងមូលបានទេ?

ទឹក​មុខ​របស់ Draco ឡើង​លើ​ពេល​គាត់​ព្យាយាម​ចងចាំ។ "ខ្ញុំមិនអាចចាំថាបានឮអ្វីអំពី Dark Lord ប៉ុន្តែខ្ញុំគិតថា Dumbledore គួរតែធ្វើអ្វីដែលអស្ចារ្យនៅលើការផ្លាស់ប្តូរ O.W.L.s របស់គាត់នៅឆ្នាំទីប្រាំ...

Harry ធ្វើ​មុខ​មាត់​ដោយ​បន្ត​ដើរ។ “ពួកគេគ្រាន់តែអាចសិក្សាយ៉ាងលំបាក។ ទោះយ៉ាងណាក៏ដោយ ប្រសិនបើសិស្សឆ្នាំទី 1 បានរៀនអក្ខរាវិរុទ្ធដូចគ្នា ហើយហាក់ដូចជាមានថាមពលខ្លាំងដូចពេលនេះ យើងអាចហៅភស្តុតាងដែលមានតម្លៃ \emph{ខ្សោយ} ដែលពេញចិត្តនឹង 1 លើ 2… រង់ចាំ រង់ចាំ។” Harry ឈប់​ត្រង់​កន្លែង​ដែល​គាត់​ឈរ។ "ខ្ញុំមានការធ្វើតេស្តមួយផ្សេងទៀតដែលអាចបែងចែករវាង 1 និង 2 ។ វាត្រូវការពេលបន្តិចដើម្បីពន្យល់ វាប្រើរឿងមួយចំនួនដែលអ្នកវិទ្យាសាស្ត្រដឹងអំពីឈាម និងមរតក ប៉ុន្តែវាជាសំណួរងាយស្រួលក្នុងការសួរ។ ហើយប្រសិនបើយើង \emph{រួមបញ្ចូលគ្នា} ការធ្វើតេស្តរបស់ខ្ញុំ និងការធ្វើតេស្តរបស់អ្នក ហើយពួកគេទាំងពីរចេញមកដូចគ្នា នោះគឺជាការណែនាំដ៏រឹងមាំចំពោះចម្លើយ។” Harry ស្ទើរតែរត់ត្រឡប់ទៅតុវិញ យកក្រដាសជូតមាត់ ហើយសរសេរថា៖

\emph{ខ។ តើ​សិស្ស​ឆ្នាំ​ទី​១​បុរាណ​ដេញ​អក្ខរាវិរុទ្ធ​ដូច​គ្នា​មាន​អំណាច​ដូច​ឥឡូវ​ទេ? (ភ័ស្តុតាងខ្សោយសម្រាប់ 1 លើ 2 ប៉ុន្តែឈាមក៏អាចបាត់បង់តែអ្នកជំនួយការដ៏មានឥទ្ធិពលប៉ុណ្ណោះ។)}

\emph{C។ ការធ្វើតេស្តបន្ថែមដែលបែងចែក 1 និង 2 ដោយប្រើចំណេះដឹងវិទ្យាសាស្ត្រនៃឈាម នឹងពន្យល់នៅពេលក្រោយ។}

Harry បាននិយាយថា "មិនអីទេ" យ៉ាងហោចណាស់យើងអាចព្យាយាមប្រាប់ពីភាពខុសគ្នារវាង 1 និង 2 និង 3 ដូច្នេះយើងទៅជាមួយវាភ្លាមៗ យើងអាចរកឃើញការសាកល្បង \emph{more} បន្ទាប់ពីយើងធ្វើអ្វីដែលយើងរួចរាល់។ មាន ឥឡូវនេះវានឹងមើលទៅចម្លែកបន្តិចប្រសិនបើ Draco Malfoy និង Harry Potter ដើរសួរសំណួរជាមួយគ្នា ដូច្នេះនេះជាគំនិតរបស់ខ្ញុំ។ អ្នកនឹងឆ្លងកាត់ Hogwarts ហើយស្វែងរករូបគំនូរចាស់ៗ ហើយសួរពួកគេអំពីអក្ខរាវិរុទ្ធអ្វីដែលពួកគេបានរៀនដើម្បីសម្ដែងក្នុងកំឡុងឆ្នាំដំបូងរបស់ពួកគេ។ ពួកគេ​ជា​រូប​បញ្ឈរ ដូច្នេះ​ពួកគេ​នឹង​មិន​ដឹង​ថា​មាន​អ្វី​ចម្លែក​ដែល​ Draco Malfoy ធ្វើ​បែប​នោះ​ទេ។ ខ្ញុំនឹងសួររូបថតថ្មីៗ និងអ្នករស់នៅអំពីអក្ខរាវិរុទ្ធដែលយើងដឹង ប៉ុន្តែមិនអាចសម្ដែងបានទេ គ្មាននរណាម្នាក់នឹងកត់សម្គាល់អ្វីដែលមិនធម្មតាទេ ប្រសិនបើ Harry Potter សួរសំណួរចំលែក។ ហើយខ្ញុំនឹងត្រូវធ្វើការស្រាវជ្រាវដ៏ស្មុគស្មាញអំពីអក្ខរាវិរុទ្ធដែលភ្លេច ដូច្នេះខ្ញុំចង់ឱ្យអ្នកជាអ្នកប្រមូលទិន្នន័យដែលខ្ញុំត្រូវការសម្រាប់សំណួរវិទ្យាសាស្ត្រផ្ទាល់ខ្លួនរបស់ខ្ញុំ។ វា​ជា​សំណួរ​សាមញ្ញ​មួយ ហើយ​អ្នក​គួរ​តែ​អាច​រក​ចម្លើយ​ដោយ​ការ​សួរ​រូប​បញ្ឈរ។ អ្នក​ប្រហែល​ជា​ចង់​សរសេរ​វា​ហើយ​ឬ​នៅ?

Draco អង្គុយចុះម្តងទៀត ហើយឆ្លៀតក្នុងកាបូបសៀវភៅរបស់គាត់ ដើម្បីទុកដាក់ក្រដាស់ និងភួយ។ នៅពេលដែលវាត្រូវបានដាក់នៅលើតុ Draco ងើបមុខឡើងដោយទឹកមុខកំណត់។ “ទៅមុខ។”

“ស្វែងរករូបភាពដែលស្គាល់គូស្នេហ៍ Squib ដែលរៀបការហើយកុំធ្វើមុខបែបនោះ Draco វាជាព័ត៌មានសំខាន់។ គ្រាន់​តែ​សួរ​អ្នក​ថត​រូប​ថ្មីៗ​ថា​នរណា​ជា Gryffindors ឬ​អ្វី​មួយ។ ស្វែងរករូបភាពដែលស្គាល់គូស្វាមីភរិយា Squib ឱ្យបានគ្រប់គ្រាន់ដើម្បីស្គាល់ឈ្មោះកូនៗរបស់ពួកគេទាំងអស់។ សរសេរឈ្មោះរបស់ក្មេងម្នាក់ៗ ហើយថាតើក្មេងនោះជាគ្រូមន្ដ ស្គីប ឬ មូហ្គែល។ ប្រសិនបើពួកគេមិនដឹងថាកូននោះជា Squib ឬ Muggle សូមសរសេរថា 'មិនមែនអ្នកជំនួយការ'។ សរសេរវាសម្រាប់ \emph{រាល់} កូនដែលប្តីប្រពន្ធមាន កុំទុកចោល។ ប្រសិនបើរូបបញ្ឈរស្គាល់តែឈ្មោះកូនអ្នកជំនួយប៉ុណ្ណោះ មិនមែនឈ្មោះកូនៗ \emph{all} ទេនោះ កុំសរសេរទិន្នន័យ \emph{ណាមួយ} ពីគូស្នេហ៍នោះ។ វាមានសារៈសំខាន់ខ្លាំងណាស់ដែលអ្នកនាំយកទិន្នន័យមកខ្ញុំពីអ្នកដែលស្គាល់ \emph{ទាំងអស់} ដែលកូនដែលប្តីប្រពន្ធ Squib មាន គ្រប់គ្រាន់ដើម្បីស្គាល់ពួកគេទាំងអស់តាមឈ្មោះ។ ព្យាយាមរកឈ្មោះយ៉ាងហោចណាស់ចំនួនសែសិប បើអ្នកអាចធ្វើបាន ហើយប្រសិនបើអ្នកមានពេលច្រើន កាន់តែប្រសើរ។ តើអ្នកបានទទួលវាទាំងអស់ហើយឬនៅ?

Draco បាននិយាយថា "ធ្វើវាម្តងទៀត" នៅពេលគាត់សរសេរចប់ ហើយ Harry ក៏និយាយម្តងទៀត។

Draco បាននិយាយថា "ខ្ញុំបានទទួលវា ប៉ុន្តែហេតុអ្វី -"

“វាទាក់ទងនឹងអាថ៌កំបាំងមួយនៃឈាម ដែលអ្នកវិទ្យាសាស្ត្របានរកឃើញរួចហើយ។ ខ្ញុំនឹងពន្យល់នៅពេលអ្នកត្រលប់មកវិញ។ បែកគ្នាហើយជួបគ្នាវិញនៅទីនេះក្នុងមួយម៉ោង 6:22\pm ដែលគួរតែជា។ តើ​យើង​ត្រៀម​ខ្លួន​ទៅ​ហើយ​ឬ​នៅ?»

Draco ងក់ក្បាលយ៉ាងម៉ឺងម៉ាត់។ វាប្រញាប់ប្រញាល់ណាស់ ប៉ុន្តែគាត់ត្រូវបានបង្រៀនជាយូរមកហើយពីវិធីប្រញាប់។

“បន្ទាប់មក \emph{go}!” បាននិយាយថា Harry Potter ហើយដោះអាវធំរបស់គាត់ ហើយបោះវាទៅក្នុងកាបូបរបស់គាត់ ដែលចាប់ផ្តើមញ៉ាំវា ហើយដោយមិនចាំថាកាបូបរបស់គាត់ចប់ គាត់ក៏បានបង្វិលជុំវិញ ហើយចាប់ផ្តើមដើរយ៉ាងលឿនឆ្ពោះទៅទ្វារថ្នាក់រៀន បុកទៅនឹងតុ ហើយស្ទើរតែដួល។ នៅក្នុងការប្រញាប់របស់គាត់។

នៅពេលដែល Draco បានដោះអាវផ្ទាល់ខ្លួនរបស់គាត់ ហើយទុកវានៅក្នុងកាបូបសៀវភៅរបស់គាត់ Harry Potter បានបាត់។

Draco ស្ទើរតែរត់ចេញពីទ្វារ។

%  LocalWords:  Oogely boogely oo ee AAAAAAAAARRRRRRGHHHH Aw Mmm Asch’s
%  LocalWords:  Whyyyyyyyyyyyyyyyy AAAAAAAAAAAAAAHHHHHHHHHHHHHHH Gendlin
