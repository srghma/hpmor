\chapter{ផ្អាកលើការស្នើសុំដំណោះស្រាយ}

\begin{chapterOpeningAuthorNote}
ដោយសារវិទ្យាសាស្រ្តនៅក្នុងរឿងនេះជាធម្មតាត្រឹមត្រូវទាំងអស់ ខ្ញុំរួមបញ្ចូលការព្រមានថានៅក្នុងជំពូក~22–25 Harry មើលរំលងលទ្ធភាពជាច្រើន ដែលសំខាន់បំផុតនោះគឺថាមានហ្សែនវេទមន្តជាច្រើន ប៉ុន្តែពួកវាទាំងអស់នៅលើក្រូម៉ូសូមតែមួយ (ដែល វា​នឹង​មិន​កើត​ឡើង​ដោយ​ធម្មជាតិ​ទេ ប៉ុន្តែ​ក្រូម៉ូសូម​ប្រហែល​ជា​ត្រូវ​បាន​បង្កើត​ឡើង​)។ ក្នុងករណីនេះ គំរូមរតកនឹងជា Mendelian ប៉ុន្តែក្រូម៉ូសូមវេទមន្តនៅតែអាចត្រូវបានបន្ទាបបន្ថោកដោយការឆ្លងកាត់ក្រូម៉ូសូមជាមួយនឹងភាពដូចគ្នាដែលមិនមែនជាវេទមន្តរបស់វា។ (Harry បានអានអំពី Mendel និងក្រូម៉ូសូមនៅក្នុងសៀវភៅប្រវត្តិសាស្រ្តវិទ្យាសាស្ត្រ ប៉ុន្តែគាត់មិនបានសិក្សាហ្សែនពិតប្រាកដគ្រប់គ្រាន់ដើម្បីដឹងអំពីការឆ្លងក្រូម៉ូសូមទេ។ ហេ គាត់មានអាយុត្រឹមតែ 11 ឆ្នាំប៉ុណ្ណោះ។) ទោះជាយ៉ាងណាក៏ដោយ ទោះបីជាទស្សនាវដ្តីវិទ្យាសាស្ត្រទំនើបអាចរកបាន \emph{ច្រើនក៏ដោយ។ } ចំណុចសំខាន់ជាច្រើនទៀតដែលត្រូវជ្រើសរើស អ្វីគ្រប់យ៉ាងដែល Harry បង្ហាញជាភស្តុតាងដ៏រឹងមាំគឺជាការពិតភស្តុតាងដ៏រឹងមាំ—លទ្ធភាពផ្សេងទៀតគឺ\emph{improbable}។
\end{chapterOpeningAuthorNote}

\section{ច្បាប់ 2}

(ព្រះអាទិត្យរះយ៉ាងត្រចះត្រចង់ចូលទៅក្នុងមហាសាល ពីពិដានមេឃដ៏ស្រស់បំព្រងខាងលើ បំភ្លឺសិស្សានុសិស្ស ហាក់បីដូចជាពួកគេអង្គុយក្រោមមេឃទទេ ភ្លឺចេញពីចាន និងចានរបស់ពួកគេ ដូចជា ស្រស់ស្រាយដោយការគេងមួយយប់ ពួកគេស្រូបអាហារពេលព្រឹកជាស្រេច។ អ្វីក៏ដោយដែលពួកគេបានធ្វើសម្រាប់ថ្ងៃអាទិត្យរបស់ពួកគេ។ )

\lettrine{S}{o.} មានរឿងតែមួយគត់ដែលធ្វើឱ្យអ្នកក្លាយជាអ្នកជំនួយការ។

វាមិនគួរឱ្យភ្ញាក់ផ្អើលទេ នៅពេលអ្នកគិតអំពីវា។ អ្វីដែល DNA ភាគច្រើនបានធ្វើគឺប្រាប់ ribosomes ពីរបៀបចងអាស៊ីដអាមីណូជាមួយគ្នាទៅជាប្រូតេអ៊ីន។ រូបវិទ្យាធម្មតាហាក់ដូចជាមានសមត្ថភាពក្នុងការពិពណ៌នាអំពីអាស៊ីតអាមីណូ ហើយមិនថាអាស៊ីតអាមីណូប៉ុន្មានដែលអ្នកចងជាប់គ្នានោះទេ រូបវិទ្យាធម្មតាបាននិយាយថា អ្នកនឹងមិនអាចទទួលបានវេទមន្តពីវាឡើយ។

ហើយ​មន្តអាគម​ហាក់​ដូច​ជា​តំណពូជ​តាម DNA។

បន្ទាប់មកប្រហែល\emph{មិនមែន}ទេ ពីព្រោះ DNA កំពុងភ្ជាប់អាស៊ីតអាមីណូដែលមិនមែនជាវេទមន្ត ទៅជាប្រូតេអ៊ីនវេទមន្ត។

ផ្ទុយទៅវិញ លំដាប់ DNA គន្លឹះមិនបានផ្តល់ឱ្យអ្នកនូវវេទមន្តរបស់អ្នកទាល់តែសោះ។

វេទមន្តបានមកពីកន្លែងផ្សេង។

(នៅតុ Ravenclaw មានក្មេងប្រុសម្នាក់ដែលកំពុងសម្លឹងមើលទៅក្នុងលំហ ខណៈដែលដៃស្តាំរបស់គាត់បានស្លាបព្រាអាហារដែលមិនសំខាន់ចូលទៅក្នុងមាត់ដោយស្វ័យប្រវត្តិពីអ្វីដែលនៅពីមុខគាត់។ អ្នកប្រហែលជាអាចជំនួសគំនរកខ្វក់បាន ហើយគាត់នឹងមិន បានកត់សម្គាល់។ )

ហើយសម្រាប់ហេតុផលមួយចំនួន ប្រភពនៃវេទមន្តបានយកចិត្តទុកដាក់លើសញ្ញាសម្គាល់ DNA ជាក់លាក់មួយក្នុងចំណោមបុគ្គលទាំងឡាយណាដែលជាមនុស្សធម្មតាដែលកើតពីសត្វស្វាតាមវិធីផ្សេងៗ។

(តាមពិតទៅ មានក្មេងប្រុស និងក្មេងស្រីជាច្រើនកំពុងសម្លឹងមើលទៅក្នុងលំហ។ វាជាតុ\emph{Ravenclaw}។

មានបន្ទាត់នៃតក្កវិជ្ជាផ្សេងទៀតដែលនាំទៅដល់ការសន្និដ្ឋានដូចគ្នា។ \emph{Complex} គ្រឿងម៉ាស៊ីនគឺតែងតែជាសកលនៅក្នុងប្រភេទបន្តពូជផ្លូវភេទ។ ប្រសិនបើហ្សែន B ពឹងផ្អែកលើហ្សែន A នោះ A ត្រូវតែមានប្រយោជន៍ដោយខ្លួនឯង ហើយកើនឡើងដល់កម្រិតសកលនៅក្នុងក្រុមហ្សែនដោយខ្លួនឯង មុនពេល B នឹងមានប្រយោជន៍ច្រើនគ្រប់គ្រាន់ដើម្បីផ្តល់អត្ថប្រយោជន៍ផ្នែកកាយសម្បទា។ បន្ទាប់មកនៅពេលដែល B ជាសកល អ្នកនឹងទទួលបានបំរែបំរួល A* ដែលពឹងផ្អែកលើ B ហើយបន្ទាប់មក C ដែលពឹងផ្អែកលើ A* និង B បន្ទាប់មក B* ដែលពឹងផ្អែកលើ C រហូតដល់ម៉ាស៊ីនទាំងមូលនឹងដាច់ ប្រសិនបើអ្នកដកដុំតែមួយចេញ។ ប៉ុន្តែវាទាំងអស់ត្រូវតែកើតឡើង\emph{incrementally}—ការវិវត្តន៍មិនដែលមើលទៅខាងមុខទេ ការវិវត្តន៍នឹងមិនចាប់ផ្តើមផ្សព្វផ្សាយ B ក្នុង\emph{preparation} សម្រាប់ A ក្លាយជាសកលនៅពេលក្រោយទេ។ ការវិវត្តន៍គឺជារឿងប្រវត្តិសាស្ត្រដ៏សាមញ្ញមួយ ដែលថា សរីរាង្គណាក៏ដោយដែលតាមពិតមានកូនច្រើនជាងគេ ហ្សែនរបស់ពួកគេនឹងកើតមានញឹកញាប់ជាងនៅជំនាន់ក្រោយ។ ដូច្នេះបំណែកនីមួយៗនៃម៉ាស៊ីនស្មុគស្មាញត្រូវតែក្លាយជាសកលស្ទើរតែទាំងអស់ មុនពេលបំណែកផ្សេងទៀតនៅក្នុងម៉ាស៊ីននឹងវិវឌ្ឍន៍អាស្រ័យលើវត្តមានរបស់វា។

ដូច្នេះ \emph{ស្មុគ្រស្មាញ, អាស្រ័យគ្នាទៅវិញទៅមក} ម៉ាស៊ីនដែលជាម៉ាស៊ីនប្រូតេអ៊ីនដ៏ទំនើបដ៏មានឥទ្ធិពលដែលជំរុញជីវិតគឺតែងតែ \emph{សកល} នៅក្នុងប្រភេទសត្វបន្តពូជដោយផ្លូវភេទ លើកលែងតែមួយក្តាប់តូច \emph{មិនមែន} - អាស្រ័យគ្នាទៅវិញទៅមក \emph{វ៉ារ្យ៉ង់} ដែលកំពុងត្រូវបានជ្រើសរើសនៅពេលណាមួយ ដោយសារតែភាពស្មុគស្មាញបន្ថែមទៀតត្រូវបានដាក់ចុះបន្តិចម្តងៗ។ នោះហើយជាមូលហេតុដែលមនុស្សទាំងអស់មានការរចនាខួរក្បាលដូចគ្នា អារម្មណ៍ដូចគ្នា ទឹកមុខដូចគ្នា ភ្ជាប់ទៅនឹងអារម្មណ៍ទាំងនោះ។ ការសម្របខ្លួនទាំងនោះមានភាពស្មុគស្មាញ ដូច្នេះពួកគេមានតម្លៃ\emph{ត្រូវតែ}មានលក្ខណៈជាសកល។

ប្រសិនបើវេទមន្តមានលក្ខណៈបែបនោះ ការសម្របខ្លួនដ៏ស្មុគ្រស្មាញដ៏ធំជាមួយនឹងហ្សែនចាំបាច់ជាច្រើន នោះអ្នកជំនួយការដែលរួមរស់ជាមួយ Muggle នឹងនាំឱ្យកុមារមានផ្នែកទាំងនោះត្រឹមតែពាក់កណ្តាល ហើយម៉ាស៊ីនពាក់កណ្តាលនឹងមិនធ្វើអ្វីច្រើននោះទេ។ ដូច្នេះហើយ វានឹងមិនមានកំណើត Muggle ដែលមិនធ្លាប់មាន។ ទោះបីជាបំណែកទាំងអស់បានចូលទៅក្នុងអាងហ្សែន Muggle ដោយឡែកពីគ្នាក៏ដោយ ពួកគេមិនដែលប្រមូលផ្តុំទាំងអស់នៅកន្លែងតែមួយដើម្បីបង្កើតអ្នកជំនួយការនោះទេ។

មិនមានជ្រលងភ្នំដាច់ស្រយាលពីហ្សែនរបស់មនុស្សដែលបានជំពប់ដួលលើផ្លូវវិវត្តន៍ដែលនាំទៅដល់ផ្នែកវេទមន្តដ៏ទំនើបនៃខួរក្បាលនោះទេ។ គ្រឿងចក្រហ្សែនដ៏ស្មុគស្មាញនោះ ប្រសិនបើអ្នកជំនួយការបង្កាត់ពូជជាមួយ Muggles នឹងមិនដែលបានផ្គុំគ្នាឡើងវិញជា Muggle-កើតទេ។

ដូច្នេះទោះជាយ៉ាងណាក៏ដោយ ហ្សែនរបស់អ្នកបានធ្វើឱ្យអ្នកក្លាយជាអ្នកជំនួយការ វា \emph{មិនមែន} ដោយវាមានប្លង់មេសម្រាប់គ្រឿងម៉ាស៊ីនដ៏ស្មុគស្មាញ។

នោះគឺជាហេតុផលផ្សេងទៀតដែល Harry បានទាយថាលំនាំ Mendelian នឹងនៅទីនោះ។ ប្រសិនបើហ្សែនវេទមន្តមិនស្មុគស្មាញ ហេតុអ្វីបានជាមានច្រើនជាងមួយ?

ប៉ុន្តែ​វេទមន្ត​ដោយ​ខ្លួន​វា​ហាក់​ដូច​ជា​ស្មុគស្មាញ​គួរ​ឱ្យ​កត់​សម្គាល់​។ អក្ខរាវិរុទ្ធនៃការចាក់សោទ្វារនឹងរារាំងទ្វារមិនឱ្យបើក\emph{និង} រារាំងអ្នកពីការផ្លាស់ប្តូរសោរ \emph{and} ទប់ទល់\emph{Finite Incantatem} និង \emph{Alohomora} ។ ធាតុជាច្រើនដែលចង្អុលទៅទិសតែមួយ៖ អ្នកអាចហៅទិសដៅគោលដៅនោះ ឬក្នុងភាសាសាមញ្ញជាងថា គោលបំណង។

មានមូលហេតុដែលគេស្គាល់តែពីរប៉ុណ្ណោះនៃភាពស្មុគស្មាញដែលមានគោលបំណង។ ការជ្រើសរើសធម្មជាតិដែលផលិតវត្ថុដូចជាមេអំបៅ។ និងវិស្វកម្មឆ្លាតវៃដែលផលិតរបស់ដូចជារថយន្ត។

វេទមន្តហាក់ដូចជាមិនដូចអ្វីដែលបានចម្លងដោយខ្លួនឯងនៅក្នុងអត្ថិភាពនោះទេ។ អក្ខរាវិរុទ្ធមានគោលបំណងស្មុគស្មាញ ប៉ុន្តែមិនមែនដូចជាមេអំបៅទេ ដែលស្មុគស្មាញក្នុងគោលបំណងបង្កើតច្បាប់ចម្លងដោយខ្លួនឯង។ អក្ខរាវិរុទ្ធមានភាពស្មុគស្មាញក្នុងគោលបំណងបម្រើអ្នកប្រើប្រាស់របស់ពួកគេ ដូចជាឡាន។

បន្ទាប់មក វិស្វករឆ្លាតវៃមួយចំនួនបានបង្កើតប្រភពវេទមន្ត ហើយបានប្រាប់វាឱ្យយកចិត្តទុកដាក់ចំពោះសញ្ញាសម្គាល់ DNA ជាក់លាក់មួយ។

គំនិតបន្ទាប់ដែលច្បាស់នោះគឺថា វាមានអ្វីដែលត្រូវធ្វើជាមួយ "Atlantis" ។

Harry បានសួរ Hermione អំពីរឿងនោះមុននេះ - នៅលើរថភ្លើងទៅ Hogwarts បន្ទាប់ពីបានលឺ Draco និយាយ - ហើយរហូតមកដល់ពេលនេះនាងដឹង គ្មានអ្វីត្រូវបានគេស្គាល់ក្រៅពីពាក្យនោះទេ។

វាប្រហែលជារឿងព្រេងនិទានសុទ្ធ។ ប៉ុន្តែវាក៏អាចជឿជាក់បានផងដែរដែលថាអារ្យធម៌នៃអ្នកប្រើប្រាស់វេទមន្ត ជាពិសេសអ្នកដែលមានតម្លៃចាប់ពី \emph{មុននឹង} Interdict of Merlin អាចនឹងផ្ទុះឡើង។

បន្ទាត់នៃហេតុផលបានបន្ត៖ អាត្លង់ទីធ្លាប់ជាអារ្យធម៌ដាច់ស្រយាលមួយដែលបាននាំយកមកជាប្រភពនៃវេទមន្ត ហើយបានប្រាប់វាឱ្យបម្រើតែមនុស្សដែលមានសញ្ញាសម្គាល់ហ្សែនអាត្លង់ទីន ដែលជាឈាមរបស់អាត្លង់ទី។

ហើយតាមតក្កវិជ្ជាស្រដៀងគ្នានេះ៖ ពាក្យដែលអ្នកជំនួយការបាននិយាយ ចលនារបស់ wand នោះមិនស្មុគស្មាញគ្រប់គ្រាន់សម្រាប់ខ្លួនគេក្នុងការបង្កើតឥទ្ធិពលអក្ខរាវិរុទ្ធពីដំបូងឡើយ មិនមែនជាវិធីដែល DNA របស់មនុស្សចំនួនបីពាន់លានគូពិតប្រាកដគឺ\emph{ } មានភាពស្មុគស្មាញគ្រប់គ្រាន់ក្នុងការសាងសង់រាងកាយមនុស្សពីដំបូង មិនមែនជាវិធីដែលកម្មវិធីកុំព្យូទ័រយកទិន្នន័យរាប់ពាន់បៃនោះទេ។

ដូច្នេះពាក្យ និងចលនារបស់ wand គ្រាន់តែជាការកេះប៉ុណ្ណោះ គ្រាប់ចុចទាញនៅលើម៉ាស៊ីនដែលលាក់ និងស្មុគស្មាញជាង។ ប៊ូតុង មិនមែនប្លង់មេទេ។

ហើយដូចជាកម្មវិធីកុំព្យូទ័រនឹងមិនចងក្រងប្រសិនបើអ្នកធ្វើកំហុសអក្ខរាវិរុទ្ធតែមួយនោះ ប្រភពនៃវេទមន្តនឹងមិនឆ្លើយតបមកអ្នកទេ លុះត្រាតែអ្នកសរសេរអក្ខរាវិរុទ្ធរបស់អ្នកតាមរបៀបត្រឹមត្រូវ។

ខ្សែសង្វាក់នៃតក្កវិជ្ជាគឺមិនអាចកាត់ថ្លៃបាន។

ហើយវាបាននាំទៅរកការសន្និដ្ឋានចុងក្រោយតែមួយដោយជៀសមិនរួច។

បុព្វបុរស​របស់​វេទមន្ត​រាប់ពាន់​ឆ្នាំមុន​បាន​ប្រាប់​ប្រភព​នៃ​វេទមន្ត​ថា​គ្រាន់តែ​ធ្វើ​ឲ្យ​មាន​រឿង​អ្វី​កើតឡើង​ប្រសិនបើ​អ្នក​និយាយថា…

"Wingardium Leviosa" ។

Harry ដួល​នៅ​តុ​អាហារ​ពេល​ព្រឹក ដោយ​សម្រាក​ថ្ងាស​យ៉ាង​នឿយ​ហត់​នៅ​ដៃ​ស្តាំ​របស់​គាត់។

មានរឿងមួយតាំងពីព្រឹកព្រលឹមនៃ Artificial Intelligence - ត្រលប់មកវិញនៅពេលដែលពួកគេទើបតែចាប់ផ្តើមដំណើរការ ហើយគ្មាននរណាម្នាក់មិនទាន់ដឹងថាបញ្ហានឹងពិបាកនោះទេ - អំពីសាស្រ្តាចារ្យដែលបានផ្ទេរសិទ្ធិឱ្យសិស្សថ្នាក់ទីមួយរបស់គាត់ដើម្បីដោះស្រាយបញ្ហានៃចក្ខុវិស័យកុំព្យូទ័រ។ .

Harry ចាប់ផ្តើមយល់ពីរបៀបដែលសិស្សថ្នាក់ទី 1 ត្រូវតែមានអារម្មណ៍។

វាអាចចំណាយពេលបន្តិច។

ហេតុអ្វី​បាន​ជា​វា​ត្រូវ​ការ​ការ​ប្រឹងប្រែង​បន្ថែម​ទៀត​ដើម្បី​បញ្ចេញ​អក្ខរាវិរុទ្ធ Alohomora បើ​វា​ដូច​ជា​ការ​ចុច​ប៊ូតុង​មួយ?

តើអ្នកណាដែលឆ្កួតល្មមអាចបង្កើតអក្ខរាវិរុទ្ធក្នុងតម្លៃ \emph{Avada Kedavra} ដែលអាចត្រូវបានដេញដោយប្រើតែការស្អប់?

ហេតុអ្វីបានជាការប្រែរូបដោយគ្មានពាក្យតម្រូវឱ្យអ្នកធ្វើការបំបែកផ្លូវចិត្តទាំងស្រុងរវាងគំនិតនៃទម្រង់ និងគំនិតនៃសម្ភារៈ?

Harry ប្រហែលជាមិនត្រូវបានបញ្ចប់ជាមួយនឹងបញ្ហានេះទេនៅពេលគាត់បញ្ចប់ការសិក្សានៅ Hogwarts ។ គាត់នៅតែអាចធ្វើការលើបញ្ហានេះនៅពេលដែលគាត់មានអាយុ \emph{សាមសិបឆ្នាំ}។ Hermione និយាយត្រូវ Harry \emph{មិនបាន} ដឹងរឿងនោះពីមុនមក។ គាត់ទើបតែបានថ្លែងសុន្ទរកថាដ៏បំផុសគំនិតអំពីការប្តេជ្ញាចិត្ត។

ចិត្តរបស់ Harry បានពិចារណាយ៉ាងខ្លីថាតើត្រូវឡើងដល់កម្រិតវៀនដែលគាត់មិនអាចដោះស្រាយបញ្ហាបានទាល់តែសោះ បន្ទាប់មកបានសម្រេចចិត្តថានឹងទៅឆ្ងាយពេក។

លើសពីនេះ ដរាបណាគាត់អាចទទួលបានភាពអមតៈក្នុងប៉ុន្មានទសវត្សរ៍ដំបូង គាត់នឹងមិនអីទេ។

តើ Dark Lord បានប្រើវិធីអ្វី? ចូរគិតមក ការពិតដែលថា Dark Lord បានគ្រប់គ្រងដើម្បីរស់រានមានជីវិតពីមរណភាពនៃរូបកាយដំបូងរបស់គាត់គឺស្ទើរតែ \emph{គ្មានកំណត់} សំខាន់ជាងការពិតដែលថាគាត់បានព្យាយាមកាន់កាប់ចក្រភពអង់គ្លេសវេទមន្តទៅទៀត។

«សុំទោស» សំឡេងដែលរំពឹងទុកពីក្រោយគាត់ក្នុងទឹកដមដែលនឹកស្មានមិនដល់។ "តាមភាពងាយស្រួលរបស់អ្នក លោក ~ Malfoy ស្នើសុំការពេញចិត្តក្នុងការសន្ទនា។"

Harry មិន​បាន​ញាក់​សាច់​ពេល​ព្រឹក​របស់​គាត់​ទេ។ ផ្ទុយទៅវិញ គាត់បានងាកមកមើល Mr~Crabbe។

Harry បាននិយាយថា “សុំទោស\emph{ខ្ញុំ}”។ «ឯងមិនមានន័យថា 'ចៅហ្វាយដាចង់និយាយជាមួយឯងឬ?'

Mr~Crabbe មើលទៅមិនសប្បាយចិត្តទេ។ "Mr~ Malfoy ណែនាំខ្ញុំឱ្យនិយាយឱ្យបានត្រឹមត្រូវ"

Harry បាននិយាយថា "ខ្ញុំមិនអាចស្តាប់អ្នកបានទេ" ។ "អ្នកនិយាយមិនត្រឹមត្រូវ" គាត់បានត្រឡប់មកចានផ្កាគ្រីស្តាល់ពណ៌ខៀវដ៏តូចរបស់គាត់ ហើយញ៉ាំស្លាបព្រាមួយទៀតដោយចេតនា។

«ចៅហ្វាយ​ដា​ចង់​និយាយ​ជាមួយ​ឯង» សំឡេង​គំរាម​មក​ពី​ក្រោយ​គាត់។ "ទៅជួបគាត់ប្រសើរជាង បើដឹងថាល្អសម្រាប់អ្នក"

នៅទីនោះ។ \emph{ឥឡូវ​នេះ} អ្វីៗ​បាន​ដំណើរការ​ទៅ​តាម​គម្រោង។

\latersection{ច្បាប់ 1}

“\emph{ហេតុផល}?” បាននិយាយថាអ្នកជំនួយការចាស់។ គាត់បានទប់កំហឹងចេញពីមុខរបស់គាត់។ ក្មេង​ប្រុស​មុន​គាត់​ជា​ជន​រង​គ្រោះ ហើយ​ប្រាកដ​ជា​មិន​បាច់​ភ័យ​ខ្លាច​អ្វី​ទៀត​ទេ។ “មាន\emph{គ្មានអ្វី}ដែលអាចដោះសារបាន—”

"អ្វីដែលខ្ញុំបានធ្វើចំពោះគាត់គឺអាក្រក់ជាង" ។

អ្នកជំនួយការចាស់បានរឹងរូសក្នុងភាពភ័យរន្ធត់ភ្លាមៗ។ “Harry \emph{តើអ្នកបានធ្វើអ្វី?}”

"ខ្ញុំបានបោកបញ្ឆោត Draco ឱ្យជឿថាខ្ញុំបានបោកបញ្ឆោតគាត់ឱ្យចូលរួមក្នុងពិធីសាសនាដែលលះបង់ជំនឿរបស់គាត់លើភាពបរិសុទ្ធនៃឈាម។ ហើយនោះមានន័យថា គាត់មិនអាចក្លាយជា Death Eater នៅពេលគាត់ធំឡើងនោះទេ។ គាត់បានបាត់បង់អ្វីៗទាំងអស់ លោកគ្រូ។

មាន​ភាព​ស្ងប់ស្ងាត់​យ៉ាង​យូរ​នៅ​ក្នុង​ការិយាល័យ​ដែល​ខូច​ដោយ​សំឡេង​តូច​ៗ​និង​ការ​ហួច​នៃ​អ្វី​ដែល​មិន​គួរ​ឱ្យ​ជឿ ដែល​ពេល​វេលា​គ្រប់គ្រាន់​បាន​មក​ហាក់​ដូច​ជា​ភាព​ស្ងៀម​ស្ងាត់។

អ្នកជំនួយការចាស់បាននិយាយថា "សូមគោរពខ្ញុំ" ខ្ញុំ \emph{do} មានអារម្មណ៍ឆ្កួត។ ហើយ\emph{នៅទីនេះ} ខ្ញុំរំពឹងថាអ្នកអាចនឹងព្យាយាមប្រោសលោះអ្នកស្នងមរតករបស់ Malfoy ដោយនិយាយថា \emph{បង្ហាញពីមិត្តភាព និងសេចក្តីសប្បុរសពិតដល់គាត់}។

“\emph{ហា!} បាទ ដូចជា \emph{នោះ} នឹងបានដំណើរការ។

អ្នកជំនួយការចាស់បានដកដង្ហើមធំ។ នេះបាននាំវាទៅឆ្ងាយ។ “ប្រាប់ខ្ញុំមក Harry ។ តើវាសូម្បីតែ\emph{{កើតឡើង}ចំពោះអ្នកទេថាមានអ្វីមួយ\emph{មិនស៊ីសង្វាក់គ្នា}អំពីការរៀបចំដើម្បីលោះនរណាម្នាក់តាមរយៈការភូតកុហក និងការបោកប្រាស់មែនទេ?

"ខ្ញុំបានធ្វើវាដោយមិននិយាយកុហកដោយផ្ទាល់ទេ ហើយចាប់តាំងពីយើងកំពុងនិយាយអំពី Draco Malfoy នៅទីនេះ ខ្ញុំគិតថាពាក្យដែលអ្នកកំពុងស្វែងរកគឺ\emph{congruous}។" ក្មេង​ប្រុស​មើល​ទៅ​គួរ​ឲ្យ​ធុញ។

អ្នកជំនួយការចាស់គ្រវីក្បាលដោយអស់សង្ឃឹម។ “ហើយ\emph{នេះ}គឺជាវីរបុរស។ យើង​ទាំង​អស់​គ្នា​ត្រូវ​វិនាស»។

\latersection{ច្បាប់ 5}

ផ្លូវរូងក្រោមដីដ៏វែង និងតូចចង្អៀតនៃថ្មរដុប ដែលមិនមានពន្លឺ លើកលែងតែដង្កៀបរបស់កុមារ ហាក់ដូចជាលាតសន្ធឹងរាប់សិបគីឡូម៉ែត្រ។

ហេតុផលសម្រាប់នេះគឺសាមញ្ញ៖ វា \emph{did} លាតសន្ធឹងសម្រាប់ម៉ាយល៍។

ពេលវេលាគឺម៉ោង 3 ព្រឹក ហើយ Fred និង George កំពុងចាប់ផ្តើមផ្លូវដ៏វែងឆ្ងាយទៅកាន់ផ្លូវសម្ងាត់ដែលនាំពីរូបសំណាករបស់មេធ្មប់ដែលមានភ្នែកម្ខាងនៅ Hogwarts ទៅកាន់បន្ទប់ក្រោមដីនៃហាងបង្អែម Honeydukes នៅ Hogsmeade ។

“យ៉ាងម៉េចហើយ?” Fred បាននិយាយដោយសំឡេងទាប។

(មិន​មែន​ថា​នឹង​មាន​អ្នក​ស្តាប់​នោះ​ទេ ប៉ុន្តែ​មាន​អ្វី​ចម្លែក​អំពី​ការ​និយាយ​ជា​សំឡេង​ធម្មតា​នៅ​ពេល​ដែល​អ្នក​ឆ្លង​កាត់​ផ្លូវ​សម្ងាត់​មួយ​។ )

George បាននិយាយថា "នៅតែមិនព្រិចភ្នែក" ។

“ទាំងពីរ ឬ —”

“ បណ្តោះអាសន្នមួយបានជួសជុលខ្លួនឯងម្តងទៀត។ មួយ​ទៀត​ក៏​ដូច​គ្នា​ដែរ»។

ផែនទីគឺជាវត្ថុបុរាណដ៏មានអានុភាពដ៏វិសេសវិសាល ដែលអាចតាមដានរាល់អារម្មណ៍ដែលស្ថិតនៅលើបរិវេណសាលា តាមពេលវេលាជាក់ស្តែង តាមឈ្មោះ។ ស្ទើរតែប្រាកដណាស់ វាត្រូវបានបង្កើតឡើងក្នុងអំឡុងពេលការចិញ្ចឹមដើមរបស់ Hogwarts ។ វាគឺ \emph{មិនល្អ} ដែលកំហុសកំពុងចាប់ផ្តើមលេចឡើង។ ឱកាសគឺថាគ្មាននរណាម្នាក់លើកលែងតែ Dumbledore អាចជួសជុលវាបានប្រសិនបើវាខូច។

ហើយកូនភ្លោះ Weasley មិនចង់បង្វែរផែនទីទៅ Dumbledore ទេ។ វានឹងក្លាយជាការប្រមាថដែលមិនអាចលើកលែងបានចំពោះ Marauders ដែលជាជនមិនស្គាល់មុខបួននាក់ដែលបានលួចផ្នែកមួយនៃ\emph{ប្រព័ន្ធសុវត្ថិភាព Hogwarts} អ្វីមួយដែលប្រហែលជាត្រូវបានក្លែងបន្លំដោយ Salazar Slytherin ខ្លួនឯង ហើយបង្វែរវាទៅជា\emph{ឧបករណ៍មួយ។ សម្រាប់ការលេងសើចរបស់សិស្ស}។

អ្នក​ខ្លះ​ប្រហែល​ជា​បាន​ចាត់​ទុក​វា​ជា​ការ​មិន​គោរព។

អ្នក​ខ្លះ​អាច​ចាត់​ទុក​វា​ជា​ឧក្រិដ្ឋកម្ម។

កូនភ្លោះ Weasley ជឿជាក់យ៉ាងមុតមាំថា ប្រសិនបើ Godric Gryffindor នៅជុំវិញដើម្បីឃើញវា គាត់នឹងយល់ព្រម។

បងប្អូន​បាន​ដើរ​បន្ត​ទៅ​មុខ ភាគច្រើន​នៅ​ស្ងៀម។ កូនភ្លោះ Weasley បាន​និយាយ​គ្នា​ទៅវិញទៅមក នៅពេល​ពួកគេ​កំពុង​គិត​តាមរយៈ​ការលេងសើច​ថ្មីៗ ឬ​នៅពេលដែល​ពួកគេ​ម្នាក់​បានដឹង​រឿង​មួយ​ផ្សេងទៀត​មិនបាន។ បើមិនដូច្នោះទេមិនមានចំណុចច្រើនទេ។ ប្រសិនបើពួកគេដឹងព័ត៌មានដូចគ្នារួចហើយ ពួកគេមានទំនោរគិតតែគំនិតដូចគ្នា និងធ្វើការសម្រេចចិត្តដូចគ្នា។

(ត្រលប់ទៅសម័យបុរាណ នៅពេលណាដែលកូនភ្លោះដូចគ្នាបេះបិទ វាជាទម្លាប់ក្នុងការសម្លាប់ពួកគេម្នាក់បន្ទាប់ពីកំណើត។ )

យូរៗទៅ ហ្វ្រេដ និង ចច បានលបចូលទៅក្នុងបន្ទប់ក្រោមដីដែលពោរពេញដោយធូលី ហើយរាយប៉ាយដោយធុង និងធុងចំរុះ។

Fred និង George បានរង់ចាំ។ វានឹងមិនមានសុជីវធម៌ក្នុងការធ្វើអ្វីផ្សេងទេ។

មុន​ពេល​យូរ​ពេក បុរស​ចំណាស់​រាង​ស្គម​ក្នុង​ឈុត​គេង​ពណ៌​ខ្មៅ​បាន​ដើរ​ចុះ​តាម​ជំហាន​ដែល​ចូល​ទៅ​ក្នុង​បន្ទប់​ក្រោមដី​ទាំង​យំ។ Ambrosius Flume បាននិយាយថា "ជំរាបសួរក្មេងប្រុស" ។ “ខ្ញុំមិនបានរំពឹងអ្នកយប់នេះទេ។ អស់ស្តុកហើយ?

Fred និង George បានសម្រេចចិត្តថា Fred នឹងនិយាយ។

Fred បាននិយាយថា "មិនច្បាស់ទេលោក ~ Flume" ។ "យើងសង្ឃឹមថាអ្នកអាចជួយពួកយើងជាមួយនឹងអ្វីមួយដែលគួរឱ្យកត់សម្គាល់បន្ថែមទៀត ... គួរឱ្យចាប់អារម្មណ៍" ។

Flume បាននិយាយថា "ឥឡូវនេះក្មេងប្រុស" Flume បាននិយាយថា "ខ្ញុំសង្ឃឹមថាអ្នកមិនបានដាស់ខ្ញុំឱ្យភ្ញាក់ដូច្នេះខ្ញុំអាចប្រាប់អ្នកម្តងទៀតថាខ្ញុំមិនលក់ទំនិញណាមួយដែលអាចធ្វើឱ្យអ្នកមានបញ្ហាពិតប្រាកដនោះទេ។ មិន​មែន​រហូត​ដល់​អ្នក​មាន​អាយុ​ដប់ប្រាំ​មួយ​ឆ្នាំ​ក៏​ដោយ—»

George បានទាញវត្ថុមួយចេញពីអាវរបស់គាត់ ហើយបានបញ្ជូនវាទៅ Flume ដោយគ្មានពាក្យ។ "តើអ្នកបានឃើញរឿងនេះទេ?" Fred បាននិយាយ។

Flume បានមើលការបោះពុម្ភកាលពីម្សិលមិញនៃ\emph{Daily Prophet} ហើយងក់ក្បាលដោយងក់ក្បាល។ ចំណងជើងនៅលើក្រដាសនោះសរសេរថា "ព្រះអម្ចាស់ងងឹតបន្ទាប់?" ហើយ​បាន​បង្ហាញ​ក្មេង​ប្រុស​ម្នាក់​ដែល​កាមេរ៉ា​របស់​សិស្ស​មួយ​ចំនួន​បាន​ចាប់​បាន​ក្នុង​ទឹក​មុខ​ត្រជាក់​និង​ក្រៀមក្រំ​ដែល​មិន​មាន​លក្ខណៈ។

Flume បាននិយាយថា "ខ្ញុំមិនអាចជឿថា Malfoy នោះទេ" ។ «តាម​ក្មេង​ប្រុស​អាយុ​តែ​១១​ឆ្នាំ! បុរស​នោះ​គួរ​តែ​ចេះ​ធ្វើ​សូកូឡា​!»

ហ្វ្រេដ និងចច ព្រិចភ្នែកដោយឯកច្ឆន្ទ។ \emph{Malfoy} នៅពីក្រោយ Rita Skeeter? Harry Potter មិន​បាន​ព្រមាន​ពួក​គេ​អំពី​រឿង​នោះ​ទេ… ដែល​ប្រាកដ​ជា​មាន​ន័យ​ថា Harry មិន​បាន​ដឹង។ គាត់​មិន​ដែល​នាំ​គេ​ចូល​ទេ បើ​គាត់​ធ្វើ…

ហ្វ្រេដ និងចច បានផ្លាស់ប្តូរការក្រឡេកមើល។ ជាការប្រសើរណាស់, Harry មិនបាន\emph{need} ដើម្បីដឹងរហូតដល់បន្ទាប់ពីការងារនេះត្រូវបានបញ្ចប់។

"លោក ~ Flume" Fred បាននិយាយដោយស្ងៀមស្ងាត់ "ក្មេងប្រុសដែលរស់នៅត្រូវការជំនួយរបស់អ្នក" ។

Flume សម្លឹងមើលពួកគេទាំងពីរ។

បន្ទាប់មកគាត់ដកដង្ហើមធំដោយដកដង្ហើមធំ។

"មិនអីទេ" Flume បាននិយាយថា "តើអ្នកចង់បានអ្វី?"

\latersection{ច្បាប់ 6}

នៅពេលដែល Rita Skeeter មានបំណងចង់ចាប់សត្វដែលមានរសជាតិឈ្ងុយឆ្ងាញ់ នាងមិនមានទំនោរក្នុងការកត់សម្គាល់ឃើញស្រមោចស្រមោចដែលបង្កើតបានជាចក្រវាឡដែលនៅសេសសល់នោះទេ ដែលជារបៀបដែលនាងស្ទើរតែបុកបុរសទំពែកដែលឈានជើងចូលផ្លូវរបស់នាង។

“កញ្ញា~Skeeter” បុរសនោះនិយាយដោយសំឡេងធ្ងន់ និងត្រជាក់ចំពោះអ្នកដែលមុខមើលទៅក្មេង។ "Fancy រត់មករកអ្នកនៅទីនេះ"

«ចេញ​ពី​ផ្លូវ​ខ្ញុំ​អ្នក​បួស!» ចាប់ Rita ហើយព្យាយាមដើរជុំវិញគាត់។

បុរសនៅក្នុងផ្លូវរបស់នាងបានផ្គូផ្គងនឹងចលនាយ៉ាងល្អឥតខ្ចោះ ដែលវាហាក់ដូចជាពួកគេទាំងពីរមិនបានរើអ្វីទាំងអស់ គ្រាន់តែឈរស្ងៀមខណៈពេលដែលផ្លូវផ្លាស់ប្តូរជុំវិញពួកគេ។

ភ្នែករបស់រីតាបានរួមតូច។ "តើអ្នកគិតថាអ្នកជានរណា?"

បុរស​នោះ​និយាយ​ទាំង​ស្ងួត​ថា​៖ «​ល្ងង់​យ៉ាង​ណា​»។ "វាជាការល្អក្នុងការទន្ទេញចាំមុខអ្នកបំផ្លិចបំផ្លាញ Death Eater ដែលបង្វឹក Harry Potter ឱ្យក្លាយជា Dark Lord បន្ទាប់។ យ៉ាងណាមិញ” ស្នាមញញឹមស្តើង “\emph{នោះ} ច្បាស់ជាស្តាប់ទៅដូចជានរណាម្នាក់ដែលអ្នកមិនចង់រត់ចូលតាមផ្លូវ ជាពិសេសបន្ទាប់ពីធ្វើការងារលើគាត់នៅក្នុងកាសែត។”

រីតាបានចំណាយពេលមួយភ្លែតដើម្បីដាក់ឯកសារយោង។ \emph{នេះ} ជា Quirinus Quirrell? គាត់មើលទៅក្មេងពេក ហើយចាស់ពេកក្នុងពេលតែមួយ។ មុខរបស់គាត់ ប្រសិនបើវាធូរស្រាលពីស្ថានភាពធ្ងន់ធ្ងរ និងចុះអន់ថយ នោះនឹងក្លាយជារបស់អ្នកណាម្នាក់ក្នុងវ័យសាមសិបចុងរបស់គាត់។ ហើយ​សក់​របស់​គាត់​បាន​ជ្រុះ​បាត់​ហើយ? តើគាត់មិនអាចមានលទ្ធភាពព្យាបាលបានទេ?

ទេ នោះមិនសំខាន់ទេ នាងមានពេលវេលា និងទីកន្លែង និងសត្វកណ្តៀរ។ នាងទើបតែទទួលបានព័ត៌មានជំនួយអនាមិកអំពី Madam Bones រកពេលវេលាជាមួយជំនួយការវ័យក្មេងរបស់នាងម្នាក់។ នោះនឹងមានតម្លៃជាប្រាក់រង្វាន់ប្រសិនបើនាងអាចគ្រប់គ្រងដើម្បីផ្ទៀងផ្ទាត់វា Bones គឺខ្ពស់នៅក្នុងបញ្ជីដ៏ល្បី។ អ្នកណែនាំបាននិយាយថា Bones និងជំនួយការវ័យក្មេងរបស់នាងត្រូវទៅញ៉ាំអាហារថ្ងៃត្រង់នៅក្នុងបន្ទប់ពិសេសមួយនៅឯ Mary's Place ដែលជាបន្ទប់ដ៏ពេញនិយមសម្រាប់គោលបំណងជាក់លាក់។ បន្ទប់ដែលនាងបានរកឃើញ មានសុវត្ថិភាពប្រឆាំងនឹងឧបករណ៍ស្តាប់ទាំងអស់ ប៉ុន្តែមិនមានភស្តុតាងប្រឆាំងនឹងសត្វល្អិតពណ៌ខៀវដ៏ស្រស់ស្អាតដែលស្ថិតនៅជាប់នឹងជញ្ជាំងមួយ…

“ចេញពី \emph របស់ខ្ញុំ!” Rita បាននិយាយ ហើយព្យាយាមរុញ Quirrell ចេញពីផ្លូវរបស់នាង។ ដៃរបស់ Quirrell បានដុសខ្លួនរបស់នាង ផ្លាត ហើយ Rita ញ័រនៅពេលដែលការរុញចូលទៅក្នុងខ្យល់ស្តើង។

Quirrell បាន​ទាញ​ដៃអាវ​នៃ​អាវ​ឆ្វេង​របស់គាត់​ឡើង ដោយ​បង្ហាញ​ដៃឆ្វេង​របស់គាត់។ Quirrell បាននិយាយថា "សង្កេត" មិនមាន Dark Mark ទេ។ ខ្ញុំ​ចង់​ឱ្យ​ក្រដាស​របស់​អ្នក​បោះពុម្ព​ផ្សាយ​ការ​ដក​ហូត​វិញ»។

រីតា សើច​មិន​គួរ​ឲ្យ​ជឿ។ ជាការពិតណាស់ បុរសនោះមិនមែនជាអ្នកស្លាប់ពិតប្រាកដនោះទេ។ កាសែតនឹងមិនបោះពុម្ពវាទេប្រសិនបើគាត់។ “ភ្លេចវាទៅ ចៅហ្វាយ ឥឡូវ​ដើរ​លេង»។

Quirrell សម្លឹងមើលនាងមួយភ្លែត។

បន្ទាប់មកគាត់ញញឹម។

"Miss~Skeeter" Quirrell បាននិយាយថា "ខ្ញុំសង្ឃឹមថានឹងរកឃើញដងថ្លឹងដែលនឹងបង្ហាញពីការបញ្ចុះបញ្ចូល។ ប៉ុន្តែ​ខ្ញុំ​យល់​ឃើញ​ថា ខ្ញុំ​មិន​អាច​បដិសេធ​ខ្លួន​ឯង​ពី​ការ​សប្បាយ​ចិត្ត​ដោយ​គ្រាន់​តែ​វាយ​អ្នក​បាន​នោះ​ទេ»។

"វាត្រូវបានសាកល្បង។ ឥឡូវ​នេះ​ចេញ​ពី​ផ្លូវ​របស់​ខ្ញុំ​ទៅ ចោរ​ប្លន់ ឬ​ខ្ញុំ​នឹង​រក​ឃើញ Aurors ខ្លះ ហើយ​ចាប់​អ្នក​ពី​បទ​រារាំង​អ្នក​សារព័ត៌មាន»។

Quirrell ចាប់ធ្នូតូចមួយរបស់នាង ហើយបន្ទាប់មកដើរកាត់។ "លាហើយ Rita Skeeter" សំលេងរបស់គាត់បាននិយាយពីខាងក្រោយនាង។

ខណៈដែល Rita ស្រែកថ្ងូរ នាងបានកត់សម្គាល់នៅក្នុងគំនិតរបស់នាងថា បុរសនោះកំពុងហួចសំនៀងពេលគាត់ដើរចេញទៅ។

ដូចជា\emph{នោះ}នឹងបំភ័យនាង។

\latersection{ច្បាប់ 4}

Lee Jordan បាននិយាយថា “សុំទោស រាប់ខ្ញុំចេញ។ "ខ្ញុំជាប្រភេទសត្វពីងពាងយក្ស"

The Boy-Who-Lived បាននិយាយថាគាត់មាន \emph{important} ការងារសម្រាប់ Order of Chaos ដែលជាអ្វីដែលធ្ងន់ធ្ងរ និងសម្ងាត់ សំខាន់ និងពិបាកជាងការលេងសើចធម្មតារបស់ពួកគេ។

ហើយបន្ទាប់មក Harry Potter បានចាប់ផ្តើមសុន្ទរកថាដែលបំផុសគំនិត ប៉ុន្តែមិនច្បាស់លាស់។ សុន្ទរកថាទៅកាន់ឥទ្ធិពលដែល Fred និង George និង Lee មានសក្តានុពលយ៉ាងខ្លាំង ប្រសិនបើពួកគេអាចរៀនក្លាយជា\emph{ចម្លែកជាង}។ ដើម្បីធ្វើឱ្យជីវិតរបស់មនុស្សមានតម្លៃ \emph{surreal} ជំនួសឱ្យការធ្វើឱ្យពួកគេភ្ញាក់ផ្អើលជាមួយនឹងធុងទឹកដែលស្មើពីលើទ្វារ។ (Fred និង George បានផ្លាស់ប្តូររូបរាងគួរឱ្យចាប់អារម្មណ៍ ពួកគេមិនដែលគិតពីវានោះទេ។) Harry Potter បានហៅរូបភាពនៃការលេងសើចដែលពួកគេបានទាញ Neville - ដែល Harry បានរៀបរាប់ដោយវិប្បដិសារីខ្លះ មួកតម្រៀបបានទំពារគាត់។ ចេញហើយ ប៉ុន្តែអ្វីដែលត្រូវតែធ្វើឱ្យ Neville \emph{សង្ស័យអនាម័យផ្ទាល់ខ្លួន}។ សម្រាប់ Neville វាហាក់ដូចជាត្រូវបានបញ្ជូនភ្លាមៗចូលទៅក្នុងសកលលោកជំនួស។ ដូច​គ្នា​នឹង​អ្នក​ដទៃ​មាន​អារម្មណ៍​ពេល​ឃើញ Snape សុំទោស។ នោះគឺជា \emph{អំណាចពិតនៃការលេងសើច}។

\emph{តើអ្នកនៅជាមួយខ្ញុំទេ?} Harry Potter បានយំ ហើយ Lee Jordan បានឆ្លើយថាទេ

Fred បាននិយាយថា "រាប់យើង\emph{in}" Fred ឬប្រហែលជា George ព្រោះវាគ្មានការងឿងឆ្ងល់ទេដែល Godric Gryffindor នឹងបាននិយាយថាបាទ។

Lee Jordan ញញឹមយ៉ាងសោកស្ដាយ ហើយក្រោកឈរឡើង ហើយចាកចេញពីច្រករបៀងស្ងាត់ជ្រងំ ដែលសមាជិកទាំងបួននាក់នៃ Order of Chaos បានជួប ហើយអង្គុយក្នុងរង្វង់មូលគំនិត។

សមាជិកទាំងបីនៃ Order of Chaos បានចុះទៅរកស៊ី។

(វាមិនមែនជា \emph{ដែល} សោកសៅនោះទេ។ Fred និង George នឹងនៅតែធ្វើការជាមួយ Lee លើការលេងសើចពីងពាងយក្សដដែល។ ពួកគេគ្រាន់តែចាប់ផ្តើមហៅវាថា Order of Chaos ដើម្បីជ្រើសរើស Harry Potter បន្ទាប់ពី Ron បានប្រាប់ពួកគេអំពី Harry ជាមនុស្សចំលែក និងអាក្រក់ ហើយ Fred និង George បានសម្រេចចិត្តជួយសង្គ្រោះ Harry ដោយបង្ហាញគាត់នូវមិត្តភាព និងសេចក្តីសប្បុរសដ៏ស្មោះស្ម័គ្រ នេះហាក់ដូចជាមិនចាំបាច់ទៀតទេ - ទោះបីជាពួកគេមិនប្រាកដអំពីរឿងនោះ \emph...)

កូនភ្លោះ​ម្នាក់​ក្នុង​ចំណោម​កូន​ភ្លោះ​បាន​និយាយ​ថា “តើ​នេះ​មាន​រឿង​អ្វី?

Harry បាននិយាយថា "Rita Skeeter" ។ "តើអ្នកដឹងថានាងជានរណាទេ?"

ហ្វ្រេដ និងចចងក់ក្បាលដោយងក់ក្បាល។

"នាងកំពុងសួរសំណួរអំពីខ្ញុំ"

នោះ​មិន​មែន​ជា​ដំណឹង​ល្អ​ទេ។

"តើអ្នកអាចទាយបានទេថាខ្ញុំចង់ឱ្យអ្នកធ្វើអ្វី?"

Fred និង George មើលមុខគ្នាដោយងឿងឆ្ងល់បន្តិច។ "អ្នកចង់ឱ្យយើងរអិលនាងនូវស្ករគ្រាប់ដែលគួរឱ្យចាប់អារម្មណ៍បន្ថែមទៀតរបស់យើង?"

"ទេ" Harry បាននិយាយ។ “ទេ ទេ \emph{ទេ}! នោះ​ជា​គំនិត​ពីងពាង​យក្ស! មក \emph{you} នឹងធ្វើអ្វី ប្រសិនបើអ្នកឮថា Rita Skeeter កំពុងស្វែងរកពាក្យចចាមអារ៉ាមអំពី\emph{you}?”

នោះបានធ្វើឱ្យវាច្បាស់។

ការញញឹមចាប់ផ្តើមយឺតៗនៅលើមុខរបស់ Fred និង George ។

ពួកគេឆ្លើយថា "ចាប់ផ្តើមពាក្យចចាមអារ៉ាមអំពីខ្លួនយើង" ។

“\emph{ពិតប្រាកដណាស់}” Harry បាននិយាយដោយញញឹមយ៉ាងពេញទំហឹង។ “ប៉ុន្តែទាំងនេះមិនអាចគ្រាន់តែជាពាក្យចចាមអារ៉ាម \emph{any} ទេ។ ខ្ញុំចង់បង្រៀនមនុស្សកុំឱ្យជឿអ្វីដែលកាសែតនិយាយអំពី Harry Potter លើសពី Muggles ជឿអ្វីដែលកាសែតនិយាយអំពី Elvis ។ ដំបូងឡើយ ខ្ញុំគ្រាន់តែគិតអំពីការជន់លិច Rita Skeeter ជាមួយនឹងពាក្យចចាមអារ៉ាមជាច្រើនដែលថានាងនឹងមិនដឹងថាត្រូវជឿអ្វី ប៉ុន្តែបន្ទាប់មកនាងនឹងគ្រាន់តែជ្រើសរើសអ្វីដែលគួរឱ្យទុកចិត្ត និងអាក្រក់ប៉ុណ្ណោះ។ ដូច្នេះ​អ្វី​ដែល​ខ្ញុំ​ចង់​ឱ្យ​អ្នក​ធ្វើ​គឺ​បង្កើត​រឿង​ក្លែងក្លាយ​អំពី​ខ្ញុំ ហើយ​ធ្វើ​ឱ្យ Rita Skeeter ជឿ​វា​តាម​វិធី​ណា​មួយ​។ ប៉ុន្តែវាត្រូវតែជាអ្វីមួយដែលក្រោយមក អ្នករាល់គ្នានឹង\emph{ដឹងថា}ក្លែងក្លាយ។ យើងចង់បញ្ឆោត Rita Skeeter និងអ្នកកែសម្រួលរបស់នាង ហើយ\emph{ក្រោយមក} មានភស្តុតាងចេញមកថាវាមិនពិត។ ហើយជាការពិតណាស់ — ដោយចាត់ទុកថាទាំងនោះគឺជាតម្រូវការ—រឿងត្រូវតែមានតម្លៃ \emph{គួរឱ្យអស់សំណើច} ដូចដែលវាអាចទៅរួច ហើយនៅតែត្រូវបានបោះពុម្ព។ តើអ្នកយល់ពីអ្វីដែលខ្ញុំចង់ឱ្យអ្នកធ្វើ?

“មិនប្រាកដទេ…” Fred ឬ George និយាយយឺតៗ។ "អ្នកចង់ឱ្យយើង\emph{បង្កើតរឿង}?"

Harry Potter បាននិយាយថា "ខ្ញុំចង់ឱ្យអ្នកធ្វើ\emph{ទាំងអស់}។ “ខ្ញុំ​រវល់​ណាស់​ឥឡូវ​នេះ បូក​នឹង​ខ្ញុំ​ចង់​និយាយ​ដោយ​ត្រង់​ថា ខ្ញុំ​មិន​ដឹង​ថា​នឹង​មាន​អ្វី​មក​ដល់។ ធ្វើឱ្យខ្ញុំភ្ញាក់ផ្អើល»។

មួយសន្ទុះ មានការញញឹមដ៏អាក្រក់នៅលើមុខរបស់ Fred និង George ។

បន្ទាប់មកពួកគេប្រែទៅជាធ្ងន់ធ្ងរ។ "ប៉ុន្តែ Harry យើងពិតជាមិនដឹងពីរបៀបធ្វើរឿងបែបនេះទេ"

លោក Harry បាននិយាយថា "ដូច្នេះសូមស្វែងយល់ពីវា" ។ “ខ្ញុំមានទំនុកចិត្តលើអ្នក។ មិនមែនជាទំនុកចិត្ត \emph{total} ទេ ប៉ុន្តែប្រសិនបើអ្នក \emph{មិនអាច} ធ្វើវាបានទេ \emph{ប្រាប់ខ្ញុំថា ខ្ញុំនឹងសាកល្បងអ្នកផ្សេង ឬធ្វើវាដោយខ្លួនឯង។ ប្រសិនបើអ្នកមានគំនិតល្អ - សម្រាប់ទាំងរឿងគួរឱ្យអស់សំណើច និងរបៀបបញ្ចុះបញ្ចូល Rita Skeeter និងអ្នកកែសម្រួលរបស់នាងឱ្យបោះពុម្ពវា - បន្ទាប់មកអ្នកអាចបន្តធ្វើវាបាន។ ប៉ុន្តែកុំទៅជាមួយអ្វីដែលមធ្យម។ ប្រសិនបើអ្នកមិនអាចទទួលបានអ្វីមួយ \emph{អស្ចារ្យ}ទេ គ្រាន់តែនិយាយដូច្នេះ។"

ហ្វ្រេដ និងចច បានផ្លាស់ប្តូរការសម្លឹងមើលដោយព្រួយបារម្ភ។

George បាននិយាយថា "ខ្ញុំមិនអាចគិតអ្វីបានទេ។

Fred បាននិយាយថា "ខ្ញុំក៏មិនអាចដែរ" ។ “សុំទោស។”

Harry សម្លឹងមើលពួកគេ។

ហើយបន្ទាប់មក Harry បានចាប់ផ្តើមពន្យល់ពីរបៀបដែលអ្នកគិតអំពីរឿង។

Harry បាននិយាយថាវាត្រូវបានគេដឹងថាវាចំណាយពេលលើសពីពីរវិនាទី។

Harry បាននិយាយថាអ្នក \emph{មិនដែល} ហៅ \emph{any} សំណួរដែលមិនអាចទៅរួចនោះទេ Harry បាននិយាយថារហូតដល់អ្នកបានយកនាឡិកាពិតប្រាកដមួយហើយគិតអំពីវាអស់រយៈពេល 5 នាទីដោយចលនានៃដៃនាទី។ មិនមែនប្រាំនាទីតាមន័យធៀបទេ ប្រាំនាទីដោយនាឡិការូបវិទ្យា។

ហើយ \emph{លើសពីនេះទៅទៀត} Harry បាននិយាយថា សំលេងរបស់គាត់សង្កត់ធ្ងន់ ហើយដៃស្តាំរបស់គាត់កំពុងគោះយ៉ាងខ្លាំងនៅលើឥដ្ឋ អ្នកបានធ្វើ \emph{not} ចាប់ផ្តើមស្វែងរកដំណោះស្រាយភ្លាមៗ។

បន្ទាប់មក Harry បានចាប់ផ្តើមការពន្យល់អំពីការធ្វើតេស្តដែលធ្វើឡើងដោយនរណាម្នាក់ដែលមានឈ្មោះថា Norman Maier ដែលជាអ្នកចិត្តសាស្រ្តនៃអង្គការ ហើយអ្នកដែលសួរក្រុមដោះស្រាយបញ្ហាពីរផ្សេងគ្នាដើម្បីដោះស្រាយបញ្ហា។

លោក Harry បាននិយាយថា បញ្ហានេះមានពាក់ព័ន្ធនឹងបុគ្គលិកបីនាក់ ដែលធ្វើការងារចំនួនបី។ បុគ្គលិកវ័យក្មេងចង់ធ្វើការងារដែលស្រួលបំផុត។ បុគ្គលិកជាន់ខ្ពស់ចង់បង្វិលរវាងការងារដើម្បីកុំឱ្យធុញទ្រាន់។ អ្នកជំនាញខាងប្រសិទ្ធភាពបានផ្តល់អនុសាសន៍ឱ្យផ្តល់ការងារងាយស្រួលបំផុតដល់មនុស្សវ័យក្មេង ហើយមនុស្សចាស់ជាការងារពិបាកបំផុត ដែលនឹងមានផលិតភាពជាង 20% ។

\emph{មួយ} សំណុំនៃក្រុមដោះស្រាយបញ្ហាត្រូវបានផ្តល់ការណែនាំ "កុំស្នើដំណោះស្រាយរហូតដល់បញ្ហាត្រូវបានពិភាក្សាឱ្យបានហ្មត់ចត់តាមដែលអាចធ្វើទៅបានដោយមិនចាំបាច់ផ្តល់យោបល់ណាមួយឡើយ។"

ក្រុមដោះស្រាយបញ្ហាផ្សេងទៀតមិនត្រូវបានណែនាំទេ។ ហើយមនុស្សទាំងនោះបានធ្វើរឿងធម្មជាតិ ហើយមានប្រតិកម្មចំពោះវត្តមាននៃបញ្ហាដោយស្នើដំណោះស្រាយ។ ហើយមនុស្សបានភ្ជាប់ជាមួយដំណោះស្រាយទាំងនោះ ហើយចាប់ផ្តើមប្រយុទ្ធគ្នាអំពីពួកគេ ហើយជជែកវែកញែកអំពីសារៈសំខាន់នៃសេរីភាពធៀបនឹងប្រសិទ្ធភាពជាដើម។

សំណុំដំបូងនៃក្រុមដោះស្រាយបញ្ហា ដែលបានផ្តល់ការណែនាំដល់ \emph{ពិភាក្សា} បញ្ហាជាមុន ហើយ\emph{បន្ទាប់មក} ដោះស្រាយវា ទំនងជាទទួលបានច្រើនជាងដំណោះស្រាយនៃការអនុញ្ញាតឱ្យបុគ្គលិកតូចរក្សា ការងារដែលងាយស្រួលបំផុត និងការបង្វិលមនុស្សពីរនាក់ផ្សេងទៀតរវាងការងារពីរផ្សេងទៀត សម្រាប់អ្វីដែលទិន្នន័យរបស់អ្នកជំនាញបាននិយាយថានឹងមានភាពប្រសើរឡើង 19\% ។

ការចាប់ផ្តើមដោយការស្វែងរកដំណោះស្រាយគឺយករបស់ដែល\emph{ទាំងស្រុងចេញពីលំដាប់}។ ដូចជាការចាប់ផ្តើមអាហារជាមួយបង្អែម ត្រឹមតែ\emph{bad}។

(Harry ក៏បានដកស្រង់អ្នកដែលមានឈ្មោះ Robyn Dawes ថាបញ្ហាកាន់តែពិបាក មនុស្សទំនងជាព្យាយាមដោះស្រាយភ្លាមៗ។ )

ដូច្នេះ Harry នឹងចាកចេញពីបញ្ហានេះទៅ Fred និង George ហើយពួកគេនឹងពិភាក្សាអំពីគ្រប់ទិដ្ឋភាពទាំងអស់របស់វា ហើយគិតគូរអំពីអ្វីដែលពួកគេគិតថាអាចពាក់ព័ន្ធពីចម្ងាយ។ ហើយពួកគេមិនគួរព្យាយាមរកដំណោះស្រាយជាក់ស្តែងរហូតដល់ពួកគេបានបញ្ចប់ការធ្វើនោះទេ លុះត្រាតែពួកគេ \emph{តើ} បានកើតឡើងដោយចៃដន្យដើម្បីគិតអំពីអ្វីដែលអស្ចារ្យ ក្នុងករណីនេះពួកគេអាចសរសេរវាទុកនៅពេលក្រោយ និង បន្ទាប់មកត្រឡប់ទៅគិត។ ហើយគាត់មិនចង់ឮពីពួកគេអំពីអ្វីដែលគេហៅថា\emph{បរាជ័យក្នុងការគិតអ្វីទាំងអស់} យ៉ាងហោចណាស់មួយសប្តាហ៍។ មនុស្សមួយចំនួនបានចំណាយ \emph{decades} ព្យាយាមគិតរឿង។

“មានសំណួរអី?” បាននិយាយថា Harry ។

Fred និង George សម្លឹងមើលគ្នាទៅវិញទៅមក។

"ខ្ញុំមិនអាចគិតអ្វីបានទេ"

"ខ្ញុំក៏មិនអាចដែរ"

Harry ក្អក​ថ្នមៗ។ "អ្នកមិនបានសួរអំពីថវិការបស់អ្នកទេ។"

\emph{ថវិកា?} ពួកគេបានគិត។

Harry បាននិយាយថា "ខ្ញុំគ្រាន់តែអាចប្រាប់អ្នកពីចំនួនទឹកប្រាក់" ។ "ប៉ុន្តែខ្ញុំគិតថា \emph{នេះ} នឹងច្រើនជាង \emph{លើកទឹកចិត្ត}។"

ដៃរបស់ Harry បានជ្រលក់ចូលទៅក្នុងអាវរបស់គាត់ ហើយបានចេញមក—

Fred និង George ស្ទើរតែដួល ទោះបីពួកគេអង្គុយចុះក៏ដោយ។

Harry បាននិយាយថា "កុំចំណាយវាសម្រាប់ជាប្រយោជន៍នៃការចំណាយវា" ។ នៅ​លើ​កម្រាល​ថ្ម​នៅ​ពី​មុខ​ពួក​គេ​បាន​បញ្ចេញ​លុយ​ដ៏​គួរ​ឱ្យ​អស់​សំណើច​ជា​ខ្លាំង។ “គ្រាន់តែចំណាយវា ប្រសិនបើភាពអស្ចារ្យទាមទារ។ ហើយអ្វីដែលអស្ចារ្យទាមទារ កុំស្ទាក់ស្ទើរក្នុងការចំណាយ។ បើមានអីនៅសេសសល់ ចាំយកមកវិញក្រោយ ខ្ញុំទុកចិត្តអ្នក។ អូ ហើយអ្នកទទួលបានដប់ភាគរយនៃអ្វីដែលនៅទីនោះ មិនថាអ្នកចំណាយអស់ប៉ុន្មាននោះទេ—

“យើង\emph{មិនអាច}!” បានធ្វើឱ្យកូនភ្លោះម្នាក់ក្នុងចំណោមកូនភ្លោះ។ «យើង​មិន​ទទួល​យក​លុយ​សម្រាប់​រឿង​ហ្នឹង​ទេ!»

(កូនភ្លោះទាំងពីរមិនដែលយកលុយសម្រាប់ការធ្វើអ្វីខុសច្បាប់ទេ។ មិនស្គាល់ចំពោះ Ambrosius Flume ទេ ពួកគេបានលក់ទំនិញទាំងអស់របស់គាត់ក្នុងអត្រាសូន្យភាគរយ។ Fred និង George ចង់អាចផ្តល់សក្ខីកម្ម - នៅក្រោម Veritaserum បើចាំបាច់ - ថាពួកគេមិនបានរកប្រាក់ចំណេញពីឧក្រិដ្ឋជន។ គ្រាន់តែផ្តល់សេវាសាធារណៈ។ )

Harry ងក់ក្បាលដាក់ពួកគេ។ “ប៉ុន្តែខ្ញុំសុំឱ្យអ្នកដាក់ការងារពិតនៅទីនេះ។ មនុស្សធំនឹងទទួលបានប្រាក់សម្រាប់ការធ្វើអ្វីមួយដូចនេះ ហើយវានឹងមានតម្លៃ \emph{នៅតែ} រាប់ថាជាការពេញចិត្តសម្រាប់មិត្តម្នាក់។ អ្នក​មិន​អាច​ជួល​មនុស្ស​សម្រាប់​រឿង​បែប​នេះ​ទេ»។

Fred និង George គ្រវីក្បាល។

Harry បាននិយាយថា "មិនអីទេ" ។ “ខ្ញុំ​នឹង​យក​អំណោយ​បុណ្យ​ណូអែល​ថ្លៃ​ៗ​ឲ្យ​អ្នក ហើយ​ប្រសិន​បើ​អ្នក​ព្យាយាម​ប្រគល់​វា​មក​ខ្ញុំ ខ្ញុំ​នឹង​ដុត​វា​ចោល។ ឥឡូវនេះ អ្នកមិនដឹងថា \emph{ដឹង} ថាតើខ្ញុំនឹងចំណាយលើអ្នកប៉ុន្មានទេ លើកលែងតែជាក់ស្តែង វានឹងមានច្រើនជាងប្រសិនបើអ្នកទើបតែយកលុយនោះ។ ហើយខ្ញុំនឹងទិញជូនអ្នកនូវកាដូទាំងនោះ \emph{anyway} ដូច្នេះសូមគិតអំពី \emph{នោះ} មុនពេលអ្នកប្រាប់ខ្ញុំ \emph{អ្នកមិនអាចគិតពីអ្វីដែលអស្ចារ្យ}។

Harry ក្រោកឈរឡើង ញញឹម ហើយងាកទៅ ខណៈពេលដែល Fred និង George នៅតែតក់ស្លុតដោយភាពតក់ស្លុត។ គាត់​ដើរ​បាន​ប៉ុន្មាន​ជំហាន​ទៀត រួច​ក៏​ត្រឡប់​មក​វិញ។

Harry បាននិយាយថា "អូ មួយចុងក្រោយ" ។ “ទុកឱ្យសាស្រ្តាចារ្យ Quirrell ចេញពីអ្វីដែលអ្នកធ្វើ។ គាត់មិនចូលចិត្តការផ្សព្វផ្សាយទេ។ ខ្ញុំដឹងថា វានឹងមានភាពងាយស្រួលក្នុងការធ្វើឱ្យមនុស្សជឿរឿងចំលែកអំពីសាស្ត្រាចារ្យការពារជាតិជាងអ្នកដ៏ទៃ ហើយខ្ញុំសុំទោសដែលត្រូវធ្វើតាមរបៀបរបស់អ្នក ប៉ុន្តែសូមទុកសាស្ត្រាចារ្យ Quirrell ចេញពីវា»។

ហើយ Harry បានងាកមកម្តងទៀត ហើយបានបោះជំហានពីរបីទៀត—

ក្រឡេកមើលមួយលើកចុងក្រោយ ហើយនិយាយយ៉ាងស្រទន់ថា “អរគុណ”។

ហើយបានចាកចេញ។

មាន​ការ​ផ្អាក​យ៉ាង​យូរ​បន្ទាប់​ពី​គាត់​បាន​ចាកចេញ។

ម្នាក់បាននិយាយថា "ដូច្នេះ" ។

អ្នក​ផ្សេង​ទៀត​បាន​និយាយ​ថា “ដូច្នេះ”។

"សាស្ត្រាចារ្យការពារជាតិមិនចូលចិត្តការផ្សព្វផ្សាយទេតើគាត់"

"Harry មិនស្គាល់យើងច្បាស់ទេតើគាត់"

“អត់ទេ គាត់អត់ទេ”

ប៉ុន្តែ​យើង​នឹង​មិន​ប្រើ​លុយ​របស់​គាត់​សម្រាប់​រឿង​នោះ​ទេ​»។

“មិន​មែន​ជា​ការ​ពិត​ទេ វា​នឹង​មិន​ត្រឹម​ត្រូវ។ យើង​នឹង​ធ្វើ​សាស្ត្រាចារ្យ​ការពារ​ដោយ​ឡែក​ពី​គ្នា»។

"យើងនឹងទទួលបាន Gryffindors ខ្លះដើម្បីសរសេរ Skeeter ហើយនិយាយថា ... "

“…ដៃអាវរបស់គាត់លើកឡើងម្តងក្នុងថ្នាក់ការពារ ហើយគេបានឃើញ Dark Mark…”

“… ហើយគាត់ប្រហែលជាកំពុងបង្រៀនរឿង Harry Potter គ្រប់ប្រភេទដែលគួរឱ្យខ្លាច…”

“… ហើយគាត់គឺជាសាស្ត្រាចារ្យការពារជាតិដ៏អាក្រក់បំផុតដែលគ្រប់គ្នាចងចាំសូម្បីតែនៅ Hogwarts គាត់មិនត្រឹមតែ \emph{បរាជ័យ} ក្នុងការបង្រៀនយើងនោះទេ គាត់កំពុងទទួលបានអ្វីគ្រប់យ៉ាងខុស ផ្ទុយពីអ្វីដែលវាគួរតែជា…”

“… ដូចជាពេលដែលគាត់អះអាងថា អ្នកអាចដេញ Killing Curse ដោយប្រើស្នេហា ដែលធ្វើឱ្យវាគ្មានប្រយោជន៍ច្រើន”។

"ខ្ញុំចូលចិត្តវា"

“អរគុណ។”

“ខ្ញុំភ្នាល់ថា សាស្ត្រាចារ្យការពារជាតិក៏ចូលចិត្តវាដែរ”។

“គាត់​មាន​អារម្មណ៍​កំប្លែង។ គាត់​នឹង​មិន​ហៅ​យើង​ពី​អ្វី​ដែល​គាត់​បាន​ធ្វើ​ទេ បើ​គាត់​មិន​មាន​ការ​លេង​សើច»។

"ប៉ុន្តែតើយើងពិតជាអាចធ្វើការងាររបស់ Harry មែនទេ?"

"Harry បាននិយាយដើម្បីពិភាក្សាបញ្ហាមុននឹងព្យាយាមដោះស្រាយវា ដូច្នេះ ចូរយើងធ្វើវាចុះ"

កូនភ្លោះ Weasley បានសម្រេចចិត្តថា George នឹងក្លាយជាមនុស្សសាទរ ខណៈដែល Fred មានការសង្ស័យ។

Fred បាននិយាយថា "វាហាក់ដូចជាផ្ទុយគ្នា។ "គាត់ចង់ឱ្យវាគួរឱ្យអស់សំណើចណាស់ដែលមនុស្សគ្រប់គ្នាសើចនឹង Skeeter ហើយដឹងថាវាខុសហើយគាត់ចង់ឱ្យ Skeeter ជឿវា។ យើង​មិន​អាច​ធ្វើ​រឿង​ទាំង​ពីរ​ក្នុង​ពេល​តែ​មួយ​បាន​ទេ»។

George បាននិយាយថា "យើងនឹងត្រូវក្លែងបន្លំភស្តុតាងមួយចំនួនដើម្បីបញ្ចុះបញ្ចូល Skeeter" ។

"តើនោះជាដំណោះស្រាយទេ?" Fred បាននិយាយ។

ពួកគេបានពិចារណារឿងនេះ។

George បាននិយាយថា "ប្រហែលជា" ប៉ុន្តែខ្ញុំគិតថាយើងមិនគួរទាំងអស់\emph{ដែល} តឹងរឹងអំពីវាមែនទេ?

កូនភ្លោះទាំងពីរគ្រវីក្បាលដោយមិនដឹងខ្លួន។

Fred បាននិយាយថា "ដូច្នេះភស្តុតាងក្លែងក្លាយត្រូវតែល្អគ្រប់គ្រាន់ដើម្បីបញ្ចុះបញ្ចូល Skeeter" ។ "តើយើងពិតជាអាចធ្វើវាដោយខ្លួនឯងបានទេ?"

George បាននិយាយថា "យើងមិនចាំបាច់ធ្វើវាដោយខ្លួនឯងទេ" ហើយចង្អុលទៅគំនរលុយ។ “យើងអាចជួលអ្នកផ្សេងមកជួយយើង”។

កូនភ្លោះបានមើលមុខគេដោយគិតគូរ។

Fred បាននិយាយថា "វាអាចប្រើប្រាស់ថវិការបស់ Harry លឿនណាស់។ "នេះ​ជា​ប្រាក់​ច្រើន​សម្រាប់​យើង ប៉ុន្តែ​វា​មិន​មែន​ជា​ប្រាក់​ច្រើន​សម្រាប់​មនុស្ស​ដូច Flume ទេ"។

George បាននិយាយថា "ប្រហែលជាមនុស្សនឹងផ្តល់ការបញ្ចុះតម្លៃប្រសិនបើពួកគេដឹងថាវាជារបស់ Harry" ។ "ប៉ុន្តែអ្វីដែលសំខាន់បំផុត អ្វីក៏ដោយដែលយើងធ្វើ វាត្រូវតែមានតម្លៃ \emph{impossible}។"

Fred ព្រិចភ្នែក។ "តើអ្នកមានន័យយ៉ាងណា \emph{impossible}?"

“មិន​អាច​ទៅ​រួច​ទេ​ដែល​យើង​មិន​ជួប​បញ្ហា ព្រោះ​គ្មាន​អ្នក​ណា​ជឿ​ថា​យើង​អាច​ធ្វើ​វា​បាន។ មិនអាចទៅរួចដែលសូម្បីតែ Harry ចាប់ផ្តើមឆ្ងល់។ វាត្រូវតែអស្ចារ្យ វាត្រូវតែធ្វើឱ្យមនុស្សសង្ស័យពីអនាម័យផ្ទាល់ខ្លួនរបស់វា វាត្រូវតែ…\emph{ប្រសើរជាង Harry ។}"

ភ្នែករបស់ហ្វ្រេដបើកដោយការភ្ញាក់ផ្អើល។ វាបានកើតឡើងពេលខ្លះរវាងពួកគេ ប៉ុន្តែមិនញឹកញាប់ទេ។ "ប៉ុន្តែហេតុអ្វី?"

“ពួកគេលេងសើច។ ពួកគេបានលេងសើច \emph{all}។ នំប៉ាវគឺជាការលេងសើច។ The Remembrall គឺជាការលេងសើច។ ឆ្មារបស់ Kevin Entwhistle គឺជាការលេងសើច។ \emph{Snape} គឺជាការលេងសើច។ \emph{យើង} គឺជាអ្នកលេងល្បែងស៊ីសងដ៏ល្អបំផុតនៅ Hogwarts តើយើងនឹងចុះចាញ់ ហើយបោះបង់ដោយគ្មានការប្រយុទ្ធទេ?

Fred បាននិយាយថា "គាត់គឺជាក្មេងប្រុសដែលរស់នៅ" ។

“ហើយ\emph{យើង} កូនភ្លោះ Weasley! គាត់គឺ\emph{ប្រកួតប្រជែង}ពួកយើង។ គាត់បាននិយាយថាយើងអាចធ្វើអ្វីដែលគាត់ធ្វើ។ ប៉ុន្តែ​ខ្ញុំ​ហ៊ាន​ភ្នាល់​ថា​គាត់​មិន​គិត​ថា​យើង​នឹង​ល្អ​ដូច​\emph{គាត់}​ទេ»។

ហ្វ្រេដ​និយាយ​ដោយ​មាន​អារម្មណ៍​ភ័យ​ខ្លាច​ថា​៖ «​គាត់​និយាយ​ត្រូវ។ កូនភ្លោះ Weasley បានធ្វើ \emph{ពេលខ្លះ} មិនយល់ស្រប សូម្បីតែនៅពេលដែលពួកគេមានព័ត៌មានដូចគ្នា ប៉ុន្តែរាល់ពេលដែលពួកគេបានធ្វើ វាហាក់ដូចជាខុសពីធម្មជាតិ ដូចជាយ៉ាងហោចណាស់មានម្នាក់ក្នុងចំនោមពួកគេត្រូវតែធ្វើអ្វីខុស។ “នេះគឺ\emph{Harry Potter} ដែលយើងកំពុងនិយាយអំពី។ គាត់អាចធ្វើអ្វីដែលមិនអាចទៅរួច។ យើងមិនអាច”

លោក George បាននិយាយថា "បាទយើងអាចធ្វើបាន" ។ "ហើយយើងត្រូវមាន\emph{ច្រើន}មិនអាចទៅរួចជាងគាត់។"

"ប៉ុន្តែ -" Fred បាននិយាយ។

George បាននិយាយថា "វាជាអ្វីដែល Godric Gryffindor នឹងធ្វើ" ។

នោះបានដោះស្រាយវា ហើយកូនភ្លោះទាំងពីរបានត្រឡប់ទៅរក... អ្វីក៏ដោយដែលវាជារឿងធម្មតាសម្រាប់ពួកគេ។

“មិនអីទេ -”

"- ចូរយើងគិតអំពីវា" ។

%  LocalWords:  wid youse Ya’d Honeydukes Ambrosius Entwhistle’s
