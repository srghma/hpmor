\partchapter{ការឆ្លុះបញ្ចាំង}{I}

\lettrine{E}{\emph{ven}} \emph{វត្ថុបុរាណដ៏អស្ចារ្យបំផុតអាចត្រូវបានកម្ចាត់ដោយវត្ថុបុរាណដែលតិចជាង ប៉ុន្តែមានឯកទេស}។

នោះហើយជាអ្វីដែលសាស្រ្តាចារ្យការពារជាតិបានប្រាប់ Harry បន្ទាប់ពីទម្លាក់ True Cloak of Invisibility ទៅអាងក្នុងផ្នត់ដ៏គួរឱ្យភ័យខ្លាចនៅជិតស្បែកជើងរបស់ Harry ។

\emph{កញ្ចក់នៃការឆ្លុះបញ្ចាំងដ៏ល្អឥតខ្ចោះមានអំណាចលើអ្វីដែលឆ្លុះបញ្ចាំងនៅក្នុងវា ហើយថាមពលនោះត្រូវបានគេនិយាយថាមិនអាចប្រកួតប្រជែងបាន។ ប៉ុន្តែដោយសារសម្លៀកបំពាក់ពិតនៃភាពមើលមិនឃើញបង្កើតភាពអវត្ដមាននៃរូបភាពដ៏ល្អឥតខ្ចោះ វាគួរតែគេចចេញពីគោលការណ៍នេះជាជាងប្រកួតប្រជែងវា។}

មានសំណួរជាបន្តបន្ទាប់នៅក្នុង Parseltongue ដែលបង្កើតថា Harry បច្ចុប្បន្ននេះមិនមានបំណងធ្វើអ្វីឆ្កួតៗ ឬព្យាយាមរត់ចេញនោះទេ ហើយការរំលឹកបន្ថែមទៀតថាសាស្រ្តាចារ្យ Quirrell អាចដឹងពីគាត់ ហើយមានអក្ខរាវិរុទ្ធដើម្បីស្វែងរក Cloak និងកំពុងចាប់ចំណាប់ខ្មាំងរាប់រយនាក់ទៀត។ Hermione

បន្ទាប់មក Harry ត្រូវ​បាន​គេ​ប្រាប់​ឱ្យ​ពាក់​អាវ​ក្រៅ បើក​ទ្វារ​ដែល​ហួស​ពី​ភ្លើង​ដែល​បាន​ពន្លត់ ហើយ​ឈាន​ទៅ​តាម​ទ្វារ​ចូល​ទៅ​ក្នុង​បន្ទប់​ចុងក្រោយ។ ខណៈដែលសាស្រ្តាចារ្យ Quirrell បានឈរត្រឡប់មកវិញយ៉ាងល្អនៅខាងក្រៅការមើលឃើញរបស់ទ្វារនោះ។

បន្ទប់ចុងក្រោយត្រូវបានបំភ្លឺដោយពន្លឺពណ៌មាសទន់ ហើយជញ្ជាំងថ្មមានពណ៌សទន់ភ្លន់ ហើយប្រឈមមុខនឹងថ្មម៉ាប។

នៅចំកណ្តាលបន្ទប់ឈរស៊ុមពណ៌មាសដ៏សាមញ្ញមួយ ហើយនៅក្នុងស៊ុមនោះមានច្រកមួយទៅកាន់បន្ទប់បំភ្លឺពណ៌មាសមួយទៀត លើសពីទ្វារដែលដាក់បន្ទប់ Potions មួយទៀត។ នេះជាអ្វីដែលខួរក្បាលរបស់ Harry បានប្រាប់គាត់។ ការបំប្លែងពន្លឺរបស់កញ្ចក់គឺល្អឥតខ្ចោះដែលការគិតប្រកបដោយមនសិការត្រូវបានទាមទារដើម្បីសន្និដ្ឋានថាបន្ទប់នៅខាងក្នុងស៊ុមគ្រាន់តែជាការឆ្លុះបញ្ចាំងជាជាងវិបផតថល។ (ទោះបីជាវាងាយស្រួលជាងក្នុងការយល់ឃើញក៏ដោយ ប្រសិនបើ Harry មិនត្រូវបានគេមើលមិនឃើញ ពេលនោះ។ )

កញ្ចក់មិនប៉ះដី; ស៊ុមមាសគ្មានជើងទេ។ វាមិនមើលទៅដូចជាវាកំពុងសំកាំងទេ; វាមើលទៅដូចជាវាត្រូវបានជួសជុលនៅនឹងកន្លែង រឹងជាង និងគ្មានចលនាជាងជញ្ជាំងខ្លួនឯង ដូចជាវាត្រូវបានដែកគោលទៅនឹងស៊ុមយោងនៃចលនារបស់ផែនដី។

"តើកញ្ចក់នៅទីនោះទេ? តើវាផ្លាស់ទីទេ? សំឡេងបញ្ជារបស់សាស្រ្តាចារ្យ Quirrell មកពីអង្គជំនុំជម្រះ Potions ។

“\parsel{តើមានទេ}” Harry តបវិញ។ “\parsel{មិនផ្លាស់ទី។}”

សំឡេងនៃពាក្យបញ្ជាបានបន្លឺឡើងម្តងទៀត។ "ដើរទៅខាងក្រោយកញ្ចក់"

ពីខាងក្រោយ ស៊ុមមាសបានលេចចេញជារូបរាងរឹង ដោយមិនបង្ហាញការឆ្លុះបញ្ចាំង ហើយ Harry បាននិយាយដូច្នេះនៅក្នុង Parseltongue ។

“ឥឡូវ​នេះ​ដោះ​អាវ​របស់​អ្នក​ចេញ” សំឡេង​របស់​សាស្ត្រាចារ្យ Quirrell បាន​បញ្ជា​ពី​ក្នុង​បន្ទប់ Potions។ "រាយការណ៍មកខ្ញុំភ្លាមៗ ប្រសិនបើកញ្ចក់រើមករកអ្នក"

Harry បានដោះអាវធំរបស់គាត់។

កញ្ចក់នៅតែជាប់នឹងស៊ុមយោងនៃចលនារបស់ផែនដី។ ហើយ Harry បានរាយការណ៍អំពីរឿងនេះ។

មិនយូរប៉ុន្មានក្រោយមក មានការស្រែកហ៊ោកញ្ជ្រៀវ និងស្រៀវស្រើប ហើយភ្លើងហ្វានីកបានរលាយតាមជញ្ជាំងថ្មម៉ាបខាងក្រោយ Harry ពន្លឺព័ទ្ធជុំវិញក្នុងបន្ទប់មានពណ៌ក្រហមនៅពេលវាចូល។ សាស្ត្រាចារ្យ Quirrell បានដើរតាមពីក្រោយវា ដោយដើរចេញពីច្រករបៀងដែលទើបបង្កើតថ្មី ដែលត្រូវបានឆ្លាក់ ស្បែកជើងផ្លូវការពណ៌ខ្មៅរបស់គាត់ ដែលមិនមានគ្រោះថ្នាក់ដោយសារផ្ទៃរលាយពណ៌ក្រហមនៅក្រោម។ សាស្រ្តាចារ្យ Quirrell បាននិយាយថា "ជាការប្រសើរណាស់" នោះគឺជាអន្ទាក់មួយដែលអាចធ្វើទៅបាន។ ហើយឥឡូវនេះ…” សាស្ត្រាចារ្យ Quirrell ដកដង្ហើមចេញ។ “ឥឡូវនេះ យើងនឹងគិតពីយុទ្ធសាស្ត្រដែលអាចមានសម្រាប់ការទាញយកថ្មពីកញ្ចក់ ហើយអ្នកនឹងសាកល្បងវា។ ព្រោះខ្ញុំមិនចង់ឱ្យរូបភាពផ្ទាល់ខ្លួនរបស់ខ្ញុំត្រូវបានឆ្លុះបញ្ចាំង។ ខ្ញុំ​ផ្តល់​ការ​ព្រមាន​ដោយ​យុត្តិធម៌​ដល់​អ្នក នេះ​ជា​ផ្នែក​ដែល​អាច​នឹង​បង្ហាញ​ថា​គួរ​ឱ្យ​ធុញ​ទ្រាន់»។

"ខ្ញុំយល់ថានេះមិនមែនជាបញ្ហាដែលអ្នកអាចដោះស្រាយជាមួយ Fiendfyre បានទេ?"

“ហា” សាស្ត្រាចារ្យ Quirrell បាននិយាយ ហើយកាយវិការ។

ភ្លើងហ្វានីកបានរុលទៅមុខក្នុងភាពភ័យស្លន់ស្លោពណ៌ក្រហមឆ្អៅ ពន្លឺពណ៌ក្រហមបញ្ចេញស្រមោលលើជញ្ជាំងថ្មម៉ាបដែលនៅសេសសល់។ Harry លោតត្រឡប់មកវិញមុនពេលគាត់អាចគិតបាន។

អណ្តាតភ្លើងពណ៌ក្រហមងងឹតដ៏គួរឱ្យភ័យខ្លាចបានហោះកាត់លោកសាស្រ្តាចារ្យ Quirrell ចូលទៅក្នុងកញ្ចក់ខាងក្រោយពណ៌មាស ហើយបាត់ទៅវិញយ៉ាងលឿនដូចដែលវាប៉ះនឹងមាស។

ពេលនោះ​ភ្លើង​បាន​រលត់​ទៅ ហើយ​បន្ទប់​ក៏​គ្មាន​ពណ៌​ក្រហម​ឆ្អៅ​ទៀត​ដែរ ។

មិនមានស្នាមឆ្កូតលើផ្ទៃមាស គ្មានពន្លឺដើម្បីសម្គាល់ការស្រូបយកកំដៅ។ កញ្ចក់​បាន​នៅ​នឹង​កន្លែង​ដោយ​មិន​បាន​ប៉ះ​ពាល់។

ភាពត្រជាក់បានធ្លាក់ចុះដល់ឆ្អឹងខ្នងរបស់ Harry ។ ប្រសិនបើគាត់កំពុងលេង Dungeons and Dragons ហើយម្ចាស់គុកងងឹតបានរាយការណ៍ពីលទ្ធផលនោះ Harry នឹងសង្ស័យថាមានការបំភាន់ផ្លូវចិត្ត ហើយមិនជឿ។

នៅចំកណ្តាលនៃខ្នងមាសបានលេចចេញជាលំដាប់នៃអក្សររត់ក្នុងអក្ខរក្រមដែលមិនស្គាល់ អវត្តមានពណ៌ខ្មៅនៃពន្លឺនៅក្នុងបន្ទាត់តូចៗ និងខ្សែកោងដែលរៀបចំជាជួរផ្ដេកកម្រិត។ គំនិតនេះបានកើតឡើងចំពោះ Harry ថាការបំភាន់លាក់កំបាំងតូចៗមួយចំនួនត្រូវបានប្រើប្រាស់នៅក្នុង Fiendfyre ដែលជាភាពទាក់ទាញតិចជាងមុនដែលត្រូវបានបន្ថែមដើម្បីការពារកុមារពីការមើលឃើញអក្សរទាំងនោះ…

"កញ្ចក់នេះអាយុប៉ុន្មាន?" Harry និយាយស្ទើរតែខ្សឹប។

"គ្មានអ្នកណាដឹងទេលោក ~ Potter" សាស្រ្តាចារ្យការពារជាតិបានលើកម្រាមដៃរបស់គាត់ឆ្ពោះទៅរក runes, មើលទៅនៃអ្វីមួយដូចជាការគោរពនៅលើមុខរបស់គាត់។ ប៉ុន្តែ ម្រាមដៃរបស់គាត់មិនបានប៉ះមាសទេ។ “ប៉ុន្តែការស្មានរបស់ខ្ញុំគឺដូចគ្នានឹងអ្នកដែរ ខ្ញុំគិតថា។ វាត្រូវបានគេនិយាយនៅក្នុងរឿងព្រេងមួយចំនួនដែលអាចឬប្រហែលជាមិនមែនជាការប្រឌិត ថាកញ្ចក់នេះឆ្លុះបញ្ចាំងពី\emph{ខ្លួនវា}យ៉ាងល្អឥតខ្ចោះ ដូច្នេះហើយអត្ថិភាពរបស់វាមានស្ថេរភាពពិតប្រាកដ។ មានស្ថេរភាពដែលកញ្ចក់អាចរស់រានមានជីវិតនៅពេលដែលគ្រប់ឥទ្ធិពលផ្សេងទៀតនៃអាត្លង់ទីត្រូវបានលុបចោល ផលវិបាកទាំងអស់របស់វាបានកាត់ចេញពីពេលវេលា។ អ្នក​អាច​ដឹង​ថា​ហេតុ​អ្វី​បាន​ជា​ខ្ញុំ​សប្បាយ​ចិត្ត​ពេល​អ្នក​បាន​ណែនាំ Fiendfyre»។ សាស្ត្រាចារ្យ​ការពារ​ជាតិ​ទុក​ដៃ​ដួល។

សូម្បី​តែ​នៅ​កណ្តាល​នៃ​អ្វី​ផ្សេង​ទៀត Harry មាន​អារម្មណ៍​ថា​គួរ​ឱ្យ​ស្ញែង​ខ្លាច​ប្រសិន​បើ​នោះ​ជា​ការ​ពិត​។ ស៊ុមមាសភ្លឺមិនភ្លឺជាងពីមុនសម្រាប់វិវរណៈទាំងអស់; ប៉ុន្តែអ្នកអាចស្រមៃថាវាត្រលប់មកវិញ ហើយត្រលប់មកវិញ ចូលទៅក្នុងអារ្យធម៌ ដែលត្រូវបានបង្កើតឡើង ដែលមិនធ្លាប់មាន… “តើកញ្ចក់ \emph{do} ជាអ្វី?

សាស្រ្តាចារ្យ Quirrell បាននិយាយថា "សំណួរដ៏ល្អមួយ" ។ “ចម្លើយគឺនៅក្នុងអក្សររត់ ដែលត្រូវបានសរសេរនៅលើខ្នងមាសរបស់កញ្ចក់។ សូម​អាន​ពួក​វា​ឲ្យ​ខ្ញុំ»។

“ពួកគេមិនមានអក្សរណាមួយដែលខ្ញុំទទួលស្គាល់ទេ។ ពួកវាមើលទៅដូចជាស្នាមឆ្កូតមាន់ដែលតម្រង់ទិសចៃដន្យដែលគូរដោយ Tolkien elves ។

"អានពួកគេយ៉ាងណាក៏ដោយ។ \parsel{មិនមានគ្រោះថ្នាក់ទេ។}”

“Runes និយាយថា \emph{noitilov detalo partxe tnere hoc ruoy tu becafruoy ton wo hsi—}” Harry ឈប់ ដោយមានអារម្មណ៍ថាមានស្នាមប្រេះនៅឆ្អឹងខ្នង។

Harry ដឹងពីអ្វីដែល rune សម្រាប់ noitilov \emph{{}}។ វាមានន័យថា Noitilov ។ ហើយ runes បន្ទាប់បាននិយាយថាដើម្បី detalo noitilov រហូតដល់វាឈានដល់ partxe បន្ទាប់មករក្សាផ្នែកដែលមានទាំង tnere និង hoc ។ ជំនឿនោះមានអារម្មណ៍ដូចជាចំណេះដឹង ដូចជាគាត់អាចឆ្លើយថា 'បាទ' ជាមួយនឹងសិទ្ធិអំណាចដែលមានទំនុកចិត្ត ប្រសិនបើនរណាម្នាក់សួរគាត់ថាតើ តោន វ៉ូ ជា រូយ៉យ ឬ បេកាហ្វរូយ។ វាគ្រាន់តែថានៅពេលដែល Harry ព្យាយាមទាក់ទងគំនិតទាំងនោះទៅនឹងគោលគំនិតផ្សេងទៀត គាត់បានទាញទទេ។

“\parsel{តើ​អ្នក​យល់​ថា​ពាក្យ​នេះ​មាន​ន័យ​យ៉ាង​ណា​ទេ?}”

“\parsel{កុំគិតដូច្នេះ។}”

សាស្ត្រាចារ្យ Quirrell ដកដង្ហើមចេញយ៉ាងស្រទន់ ភ្នែករបស់គាត់មិនចាកចេញពីស៊ុមមាសទេ។ "ខ្ញុំបានឆ្ងល់ថាតើពាក្យនៃការយល់ដឹងមិនពិតអាចយល់បានចំពោះសិស្សនៃវិទ្យាសាស្ត្រ Muggle ដែរឬទេ។ ជាក់​ស្តែង​មិន​បាន»។

"ប្រហែលជា" Harry បានចាប់ផ្តើម។

\emph{ពិតជា Ravenclaw?} បាននិយាយថា Slytherin ។ \emph{អ្នកកំពុងទាញនេះ \emph{ឥឡូវនេះ}?}

"ប្រហែលជាខ្ញុំអាចព្យាយាមម្តងទៀតដើម្បីយល់ពីពាក្យប្រសិនបើខ្ញុំដឹងបន្ថែមអំពីកញ្ចក់?" បាននិយាយថាផ្នែក Ravenclaw របស់ Harry ដែលបានគ្រប់គ្រងដោយផ្ទាល់។

បបូរមាត់របស់សាស្រ្តាចារ្យ Quirrell រមួលឡើងលើ។ “ដូចទៅនឹងរឿងបុរាណភាគច្រើនដែរ អ្នកប្រាជ្ញបានកត់ត្រាការភូតភរគ្រប់គ្រាន់ ដែលថាវាពិបាកនឹងប្រាកដណាស់នៅពេលឥឡូវនេះ។ វាច្បាស់ណាស់ថា Mirror យ៉ាងហោចណាស់មានអាយុដូច Merlin ព្រោះវាត្រូវបានគេស្គាល់ថា Merlin បានប្រើវាជាឧបករណ៍។ វាត្រូវបានគេដឹងផងដែរថាបន្ទាប់ពីការស្លាប់របស់គាត់ Merlin បានបន្សល់ទុកនូវការណែនាំជាលាយលក្ខណ៍អក្សរដែលកញ្ចក់មិនចាំបាច់ត្រូវបានផ្សាភ្ជាប់ទោះបីជាវាមានថាមពលជាក់លាក់ដែលជាធម្មតាអាចធ្វើឱ្យមនុស្សម្នាក់ព្រួយបារម្ភក៏ដោយ។ គាត់បានសរសេរថា ដោយសារកញ្ចក់ឆ្លុះត្រូវបានបង្កើតយ៉ាងប៉ិនប្រសប់ ដើម្បីមិនបំផ្លាញពិភពលោក វានឹងកាន់តែងាយស្រួលក្នុងការបំផ្លាញពិភពលោកដោយប្រើឈីសមួយដុំ»។

សេចក្តីថ្លែងការណ៍នេះបានវាយប្រហារ Harry ថាមិនមានការធានាទាំងស្រុងនោះទេ។

"ការពិតមួយចំនួនទៀតអំពីកញ្ចក់ត្រូវបានបញ្ជាក់ដោយអ្នកជំនួយការដ៏ល្បីល្បាញដែលមានការសង្ស័យដោយសមហេតុផល ហើយពាក្យរបស់ពួកគេបានបង្ហាញឱ្យឃើញពីភាពជឿជាក់។ អំណាចលក្ខណៈពិសេសបំផុតរបស់ Mirror គឺបង្កើតអាណាចក្រជំនួសនៃអត្ថិភាព ទោះបីជាអាណាចក្រទាំងនេះមានទំហំប៉ុននឹងអ្វីដែលអាចមើលឃើញនៅក្នុងកញ្ចក់ក៏ដោយ។ វាត្រូវបានគេដឹងថាមនុស្ស និងវត្ថុផ្សេងទៀតអាចត្រូវបានរក្សាទុកនៅទីនោះ។ វាត្រូវបានអះអាងដោយអាជ្ញាធរមួយចំនួនថាកញ្ចក់តែមួយគត់នៃមន្តអាគមទាំងអស់មានទិសដៅសីលធម៌ពិតប្រាកដ ទោះបីជាខ្ញុំមិនប្រាកដថាអ្វីដែលអាចមានន័យនៅក្នុងន័យជាក់ស្តែងក៏ដោយ។ ខ្ញុំនឹងរំពឹងថាអ្នកសីលធម៌នឹងហៅ Cruciatus បណ្តាសាដោយឈ្មោះរបស់ពួកគេនៃ 'អាក្រក់' និង Patronus Charm ដោយឈ្មោះរបស់ពួកគេនៃ 'ល្អ'; ខ្ញុំមិនអាចទស្សន៍ទាយបានថា អ្នកសីលធម៌នឹងគិតអ្វីជាសីលធម៌ \emph{ច្រើន}ជាងនោះ។ ប៉ុន្តែ​ជា​ឧទាហរណ៍​គេ​អះអាង​ថា phœnixes បាន​ចូល​មក​ក្នុង​ពិភពលោក​របស់​យើង​ពី​អាណាចក្រ​មួយ​ដែល​បាន​បង្កើត​ឡើង​នៅ​ក្នុង​កញ្ចក់​នេះ»។

ពាក្យដូចជា \emph{crikey} និងអ្វីដែលឪពុកម្តាយរបស់គាត់នឹងហៅជាភាសាមិនសមរម្យនោះ សុទ្ធតែកំពុងដំណើរការពេញក្បាលរបស់ Harry ដោយមិនមានភាពស៊ីសង្វាក់គ្នានោះទេ នៅពេលដែលគាត់បានសម្លឹងមើលទៅខ្នងមាសនៃកញ្ចក់។

សាស្ត្រាចារ្យ Quirrell បាននិយាយថា "ខ្ញុំបានវង្វេងពិភពលោក ហើយបានជួបប្រទះរឿងជាច្រើនដែលមិនធ្លាប់ឮ" ។ “ពួកគេភាគច្រើនហាក់ដូចជាខ្ញុំកុហក ប៉ុន្តែមានមនុស្សមួយចំនួនតូចមានរង្វង់ប្រវត្តិសាស្ត្រ ជាជាងការនិទានរឿង។ នៅលើជញ្ជាំងដែកក្នុងកន្លែងដែលគ្មាននរណាម្នាក់បានមកអស់រយៈពេលជាច្រើនសតវត្ស ខ្ញុំបានរកឃើញការអះអាងដែលថាជនជាតិ Atlante មួយចំនួនបានមើលឃើញពីអវសាននៃពិភពលោករបស់ពួកគេ ហើយបានស្វែងរកការបង្កើតឧបករណ៍ដ៏មានថាមពលដើម្បីបញ្ចៀសគ្រោះមហន្តរាយដែលជៀសមិនរួច។ ប្រសិនបើឧបករណ៍នោះត្រូវបានបញ្ចប់ រឿងនេះបានអះអាងថា វានឹងក្លាយជាអត្ថិភាពដ៏មានស្ថេរភាព ដែលអាចទប់ទល់នឹងការបញ្ជូនវេទមន្តគ្មានដែនកំណត់ ដើម្បីផ្តល់សេចក្តីប្រាថ្នា។ ហើយ​ផងដែរ—នេះ​ត្រូវ​បាន​គេ​និយាយ​ថា​ជា​កិច្ចការ​ដ៏​លំបាក​ដ៏​ធំ—ឧបករណ៍​នឹង​ជៀស​វាង​ពី​មហន្តរាយ​ដែល​ជៀស​មិន​រួច​ដែល​មនុស្ស​មាន​ប្រាជ្ញា​នឹង​រំពឹង​ថា​នឹង​ធ្វើ​តាម​ពី​ទីតាំង​នោះ។ ទិដ្ឋភាពដែលខ្ញុំចាប់អារម្មណ៍នោះគឺថា យោងទៅតាមរឿងនិទានដែលសរសេរនៅលើបន្ទះដែកទាំងនោះ អ្នកដែលនៅសល់នៃអាត្លង់ទីបានមិនអើពើនឹងគម្រោងនេះ ហើយបន្តដំណើររបស់ពួកគេ។ ពេលខ្លះវាត្រូវបានសរសើរថាជាការខិតខំប្រឹងប្រែងជាសាធារណៈដ៏ថ្លៃថ្នូ ប៉ុន្តែស្ទើរតែទាំងអស់នៅ Atlanteans ផ្សេងទៀតបានរកឃើញកិច្ចការសំខាន់ៗដែលត្រូវធ្វើនៅថ្ងៃណាមួយជាជាងជំនួយ។ សូម្បីតែពួកអភិជននៅអាត្លង់ទីនបានព្រងើយកន្តើយនឹងការរំពឹងទុករបស់នរណាម្នាក់ក្រៅពីខ្លួនពួកគេក្នុងការទទួលបានអំណាចដែលមិនអាចប្រកែកបាន ដែលមនុស្សដែលមានបទពិសោធន៍តិចអាចរំពឹងថានឹងទាក់ទាញចំណាប់អារម្មណ៍របស់ពួកគេ។ ដោយមានការគាំទ្រតិចតួច អ្នកផលិតឧបករណ៍នេះមួយចំនួនតូចបានធ្វើការក្រោមលក្ខខណ្ឌការងារដែលមិនមានភាពលំបាកខ្លាំងពេក ដែលជាការរំខានដោយគ្មានន័យ។ នៅទីបំផុតពេលវេលាបានអស់ហើយ Atlantis ត្រូវបានបំផ្លាញដោយឧបករណ៍នៅតែនៅឆ្ងាយពីពេញលេញ។ ខ្ញុំ​ទទួល​ស្គាល់​ការ​បន្ទរ​ខ្លះ​នៃ​បទ​ពិសោធ​ផ្ទាល់​ខ្លួន​របស់​ខ្ញុំ ដែល​ជា​ធម្មតា​គេ​មិន​ឃើញ​មាន​ការ​បង្កើត​ឡើង​ក្នុង​រឿង​តែ​ប៉ុណ្ណោះ»។ ស្នាមញញឹមស្ងួត។ “ប៉ុន្តែប្រហែលជាវាគ្រាន់តែជាចំណង់ចំណូលចិត្តផ្ទាល់ខ្លួនរបស់ខ្ញុំសម្រាប់រឿងនិទានមួយក្នុងចំណោមរឿងព្រេងរាប់រយផ្សេងទៀត។ ទោះជាយ៉ាងណាក៏ដោយ អ្នកយល់ឃើញថា បន្ទរនៃសេចក្តីថ្លែងការណ៍របស់ Merlin អំពីអ្នកបង្កើតរបស់ Mirror ដែលបង្កើតវាដើម្បីមិនបំផ្លាញពិភពលោក។ សំខាន់បំផុតសម្រាប់គោលបំណងរបស់យើង វាអាចពន្យល់ពីមូលហេតុដែល Mirror នឹងមានសមត្ថភាពមិនស្គាល់ពីមុន ដែល Dumbledore ឬ Perenelle ហាក់ដូចជាបានបំផុសឡើង បង្ហាញពីបុគ្គលណាម្នាក់ដែលឈានជើងមុនវា ជាការបំភាន់នៃពិភពលោកដែលបំណងប្រាថ្នារបស់ពួកគេត្រូវបានបំពេញ។ វាគឺជាប្រភេទនៃការប្រយ័ត្នប្រយែងដ៏សមហេតុសមផល ដែលអ្នកអាចស្រមៃថាមាននរណាម្នាក់កំពុងកសាងខ្លួនចូលទៅក្នុងការបង្កើតដែលផ្តល់ដោយក្តីប្រាថ្នា មានន័យថាមិនខុសអ្វីទាំងអស់។

“អីយ៉ា” Harry ខ្សឹប ហើយចង់មានន័យ។ នេះគឺជាវេទមន្តដែលមានអក្សរធំ M ដែលជាប្រភេទវេទមន្តដែលបានបង្ហាញខ្លួនក្នុង \emph{So You Want To Be A Wizard} មិនមែនគ្រាន់តែជាបណ្តុំនៃវត្ថុដែលបំពានរូបវិទ្យាចៃដន្យទេដែលអ្នកអាចធ្វើបានដោយប្រើដំបង។

សាស្ត្រាចារ្យ Quirrell ធ្វើកាយវិការនៅខាងក្រោយមាស។ “ទ្រព្យសម្បត្តិចុងក្រោយដែលរឿងនិទានភាគច្រើនយល់ស្របនោះគឺថា អ្វីក៏ដោយដែលមិនស្គាល់មធ្យោបាយនៃការបញ្ជាកញ្ចក់ — នៃកូនសោនោះមិនមានគណនីដែលអាចជឿជាក់បាននោះទេ—ការណែនាំរបស់កញ្ចក់មិនអាចមានរាងដើម្បីប្រតិកម្មចំពោះមនុស្សម្នាក់ៗនោះទេ។ ដូច្នេះវាមិនអាចទៅរួចទេសម្រាប់ Perenelle ដើម្បីបញ្ជាកញ្ចក់នេះ "គ្រាន់តែផ្តល់ឱ្យថ្មទៅ Perenelle" ។ Dumbledore មិន​អាច​និយាយ​ថា 'តែ​ផ្តល់​ថ្ម​ដល់​អ្នក​ដែល​ចង់​ប្រគល់​វា​ទៅ Nicholas Flamel'។ មាននៅក្នុងកញ្ចក់ ភាពពិការភ្នែកដូចជាទស្សនវិទូបានសន្មតថាជាឧត្តមគតិយុត្តិធម៌។ ត្រូវតែប្រព្រឹត្តចំពោះអស់អ្នកដែលមកមុនដោយច្បាប់ដូចគ្នា ទោះជាច្បាប់ណាក៏ដោយអាចចូលជាធរមាន។ ដូច្នេះ ត្រូវតែមានច្បាប់មួយចំនួនសម្រាប់ទៅដល់កន្លែងលាក់ខ្លួនរបស់ថ្ម ដែលអ្នកណាម្នាក់អាចហៅបាន។ ហើយឥឡូវនេះ អ្នកឃើញហើយថាហេតុអ្វីបានជា \emph{you} ហៅថា Boy-Who-Lived នឹងអនុវត្តយុទ្ធសាស្រ្តអ្វីក៏ដោយដែលយើងទាំងពីរបង្កើត។ ត្បិត​មាន​គេ​និយាយ​ថា​រឿង​នេះ​មាន​ការ​តម្រង់​ទិស​ខាង​សីលធម៌ ហើយ​វា​ប្រហែល​ជា​ត្រូវ​បាន​គេ​ឲ្យ​បញ្ជា​ដែល​ឆ្លុះ​បញ្ចាំង​ដូច​គ្នា។ ខ្ញុំ​ដឹង​ច្បាស់​ហើយ​ថា តាម​ពាក្យ​ធម្មតា អ្នក​ត្រូវ​បាន​គេ​និយាយ​ថា​ល្អ ដូច​ខ្ញុំ​ត្រូវ​គេ​និយាយ​ថា​អាក្រក់»។ សាស្រ្តាចារ្យ Quirrell ញញឹមយ៉ាងងងឹត។ "ដូច្នេះជាការប៉ុនប៉ងលើកដំបូងរបស់យើង - ទោះបីជាមិនមែនជាចុងក្រោយរបស់យើងក៏ដោយ - ធានាឱ្យយើងមើលថាតើកញ្ចក់នេះបង្កើតអ្វីខ្លះពីការប៉ុនប៉ងរបស់អ្នកដើម្បីទាញយកថ្មដើម្បីជួយសង្គ្រោះជីវិតរបស់ Hermione Granger និងសិស្សរាប់រយនាក់របស់អ្នក" ។

Harry ដែលកំពុងចាប់ផ្តើមយល់នៅទីបំផុតបាននិយាយថា "ហើយកំណែ \emph{first} នៃផែនការនោះ គឺជាអ្វីដែលអ្នកបានបង្កើតកាលពីថ្ងៃសុក្រក្នុងសប្តាហ៍ដំបូងរបស់ខ្ញុំនៅ Hogwarts បានអំពាវនាវឱ្យយកថ្មមកដោយកូនមាសរបស់ Dumbledore Boy-Who-Lived ព្យាយាមមិនគិតតែពីប្រយោជន៍ផ្ទាល់ខ្លួន និងដ៏ថ្លៃថ្នូ ដើម្បីជួយសង្រ្គោះជីវិតរបស់គ្រូការពារជាតិដែលបានស្លាប់របស់គាត់ គឺសាស្រ្តាចារ្យ Quirrell ។

សាស្រ្តាចារ្យ Quirrell បាននិយាយថា "ពិតណាស់" ។

វាជារឿងបែបកំណាព្យដែល Harry សន្មត់ថា ប៉ុន្តែការកោតសរសើររបស់គាត់ចំពោះភាពឆើតឆាយនោះត្រូវបានរារាំងដោយកាលៈទេសៈជុំវិញ។

បន្ទាប់មក មានគំនិតមួយទៀតកើតឡើងចំពោះ Harry ។

"អឹម" Harry បាននិយាយ។ “អ្នកគិតថាកញ្ចក់នេះគឺជាអន្ទាក់សម្រាប់អ្នក…”

«​គ្មាន​ផ្លូវ​នៅ​ក្រោម​មេឃ​ដែល​មិន​មាន​ន័យ​ថា​ជា​អន្ទាក់​នោះ​ទេ​»។

“នោះមានន័យថា វាជាអន្ទាក់សម្រាប់ Lord Voldemort ។ មានតែវាទេដែលមិនអាចជាអន្ទាក់សម្រាប់គាត់ផ្ទាល់។ វាត្រូវតែមានច្បាប់ទូទៅមួយដែលគូសបញ្ជាក់វា គុណភាពដែលអាចយល់បានទូទៅមួយចំនួនរបស់ Lord Voldemort ដែលជំរុញវា”។ ដោយ​គ្មាន​ការ​ដឹង​ខ្លួន Harry កំពុង​តែ​ងឿង​ឆ្ងល់​ចំពោះ​ខ្នង​មាស​របស់ Mirror។

សាស្ត្រាចារ្យ Quirrell ដែលចាប់ផ្តើមងឿងឆ្ងល់ចំពោះមុខ Harry បាននិយាយថា "ដូចដែលអ្នកនិយាយ។

“មែនហើយ នៅថ្ងៃព្រហស្បត្តិ៍ទី 1 នៃឆ្នាំនេះ ចៅហ្វាយនាយឆ្កួត Dumbledore ដែលខ្ញុំទើបតែឃើញដុតមាន់ បានប្រាប់ខ្ញុំថា ខ្ញុំគ្មានឱកាសអ្វីក្នុងការចូលទៅច្រករបៀងហាមឃាត់របស់គាត់ទេ ព្រោះខ្ញុំមិនដឹងអក្ខរាវិរុទ្ធទេ € 24€{Alohomora។}”

សាស្រ្តាចារ្យ Quirrell បាននិយាយថា "ខ្ញុំ \emph{មើល}"។ “អូ សម្លាញ់។ ខ្ញុំ​ចង់​ឲ្យ​អ្នក​គិត​ចង់​និយាយ​រឿង​នេះ​ដល់​ខ្ញុំ​មុន​នេះ​បន្តិច»។

ពួកគេទាំងពីរនាក់មិនចាំបាច់និយាយឱ្យច្បាស់ថា ចិត្តវិទ្យាបញ្ច្រាសបញ្ច្រាសនេះបានធានាដោយជោគជ័យថា Harry នឹងនៅឆ្ងាយពីច្រករបៀងហាមឃាត់របស់ Dumbledore ។

Harry នៅតែផ្តោតអារម្មណ៍។ "តើអ្នកគិតថា Dumbledore សង្ស័យថាខ្ញុំជា horcrux របស់ Lord Voldemort ឬជាទូទៅថាទិដ្ឋភាពមួយចំនួននៃបុគ្គលិកលក្ខណៈរបស់ខ្ញុំត្រូវបានចម្លងពី Lord Voldemort?" សូម្បីតែ Harry បានសួរវាខ្លាំងៗក៏ដោយ គាត់បានដឹងថាវាជាសំណួរដ៏ល្ងង់ខ្លៅមួយ ហើយតើភស្តុតាងដែលគួរឱ្យខ្ពើមរអើមទាំងស្រុងប៉ុណ្ណា ដែលគាត់បានឃើញរួចហើយ—

សាស្ត្រាចារ្យ Quirrell បាននិយាយថា "Dumbledore មិនអាច \emph{ssibly} បានខកខានវាទេ។ “វា​មិន​ច្បាស់​លាស់​ទេ។ តើ Dumbledore គិតអ្វីទៀតដែលអ្នកជាតួសម្តែងក្នុងរឿងដែលអ្នកនិពន្ធដ៏ល្ងង់ខ្លៅមិនដែលជួបក្មេងអាយុ 11 ឆ្នាំពិតប្រាកដ? មាន​តែ​មនុស្ស​ខ្ជិល​ច្រអូស​ប៉ុណ្ណោះ​ដែល​នឹង​ជឿ​រឿង​នោះ—បាទ មិន​អី​ទេ»។

ពួកគេទាំងពីរសម្លឹងមើលកញ្ចក់ដោយស្ងៀមស្ងាត់។

ទីបំផុត សាស្ត្រាចារ្យ Quirrell ដកដង្ហើមធំ។ “ខ្ញុំ​ខ្លាច​ខ្លួន​ឯង​ខុស។ ទាំងអ្នក និងខ្ញុំក៏មិនហ៊ានឆ្លុះបញ្ចាំងនៅក្នុងកញ្ចក់នេះដែរ។ ខ្ញុំគិតថាខ្ញុំត្រូវតែបញ្ជាឱ្យសាស្រ្តាចារ្យ Sprout លុបចោលនូវ Obliviations របស់ខ្ញុំចំពោះ Mr~Nott និង Miss~Greengrass... អ្នកឃើញហើយ ការលំបាកដ៏អស្ចារ្យមួយទៀតរបស់ Mirror គឺថា ច្បាប់ដែលវាប្រព្រឹត្តចំពោះអ្នកដែលឆ្លុះបញ្ចាំងនឹងមិនយកចិត្តទុកដាក់លើកម្លាំងខាងក្រៅ ដូចជា ការចងចាំមិនពិត ឬ មន្តស្នេហ៍ Confundus ។ កញ្ចក់ឆ្លុះបញ្ជាំងតែកម្លាំងទាំងនោះដែលកើតចេញពីក្នុងខ្លួនមនុស្ស ស្ថានភាពនៃចិត្តដែលពួកគេមកដល់តាមរយៈជម្រើសរបស់ពួកគេផ្ទាល់។ ដូច្នេះវាត្រូវបានគេនិយាយនៅកន្លែងជាច្រើន។ នោះហើយជាមូលហេតុដែលខ្ញុំមាន Mr~Nott និង Miss~Greengrass ដោយជឿលើរឿងរ៉ាវផ្សេងៗគ្នាអំពីមូលហេតុដែលការស្រង់ចេញរបស់ថ្មគឺជាការចាំបាច់ ត្រៀមខ្លួនដើម្បីបង្ហាញខ្លួននៅមុខកញ្ចក់នេះ។ សាស្រ្តាចារ្យ Quirrell បានត្រដុសនៅស្ពានច្រមុះរបស់គាត់។ “ខ្ញុំបានបង្កើតរឿងផ្សេងទៀតសម្រាប់សិស្សផ្សេងទៀត ដោយត្រៀមខ្លួនសម្រាប់ខ្ញុំក្នុងចលនាជាមួយនឹងគន្លឹះដែលបានជ្រើសរើស… ប៉ុន្តែនៅពេលថ្ងៃនេះជិតមកដល់ ខ្ញុំចាប់ផ្តើមមានអារម្មណ៍ទុទិដ្ឋិនិយមចំពោះគម្រោងនេះ។ ដូចជា Nott និង Greengrass នៅតែមានតម្លៃសាកល្បង ប្រសិនបើយើងមិនអាចគិតពីអ្វីដែលប្រសើរជាងនេះ។ ប៉ុន្តែខ្ញុំឆ្ងល់ថាតើ Dumbledore បានព្យាយាមបង្កើតល្បែងផ្គុំរូបនេះដើម្បីទប់ទល់នឹងល្បិចកលរបស់ Voldemort យ៉ាងជាក់លាក់ដែរឬទេ។ ខ្ញុំឆ្ងល់ថាតើគាត់អាចជោគជ័យឬអត់? ប្រសិនបើអ្នកបង្កើតផែនការជំនួសដែលខ្ញុំយល់ព្រមគ្រប់គ្រាន់ដើម្បីសាកល្បង \parsel{ខ្ញុំសន្យាថាអ្វីក៏ដោយដែលខ្ញុំផ្ញើមកនឹងមិនបង្កគ្រោះថ្នាក់ដល់ខ្ញុំទេ ពេលនោះ ឬមិនធ្លាប់មាន។ ខ្ញុំក៏មិនរំពឹងថានឹងបំពានការសន្យានោះដែរ។ ហើយខ្ញុំសូមរំលឹកអ្នកម្តងទៀតអំពីចំណាប់ខ្មាំងដែលខ្ញុំជាប់នឹងការបរាជ័យរបស់ខ្ញុំ ទាំង Miss~Granger និងអ្នកផ្សេងទៀតទាំងអស់”។

ជាថ្មីម្តងទៀតពួកគេបានសម្លឹងមើលកញ្ចក់ក្នុងភាពស្ងៀមស្ងាត់ ចាស់ទុំ Tom Riddle និងក្មេងជាង។

"ខ្ញុំសង្ស័យថាសាស្រ្តាចារ្យ" Harry បាននិយាយបន្ទាប់ពីមួយរយៈថា "សម្មតិកម្មថ្នាក់ទាំងមូលរបស់អ្នកអំពីនរណាម្នាក់ដែលចង់បានថ្មសម្រាប់គោលបំណងល្អឬស្មោះត្រង់គឺខុស។ នាយក​សាលា​នឹង​មិន​កំណត់​ច្បាប់​ទាញ​យក​បែប​នេះ​ទេ»។

“ហេតុអ្វី?”

"ដោយសារតែ Dumbledore ដឹងថាវាងាយស្រួលប៉ុណ្ណាក្នុងការបញ្ចប់ការជឿថាអ្នកកំពុងធ្វើរឿងត្រឹមត្រូវនៅពេលដែលអ្នកមិនធ្វើ។ វា​ជា​លទ្ធភាព​ដំបូង​ដែល​គាត់​បាន​ស្រមៃ»។

“\parsel{តើវាជាការពិត ឬល្បិចដែលខ្ញុំឮ?}”

Harry បាននិយាយថា “\parsel{ខ្ញុំនិយាយដោយស្មោះត្រង់}”។

សាស្ត្រាចារ្យ Quirrell ងក់ក្បាល។ "បន្ទាប់មកចំណុចរបស់អ្នកត្រូវបានយកល្អ" ។

Harry បាននិយាយថា "ខ្ញុំមិនប្រាកដថាហេតុអ្វីបានជាអ្នកគិតថាល្បែងផ្គុំរូបនេះអាចដោះស្រាយបាន" ។ "គ្រាន់តែកំណត់ច្បាប់ដូចជា ដៃឆ្វេងរបស់អ្នកត្រូវតែកាន់សាជីជ្រុងពណ៌ខៀវតូចមួយ និងពីរ៉ាមីតក្រហមធំពីរ ហើយដៃស្តាំរបស់អ្នកត្រូវច្របាច់ mayonnaise លើ Hamster មួយ -"

សាស្រ្តាចារ្យ Quirrell បាននិយាយថា "ទេ" ។ “ទេ ខ្ញុំគិតថាមិនមែនទេ។ រឿងព្រេងមិនច្បាស់អំពីច្បាប់អ្វីដែលអាចត្រូវបានផ្តល់ឱ្យ ប៉ុន្តែខ្ញុំគិតថាវាត្រូវតែមានអ្វីដែលត្រូវធ្វើជាមួយការប្រើប្រាស់ដែលមានបំណងដើមរបស់ Mirror - វាត្រូវតែមានអ្វីដែលត្រូវធ្វើជាមួយបំណងប្រាថ្នា និងបំណងប្រាថ្នាដ៏ជ្រាលជ្រៅដែលកើតចេញពីខាងក្នុងមនុស្ស។ ការច្របាច់ mayonnaise ទៅលើ hamster នឹងមិនមានលក្ខណៈគ្រប់គ្រាន់នោះទេ សម្រាប់មនុស្សភាគច្រើន។

“ហ៊ឺ” ហារី និយាយ។ "ប្រហែលជាច្បាប់គឺថាមនុស្សមិនត្រូវចង់ប្រើថ្មទាល់តែសោះ - ទេវាងាយស្រួលពេករឿងដែលអ្នកបានផ្តល់ឱ្យលោក ~ Nott ដោះស្រាយវា" ។

សាស្ត្រាចារ្យ Quirrell បាននិយាយថា "តាមវិធីខ្លះ អ្នកអាចយល់ Dumbledore បានល្អជាងខ្ញុំ" ។ "ដូច្នេះឥឡូវនេះខ្ញុំសួរអ្នកថា: តើ Dumbledore នឹងប្រើគំនិតរបស់គាត់អំពីការទទួលយកសេចក្តីស្លាប់ដើម្បីការពារថ្មនេះដោយរបៀបណា? លើសពីនេះ គាត់គិតថាខ្ញុំមិនអាចយល់បាន ហើយគាត់ក៏មិនខុសឆ្ងាយដែរ»។

Harry បានគិតអំពីរឿងនេះមួយសន្ទុះ ដោយពិចារណាលើគំនិតជាច្រើន ហើយបោះវាចោល។ ហើយបន្ទាប់មក ដោយបានគិតពីអ្វីមួយ ហារីបានពិចារណានៅស្ងៀម… មុននឹងបង្ហាញផ្នែកជាក់ស្តែងនៃការសន្ទនានាពេលអនាគត ដែលសាស្រ្តាចារ្យ Quirrell បានសុំឱ្យគាត់និយាយជាភាសា Parseltongue ប្រសិនបើគាត់គិតអ្វីមួយ។

Harry និយាយដោយស្ទាក់ស្ទើរ។ "តើ Dumbledore គិតថា Mirror នេះអាចទៅដល់ជីវិតចុងក្រោយបានទេ? តើគាត់អាចដាក់ថ្មចូលទៅក្នុងអ្វីមួយដែលគាត់ \emph{គិតថា} គឺជាជីវិតបន្ទាប់បន្សំ ដូច្នេះមានតែមនុស្សដែលជឿលើជីវិតក្រោយប៉ុណ្ណោះដែលអាចឃើញវា?

“ហ៊ឹម…” សាស្ត្រាចារ្យ Quirrell បាននិយាយ។ “ប្រហែលជា… បាទ មានភាពជឿជាក់ជាក់លាក់មួយចំពោះវា។ ការប្រើការកំណត់នៃកញ្ចក់នេះ ដើម្បីបង្ហាញមនុស្សពីបំណងប្រាថ្នារបស់ពួកគេ...Albus Dumbledore នឹងឃើញខ្លួនឯងបានជួបជុំគ្រួសាររបស់គាត់ឡើងវិញ។ គាត់នឹងឃើញខ្លួនឯងរួបរួមគ្នាជាមួយពួកគេ\emph{in death} ដោយចង់ស្លាប់ខ្លួនឯងជាជាងចង់ឱ្យពួកគេរស់ឡើងវិញ។ បងប្រុសរបស់គាត់ Aberforth ប្អូនស្រីរបស់គាត់ Ariana ឪពុកម្តាយរបស់គាត់ Kendra និង Percival... វានឹងក្លាយជា Aberforth ដែល Dumbledore បានផ្តល់ថ្ម។ ខ្ញុំគិតថា។ តើ Mirror ទទួលស្គាល់ថា Aberforth ជាពិសេសត្រូវបានផ្តល់ឱ្យ Stone ដែរឬទេ? ឬ​មួយ​សាច់​ញាតិ​របស់​អ្នក​ណា​ម្នាក់​ដែល​ស្លាប់​នឹង​ធ្វើ បើ​មនុស្ស​នោះ​ជឿ​ថា​វិញ្ញាណ​របស់​ញាតិ​សន្ដាន​នឹង​ប្រគល់​ថ្ម​មក​វិញ?»។ សាស្ត្រាចារ្យ Quirrell កំពុងដើរក្នុងរង្វង់ខ្លី ដោយរក្សាឱ្យឆ្ងាយពី Harry និង Mirror នៅពេលគាត់ផ្លាស់ទី។ "ប៉ុន្តែទាំងអស់នេះគ្រាន់តែជាគំនិតមួយ។ ចូរ​យើង​បង្កើត​រឿង​មួយ​ទៀត»។

Harry ចាប់​ផ្តើម​គោះ​ថ្ពាល់​របស់​គាត់ បន្ទាប់​មក​ឈប់​ភ្លាមៗ​ពេល​គាត់​ដឹង​ថា​គាត់​យក​កាយវិការ​នោះ​មក​ពី​ណា។ “ចុះបើ Perenelle ជាអ្នកដាក់ថ្មនៅទីនេះ? ប្រហែល​ជា​នាង​បាន​ចាក់សោ​កញ្ចក់​ដើម្បី​ឲ្យ​ថ្ម​ទៅ​តែ​អ្នក​ដែល​ដាក់​វា​ពី​ដើម​ប៉ុណ្ណោះ»។

សាស្រ្តាចារ្យ Quirrell បាននិយាយថា "Perenelle បានរស់នៅបានយូរដោយដឹងពីដែនកំណត់របស់នាង" ។ “នាង​មិន​វាយតម្លៃ​លើ​បញ្ញា​របស់​ខ្លួន​ឯង​ពេក​ទេ នាង​មិន​មាន​មោទនភាព​ឡើយ បើ​ដូច្នោះ​នាង​នឹង​បាត់​បង់​ថ្ម​ជា​យូរ​មក​ហើយ។ Perenelle នឹងមិនព្យាយាមគិតពី Mirror-rule ដ៏ល្អសម្រាប់ខ្លួនឯងនោះទេ មិនមែននៅពេលដែល Master Flamel អាចទុកបញ្ហានៅក្នុងដៃដ៏ឆ្លាតវៃរបស់ Dumbledore នោះទេ… ប៉ុន្តែច្បាប់នៃការប្រគល់ Stone ទៅកាន់អ្នកដែលចងចាំការដាក់វាក៏ដំណើរការផងដែរ ប្រសិនបើ Dumbledore ខ្លួនឯងបានដាក់ ថ្ម។ វា​ជា​ច្បាប់​មួយ​ដ៏​ពិបាក​ក្នុង​ការ​រំលង ព្រោះ​ខ្ញុំ​មិន​អាច​បំភាន់​អ្នក​ណា​ម្នាក់​ឱ្យ​ជឿ​ថា​គេ​ដាក់​ក្នុង​ថ្ម​នោះ​ទេ… ខ្ញុំ​នឹង​ត្រូវ​បង្កើត​ថ្ម​ក្លែងក្លាយ និង​កញ្ចក់​ក្លែងក្លាយ ហើយ​រៀបចំ​រឿង​នោះ…” សាស្ត្រាចារ្យ Quirrell កំពុង​តែ​ងឿង​ឆ្ងល់។ . "ប៉ុន្តែវានៅតែជាអ្វីមួយដែល Dumbledore នឹងស្រមៃថា Voldemort អាចរៀបចំពេលវេលាដែលបានផ្តល់ឱ្យ។ ប្រសិនបើអាចធ្វើបានទាំងអស់ Dumbledore នឹងចង់ធ្វើឱ្យកូនសោទៅកាន់ Mirror ក្លាយជាស្ថានភាពនៃចិត្តដែលគាត់គិតថាខ្ញុំ \emph{មិនអាច} រៀបចំជាបញ្ចាំ—ឬច្បាប់ដែល Dumbledore គិតថា Voldemort មិនអាចយល់បាន ដូចជាច្បាប់ដែលទាក់ទងនឹង ការទទួលយកការស្លាប់របស់ខ្លួនឯង។ ហេតុ​ដូច្នេះ​ហើយ​បាន​ជា​ខ្ញុំ​ចាត់​ទុក​គំនិត​មុន​របស់​អ្នក​គឺ​អាច​ជឿ​បាន»។

បន្ទាប់មក Harry មានគំនិតមួយ។

គាត់​មិន​ប្រាកដ​ថា​វា​ជា​គំនិត​ល្អ​ឬ​អត់។

… វាមិនដូចជា Harry មានជម្រើសច្រើននៅទីនេះទេ។

Harry បាននិយាយថា "Arguendo" ។ “យើង​មិន​ប្រាកដ​ថា​មាន​អ្វី​ចាំបាច់​ដើម្បី​យក​ថ្ម​មក​វិញ​ទេ។ ប៉ុន្តែលក្ខខណ្ឌ \emph{គ្រប់គ្រាន់} គួរតែពាក់ព័ន្ធនឹង Albus Dumbledore ឬប្រហែលជាមាននរណាម្នាក់ផ្សេងទៀតនៅក្នុងស្ថានភាពនៃចិត្តដែលពួកគេជឿថា Dark Lord ត្រូវបានចាញ់ ថាការគំរាមកំហែងបានបញ្ចប់ ហើយថាវាដល់ពេលដែលត្រូវដកចេញ។ ថ្មហើយប្រគល់វាទៅ Nicholas Flamel ។ យើង​មិន​ប្រាកដ​ថា​ផ្នែក​ណា​នៃ​ចិត្ត​គំនិត​របស់​មនុស្ស​នោះ​ទេ ចូរ​និយាយ​ថា Dumbledore's នឹង​ជា​ផ្នែក​ចាំបាច់​ដែល​គាត់​គិត​ថា Lord Voldemort មិន​អាច​យល់​បាន ឬ​ចម្លង​ពី​គ្នា​នោះ​ទេ។ ប៉ុន្តែនៅក្រោមលក្ខខណ្ឌទាំងនោះ ស្ថានភាពចិត្តទាំងមូលរបស់ Dumbledore នឹងមាន\emph{គ្រប់គ្រាន់}។

សាស្រ្តាចារ្យ Quirrell បាននិយាយថា "សមហេតុផល" ។ “ដូច្នេះ?”

Harry បាននិយាយដោយប្រុងប្រយ័ត្នថា "យុទ្ធសាស្រ្តដែលត្រូវគ្នា" គឺធ្វើត្រាប់តាមស្ថានភាពចិត្តរបស់ Dumbledore នៅក្រោមលក្ខខណ្ឌទាំងនោះឱ្យបានលម្អិតតាមដែលអាចធ្វើទៅបាន ខណៈពេលដែលឈរនៅមុខកញ្ចក់។ ហើយ​ចិត្ត​គំនិត​នេះ​ត្រូវ​តែ​ត្រូវ​បាន​បង្កើត​ឡើង​ដោយ​កម្លាំង​ផ្ទៃក្នុង មិន​មែន​ជា​មនុស្ស​ខាង​ក្រៅ​ទេ»។

"ប៉ុន្តែតើយើងត្រូវធ្វើដូចម្តេចដើម្បីទទួលបានវាដោយគ្មានភាពស្របច្បាប់ឬមន្តស្នេហ៍ Confundus ដែលទាំងពីរនេះពិតជាខាងក្រៅ - ហា។ ខ្ញុំ\emph{មើល}។ ភ្នែកស្លេករបស់សាស្រ្តាចារ្យ Quirrell បានទម្លុះភ្លាមៗ។ "អ្នកណែនាំថាខ្ញុំ Confund \emph{myself} ដូចដែលអ្នកបានបោះ hex នោះមកលើខ្លួនអ្នកក្នុងអំឡុងពេលថ្ងៃដំបូងរបស់អ្នកនៅក្នុង Battle Magic ។ ដូច្នេះ​ថា​វា​ជា​កម្លាំង​ខាងក្នុង ហើយ​មិន​មែន​ជា​ខាងក្រៅ​ទេ គឺ​ជា​ស្ថានភាព​នៃ​ចិត្ត​ដែល​កើត​ឡើង​ដោយ​ការ​ជ្រើសរើស​របស់​ខ្ញុំ​តែ​ប៉ុណ្ណោះ។ ប្រាប់ខ្ញុំថាតើអ្នកបានធ្វើការណែនាំនេះដោយចេតនាដើម្បីចាប់ខ្ញុំទេ ក្មេងប្រុស។ និយាយវាមកខ្ញុំជាភាសា Parseltongue ។

“\parsel{គំនិតរបស់ខ្ញុំដែលអ្នកបានស្នើសុំឱ្យបង្កើតយុទ្ធសាស្ត្រ ប្រហែលជាត្រូវបានជះឥទ្ធិពលដោយចេតនាបែបនេះ—តើអ្នកណាដឹង? ដឹង​ថា​អ្នក​នឹង​មាន​ការ​សង្ស័យ​, សួរ​សំណួរ​នេះ​យ៉ាង​ខ្លាំង​។ ការ​សម្រេច​ចិត្ត​គឺ​អាស្រ័យ​លើ​អ្នក​គ្រូ។ ខ្ញុំ​មិន​ដឹង​អ្វី​ដែល​អ្នក​មិន​ដឹង​អំពី​ថា​តើ​នេះ​ទំនង​ជា​នឹង​ចាប់​អ្នក​។ កុំ​ហៅ​វា​ថា​ក្បត់​ដោយ​ខ្ញុំ​ប្រសិន​បើ​អ្នក​ជ្រើស​រើស​នេះ​សម្រាប់​ខ្លួន​អ្នក​ហើយ​វា​បរាជ័យ​។

សាស្រ្តាចារ្យ Quirrell ដែល\emph{កំពុង}ញញឹមបាននិយាយថា "គួរឱ្យស្រឡាញ់"។ "ខ្ញុំគិតថាមានការគម្រាមកំហែងមួយចំនួនពីគំនិតច្នៃប្រឌិតដែលសូម្បីតែការសាកសួរនៅក្នុង Parseltongue ក៏មិនអាចបន្សាបបានដែរ។"

\later

Harry ពាក់អាវបិទបាំងភាពមើលមិនឃើញ តាមបញ្ជារបស់សាស្រ្តាចារ្យ Quirrell ដល់\parsel{បញ្ឈប់បុរសដែលនឹងជឿថាខ្លួនឯងជាគ្រូសាលាមិនអោយឃើញអ្នក} ដូចដែលសាស្រ្តាចារ្យ Quirrell បាននិយាយនៅក្នុង Parseltongue។

សាស្ត្រាចារ្យ Quirrell បាននិយាយថា "ពាក់អាវ ឬអត់ អ្នកនឹងឈរនៅជួរកញ្ចក់ដោយខ្លួនឯង" ។ “ប្រសិនបើមានកម្អែភ្នំភ្លើងចេញមក អ្នកក៏នឹងឆេះដែរ។ ខ្ញុំ​មាន​អារម្មណ៍​ថា​ស៊ីមេទ្រី​ច្រើន​គួរ​តែ​អនុវត្ត»។

សាស្ត្រាចារ្យ Quirrell បាន​ចង្អុល​ទៅ​កន្លែង​មួយ​នៅ​ជិត​ទ្វារ​ខាងស្តាំ ដែល​ពួកគេ​បាន​ចូល​ទៅ​ក្នុង​បន្ទប់​នោះ មុន​នឹង​កញ្ចក់ និង​នៅ​ខាង​ក្រោយ​វា​។ Harry ពាក់អាវក្រោះ បានទៅកន្លែងដែលសាស្រ្តាចារ្យ Quirrell បានចង្អុលគាត់ ហើយមិនបានប្រកែកទេ។ កាន់តែមានភាពមិនច្បាស់លាស់ចំពោះ Harry ថាតើ Riddles ទាំងពីរនាក់ដែលស្លាប់នៅទីនេះអាចជារឿងអាក្រក់ សូម្បីតែចំណាប់ខ្មាំងសិស្សរាប់រយនាក់ផ្សេងទៀតនៅក្នុងភាគហ៊ុនក៏ដោយ។ ចំពោះចេតនាល្អទាំងអស់របស់ Harry ភាគច្រើនគាត់បានបង្ហាញខ្លួនឯងថាជាមនុស្សល្ងង់ ហើយ Lord Voldemort ត្រឡប់មកវិញគឺជាការគំរាមកំហែងដល់ពិភពលោកទាំងមូល។

(ទោះបីជាវិធីណាក៏ដោយ Harry មិនអាចឃើញ Dumbledore ធ្វើរឿងកម្អែរនោះទេ។ Dumbledore ប្រហែលជាខឹងនឹង Voldemort ដើម្បីបោះបង់ការអត់ធ្មត់ធម្មតារបស់គាត់ ប៉ុន្តែ lava នឹងមិនបញ្ឈប់អង្គភាពដែល Dumbledore ជឿថាជាព្រលឹងបំបែកខ្លួនឡើយ។)

បន្ទាប់មកសាស្ត្រាចារ្យ Quirrell បានចង្អុលដៃរបស់គាត់ ហើយរង្វង់ភ្លឺចាំងមួយបានលេចឡើងនៅជុំវិញកន្លែងដែល Harry កំពុងឈរនៅលើឥដ្ឋ។ នេះ សាស្ត្រាចារ្យ Quirrell បាននិយាយថា ឆាប់ៗនេះនឹងក្លាយទៅជារង្វង់ធំនៃការលាក់បាំង ដែលគ្មានអ្វីនៅក្នុងរង្វង់នោះអាចត្រូវបានគេឮ ឬមើលឃើញពីខាងក្រៅនោះទេ។ Harry នឹង​មិន​អាច​បង្ហាញ​ខ្លួន​គាត់​ចំពោះ Dumbledore ក្លែងក្លាយ​ដោយ​ការ​ដោះ​អាវ​ក្រៅ​ឬ​ដោយ​ការ​ស្រែក​ទេ។

សាស្រ្តាចារ្យ Quirrell បាននិយាយថា "អ្នក \emph{នឹងមិន} ឆ្លងកាត់រង្វង់នេះទេ នៅពេលដែលវាសកម្ម។ "នោះនឹងធ្វើឱ្យអ្នកប៉ះវេទមន្តរបស់ខ្ញុំ ហើយខណៈពេលដែលមានការខ្មាស់អៀន ខ្ញុំប្រហែលជាមិនចាំពីរបៀបដើម្បីបញ្ឈប់ការបន្លឺសំឡេងដែលនឹងបំផ្លាញពួកយើងទាំងពីរនោះទេ។ ហើយលើសពីនេះទៅទៀត ដោយសារខ្ញុំមិនចង់ឱ្យអ្នកបោះស្បែកជើង—” សាស្ត្រាចារ្យ Quirrell បានធ្វើកាយវិការមួយផ្សេងទៀត ហើយគ្រាន់តែនៅក្នុងរង្វង់ដ៏អស្ចារ្យនៃការលាក់បាំងនោះ ពន្លឺចែងចាំងបន្តិចបានលេចឡើងនៅលើអាកាស ដែលជាការបង្ខូចទ្រង់ទ្រាយរាងដូចផែនដី។ “\parsel{របាំងនេះនឹងផ្ទុះឡើង ប្រសិនបើប៉ះអ្នក ឬវត្ថុផ្សេងៗ។} សំឡេងបន្លឺឡើងអាចនឹងវាយប្រហារមកលើខ្ញុំនៅពេលក្រោយ ប៉ុន្តែអ្នកក៏នឹងត្រូវស្លាប់ដែរ។ ឥឡូវនេះប្រាប់ខ្ញុំជាភាសា Parseltongue ថាអ្នកមិនមានបំណងឆ្លងរង្វង់នេះ ឬដោះអាវរបស់អ្នកចេញ ឬធ្វើ\emph{អ្វីទាំងអស់} ដោយចេតនា ឬឆោតល្ងង់។ ប្រាប់​ខ្ញុំ​ថា​អ្នក​នឹង​រង់ចាំ​យ៉ាង​ស្ងៀម​ស្ងាត់​នៅ​ទី​នេះ​ក្រោម​អាវ​ធំ​រហូត​ដល់​វា​ចប់»។

នេះ Harry បាននិយាយឡើងវិញ។

បន្ទាប់មក អាវផាយរបស់សាស្រ្តាចារ្យ Quirrell ប្រែជាពណ៌ខ្មៅលាយមាស អាវផាយដូចជា Dumbledore អាចពាក់ក្នុងឱកាសផ្លូវការ។ ហើយសាស្រ្តាចារ្យ Quirrell បានចង្អុលដៃរបស់គាត់នៅលើក្បាលរបស់គាត់។

សាស្ត្រាចារ្យ Quirrell បាន​នៅ​ស្ងៀម​អស់​រយៈ​ពេល​ជា​យូរ​មក​ហើយ ដោយ​នៅ​តែ​កាន់​ដំបង​នៅ​នឹង​ក្បាល។ ភ្នែករបស់គាត់បានបិទដោយផ្តោតអារម្មណ៍។

ហើយបន្ទាប់មកសាស្រ្តាចារ្យ Quirrell បាននិយាយថា “\emph{Confundus។}”

ភ្លាមៗនោះ ការបញ្ចេញមតិរបស់បុរសដែលឈរនៅទីនោះបានផ្លាស់ប្តូរ។ គាត់​ព្រិច​ភ្នែក​ពីរបី​ដង​ដូច​ជា​ច្របូកច្របល់ ដោយ​បន្ទាប​ដំបង​របស់គាត់។

ការនឿយហត់យ៉ាងខ្លាំងបានរាលដាលលើផ្ទៃមុខដែលសាស្រ្តាចារ្យ Quirrell បានពាក់។ ដោយគ្មានការផ្លាស់ប្តូរដែលអាចមើលឃើញ ភ្នែករបស់គាត់ហាក់ដូចជាចាស់ជាងនេះ បន្ទាត់ពីរបីនៅលើមុខរបស់គាត់បានទាក់ទាញចំណាប់អារម្មណ៍ចំពោះខ្លួនឯង។

បបូរមាត់​របស់​គាត់​ត្រូវ​បាន​ដាក់​ក្នុង​ស្នាម​ញញឹម​សោកសៅ។

ដោយ​មិន​ប្រញាប់​ប្រញាល់​នោះ​ទេ បុរស​នោះ​ក៏​ដើរ​ទៅ​កញ្ចក់​ដោយ​ស្ងាត់​ៗ ហាក់​ដូច​ជា​គាត់​មាន​ពេល​វេលា​ទាំង​អស់​ក្នុង​លោក។

គាត់បានឆ្លងចូលទៅក្នុងជួរនៃការឆ្លុះបញ្ចាំងរបស់ Mirror ដោយមិនមានអ្វីកើតឡើង ហើយសម្លឹងមើលទៅលើផ្ទៃ។

អ្វី​ដែល​បុរស​នោះ​អាច​នឹង​កំពុង​ឃើញ​នៅ​ទី​នោះ Harry មិន​អាច​ប្រាប់​បាន; ចំពោះ Harry វាហាក់ដូចជាថាផ្ទៃរាបស្មើ និងល្អឥតខ្ចោះនៅតែឆ្លុះបញ្ចាំងពីបន្ទប់នៅពីក្រោយវា ដូចជាច្រកទៅកាន់កន្លែងផ្សេង។

"Ariana" បុរសនោះដកដង្ហើមធំ។ “ ម្តាយឪពុក។ ហើយ​ប្អូន​ប្រុស​របស់​ខ្ញុំ​បាន​សម្រេច​ហើយ»។

បុរសនោះឈរស្ងៀម ហាក់ដូចជាកំពុងស្តាប់។

បុរស​នោះ​បាន​និយាយ​ថា​៖ «​បាទ រួចរាល់​ហើយ»។ "Voldemort បានមកមុនកញ្ចក់នេះហើយត្រូវបានជាប់ដោយវិធីសាស្ត្ររបស់ Merlin ។ គាត់​គឺ​ជា​ភាព​ភ័យ​រន្ធត់​តែ​មួយ​គត់​ទៀត​ហើយ​ឥឡូវ​នេះ»។

ភាពស្ងប់ស្ងាត់នៃការស្តាប់ម្តងទៀត។

“ខ្ញុំ​ចង់​ឲ្យ​ខ្ញុំ​ស្តាប់​បង្គាប់​បង​ប្អូន ប៉ុន្តែ​វា​ល្អ​ជាង​តាម​វិធី​នេះ”។ បុរសនោះឱនក្បាល។ "គាត់ត្រូវបានបដិសេធមិនស្លាប់របស់គាត់ជារៀងរហូត; ការ​សងសឹក​នោះ​គឺ​ជា​រឿង​ដ៏​គួរ​ឲ្យ​ខ្លាច​ណាស់»។

Harry មានអារម្មណ៍ស្រើបស្រាល ដោយមើលរឿងនេះ យល់ថានេះគឺ\emph{មិនមែន} នូវអ្វីដែល Dumbledore នឹងនិយាយ វាហាក់បីដូចជាមនុស្សចំបើង ជាគំរូរាក់ៗ… ប៉ុន្តែបន្ទាប់មកនេះមិនមែនជាវិញ្ញាណរបស់ Aberforth ពិតប្រាកដនោះទេ នេះគឺ តើអ្នកណាដែលសាស្រ្តាចារ្យ Quirrell ស្រមៃថា Dumbledore ស្រមៃថា Aberforth គឺជា ហើយរូបភាពដែលឆ្លុះបញ្ចាំងទ្វេដងនៃ Aberforth នឹងមិនកត់សំគាល់អ្វីខុសឡើយ...

បុរស​ដែល​គិត​ថា​គាត់​ជា Dumbledore បាន​និយាយ​ថា៖ «វា​ដល់​ពេល​ត្រូវ​ប្រគល់​ថ្ម​របស់​ទស្សនវិទូ​មក​វិញ»។ "វាត្រូវតែត្រលប់ទៅការរក្សារបស់ Master Flamel ឥឡូវនេះ។"

ភាពស្ងប់ស្ងាត់ក្នុងការស្តាប់។

បុរសនោះបាននិយាយថា “ទេ” លោកម្ចាស់ Flamel បានរក្សាវាឱ្យមានសុវត្ថិភាពជាច្រើនឆ្នាំមកនេះ ពីអស់អ្នកដែលចង់ស្វែងរកអមតៈ ហើយខ្ញុំគិតថាវានឹងមានសុវត្ថិភាពបំផុតនៅក្នុងដៃរបស់គាត់… ទេ អាបឺហ្វត ខ្ញុំគិតថាបំណងរបស់គាត់គឺល្អ”។

Harry មិនអាចគ្រប់គ្រងភាពតានតឹងដែលកំពុងរត់កាត់គាត់ដូចជាខ្សែភ្លើងផ្ទាល់ទេ។ គាត់មានបញ្ហាក្នុងការដកដង្ហើម។ មិនល្អឥតខ្ចោះ មន្តស្នេហ៍ Confundus របស់សាស្រ្តាចារ្យ Quirrell មានភាពមិនល្អឥតខ្ចោះ។ បុគ្គលិកលក្ខណៈមូលដ្ឋានរបស់សាស្រ្តាចារ្យ Quirrell កំពុងតែលេចធ្លាយ ហើយឃើញសំណួរជាក់ស្តែង៖ ហេតុអ្វីបានជាលោក Nicholas Flamel ខ្លួនឯងមានថ្ម ប្រសិនបើអមតៈពិតជាអាក្រក់ម្ល៉េះ? ទោះបីជាសាស្រ្តាចារ្យ Quirrell ស្រមៃថា Dumbledore ពិការភ្នែកចំពោះសំណួរក៏ដោយ គាត់មិនបានបញ្ចូលឃ្លានៅក្នុង Confundus ដែលនិយាយថា \emph{រូបភាពរបស់ Dumbledore នៃ Aberforth} នឹងមិនគិតពីវាទេ។ ហើយ​អ្វី​ទាំង​អស់​នេះ​នៅ​ទី​បំផុត​គឺ​ជា​ការ​ឆ្លុះ​បញ្ចាំង​ពី​គំនិត​ផ្ទាល់​ខ្លួន​របស់​សាស្ត្រាចារ្យ Quirrell ជា​រូបភាព​ពី​ក្នុង​ភាព​វៃឆ្លាត​របស់ Tom Riddle…

"បំផ្លាញវា?" បុរសនោះបាននិយាយ។ “ប្រហែល។ ខ្ញុំមិនប្រាកដថា \emph{អាច}ត្រូវបានបំផ្លាញ ឬ Master Flamel នឹងធ្វើវាតាំងពីយូរយារណាស់មកហើយ។ ខ្ញុំគិតថាជាច្រើនដងហើយ ដែលគាត់បានសោកស្តាយក្នុងការធ្វើវា... អាប៊ើហ្វត ខ្ញុំបានសន្យាជាមួយគាត់ ហើយយើងមិនមែនជាមនុស្សចាស់ ឬឆ្លាតពេកនោះទេ។ ថ្មរបស់ទស្សនវិទូត្រូវតែត្រឡប់ទៅរកអ្នកដែលបានបង្កើតវាវិញ»។

ហើយដង្ហើមរបស់ Harry បានឈប់។

បុរស​នោះ​កំពុង​កាន់​ដុំ​កញ្ចក់​ពណ៌​ក្រហម​ឆ្អៅ​មិន​ទៀងទាត់​នៅ​ដៃ​ឆ្វេង​ដែល​ទំហំ​ប្រហែល​មេដៃ Harry ចាប់​ពី​ក្រចកដៃ​ដល់​សន្លាក់​ដំបូង។ ផ្ទៃនៃកញ្ចក់ពណ៌ក្រហមធ្វើឱ្យវាមើលទៅសើម; រូបរាង​គឺ​ចេញ​ពី​ឈាម ផ្អាក​ទាន់​ពេល​វេលា ហើយ​ធ្វើ​ជា​ផ្ទៃ​ប្រឡាក់។

បុរស​នោះ​និយាយ​ស្ងាត់ៗ​ថា «​អរគុណ​បង​ប្រុស​»។

\emph{តើនោះជាអ្វីដែលថ្មគួរមើលទៅ? តើសាស្រ្តាចារ្យ Quirrell ដឹងថាថ្មពិតគួរមានលក្ខណៈដូចម្តេច? តើកញ្ចក់នឹងផ្តល់ថ្មពិតមកវិញក្រោមលក្ខខណ្ឌទាំងនេះ ឬធ្វើត្រាប់តាមហើយប្រគល់វាវិញ?}

ហើយបន្ទាប់មក -

“ទេ អារីយ៉ាណា” បុរសនោះនិយាយដោយញញឹមថ្នមៗ “ខ្ញុំខ្លាចខ្ញុំត្រូវតែទៅឥឡូវនេះ។ អត់ធ្មត់ អូនសម្លាញ់ វានឹងឆាប់គ្រប់គ្រាន់ដែលខ្ញុំចូលរួមជាមួយអ្នកដោយការពិត… ហេតុអ្វី? ហេតុអ្វីខ្ញុំមិនច្បាស់ថាហេតុអ្វីខ្ញុំត្រូវទៅ… ពេលខ្ញុំកាន់ថ្ម ខ្ញុំត្រូវដើរចេញពីកញ្ចក់ ហើយរង់ចាំ Master Flamel ទាក់ទងមកខ្ញុំ ប៉ុន្តែខ្ញុំមិនច្បាស់ថាហេតុអ្វីបានជាខ្ញុំត្រូវដើរចេញពីកញ្ចក់ដើម្បីធ្វើរឿងនោះ។ …” បុរសនោះដកដង្ហើមធំ។ “អេ ខ្ញុំ​ចាស់​ទៅ​ហើយ។ ជាការប្រសើរណាស់ សង្រ្គាមដ៏គួរឱ្យភ័យខ្លាចនេះបានបញ្ចប់នៅពេលដែលវាបានកើតឡើង។ ខ្ញុំ​ស្មាន​ថា​មិន​មាន​គ្រោះ​ថ្នាក់​ទេ បើ​ខ្ញុំ​និយាយ​ទៅ​កាន់​អ្នក​មួយ​គ្រា​ដែល​ជា​ទី​ស្រឡាញ់​របស់​ខ្ញុំ បើ​អ្នក​ប្រាថ្នា​ដូច្នេះ»។

ការឈឺក្បាលចាប់ផ្តើមនៅពីក្រោយភ្នែករបស់ Harry; ផ្នែកខ្លះរបស់ Harry កំពុងព្យាយាមផ្ញើសារអំពីការមិនដកដង្ហើមមួយរយៈ ប៉ុន្តែគ្មាននរណាម្នាក់ស្តាប់ឡើយ។ \emph{Imperfect}, មន្តស្នេហ៍ Confundus របស់សាស្រ្តាចារ្យ Quirrell មានភាពមិនល្អឥតខ្ចោះ រូបភាពរបស់សាស្រ្តាចារ្យ Quirrell នៃរូបភាពរបស់ Dumbledore របស់ Ariana ចង់និយាយជាមួយ Dumbledore ហើយប្រហែលជាមិនចង់រង់ចាំទេ ព្រោះសាស្រ្តាចារ្យ Quirrell ដឹងក្នុងកម្រិតខ្លះថា ពិតជាមិនមាន ជីវិតបន្ទាប់បន្សំ និងកម្លាំងរុញច្រានដែលបានដាក់ពីមុនដើម្បីចាកចេញបន្ទាប់ពីទទួលបាន Stone \emph{មិនឈរតាមទឡ្ហីករណ៍របស់ Riddle-Ariana…}

ហើយបន្ទាប់មក Harry មានអារម្មណ៍ថាខ្លួនគាត់ស្ងប់ស្ងាត់ណាស់។ គាត់ចាប់ផ្តើមដកដង្ហើមម្តងទៀត។

ទោះយ៉ាងណាក៏ដោយ មិនមាន Harry អាចធ្វើបានច្រើនអំពីវាទេ។ សាស្រ្តាចារ្យ Quirrell បានបញ្ឈប់ Harry ពីការអន្តរាគមន៍។ ជាការប្រសើរណាស់, សាស្រ្តាចារ្យ Quirrell ត្រូវបានស្វាគមន៍ក្នុងការប្រមូលផលនៃការសម្រេចចិត្តនោះ។ ប្រសិនបើ​លទ្ធផល​ចាប់​បាន Harry ក៏​ដូច្នោះ​ដែរ។

បុរសដែលគិតថាគាត់ជា Dumbledore ភាគច្រើនងក់ក្បាលដោយអត់ធ្មត់ ពេលខ្លះឆ្លើយតបទៅបងស្រីជាទីស្រឡាញ់របស់គាត់។ ពេល​ខ្លះ​បុរស​នោះ​មើល​ទៅ​ម្ខាង​មិន​ស្រួល; ហាក់បីដូចជាមានអារម្មណ៍ថាមានកម្លាំងចិត្តចង់ទៅ ប៉ុន្តែការបង្រ្កាបនូវកម្លាំងរុញច្រាននោះដោយភាពអត់ធ្មត់ និងភាពគួរសម និងការយកចិត្តទុកដាក់ចំពោះប្អូនស្រីរបស់គាត់ដែលសាស្រ្តាចារ្យ Quirrell ស្រមៃថា Albus Dumbledore មាន។

Harry បាន​ឃើញ​វា​ភ្លាម​ដែល Confundus ពាក់​ចេញ ហើយ​ការ​បញ្ចេញ​មតិ​របស់​បុរស​នោះ​បាន​ផ្លាស់​ប្តូរ ក្លាយ​ជា​មុខ​របស់​សាស្ត្រាចារ្យ Quirrell ម្ដង​ទៀត។

ហើយក្នុងពេលជាមួយគ្នានោះ កញ្ចក់បានផ្លាស់ប្តូរ លែងបង្ហាញ Harry ពីការឆ្លុះបញ្ជាំងបន្ទប់ ដោយបង្ហាញទម្រង់នៃ Albus Dumbledore ពិតប្រាកដ ហាក់ដូចជាគាត់ឈរនៅពីក្រោយកញ្ចក់ ហើយអាចមើលឃើញតាមរយៈវា។

មុខ​របស់ Dumbledore ពិត​ប្រាកដ​ត្រូវ​បាន​កំណត់ ហើយ​ក្រៀម​ក្រំ។

Albus Dumbledore បាននិយាយថា "ជំរាបសួរ Tom" ។

%  LocalWords:  ven balefire noitilov detalo partxe tnere ruoy becafruoy wo
%  LocalWords:  hsi
