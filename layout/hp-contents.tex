\cleartorecto
\renewcommand*{\printtoctitle}[1]{\centering\Huge{\hpfont\MakeUppercase{#1}}}
\setlength{\cftbeforechapterskip}{.5\baselineskip plus 12pt minus 3pt}
\setlength{\cftbeforepartskip}{1\baselineskip plus 12pt minus 3pt}
\renewcommand*{\cftpartleader}{\space—\space}
\renewcommand*{\cftpartfillnum}[1]{%
 \cftpartafterpnum\par}
\renewcommand*{\cftchapterleader}{\space—\space}
\renewcommand*{\cftchapterfillnum}[1]{%
{\cftchapterleader}\nobreak%
{#1}%
 \cftchapterafterpnum\par}
\setrmarg{0em}
\setlength\cftpartindent{0pt}
\setlength\cftpartnumwidth{0pt}
\setlength\cftchapterindent{0pt}
\setlength\cftchapternumwidth{0pt}
\renewcommand*{\cftpartafterpnum}{\cftparfillskip}
\renewcommand*{\cftchapterafterpnum}{\cftparfillskip}
\renewcommand*{\cftpartfont}{}
\renewcommand*{\cftchapterfont}{}

\renewcommand\partnumberline[1]{\hfil\thispagestyle{empty}{\hpfont\textls[30]{\large\MakeUppercase{\partname} #1}}\hfil\strut\par\nopagebreak\hfil}
\renewcommand\chapternumberline[1]{\hfil\thispagestyle{empty}{\hpfont\textls[30]{\IfInteger{#1}{\NUMBERstringnum{#1}}{ឧបសម្ព័ន្ធ #1}}}\hfil\strut\par\nopagebreak\hfil}

\settocdepth{chapter}
\phantomsection
\label{contents}

\thispagestyle{empty}

\tableofcontents*

\clearpage
\thispagestyle{empty}
\hbox{}
